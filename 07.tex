\chapter{}

\topic{McKay's conjecture}
\label{McKay}

Let $G$ be a finite group and let $p$ be a prime number dividing
$|G|$. Write $\Syl_p(G)$ to denote the (non-empty) set of Sylow 
$p$-subgroups of $G$. Recall that 
the \emph{normalizer} of $P$ is the subgroup
\[
N_G(P)=\{g\in G:gPg^{-1}=P\}.
\]

The following conjecture was made by McKay for the prime $p=2$ and simple groups 
and later generalized by Alperin in~\cite{MR0404417} and independently
by Isaacs en~\cite{MR332945}. 

\begin{conjecture}[McKay]
\index{McKay's conjecture}
Let $p$ be a prime. If  
$G$ is a finite group and $P\in\Syl_p(G)$, then 
\[
|\{\chi\in\Irr(G):p\nmid \chi(1)\}|
=|\{\psi\in\Irr(N_G(P))|:p\nmid\psi(1)\}|.
\]
\end{conjecture}

McKay's conjecture is still open and is an important problem in representation theory. 
The conjecture was proved for several classes of groups. Isaacs 
proved the conjecture for solvable groups, see for example~\cite{MR332945,MR3791517}. 
Malle and Sp\"ath prove the conjecture for $p=2$. 

\begin{theorem}[Malle--Sp\"ath]
\index{Malle--Sp\"ath's theorem}
If $G$ is finite and $P\in\Syl_2(G)$,
then 
\[
|\{\chi\in\Irr(G):2\nmid \chi(1)\}|
=|\{\psi\in\Irr(N_G(P))|:2\nmid\psi(1)\}|.
\]
\end{theorem}

The proof appears in~\cite{MR3549625} and uses the classification of 
finite simple groups. It uses a deep result of 
Isaacs, Malle and Navarro~\cite{MR2336079}. 

We cannot prove Malle--Sp\"ath theorem here. However, 
we can use the computer to prove some particular cases
with the following cuntion: 

\begin{lstlisting}
gap> McKay := function(G, p)
> local N, n, m;
> N := Normalizer(G, SylowSubgroup(G, p));
> n := Number(Irr(G), x->Degree(x) mod p <> 0);
> m := Number(Irr(N), x->Degree(x) mod p <> 0);
> if n = m then
> return true;
> else
> return false;
> fi;
> end;
function( G, p ) ... end
\end{lstlisting}

As a concrete example, let us 
verify McKay's conjecture for the Mathieu simple group 
$M_{11}$ of order 7920. 

\begin{lstlisting}
gap> M11 := MathieuGroup(11);;
gap> PrimeDivisors(Order(M11));
[ 2, 3, 5, 11 ]
gap> McKay(M11,2);
true
gap> McKay(M11,3);
true
gap> McKay(M11,5);
true
gap> McKay(M11,11);
true
\end{lstlisting}

The following conjecture refines McKay's conjecture. It was
formulated by Isaacs and Navarro:

\begin{conjecture}[Isaacs--Navarro]
\index{Isaacs--Navarro's conjecture}
Let $p$ be a prime and $k\in\Z$. 
If $G$ is a finite group and $P\in\Syl_p(G)$,
then
\begin{align*}
|\{\chi\in\Irr(G):&p\nmid \chi(1)\text{ y }\chi(1)\equiv\pm k\bmod p\}|\\
&=|\{\psi\in\Irr(N_G(P))|:p\nmid\psi(1)\text{ y }\psi(1)\equiv\pm k\bmod p\}|.
\end{align*}
\end{conjecture}

Isaacs--Navarro's conjecture is still open. However, 
it is known to be true for solvable groups, 
sporadic simple groups and 
symmetric groups, see~\cite{MR1935849}. 

\begin{lstlisting}
gap> IsaacsNavarro := function(G, k, p)
> local mG, mN, N;
> N := Normalizer(G, SylowSubgroup(G, p));
> mG := Number(Filtered(Irr(G), x->Degree(x)\
> mod p <> 0), x->Degree(x) mod p in [-k,k] mod p);
> mN := Number(Filtered(Irr(N), x->Degree(x)\
> mod p <> 0), x->Degree(x) mod p in [-k,k] mod p);
> if mG = mN then
> return mG;
> else
> return false;
> fi;
> end;
function( G, k, p ) ... end
\end{lstlisting}

It is an exercise to verify Isaacs--Navarro's conjecture in some
small cases, for example Mathieu simple group $M_{11}$. 

\topic{Commutators}
\label{commutators}

Let $G$ be a finite group with conjugacy classes $C_1,\dots,C_s$. For
$i\in\{1,\dots,s\}$ and $\chi\in\Irr(G)$ let  
\[
\omega_{\chi}(C_i)=\frac{|C_i|\chi(C_i)}{\chi(1)}\in\A.
\]
In the proof of Theorem \ref{thm:A}, Equality \eqref{eq:omega}, 
we obtained
that 
\begin{equation}
\label{eq:again_omega}
\omega_\chi(C_i)\omega_\chi(C_j)=\sum_{k=1}^ka_{ijk}\omega_{\chi}(C_k),
\end{equation}
where $a_{ijk}$ is the number of solutions 
of $xy=z$ with $x\in C_i$, $y\in c_j$ and $z\in C_k$. 

\begin{theorem}[Burnside]
    \index{Burnside's theorem}
    Let $G$ be a finite group with conjugacy classes $C_1,\dots,C_s$. 
    Then
    \[
    a_{ijk}
    =\frac{|C_i||C_j|}{|G|}
    \sum_{\chi\in\Irr(G)}\frac{\chi(C_i)\chi(C_j)\overline{\chi(C_k)}}{\chi(1)}.
    \]
\end{theorem}

\begin{proof}
    By \eqref{eq:again_omega}, 
    \[
    \frac{|C_i||C_j|}{\chi(1)}\chi(C_i)\chi(C_j)
    =\sum_{k=1}^sa_{ijk}|C_k|\chi(C_k).
    \]
    Multiply by $\overline{\chi(C_l)}$ and sum over all
    $\chi\in\Irr(G)$ to obtain 
    \begin{align*}
    |C_i||C_j|\sum_{\chi\in\Irr(G)}\frac{\overline{\chi(C_l)}}{\chi(1)}\chi(C_i)\chi(C_j)
    &=\sum_{\chi\in\Irr(G)}\sum_{k=1}^sa_{ijk}|C_k|\chi(C_k)\overline{\chi(C_l)}\\
    &=\sum_{k=1}^sa_{ijk}|C_k|\sum_{\chi\in\Irr(G)}\chi(C_k)\overline{\chi(C_l)}\\
    &=a_{ijk}|G|,
    \end{align*}
    because 
    \[
    \sum_{\chi\in\Irr(G)}\chi(C_k)\overline{\chi(C_l)}=\begin{cases}
        \frac{|G|}{|C_l|} & \text{if $k=l$},\\
        0 & \text{otherwise}.
        \end{cases}\qedhere
    \]
\end{proof}

\topic{Ore's conjecture}
\label{Ore}

In 1951 Ore and independently Ito
proved that every element of any alternating simple group is a commutator. 
Ore also mentioned that ``it is possible that a similar theorem holds for any simple group of finite order, but it seems that at present we do not have the necessary methods to investigate the question". 

\begin{conjecture}[Ore]
\index{Ore's conjecture}
    Let $G$ be a finite simple non-abelian group. 
    Then every element of $G$ is a commutator. 
\end{conjecture}

Ore's conjecture was proved in 2010:

\begin{theorem}[Liebeck--O'Brien--Shalev--Tiep]
\index{Liebeck--O'Brien--Shalev--Tiep's theorem}
    Every element of a non-abelian finite simple group is a commutator.     
\end{theorem}

The proof appears in~\cite{MR2654085}. It needs about 70 pages and
uses the classification of finite simple groups (CFSG) and character theory.
See \cite{MR3289286} for more information on Ore's conjecture and its proof 
a~\cite{MR3289286}. 

Despite the fact that the proof of Ore's conjecture is too complicated for 
this course, we can use the computer to 
prove the conjecture in some particular cases:

\begin{proposition}
    Ore's conjecture is true 
    for sporadic simple groups.
\end{proposition}

\begin{proof}
    Let $G$ be a finite simple group. 
    We now that $g\in G$ is a commutator if and only if 
    $\sum_{\chi\in\Irr(G)}\frac{\chi(g)}{\chi(1)}\ne 0$. Let us write
    a computer script to check whether every element in a group 
    is a commutator. Our
    function needs the character table of a group and returns 
    \lstinline{true} if every element of the group is a commutator and
    \lstinline{false} otherwise. 
\begin{lstlisting}
gap> Ore := function(char) 
> local s,f,k;
> for k in [1..NrConjugacyClasses(char)] do
> s := 0;
> for f in Irr(char) do
> s := s+f[k]/Degree(f);  
> od;
> if s<=0 then
> return false;
> fi;
> od;
> return true;
> end;
function( char ) ... end
\end{lstlisting}
Now we check Ore's conjecture for Mathieu simple groups
and for the Monster group: 
\begin{lstlisting}
gap> Ore(CharacterTable("M11"));
true
gap> Ore(CharacterTable("M12"));
true
gap> Ore(CharacterTable("M22"));
true
gap> Ore(CharacterTable("M23"));
true
gap> Ore(CharacterTable("M24"));
true
gap> Ore(CharacterTable("M"));
true
\end{lstlisting}
It is an exercise to check the conjecture for the other finite sporadic 
simple groups $McL$, $Ru$, $Ly$, $Suz$, $He$, $HN$, $Th$, $Fi_{22}$, $Fi_{23}$, $Fi_{24}'$, $B$, $M$ 
\end{proof}

See \cite{MR3821142} for other applications of character theory. 
