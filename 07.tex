\chapter{}

\topic{McKay's conjecture}
\label{McKay}

Let $G$ be a finite group and let $p$ be a prime number dividing
$|G|$. Write $\Syl_p(G)$ to denote the (non-empty) set of Sylow 
$p$-subgroups of $G$. Recall that 
the \emph{normalizer} of $P$ is the subgroup
\[
N_G(P)=\{g\in G:gPg^{-1}=P\}.
\]

The following conjecture was made by McKay for the prime $p=2$ and simple groups 
and later generalized by Alperin in~\cite{MR0404417} and independently
by Isaacs in~\cite{MR332945}. 

\begin{conjecture}[McKay]
\index{McKay's conjecture}
\label{conjecture:McKay}
Let $p$ be a prime. If  
$G$ is a finite group and $P\in\Syl_p(G)$, then 
\[
|\{\chi\in\Irr(G): p\nmid \chi(1)\}|
=|\{\psi\in\Irr(N_G(P)): p\nmid\psi(1)\}|.
\]
\end{conjecture}

McKay's conjecture is still open and is a critical problem in representation theory. 
The conjecture was proved for several classes of groups. Isaacs 
proved the conjecture for solvable groups; see~\cite{MR332945,MR3791517}. 
Malle and Sp\"ath prove the conjecture for $p=2$. 

\begin{theorem}[Malle--Sp\"ath]
\index{Malle--Sp\"ath theorem}
If $G$ is finite and $P\in\Syl_2(G)$,
then 
\[
|\{\chi\in\Irr(G): 2\nmid \chi(1)\}|
=|\{\psi\in\Irr(N_G(P)): 2\nmid\psi(1)\}|.
\]
\end{theorem}

The proof appears in~\cite{MR3549625} and uses the classification of 
finite simple groups. It uses a deep result of 
Isaacs, Malle and Navarro~\cite{MR2336079}. 

We cannot prove Malle--Sp\"ath theorem here. However, 
we can use the computer to prove some particular cases
with the following function: 

\begin{lstlisting}
gap> McKay := function(G, p)
> local N, n, m;
> N := Normalizer(G, SylowSubgroup(G, p));
> n := Number(Irr(G), x->Degree(x) mod p <> 0);
> m := Number(Irr(N), x->Degree(x) mod p <> 0);
> return n = m;
> end;
function( G, p ) ... end
\end{lstlisting}

As a concrete example, let us 
verify McKay's conjecture for the Mathieu simple group 
$M_{11}$ of order 7920. 

\begin{lstlisting}
gap> M11 := MathieuGroup(11);;
gap> PrimeDivisors(Order(M11));
[ 2, 3, 5, 11 ]
gap> McKay(M11,2);
true
gap> McKay(M11,3);
true
gap> McKay(M11,5);
true
gap> McKay(M11,11);
true
\end{lstlisting}

The following conjecture refines McKay's conjecture. It was
formulated by Isaacs and Navarro:

\begin{conjecture}[Isaacs--Navarro]
\index{Isaacs--Navarro conjecture}
\label{conjecture:IsaacsNavarro}
Let $p$ be a prime and $k\in\Z$. 
If $G$ is a finite group and $P\in\Syl_p(G)$,
then
\begin{align*}
|\{\chi\in\Irr(G):&p\nmid \chi(1)\text{ and }\chi(1)\equiv\pm k\bmod p\}|\\
&=|\{\psi\in\Irr(N_G(P)):p\nmid\psi(1)\text{ and }\psi(1)\equiv\pm k\bmod p\}|.
\end{align*}
\end{conjecture}

Isaacs--Navarro conjecture is still open. However, 
it is known to be true for solvable groups, 
sporadic simple groups and 
symmetric groups, see~\cite{MR1935849}. 

\begin{lstlisting}
gap> IsaacsNavarro := function(G, k, p)
> local m, n, N;
> N := Normalizer(G, SylowSubgroup(G, p));
> m := Number(Filtered(Irr(G), x->Degree(x)\
> mod p <> 0), x->Degree(x) mod p in [-k,k] mod p);
> n := Number(Filtered(Irr(N), x->Degree(x)\
> mod p <> 0), x->Degree(x) mod p in [-k,k] mod p);
> return n=m;
> end;
function( G, k, p ) ... end
\end{lstlisting}

It is an exercise to verify Isaacs--Navarro conjecture in some
small cases, for example, Mathieu simple group $M_{11}$. 

\topic{Commutators}
\label{commutators}

Let $G$ be a finite group with conjugacy classes $C_1,\dots,C_s$. For
$i\in\{1,\dots,s\}$ and $\chi\in\Irr(G)$ let  
\[
\omega_{\chi}(C_i)=\frac{|C_i|\chi(C_i)}{\chi(1)}\in\A.
\]
In the proof of Theorem \ref{thm:B}, Equality \eqref{eq:omega}, 
we obtained
that 
\begin{equation}
\label{eq:again_omega}
\omega_\chi(C_i)\omega_\chi(C_j)=\sum_{k=1}^sa_{ijk}\omega_{\chi}(C_k),
\end{equation}
where $a_{ijk}$ is the number of solutions 
of $xy=z$ with $x\in C_i$, $y\in C_j$ and $z\in C_k$. 

\begin{theorem}[Burnside]
    \index{Burnside's theorem}
    Let $G$ be a finite group with conjugacy classes $C_1,\dots,C_s$. 
    Then
    \[
    a_{ijk}
    =\frac{|C_i||C_j|}{|G|}
    \sum_{\chi\in\Irr(G)}\frac{\chi(C_i)\chi(C_j)\overline{\chi(C_k)}}{\chi(1)}.
    \]
\end{theorem}

\begin{proof}
    By \eqref{eq:again_omega}, 
    \[
    \frac{|C_i||C_j|}{\chi(1)}\chi(C_i)\chi(C_j)
    =\sum_{k=1}^sa_{ijk}|C_k|\chi(C_k).
    \]
    Multiply by $\overline{\chi(C_l)}$ and sum over all
    $\chi\in\Irr(G)$ to obtain 
    \begin{align*}
    |C_i||C_j|\sum_{\chi\in\Irr(G)}\frac{\overline{\chi(C_l)}}{\chi(1)}\chi(C_i)\chi(C_j)
    &=\sum_{\chi\in\Irr(G)}\sum_{k=1}^sa_{ijk}|C_k|\chi(C_k)\overline{\chi(C_l)}\\
    &=\sum_{k=1}^sa_{ijk}|C_k|\sum_{\chi\in\Irr(G)}\chi(C_k)\overline{\chi(C_l)}\\
    &=a_{ijl}|G|,
    \end{align*}
    because 
    \[
    \sum_{\chi\in\Irr(G)}\chi(C_k)\overline{\chi(C_l)}=\begin{cases}
        \frac{|G|}{|C_l|} & \text{if $k=l$},\\
        0 & \text{otherwise}.
        \end{cases}\qedhere
    \]
\end{proof}

\begin{theorem}[Burnside]
\index{Burnside's theorem}
    Let $G$ be a finite group and $g,x\in G$. Then
    $g$ and $[x,y]$ are conjugate for some $y\in G$ if and only if
    \[
    \sum_{\chi\in\Irr(G)}\frac{|\chi(x)|^2\chi(g)}{\chi(1)}>0.
    \]
\end{theorem}

\begin{proof}
    Let $C_1,\dots,C_s$ be the conjugacy classes of $G$. Assume that
    $x\in C_i$ and $g\in C_k$ for some $i$ and $k$. Then
    $C_i^{-1}=\{z^{-1}:z\in C_i\}=C_j$ for some $j$. By Burnside's theorem,
    \[
    a_{ijk}
    =\frac{|C_i|^2}{|G|}\sum_{\chi\in\Irr(G)}\frac{|\chi(C_i)|^2\overline{\chi(C_k)}}{\chi(1)}.
    \]
    We first prove $\impliedby$. Since $a_{ijk}>0$, 
    there exist $u\in C_i$ and $v\in C_j$ such that 
    $g=uv$ (since $zgz^{-1}=u_1v_1$ for some $u_1\in C_i$ and
    $v_1\in C_j$, it follows that 
    $g=(z^{-1}u_1z)(z^{-1}v_1z)$, so take $u=z^{-1}u_1z\in C_i$ and 
    $v=z^{-1}v_1z\in C_j$). If $x$ and $u$ are conjugate, say 
    $u=zxz^{-1}$ for some $z$, then $x^{-1}$ and 
    $v$ are conjugate, as 
    \[
    zxz^{-1}=u\implies zx^{-1}z^{-1}=u^{-1}\in C_i^{-1}=C_j.
    \]
    Let $z_2\in G$ be such that $z_2x^{-1}z_2^{-1}=v$. 
    If $y=z^{-1}z_2$, then $g$ and $[x,y]$ are conjugate, as 
    \[
    g=uv=(zxz^{-1})(z_2x^{-1}z_2^{-1})=(zxyx^{-1}y^{-1})yz_2^{-1}
    =z[x,y]z^{-1}.
    \]
    
    We now prove $\implies$. Let $y\in G$ be such that $g$ and
    $[x,y]$ are conjugate, say $g=z[x,y]z^{-1}$ for some $z\in G$. Let
    $v=yxy^{-1}$. Then 
    $g$ and $xv^{-1}=xyx^{-1}y^{-1}=[x,y]$ are conjugate. In particular, 
    since $g\in C_iC_j$, $a_{ijk}>0$. 
\end{proof}    

\begin{exercise}
    Let $G$ be a finite group, $g\in G$ and $\chi\in\Irr(G)$. 
    Prove that 
    \begin{align*}
        &\sum_{h\in G}\chi([g,h])=\frac{|G|}{\chi(1)}|\chi(g)|^2.
    \shortintertext{Prove also that }
        &\chi(g)\chi(h)=\frac{\chi(1)}{|G|}\sum_{z\in G}\chi(zgz^{-1}h)
    \end{align*}
    holds for all $h\in G$. 
\end{exercise}

We now prove a theorem of Frobenius that 
uses character tables to recognize commutators. For that purpose, 
let 
\[
\tau(g)=|\{(x,y)\in G\times G:[x,y]=g\}|.
\]

\begin{theorem}[Frobenius]
    \index{Frobenius' theorem}
    Let $G$ be a finite group. Then
    \[
    \tau(g)=|G|\sum_{\chi\in\Irr(G)}\frac{\chi(g)}{\chi(1)}.
    \]
\end{theorem}

\begin{proof}
    Let $\chi\in\Irr(G)$. Since $\chi$ is irreducible, 
    \[
    1=\langle\chi,\chi\rangle
    =\frac{1}{|G|}\sum_{z\in G}\chi(z)\overline{\chi(z)}
    =\frac{1}{|G|}\sum_{C}|C|\chi(C)\overline{\chi(C)},
    \]
    where the last sum is taken over all conjugacy classes of $G$. 
    Let $g\in G$ and $C$ be the conjugacy class of $g$. The equation
    $xu^{-1}=g$ with $x\in C$ and $u\in C^{-1}$ has 
    \[
        \frac{|C||C^{-1}|}{|G|}\sum_{\chi\in\Irr(G)}\frac{\chi(C)\chi(C^{-1})\chi(g^{-1})}{\chi(1)}
    \]
    solutions. If $(x,u)$ is a solution of $xu^{-1}=g$, then
    there are $|C_G(x)|$ elements $y$ such that $yxy^{-1}=u$. ($yxy^{-1}=u=y_1xy_1^{-1}$ implies that $y_1^{-1}y\in C_G(x)$ which implies $yC_G(x)=y_1C_G(x)$.) Now 
    $[x,y]=x(yx^{-1}y^{-1})=g$ has 
    \[
    |C|\sum_{\chi}\frac{\chi(C)\chi(C^{-1})\chi(g^{-1})}{\chi(1)}
    \]
    solutions, where the sum is taken over all irreducible characters of $G$. 
    Now we sum over all conjugacy classes of $G$:
    \begin{align*}
        \sum_{C}\sum_{\chi}|C|\frac{\chi(C)\chi(C^{-1})\chi(g^{-1})}{\chi(1)}
        &=\sum_{\chi}\frac{\chi(g^{-1})}{\chi(1)}\left(\sum_C|C|\chi(C)\chi(C^{-1})\right)\\
        &=|G|\sum_{\chi}\frac{\chi(g^{-1})}{\chi(1)}.
    \end{align*}
    From this, the formula follows. 
\end{proof}

Application:

\begin{corollary}
    Let $G$ be a finite group and $g\in G$. Then $g$ 
    is a commutator if and only if 
    \[
    \sum_{\chi\in\Irr(G)}\frac{\chi(g)}{\chi(1)}\ne0.
    \]
\end{corollary}

\topic{Ore's conjecture}
\label{Ore}

In 1951 Ore and independently It\^o
proved that every element of any alternating simple group is a commutator. 
Ore also mentioned that ``it is possible that a similar theorem holds for any simple group of finite order, but it seems that at present we do not have the necessary methods to investigate the question". 

\begin{conjecture}[Ore]
\index{Ore's conjecture}
\label{conjecture:Ore}
    Let $G$ be a finite simple non-abelian group. 
    Then every element of $G$ is a commutator. 
\end{conjecture}

Ore's conjecture was proved in 2010:

\begin{theorem}[Liebeck--O'Brien--Shalev--Tiep]
\index{Liebeck--O'Brien--Shalev--Tiep theorem}
    Every element of a non-abelian finite simple group is a commutator.     
\end{theorem}

The proof appears in~\cite{MR2654085}. It needs about 70 pages and
uses the classification of finite simple groups (CFSG) and character theory.
See \cite{MR3289286} for more information on Ore's conjecture and its proof. 

Despite the fact that the proof of Ore's conjecture is too complicated for 
this course, we can use the computer to 
prove the conjecture in some particular cases:

\begin{proposition}
    Ore's conjecture is true 
    for sporadic simple groups.
\end{proposition}

\begin{proof}
    Let $G$ be a finite simple group. 
    We know that $g\in G$ is a commutator if and only if 
    $\sum_{\chi\in\Irr(G)}\frac{\chi(g)}{\chi(1)}\ne 0$. Let us write
    a computer script to check whether every element in a group 
    is a commutator. Our
    function needs the character table of a group and returns 
    \lstinline{true} if every element of the group is a commutator and
    \lstinline{false} otherwise. 
\begin{lstlisting}
gap> Ore := function(char) 
> local s,f,k;
> for k in [1..NrConjugacyClasses(char)] do
> s := 0;
> for f in Irr(char) do
> s := s+f[k]/Degree(f);  
> od;
> if s<=0 then
> return false;
> fi;
> od;
> return true;
> end;
function( char ) ... end
\end{lstlisting}
Now we check Ore's conjecture for Mathieu simple groups
and for the Monster group: 
\begin{lstlisting}
gap> Ore(CharacterTable("M11"));
true
gap> Ore(CharacterTable("M12"));
true
gap> Ore(CharacterTable("M22"));
true
gap> Ore(CharacterTable("M23"));
true
gap> Ore(CharacterTable("M24"));
true
gap> Ore(CharacterTable("M"));
true
\end{lstlisting}
It is an exercise to check the conjecture for the other finite sporadic 
simple groups $McL$, $Ru$, $Ly$, $Suz$, $He$, $HN$, $Th$, $Fi_{22}$, $Fi_{23}$, $Fi_{24}'$, $B$, $M$ 
\end{proof}

See \cite{MR3821142} for other applications of character theory. 

\topic{Cauchy--Frobenius--Burnside theorem}

\begin{theorem}[Cauchy--Frobenius--Burnside]
\label{thm:CFB}
\index{Cauchy--Frobenius--Burnside theorem}
    Let $G$ be a finite group that acts on a finite set $X$. 
    If $m$ is the number of orbits, then 
    \[
    m=\frac{1}{|G|}\sum_{g\in G}|\Fix(g)|,
    \]
    where $\Fix(g)=\{x\in X:g\cdot x=x\}$. 
\end{theorem}

\begin{proof}
    Let $X=\{x_1,\dots,x_n\}$ and $V$ be the complex vector space with basis $\{x_1,\dots,x_n\}$. 
    Let $\rho\colon G\to\GL_n(\C)$, $g\mapsto\rho_g$, be the representation
    \[
    (\rho_g)_{ij}=\begin{cases}
        1 & \text{if $g\cdot x_j=x_i$},\\
        0 & \text{otherwise}.
        \end{cases}
    \]
    In particular, $(\rho_g)_{ii}=1$ if $x_i\in\Fix(g)$ and 
    $(\rho_g)_{ii}=0$ if $x_i \notin \Fix(g)$. Thus
    \[
    \chi_\rho(g)=\trace\rho_g=\sum_{i=1}^n(\rho_g)_{ii}=|\Fix(g)|.
    \]
    
    Recall that 
    \begin{gather*}
        V^G=\{v\in V:g\cdot v=v\text{ for all $g\in G$}\}
    \shortintertext{and that}    
        \dim V^G=\frac{1}{|G|}\sum_{z\in G}\chi_{\rho}(z)=\langle\chi_\rho,\chi_1\rangle
    \end{gather*}
    where $\chi_1$ is the trivial character of $G$. 
    
    We can assume that, after a possible re-enumeration,
    $x_1,\dots,x_m$ are the representatives of the orbits 
    of $G$ on $X$. For $i\in\{1,\dots,m\}$, let
    $v_i=\sum_{x\in G\cdot x_i}x$.
    
    \begin{claim}
        $\{v_1,\dots,v_m\}$ is a basis of $V^G$. 
    \end{claim}
    
    If $g\in G$, then $g\cdot v_i=\sum_{x\in G\cdot x_i}g\cdot x=
    \sum_{y\in G\cdot x_i}y=v_i$. Hence $\{v_1,\dots,v_m\}\subseteq V^G$. Moreover, 
    $\{v_1,\dots,v_m\}$ is linearly independent because the $v_j$ are
    orthogonal and non-zero:
    \[
    \langle v_i,v_j\rangle=\begin{cases}
        |G\cdot x_i| & \text{if $i=j$},\\
        0 & \text{otherwise}.
        \end{cases}
    \]
    We now prove that $V^G=\langle v_1,\dots,v_m\rangle$. Let $v\in V^G$.
    Then $v=\sum_{x\in X}\lambda_xx$ for some coefficients $\lambda_x\in\C$.
    If $g\in G$, then $g\cdot v=v$. Since 
    \[
    \sum_{x\in X}\lambda_xx=v=g\cdot v
    =\sum_{x\in X}\lambda_x(g\cdot x)
    =\sum_{x\in X}\lambda_{g^{-1}\cdot x}x,
    \]
    it follows that $\lambda_x=\lambda_{g^{-1}\cdot x}$ for all $x\in X$ and 
    $g\in G$. This means that if $y,z\in X$ and $g\in G$ is such that
    $g\cdot y=z$, then $\lambda_y=\lambda_z$. Thus 
    \[
    v=\sum_{x\in X}\lambda_xx=\sum_{i=1}^m\lambda_{x_i}\sum_{y\in G\cdot x_i}y
    =\sum_{i=1}^m \lambda_{x_i}v_i.
    \]
    
    Hence 
    \[
    m=\dim V^G=\langle\chi_\rho,\chi_1\rangle=\frac{1}{|G|}\sum_{z\in G}\chi_\rho(z)
    =\frac{1}{|G|}\sum_{z\in G}|\Fix(z)|.\qedhere 
    \]
\end{proof}

It is possible to give an alternative short proof of the theorem. For example, 
for transitive actions (i.e. $m=1$), we proceed as follows:
\[
\sum_{g\in G}|\Fix(g)|=\sum_{g\in G}\sum_{\substack{x\in X\\g\cdot x=x}}1
=\sum_{x\in X}\sum_{\substack{g\in G\\g\cdot x=x}}1
=\sum_{x\in X}|G_x|=|G_x||X|=|G|.
\]

\begin{exercise}
\label{xca:CFB}
    Use the previous idea to prove Theorem \ref{thm:CFB}. 
\end{exercise}

\index{Orbital}
\index{Rank}
Let $G$ act on a finite set $X$. Then $G$ acts
on $X\times X$ by
\begin{equation}
    \label{eq:orbitals}
    g\cdot (x,y)=(g\cdot x,g\cdot y).
\end{equation}
The orbits of this action are called
the \textbf{orbitals} of $G$ on $X$. The \textbf{rank} 
of $G$ on $X$ is the number of orbitals. 

\begin{proposition}
    Let $G$ be a group that acts on a finite set $X$.
    The rank of $G$ on $X$ is 
    \[
    \frac{1}{|G|}\sum_{g\in G}|\Fix(g)|^2.
    \]
\end{proposition}

\begin{proof}
    The action \eqref{eq:orbitals} has 
    $\Fix(g)\times\Fix(g)$ as fixed points, as 
    \begin{align*}
        g\cdot (x,y)=(x,y)&\Longleftrightarrow
        (g\cdot x,g\cdot y)=(x,y)\\
        &\Longleftrightarrow g\cdot x=x\text{ and }g\cdot y=y\Longleftrightarrow
        (x,y)\in\Fix(g)\times\Fix(g).
    \end{align*}
    Now the claim follows from Cauchy--Frobenius--Burnside theorem. 
\end{proof}

\begin{definition}
    Let $G$ act on a finite set $X$. 
    We say that $G$ is \textbf{2-transitive} on $X$ 
    if given $x,y\in X$ with $x\ne y$ and 
    $x_1,y_1\in X$ with $x_1\ne y_1$ there exists 
    $g\in G$ such that $g\cdot x=y$ and $g\cdot x_1=y_1$. 
\end{definition}

The symmetric group $\Sym_n$ acts 2-transitively on $\{1,\dots,n\}$. 

\begin{proposition}
    If $G$ is 2-transitive on $X$, then the rank of $G$ on $X$ is two. 
\end{proposition}

\begin{proof}
    The set $\Delta=\{(x,x):x\in X\}$ is an orbital. The complement
    $X\times X\setminus\Delta$ is another orbital: if $x,x_1,y,y_1\in X$
    are such that $x\ne y$ 
    and $x_1\ne y_1$, then there exists $g\in G$ such that 
    $g\cdot x=y$ and $g\cdot x_1=y_1$, so $g\cdot (x,y)=(x_1,y_1)$. 
\end{proof}
