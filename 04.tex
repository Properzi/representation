\section{Lecture: Week 4}

We will use the following notation: if $\chi$ is a character
of a group $G$ 
and $C$ is a conjugacy class of $G$, then 
$\chi(g)=\chi(xgx^{-1})$ for all $x\in G$. We write 
$\chi(C)$ to denote the value $\chi(g)$ for any $g\in C$. 

\begin{theorem}
\label{thm:B}
    Let $G$ be a finite group, $\chi\in\Irr(G)$ 
    and $K$ be a conjugacy class of $G$. Then 
    \[
    \frac{\chi(K)}{\chi(1)}|K|\in\A. 
    \]
\end{theorem}

We need a lemma. 

\begin{lemma}
    Let $x\in\C$. Then $x\in\A$ if and only if 
    there exist $z_1,\dots,z_k\in\C$ not all zero such that 
    $xz_i=\sum_{j=1}^ka_{ij}z_j$ for some $a_{ij}\in\Z$ and 
    all $i\in\{1,\dots,k\}$. 
\end{lemma}

\begin{proof}
    Let us first prove $\implies$. Let $f=X^k+a_{k-1}X^{k-1}+\cdots+a_1X+a_0\in\Z[X]$
    be such that $f(x)=0$. For $i\in\{1,\dots,k\}$ let 
    $z_i=x^{i-1}$. Then 
    $xz_i=x^i=z_{i+1}$ for all $i\in\{1,\dots,k-1\}$. Moreover, 
    $xz_k=x^k=-a_0-a_1x-\cdots-a_{k-1}x^{k-1}$.
    
    We now prove $\impliedby$. Let $A=(a_{ij})\in\Z^{k\times k}$ and 
    $Z$ be the column vector 
    $Z=\begin{pmatrix}z_1\\\vdots\\z_k\end{pmatrix}$. Note that $Z$ is non-zero. 
    Moreover, $AZ=xZ$, as 
    \[
    (AZ)_i=\sum_{j=1}^ka_{ij}z_j=xz_i=(xZ)_i
    \]
    for all $i$. Thus $x$ is an eigenvalue of $A\in\Z^{k\times k}$ and
    hence $x\in\A$. 
\end{proof}

The previous lemma could be used to give an alternative proof of the fact 
that the algebraic integers form a ring. 

\begin{proof}[Proof of Theorem \ref{thm:B}]
    Let $\varphi$ be a representation of $G$ and 
    $\chi$ be its character. Note that $\varphi$ is irreducible. 
    Let $C_1,\dots,C_r$ be the conjugacy classes of $G$ 
    and for every $i\in\{1,\dots,r\}$ let 
    \[
    T_i=\sum_{x\in C_i}\varphi_x. 
    \]
    
    \begin{claim}
        $T_i=\left(\frac{|C_i|}{\chi(1)}\chi(C_i)\right)\id$. 
    \end{claim}
    
    We proceed in several steps. First, we prove that 
    $T_i=\lambda\id$ for some $\lambda\in\C$. 
    We prove that $T_i$ is a morphism of representations:
    \[
    \varphi_gT_i\varphi_g^{-1}=\sum_{x\in C_i}\varphi_g\varphi_x\varphi_g^{-1}
    =\sum_{x\in C_i}\varphi_{gxg^{-1}}=\sum_{y\in C_i}\varphi_y=T_i.
    \]
    Now Schur's lemma implies that $T_i=\lambda\id$ for some
    $\lambda\in\C$. 
    
    We now prove that 
    \[
    \lambda=\frac{|C_i|\chi(C_i)}{\chi(1)}.
    \]
    To prove
    this we compute $\lambda$:
    \[
    \lambda\chi(1)=\trace(\lambda\id)
    =\trace T_i
    =\sum_{x\in C_i}\trace\varphi_x
    =\sum_{x\in C_i}\chi(x)
    =|C_i|\chi(C_i).
    \]
    Then the claim follows. 
    
    Now we claim that 
    \[
    T_iT_j=\sum_{k=1}^r a_{ijk}T_k
    \]
    for some $a_{ijk}\in\Z_{\geq0}$. In fact, 
    \begin{align*}
        T_iT_j &= \sum_{x\in C_i}\sum_{y\in C_j}\varphi_x\varphi_y
        =\sum_{x\in C_i}\sum_{y\in C_j}\varphi_{xy}
        =\sum_{g\in G}a_{ijg}\varphi_g,
    \end{align*}
    where $a_{ijg}$ is the number of elements $(x,y)\in C_i\times C_j$ 
    such that $g=xy$. 
    
    \begin{claim}
        Once $i$ and $j$ are fixed, $a_{ijg}$ depends only on the conjugacy class of $g$. 
    \end{claim}
    
    Let $X_g=\{(x,y)\in C_i\times C_j:g=xy\}$. If $h=kgk^{-1}$, the map
    \[
    X_g\to X_h,\quad (x,y)\mapsto (kxk^{-1},kyk^{-1}),
    \]
    is well-defined. It is bijective with inverse
    \[
    X_h\to X_g,\quad
    (a,b)\mapsto (k^{-1}ak,k^{-1}bk).
    \]
    Hence $|X_g|=|X_h|$. 

    Let $a_{ijk}$ be the number of elements 
    $(x,y)\in C_i\times C_j$ such that $xy=g$ for some $g\in C_k$. 
    Then 
    \begin{align*}
        T_iT_j & 
        =\sum_{g\in G}a_{ijg}\varphi_g
        =\sum_{k=1}^r\sum_{g\in C_k}a_{ijg}\varphi_g
        =\sum_{k=1}^ra_{ijk}\sum_{g\in C_k}\varphi_g
        =\sum_{k=1}^ra_{ijk}T_k.
    \end{align*}
    Therefore 
    \begin{equation}
        \label{eq:omega}
    \left(\frac{|C_i|}{\chi(1)}\chi(C_i)\right)
    \left(\frac{|C_j|}{\chi(1)}\chi(C_j)\right)
    =\sum_{k=1}^r a_{ijk}\left(\frac{|C_k|}{\chi(1)}\chi(C_k)\right).
    \end{equation}
    By the previous lemma, $x=\frac{|C_j|}{\chi(1)}\chi(C_j)\in\A$.
\end{proof}

\subsection{Frobenius' theorem}
\label{degree}

\begin{theorem}[Frobenius]
\index{Frobenius' theorem}
\label{thm:Frobenius_chi(1)}
    Let $G$ be a finite group and $\chi\in\Irr(G)$. 
    Then $\chi(1)$ divides~$|G|$. 
\end{theorem}

\begin{proof}
    Let $\varphi$ be an irreducible representation with character $\chi$. 
    Since $\langle\chi,\chi\rangle=1$, 
    \[
    \frac{|G|}{\chi(1)}=\frac{|G|}{\chi(1)}\langle\chi,\chi\rangle
    =\sum_{g\in G}\frac{\chi(g)}{\chi(1)}\overline{\chi(g)}.
    \]
    Note that this is a rational number. 
    Let $C_1,\dots,C_r$ be the conjugacy classes of $G$. 
    Then 
    \[
        \frac{|G|}{\chi(1)}
        =\sum_{i=1}^r\sum_{g\in C_i}\frac{\chi(g)}{\chi(1)}\overline{\chi(g)}
        =\sum_{i=1}^r\left(\frac{|C_i|}{\chi(1)}\chi(C_i)\right)\overline{\chi(C_i)}\in\A\cap\Q=\Z,
    \]
    as $\overline{\chi(C_i)}\in\A$. This implies that $\chi(1)$ divides $|G|$. 
\end{proof}

The character table gives information on the structure of the group. For example,
with the previous result, one can easily prove that
groups of order $p^2$ (where $p$ is a prime number) are abelian. 

\begin{exercise}
    Let $p$ and $q$ be prime numbers such that $p<q$.
    If $q\not\equiv1\bmod p$, then a group of order $pq$ is abelian. 
\end{exercise}

Another application:

\begin{theorem}
\label{thm:simple}
    Let $G$ be a finite simple group. 
    Then $\chi(1)\ne2$ for all $\chi\in\Irr(G)$. 
\end{theorem}

\begin{proof}
    Let $\chi\in\Irr(G)$ be such that $\chi(1)=2$. Let $\rho\colon G\to\GL_2(\C)$
    be an irreducible representation of $G$ with character $\chi$. Since 
    $G$ is simple, $\ker\rho=\{1\}$. Since $\chi(1)=2$, 
    $G$ is non-abelian and hence $[G,G]=G$. Since 
    $G$ has $(G:[G,G])=1$ degree-one characters, it follows that
    $G$ has only one degree-one character, the trivial one. The composition
    \[
    \begin{tikzcd}
    	G & {\GL_2(\C)} & {\C^{\times}}
    	\arrow["{\rho }", hook, from=1-1, to=1-2]
    	\arrow["{\det }", from=1-2, to=1-3]
    \end{tikzcd}
    \]
    is a degree-one representation, which means that $\det\rho_g=1$ for all $g\in G$. 
    By Frobenius' theorem, $|G|$ is even (because 
    $2=\chi(1)$ divides $|G|$). Let $x\in G$ be such that $|x|=2$ (Cauchy's theorem). 
    Then $|\rho_x|=2$, as $\rho$ is injective. Since $\rho_x$ is diagonalizable, 
    there exists $C\in\GL_2(\C)$ such that
    \[
    C\rho_xC^{-1}=\begin{pmatrix}
    \lambda&0\\
    0&\mu
    \end{pmatrix}
    \]
    for some $\lambda,\mu\in\{-1,1\}$. Since $1=\det\rho_x=\lambda\mu$ and
    $\rho_x$ is not the identity matrix, $\lambda=\mu=-1$. In particular, $C\rho_xC^{-1}$ is central
    and hence $\rho_x$ is central. Since $\rho$ is injective, $x$ is central 
    and thus $Z(G)\ne\{1\}$, a contradiction. 
\end{proof}


\begin{theorem}[Schur]
\index{Schur's theorem}
\label{thm:Schur_chi(1)}
    Let $G$ be a finite group and $\chi\in\Irr(G)$. 
    Then $\chi(1)$ divides $(G:Z(G))$. 
\end{theorem}

Let $G$ and $G_1$ be groups. If $V$ is a $\C[G]$-module and 
$V_1$ is a $\C[G_1]$-module, then 
$V\otimes V_1$ is a $\C[G\times G_1]$-module 
with 
\[
(g,g_1)\cdot v\otimes v_1=(g\cdot v)\otimes (g_1\cdot v_1)
\]
for $(g,g_1)\in G\times G_1$, $v\in V$ and $v_1\in V_1$. 

\begin{lemma}
    Let $G$ and $G_1$ be finite groups. If $\rho$ is an irreducible
    representation of $G$ and $\rho_1$ is an irreducible representation
    of $G_1$, then 
    $\rho\otimes\rho_1$ is an irreducible representation of $G\times G_1$. 
\end{lemma}

\begin{proof}
    Write $\chi=\chi_{\rho}$ and $\chi_1=\chi_{\rho_1}$. Since
    $\chi$ is irreducible, $\langle\chi,\chi\rangle=1$. Similarly, 
    $\langle\chi_1,\chi_1\rangle=1$. Now
    $\rho\otimes\rho_1$ is irreducible, as 
    \begin{align*}
    \langle\chi\chi_1,\chi\chi_1\rangle
    &=\frac{1}{|G\times G_1|}\sum_{(g,g_1)\in G\times G_1}(\chi\chi_1)(g,g_1)\overline{(\chi\chi_1)(g,g_1)}\\
    &=\frac{1}{|G||G_1|}\sum_{g\in G}\sum_{g_1\in G}\chi(g)\chi_1(g_1)\overline{\chi(g)}\overline{\chi_1(g_1)}\\
    &=\frac{1}{|G||G_1|}\sum_{g\in G}\chi(g)\overline{\chi(g)}\sum_{g_1\in G}\chi_1(g_1)\overline{\chi_1(g_1)}\\
    &=\langle\chi,\chi\rangle\langle\chi_1,\chi_1\rangle=1.\qedhere 
    \end{align*}
\end{proof}

\begin{exercise}
    Let $G$ and $G_1$ be finite groups. 
    Prove that irreducible characters of $G\times G_1$ 
    are of the form $\chi\chi_1$ for  
    $\chi\in\Irr(G)$ and $\chi_1\in\Irr(G_1)$. 
\end{exercise}

\index{Tensor power trick}
We now prove Schur's theorem. The proof goes back to Tate; it uses the 
\emph{tensor power trick}. See
Tao's blog  
\url{https://terrytao.wordpress.com} for other applications of this powerful
trick. 

\begin{proof}[Proof of Theorem \ref{thm:Schur_chi(1)}]
    Let $\rho\colon G\to\GL(V)$ be an irreducible representation 
    with character $\chi$. Let $z\in Z(G)$. Then $\rho_z$ commutes
    with $\rho_g$ for all $g\in G$. By Schur's lemma, 
    $\rho_z(v)=\lambda(z)v$ for all $v\in V$. Note that
    $\lambda\colon Z(G)\to\C^{\times}$, $z\mapsto\lambda(z)$, 
    is a well-defined group homomorphism, as 
    \[
    \lambda(z_1z_2)v=\rho_{z_1z_2}(v)=\rho_{z_1}\rho_{z_2}(v)
    =\lambda(z_2)\rho_{z_1}(v)=\lambda(z_1)\lambda(z_2)v
    \]
    for all $v\in V$ and $z_1,z_2\in Z(G)$. 
    
    Let $n\in\Z_{\geq1}$. Write $G^n=G\times\cdots\times G$ ($n$-times). Let
    \[
    \sigma\colon G^n\to\GL(V^{\otimes n}),\quad
    (g_1,\dots,g_n)\mapsto \rho_{g_1}\otimes\cdots\otimes\rho_{g_n}.
    \]
    Then $\sigma$ is a representation. 
    The character of $\sigma$ is $\chi^n$. By the previous lemma, 
    $\sigma$ is
    irreducible. For $z_1,\dots,z_n\in Z(G)$, we compute
    \begin{align*}   
    \sigma(z_1,\dots,z_n)(v_1\otimes\cdots\otimes v_n)&=\rho_{z_1}v_1\otimes\cdots\otimes \rho_{z_n}v_n\\
    &=\lambda(z_1)\cdots\lambda(z_n)v_1\otimes\cdots\otimes v_n\\
    &=\lambda(z_1\cdots z_n)v_1\otimes\cdots\otimes v_n.
    \end{align*}
    Let 
    \[
    H=\{(z_1,\dots,z_n)\in Z(G)^n:z_1\cdots z_n=1\}.
    \]  
    Then $H$ is a central subgroup of $G^n$. Moreover, 
    $H$ acts trivially on $V^{\otimes n}$, so there exists
    a group homomorphism $\sigma$ that makes the diagram 
    \[\begin{tikzcd}
	{G^n} && {\GL(V^{\otimes n})} \\
	{G^n/H}
	\arrow["\sigma", from=1-1, to=1-3]
	\arrow[from=1-1, to=2-1]
	\arrow["\tau"', dashed, from=2-1, to=1-3]
    \end{tikzcd}\]
    commutative. Thus  
    \[
    \tau\colon G^n/H\to\GL(V^{\otimes n}),
    \]
    is a representation 
    of degree $\chi(1)^n$:
    Since $\sigma$ is irreducible, so is $\tau$. 
    By Frobenius' theorem, $\chi(1)$ divides $|G|$ 
    and $\chi(1)^n$ divides $|G^n/H|=\frac{|G|^n}{|Z(G)|^{n-1}}$. 
    Write 
    \[
    |G|=\chi(1)s\quad\text{ and }\quad 
    |G|(G:Z(G))^{n-1}=\chi(1)^nr
    \]
    for some $r,s\in\Z$. Let $a$ and $b$ be such that 
    $\gcd(a,b)=1$ and 
    $\frac{a}{b}=\frac{(G:Z(G))}{\chi(1)}$. Then
    \[
    s\left(\frac{a}{b}\right)^{n-1}=s\frac{(G:Z(G))^{n-1}}{\chi(1)^{n-1}}
    =\frac{|G|}{\chi(1)}\frac{(G:Z(G))^{n-1}}{\chi(1)^{n-1}}=r\in\Z.
    \]
    Thus $b^{n-1}$ divides $s$ and hence $b=1$ (because $n$ is arbitrary).  
\end{proof}

\begin{theorem}[It\^o]
\index{It\^o's theorem}
\label{thm:Ito}
Let $G$ be a finite group and $\chi\in\Irr(G)$. Then 
$\chi(1)$ divides $(G:A)$ for all normal abelian subgroup $A$ of $G$.  
\end{theorem}

The proof of Theorem~\ref{thm:Ito} is no more difficult than that of Schur's Theorem~\ref{thm:Schur_chi(1)}. For a proof, see \cite[\S8.1]{MR0450380}.


\subsection{Examples of character tables}

Let $G$ be a finite group and $\chi_1,\dots,\chi_r$ be the irreducible characters of $G$. Without loss of generality
we may assume that $\chi_1$ is the trivial character, i.e. $\chi_1(g)=1$ for all $g\in G$. 
Recall that $r$ is the number of conjugacy classes of $G$. Each $\chi_j$ is constant on conjugacy classes. 
The \emph{character table} of 
$G$ is given by 
\begin{center}
\begin{tabular}{|c|cccc|}
\hline 
 & $1$ & $k_{2}$ & $\cdots$ & $k_{r}$\tabularnewline
 & $1$ & $g_{2}$ & $\cdots$ & $g_{r}$\tabularnewline
\hline 
$\chi_{1}$ & $1$ & $1$ & $\cdots$ & $1$\tabularnewline
$\chi_{2}$ & $n_{2}$ & $\chi_{2}(g_{2})$ & $\cdots$ & $\chi_{2}(g_{r})$\tabularnewline
$\vdots$ & $\vdots$ & $\vdots$ & $\ddots$ & $\vdots$\tabularnewline
$\chi_{r}$ & $n_{r}$ & $\chi_{r}(g_{2})$ & $\cdots$ & $\chi_{r}(g_{r})$\tabularnewline
\hline
\end{tabular}
\end{center}
where the $n_j$ are the degrees of the irreducible representations of $G$ and each $k_j$ is 
the size of the conjugacy class of the element $g_j$. By convention, the character table
contains only the values of the irreducible characters of the group. 





\begin{example}
	Let $G=\langle g:g^4=1\rangle$ 
	be the cyclic group of order four. The character table of $G$ is given by
	\begin{center}
		\begin{tabular}{|c|cccc|}
			\hline 
			& 1 & 1 & 1 & 1\tabularnewline
			& $1$ & $g$ & $g^2$ & $g^{3}$\tabularnewline
			\hline 
			$\chi_{1}$ & $1$ & $1$ & $1$ & $1$\tabularnewline
			$\chi_{2}$ & $1$ & $1$ & $-1$ & $-1$ \tabularnewline
			$\chi_{3}$ & $1$ & $-1$ & $i$ & $-i$\tabularnewline
			$\chi_{4}$ & $1$ & $-1$ & $-i$ & $i$\tabularnewline
			\hline
		\end{tabular}
	\end{center}
 Let us see how to see this calculation on the computer:
% \begin{lstlisting}
% gap> C4 := CyclicGroup(4);;                       
% gap> T := CharacterTable(C4);;
% gap> Display(T);
% CT1

%      2  2  2  2  2

%       1a 4a 2a 4b

% X.1     1  1  1  1
% X.2     1 -1  1 -1
% X.3     1  A -1 -A
% X.4     1 -A -1  A

% A = E(4)
%   = Sqrt(-1) = i
% \end{lstlisting}
\begin{lstlisting}
> C4 := CyclicGroup(4);
> T := CharacterTable(C4);
> T;


Character Table of Group C4
---------------------------


--------------------
Class |   1  2  3  4
Size  |   1  1  1  1
Order |   1  2  4  4
--------------------
p  =  2   1  1  2  2
--------------------
X.1   +   1  1  1  1
X.2   +   1  1 -1 -1
X.3   0   1 -1  I -I
X.4   0   1 -1 -I  I


Explanation of Character Value Symbols
--------------------------------------

I = RootOfUnity(4)    
\end{lstlisting}
%\begin{lstlisting}
%julia> G = cyclic_group(4);
%
%julia> T = character_table(G)
%<pc group of size 4 with 2 generators>
%
%  2  2    2  2    2
%                   
%    1a   4a 2a   4b
%                   
%X_1  1    1  1    1
%X_2  1  z_4 -1 -z_4
%X_3  1   -1  1   -1
%X_4  1 -z_4 -1  z_4    
%\end{lstlisting}
%\begin{lstlisting}
%julia> G = cyclic_group(4);
%
%julia> Oscar.with_unicode() do
%       show(character_table(G))
%       end;
%<pc group of size 4 with 2 generators>
%
% 2  2   2  2   2
%                
%   1a  4a 2a  4b
%                
%χ₁  1   1  1   1
%χ₂  1  ζ₄ -1 -ζ₄
%χ₃  1  -1  1  -1
%χ₄  1 -ζ₄ -1  ζ₄  
%\end{lstlisting}
Some remarks: 
 \begin{enumerate}
     \item The symbol \lstinline{I} denotes a primitive fourth root of 1.
     \item 	The function \lstinline{CharacterTable} computes more than just the character table of the group; it also provides additional information.
 \end{enumerate}
%\begin{lstlisting}
%julia> orders_class_representatives(T)
%4-element Vector{Int64}:
% 1
% 4
% 2
% 4
%
%julia> class_lengths(T)
%4-element Vector{fmpz}:
% 1
% 1
% 1
% 1
%
%julia> orders_centralizers(T)
%4-element Vector{fmpz}:
% 4
% 4
% 4
% 4
%\end{lstlisting}
\begin{lstlisting}
> T[1];
( 1, 1, 1, 1 )
> Degree(T[1]);
1
> Degree(T[2]);
1
> Degree(T[3]);
1
> Degree(T[4]);
1    
\end{lstlisting}
% \begin{lstlisting}
% gap> OrdersClassRepresentatives(T);
% [ 1, 4, 2, 4 ]
% gap> SizesCentralizers(T);
% [ 4, 4, 4, 4 ]
% gap> SizesConjugacyClasses(T);
% [ 1, 1, 1, 1 ]
% \end{lstlisting}
\end{example}


\begin{example}
    For $n\geq2$, let  
	$G=\langle g\rangle$ be the cyclic group of order $n$. Let $\lambda$ be a primitive $n$-th root of one. For each $i$, 
    let $V_i$ be a one-dimensional vector space with basis 
	$\{v\}$. Each $V_i$ is a $\C[G]$-module with 
    \[
		g\cdot v=\lambda^{i-1}v.
	\]
	Moreover, each $V_i$ is simple, as $\dim V_i=1$. The character $\chi_i$ associated with 
	$V_i$ is given by $\chi_i(g^m)=\lambda^{m(i-1)}$ for all 
	$m\in\{1,\dots,n\}$. Since the $\chi_1,\dots,\chi_n$ are all different and $G$ admits $n$ irreducible representations,
    it follows that $\Irr(G)=\{\chi_1,\dots,\chi_n\}$. The character
    table of $G$ is the following: 
	\begin{center}
		\begin{tabular}{|c|ccccc|}
			\hline 
			& 1 & 1 & 1 & $\cdots$ & 1\tabularnewline
			& $1$ & $g$ & $g^2$ & $\cdots$ & $g^{n-1}$\tabularnewline
			\hline 
			$\chi_{1}$ & $1$ & $1$ & $1$ & $\cdots$ & $1$\tabularnewline
			$\chi_{2}$ & $1$ & $\lambda$ & $\lambda^2$ & $\cdots$ & $\lambda^{n-1}$\tabularnewline
			$\chi_{3}$ & $1$ & $\lambda^2$ & $\lambda^4$ & $\cdots$ & $\lambda^{n-2}$\tabularnewline
			$\vdots$ & $\vdots$ & $\vdots$ & $\vdots$ & $\ddots$ & $\vdots$\tabularnewline
			$\chi_{n}$ & $1$ & $\lambda^{n-1}$ & $\lambda^{n-2}$ & $\cdots$ & $\lambda$\tabularnewline
			\hline
		\end{tabular}
	\end{center}
\end{example}

\begin{example}
	The character table of the group $C_2\times C_2=\{1,a,b,ab\}$ is 
	\begin{center}
		\begin{tabular}{|c|rrrr|}
			\hline 
			& 1 & 1 & 1 & 1\tabularnewline
			& $1$ & $a$ & $b$ & $ab$\tabularnewline
			\hline 
			$\chi_{1}$ & $1$ & $1$ & $1$ & $1$\tabularnewline
			$\chi_{2}$ & $1$ & $1$ & $-1$ & $-1$\tabularnewline
			$\chi_{3}$ & $1$ & $-1$ & $1$ & $-1$\tabularnewline
			$\chi_{4}$ & $1$ & $-1$ & $-1$ & $1$\tabularnewline
			\hline
		\end{tabular}
	\end{center}
	Let us do this by computer:
\begin{lstlisting}
> C2xC2 := AbelianGroup([2,2]);
> T := CharacterTable(C2xC2);
> T;


Character Table of Group C2xC2
------------------------------


--------------------
Class |   1  2  3  4
Size  |   1  1  1  1
Order |   1  2  2  2
--------------------
p  =  2   1  1  1  1
--------------------
X.1   +   1  1  1  1
X.2   +   1 -1  1 -1
X.3   +   1  1 -1 -1
X.4   +   1 -1 -1  1    
\end{lstlisting}
% \begin{lstlisting}
% gap> Display(CharacterTable(AbelianGroup([2,2])));
% CT2

%      2  2  2  2  2

%       1a 2a 2b 2c

% X.1     1  1  1  1
% X.2     1 -1  1 -1
% X.3     1  1 -1 -1
% X.4     1 -1 -1  1
% \end{lstlisting}
%\begin{lstlisting}
%julia> A = abelian_group(PcGroup, [2,2]);
%
%julia> character_table(A)
%<pc group of size 4 with 2 generators>
%
%  2  2  2  2  2
%               
%    1a 2a 2b 2c
%               
%X_1  1  1  1  1
%X_2  1 -1  1 -1
%X_3  1  1 -1 -1
%X_4  1 -1 -1  1
%\end{lstlisting}
\end{example}

% \begin{exercise}
%     Let $A$ and $B$ be abelian groups. 
%     We write $\Irr(A)=\{\rho_1,\dots,\rho_r\}$ and 
%     $\Irr(B)=\{\phi_1,\dots,\phi_s\}$. Prove
%     that the maps 
%     \[
%     \varphi_{ij}\colon A\times B\to\C^\times,\quad
%     (a,b)\mapsto\rho_i(a)\phi_j(b),
%     \]
%     where $i\in\{1,\dots,r\}$ and $j\in\{1,\dots,s\}$, are the irreducible representations of $A\times B$. 
% \end{exercise}

\begin{example}
	The character table of $\Sym_3$ is given by 
	\begin{center}
		\begin{tabular}{|c|rrr|}
			\hline
			& $1$ & $3$ & $2$\tabularnewline
			& $1$ & $(12)$ & $(123)$ \tabularnewline
			\hline 
			$\chi_{1}$ & $1$ & $1$ & $1$\tabularnewline
			$\chi_{2}$ & $1$ & $-1$ & $1$ \tabularnewline
			$\chi_{3}$ & $2$ & $0$ & $-1$ \tabularnewline
			\hline
		\end{tabular}
	\end{center}
	Let us recall one possible way to compute this table. 
	Degree-one characters were easy to compute. 
	To compute the third row of the table, one possible approach is to use
	the irreducible representation  
	\[
	(12)\mapsto \begin{pmatrix}-1&1\\0&1\end{pmatrix},
	\quad
	(123)\mapsto \begin{pmatrix}0&-1\\1&-1\end{pmatrix}.
	\]
    Then	
    \begin{align*}
		&\chi_3\left( (12) \right)=\trace\begin{pmatrix}-1&1\\0&1\end{pmatrix}=0,\\
		&\chi_3\left( (123) \right)=\chi_3\left( (12)(23)\right)=\trace\begin{pmatrix}0&-1\\1&-1\end{pmatrix}=-1.
	\end{align*}

	We should remark that the irreducible representation 
	mentioned is not needed to
	compute the third row of the character table. 
\begin{lstlisting}
> S3 := Sym(3);
> T := CharacterTable(S3);
> T;


Character Table of Group S3
---------------------------


-----------------
Class |   1  2  3
Size  |   1  3  2
Order |   1  2  3
-----------------
p  =  2   1  1  3
p  =  3   1  2  1
-----------------
X.1   +   1  1  1
X.2   +   1 -1  1
X.3   +   2  0 -1    
\end{lstlisting}
% \begin{lstlisting}
% gap> S3 := SymmetricGroup(3);;
% gap> T := CharacterTable(S3);;
% gap> Display(T);
% CT3

%      2  1  1  .
%      3  1  .  1

%       1a 2a 3a
%     2P 1a 1a 3a
%     3P 1a 2a 1a

% X.1     1 -1  1
% X.2     2  . -1
% X.3     1  1  1
% \end{lstlisting}
%\begin{lstlisting}
%julia> S3 = symmetric_group(3);
%
%julia> T = character_table(S3)
%Sym( [ 1 .. 3 ] )
%
%  2  1  1  .
%  3  1  .  1
%            
%    1a 2a 3a
% 2P 1a 1a 3a
% 3P 1a 2a 1a
%            
%X_1  1 -1  1
%X_2  2  . -1
%X_3  1  1  1
%\end{lstlisting}
%As we did before, some extra information was computed:
%\begin{lstlisting}
%julia> orders_class_representatives(T)
%3-element Vector{Int64}:
% 1
% 2
% 3
%
%julia> class_lengths(T)
%3-element Vector{fmpz}:
% 1
% 3
% 2
%
%julia> orders_centralizers(T)
%3-element Vector{fmpz}:
% 6
% 2
% 3
%\end{lstlisting}
%julia> GAP.Globals.SizesConjugacyClasses(T.GAPTable)
%GAP: [ 1, 3, 2 ]
%julia> GAP.Globals.SizesCentralizers(T.GAPTable)
%GAP: [ 6, 2, 3 ]    
% \begin{lstlisting}
% gap> SizesConjugacyClasses(T);
% [ 1, 3, 2 ]
% gap> SizesCentralizers(T);
% [ 6, 2, 3 ]
% gap> OrdersClassRepresentatives(T);
% [ 1, 2, 3 ]
% \end{lstlisting}
\end{example}

% \begin{exercise}
% \label{xca:S4}
%     Compute the character table of $\Sym_4$. 
% \end{exercise}

\begin{example} 
Let us compute the character table of $\Sym_4$. 
We know that $|\Sym_4|=24$ and that 
$\Sym_4$ has five conjugacy classes:
	\begin{center}
		\begin{tabular}{c|ccccc}
			Representative & $\id$ & $(12)$ & $(12)(34)$ & $(123)$ & $(1234)$\tabularnewline
			\hline
			Size & $1$ & $6$ & $3$ & $8$ & $6$
		\end{tabular}
	\end{center}
Thus $\Irr(\Sym_4)=\{\chi_1,\chi_2,\dots,\chi_5\}$. We may 
assume that $\chi_1$ is the trivial character and
that $\chi_2$ is the sign. Since 
$[\Sym_4,\Sym_4]\simeq\Alt_4$, the quotient 
$\Sym_4/[\Sym_4,\Sym_4]$ has order two and hence $\Sym_4$ 
admits exactly two degree-one irreducible representations. Hence
we know two rows of the character table of $\Sym_4$: 
	\begin{center}
		\begin{tabular}{|c|rrrrr|}
			\hline
			& $\id$ & $(12)$ & $(12)(34)$ & $(123)$ & $(1234)$\tabularnewline
%			& $1$ & $6$ & $3$ & $8$ & $6$\tabularnewline
			\hline
			$\chi_1$ & $1$ & $1$ & $1$ & $1$ & $1$\tabularnewline
			$\chi_2$ & $1$ & $-1$ & $1$ & $1$ & $-1$\tabularnewline
			\hline
		\end{tabular}
	\end{center}
    
	There exist $n_3,n_4,n_5\in\{2,3,4\}$ such that 
	$24=1+1+n_3^2+n_4^2+n_5^2$. A direct calculation shows that $(n_3,n_4,n_5)=(2,3,3)$ is the only solution with
    $n_3\leq n_4\leq n_5$.

To find the other characters, it is useful to use the action of $\Sym_4$ on the vector space 
	\[
		V=\{(x_1,x_2,x_3,x_4)\in\R^4:x_1+x_2+x_3+x_4=0\},
	\]
given by 
\[
g\cdot (x_1,x_2,x_3,x_4)=(x_{g^{-1}(1)},x_{g^{-1}(2)},x_{g^{-1}(3)},x_{g^{-1}(4)}).
\]
Let \[
		v_1=(1,0,0,-1),
		\quad
		v_2=(0,1,0,-1),
		\quad
		v_3=(0,0,1,-1).
	\]
	Then $\{v_1,v_2,v_3\}$ is a basis of $V$ and 
	\begin{align*}
		&(12)\cdot v_1=v_2,&&
		(12)\cdot v_2=v_1,&&
		(12)\cdot v_3=v_3,\\
		&(1432)\cdot v_1=-v_3,&&
		(1432)\cdot v_2=v_1-v_3,&&
		(1432)\cdot v_3=v_2-v_3.
	\end{align*}
	Since $\Sym_4=\langle (12),(1432)\rangle$, this 
    is enough to know how any element 
    $g\in\Sym_4$ acts on any $v\in V$. 
    This action yields a representation 
    $\rho\colon\Sym_4\to\GL(V)$: 
    \[
		\rho_{(12)}=\begin{pmatrix}
			0 & 1 & 0\\
			1 & 0 & 0\\
			0 & 0 & 1
		\end{pmatrix},\quad
		\rho_{(1432)}=\begin{pmatrix}
			0 & 1 & 0\\
			0 & 0 & 1\\
			-1 & -1 & -1
		\end{pmatrix}.
	\]
    Let $\chi$ be the character of $\rho$.  
	Then 
	$\chi(\id)=3$, $\chi\left( (12) \right)=1$, $\chi\left( (1234) \right)=-1$. How to compute the value of $\chi$ on 3-cycles? Here is the trick: 
    \begin{align*}
		&\chi\left( (234) \right)=\chi\left( (12)(1234) \right)=\trace(\rho_{(12)}\rho_{(1234)})=\trace\begin{pmatrix}
			0 & 0 & 1\\
			0 & 1 & 0\\
			1 & -1 & -1
		\end{pmatrix}
		=0.
	\end{align*}
	Similarly, to compute $\chi$ on products of two transpositions, 
    we note that 
    \begin{align*}
		&\chi\left( (13)(24) \right)=\chi\left( (1234)(1234) \right)=\trace(\rho_{(1234)}^2)=\trace\begin{pmatrix}
			0 & 0 & 1\\
			1 & -1 & -1\\
			-1 & 2 & 0
		\end{pmatrix}
		=-1.
	\end{align*}
     Now is an easy exercise to check that this $\chi$ is irreducible:
      \[
	 	\langle \chi,\chi\rangle=\frac{1}{24}(3^2+6+0+6+3)=1.
	 \]
     Moreover, $\sgn\otimes\chi$ is also an irreducible representation:
     	\[
		\langle \sgn\otimes\chi,\sgn\otimes\chi\rangle=\frac{1}{24}(3^2+(-1)^26+(-1)^23+6)=1.
	\]
     With the trivial representation $\chi_1$, the sign representation $\chi_2$ and these two new characters, namely $\chi_3=\chi$ and $\chi_4=\sgn\otimes\chi$, we are almost done. Only one irreducible character is missing. Let us call this
     character $\chi_5$. This character can be determined  
     using the left regular representation $L$: 
     \begin{align*}
		0 &= \chi_L\left( (12) \right)=1+(-1)+3+3(-1)+2\chi_5\left( (12) \right),\\
		0 &= \chi_L\left( (12)(34) \right)=1+1+3(-1)+3(-1)+2\chi_5\left( (12)(34) \right),\\
		0 &= \chi_L\left( (123) \right)=1+1+0+0+2\chi_5\left( (123) \right),\\
		0 &= \chi_L\left( (1234) \right)=1+(-1)+3(-1)+3+2\chi_5\left( (1234) \right)=0,
	\end{align*}
	
    Now we are ready to compute the character table of $\Sym_4$: 
	% Tenemos así cuatro de los cinco caracteres irreducibles de $G$. Nos falta
	% uno, digamos $\chi_5$.  Para calcular $\chi_5$ usamos el carácter de la
	% representación regular $L$:
	% \begin{align*}
	% 	0 &= \chi_L\left( (12) \right)=1+(-1)+3+3(-1)+2\chi_5\left( (12) \right),\\
	% 	0 &= \chi_L\left( (12)(34) \right)=1+1+3(-1)+3(-1)+2\chi_5\left( (12)(34) \right),\\
	% 	0 &= \chi_L\left( (123) \right)=1+1+0+0+2\chi_5\left( (123) \right),\\
	% 	0 &= \chi_L\left( (1234) \right)=1+(-1)+3(-1)+3+2\chi_5\left( (1234) \right)=0,
	% \end{align*}
	% de donde obtenemos los valores de $\chi_5$. Nos queda así la siguiente tabla:
	\begin{center}
		\begin{tabular}{|c|rrrrr|}
			\hline
			& $\id$ & $(12)$ & $(12)(34)$ & $(123)$ & $(1234)$\tabularnewline
			\hline
			$\chi_1$ & $1$ & $1$ & $1$ & $1$ & $1$\tabularnewline
			$\sgn$ & $1$ & $-1$ & $1$ & $1$ & $-1$\tabularnewline
			$\chi$ & $3$ & $1$ & $-1$ & $0$ & $-1$\tabularnewline
			$\sgn\otimes\chi$ & $3$ & $-1$ & $-1$ & $0$ & $1$\tabularnewline
			$\chi_5$ & $2$ & $0$ & $2$ & $-1$ & $0$\tabularnewline
			\hline
		\end{tabular}
	\end{center}
\end{example}


\begin{exercise}
    Compute the character table of $\Alt_4$. 
\end{exercise}

\begin{exercise}
\label{xca:order5}
    Compute the character table of a non-abelian group of order eight.
\end{exercise}

There are two non-isomorphic non-abelian groups of order eight: the dihedral group $\D_4$ and the quaternion group $Q_8$. One does not need
to use this information to solve Exercise~\ref{xca:order5}.

% \begin{example}
% Let us compute the character table of a 
% non-abelian group of order eight. (There are two non-isomorphic non-abelian groups of order eight: the dihedral group $\D_4$ and the quaternion group $Q_8$. We will not use this information.) 

% Let $G$ be a non-abelian group of order eight. Since $G$
% is a $2$-group, $Z(G)$ is non-trivial. Moreover, 
% since $G$ is non-abelian, $G/Z(G)$ is non-cyclic. Thus
% $|Z(G)|=2$. Since $G/Z(G)$ has four elements, it is
% abelian. Thus $[G,G]\subseteq Z(G)$ and hence
% $[G,G]=Z(G)$. This means that 
% $|G/[G,G]|=4$, so there are 
% exactly four degree-one characters. Since 
% \[
% 8=1+1+1+1+n_5^2+\cdots+n_r^2,
% \]
% we conclude that 
% $r=5$ and $n_5=2$. We now know that 
% $G$ has give conjugacy classes, say 
% with representatives 
% $1,x,a,b,c$, where $[G,G]=Z(G)=\langle x\rangle$.  
% The class equation implies that 
% the conjugacy classes of $a$, $b$ and $c$ have
% two elements. 

% Since $G/[G,G]\simeq C_2\times C_2$, . Por la
% 	proposición~\ref{proposition:Lin(G)}, toda representación de grado uno de
% 	$G$ es de la forma $\chi_j\circ\pi$, donde $\chi_j$ es una representación
% 	de grado uno de $C_2\times C_2$ y $\pi\colon G\to G/[G,G]$ es el morfismo
% 	canónico. Esto nos permite calcular gran parte de los valores de los
% 	caracteres de grado uno:
% 	\begin{center}
% 		\begin{tabular}{|c|rrrrr|}
% 			\hline
% 			& $1$ & $x$ & $a$ & $b$ & $c$\tabularnewline
% 			\hline
% 			$\chi_1$ & $1$ & $1$ & $1$ & $1$ & $1$\tabularnewline
% 			$\chi_2$ & $1$ & $?$ & $-1$ & $1$ & $-1$\tabularnewline
% 			$\chi_3$ & $1$ & $?$ & $1$ & $-1$ & $-1$\tabularnewline
% 			$\chi_4$ & $1$ & $?$ & $-1$ & $-1$ & $1$\tabularnewline
% 			\hline
% 		\end{tabular}
% 	\end{center}
% 	Como $0=\langle \chi_1,\chi_2\rangle=\frac18(1+x+2+2(-1)+2(-1))$, se
% 	concluye que $\chi_2(x)=1$. De la misma forma probamos que $\chi_j(x)=1$
% 	para todo $j\in\{3,4\}$. 
	
% 	Nos falta calcular el valor del caracter de grado dos. Para eso usamos la
% 	representación regular $L$. Al resolver el sistema 
% 	\begin{align*}
% 		0&=\chi_L(x)=1+1+1+1+2\chi_5(x),\\
% 		0&=\chi_L(a)=1+1+-1-1+2\chi_5(a),\\
% 		0&=\chi_L(b)=1-1+1-1+2\chi_5(b),\\
% 		0&=\chi_L(c)=1-1-1+1+2\chi_5(c),
% 	\end{align*}
% 	obtenemos $\chi_5(x)=-2$ y $\chi_5(a)=\chi_5(b)=\chi_5(c)=0$. Luego la
% 	tabla de caracteres de $G$ es 
% 	\begin{center}
% 		\begin{tabular}{|c|rrrrr|}
% 			\hline
% 			& $1$ & $x$ & $a$ & $b$ & $c$\tabularnewline
% 			\hline
% 			$\chi_1$ & $1$ & $1$ & $1$ & $1$ & $1$\tabularnewline
% 			$\chi_2$ & $1$ & $1$ & $-1$ & $1$ & $-1$\tabularnewline
% 			$\chi_3$ & $1$ & $1$ & $1$ & $-1$ & $-1$\tabularnewline
% 			$\chi_4$ & $1$ & $1$ & $-1$ & $-1$ & $1$\tabularnewline
% 			$\chi_5$ & $2$ & $-2$ & $0$ & $0$ & $0$\tabularnewline
% 			\hline
% 		\end{tabular}
% 	\end{center}
%\end{example}

