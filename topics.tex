\chapter*{Some topics for final projects}

\pagestyle{plain}
\fancyhf{}
\fancyhead[LE,RO]{Representation theory of groups}
\fancyhead[RE,LO]{Some topics}
\fancyfoot[CE,CO]{\leftmark}
\fancyfoot[LE,RO]{\thepage}

\addcontentsline{toc}{chapter}{Some topics for final projects}

We collect here some topics for final presentations. Some topics
can also be used as bachelor or master theses. 

\subsection*{Staircase groups}

This topic describes a situation similar to that of \S\ref{Kolchin}, but
more general. See \cite[Chapter 5]{MR1369573}.

\subsection*{Solvable and nilpotent groups}

The character table of a finite group
detects solvability and nilpotency of groups, see
\cite[Chapter 6]{MR1369573}.

\subsection*{Kegel--Wielandt theorem}

Prove Kegel--Wielandt theorem \ref{thm:KegelWielandt}. 
For the proof see \cite[Theorem 2.13]{MR1211633}. 

\subsection*{The Drinfeld double of a finite group}

See \cite[Chapter IX]{MR1321145} and 
\cite[Chapter 8]{MR3752618}.

\subsection*{Ito's theorem}

Ito's theorem generalize Frobenius' theorem
(Theorem \ref{thm:Frobenius_chi(1)})  
and Schur's theorem (Theorem \ref{thm:Schur_chi(1)}). 
The theorem states that if $\chi$ is an irreducible character
of a finite group $G$, then $\chi(1)$ divides 
$(G:A)$ for every normal abelian subgroup $A$ of $G$. 
See \cite[\S8.1]{MR0450380}. 

\subsection*{Characters of $\GL_2(q)$ and $\SL_2(q)$}

One possible topic is the character table of $\GL_2(q)$, see
\cite[\S5.2]{MR2867444}. Alternatively, one can 
present the character table of the group $\SL_2(p)$  
following Humphreys's paper \cite{MR364478}. 
The character theory of $\SL_2(q)$ appears in 
\cite[\S5.2]{MR2867444}, see 
\cite[Chapter 20]{MR1650707} for details. 

\subsection*{Representations of the symmetric group}

See for example \cite[\S10]{MR2867444} and 
\cite{MR1153249}. 

\subsection*{Random walks on finite groups}

The goal is to construct the character table or 
the irreducible representations of the symmetric group. 
The topic has connections with combinatorics and applications 
to voting and card shuffling. 
See \cite[4]{MR1153249} and \cite[\S11]{MR2867444}.

\subsection*{Fourier analysis on finite groups}

See \cite[\S5]{MR2867444} for a very elementary approach and some
basic applications. Other applications 
appear in \cite{MR1695775}.

\subsection*{Restriction and induction of modules}

\begin{definition}
    \index{Restriction}
    Let $G$ be a finite group. 
    If $U$ is a $\C[G]$-module and $H$ is a subgroup of $G$, 
    then $U$ is a $\C[H]$-module by restricting to the action of $H$.  
    This module will be denoted by $\Res_H^GU$. It will be called
    the \textbf{restriction} of $U$ to $H$.
\end{definition}

The restriction of a simple module may not be a simple module. 

\begin{example}
    Let $G=\D_4=\langle r,s:r^4=s^2=1,\,srs=r^{-1}\rangle$ be the dihedral
    group of eight elements and 
    $V$ be a vector space with basis $\{v_1,v_2\}$. Then 
    $V$ is a $\C[\D_4]$-module with 
    \[
    r\cdot v_1=v_2,\quad
    r\cdot v_2=-v_1,\quad
    s\cdot v_1=v_1,\quad
    s\cdot v_2=-v_2.
    \]
    The character of $V$ is 
    \[
    \chi(g)=\begin{cases}
    2 & \text{si $g=1$},\\
    -2 & \text{si $g=r^2$},\\
    0 & \text{en otro caso}.
    \end{cases}
    \]
    Note that $\chi$ is irreducible, as $\langle\chi,\rangle\chi=1$.
    Let 
    $H=\langle r^2,s\rangle=\{1,r^2,s,r^2s\}$. Then $\Res_H^GV$ is $V$ as
    a vector space with the structure of 
    $\C[H]$-module given by 
    \[
    r^2\cdot v_1=-v_1,\quad
    r^2\cdot v_2=-v_1,\quad
    s\cdot v_1=-v_1,\quad
    s\cdot v_2=-v_2.
    \]
    The character of $\Res_H^GV$ is
    \[
    \chi_H(h)=\chi|_H(h)
    =\begin{cases}
    2 & \text{if $h=1$},\\
    -2 & \text{if $h=r^2$},\\
    0 & \text{otherwise}.
    \end{cases}
    \]
    The character $\chi_H$ is not irreducible, as 
    $\langle\chi_H,\chi_H\rangle=0$. 
\end{example}

Let $G$ be a finite group and 
$H$ be a subgroup of $G$. Write 
$\Irr(H)=\{\phi_1,\dots,\phi_l\}$.
If $\chi\in\Char(G)$, then
\[
    \chi|_H=\sum_{i=1}^ld_i\phi_i
\]
for integers $d_1,\dots,d_l\geq 0$. 
Each $\phi_i$ with $d_i=\langle\chi|_H,\phi_i\rangle\ne 0$ 
is an \textbf{irreducible component} of $\chi|_H$ and these 
$\phi_i$'s are the \textbf{irreducible components} of 
$\chi|_H$. 

\begin{proposition}
    Let $G$ be a finite group. 
    If $H$ is a subgroup of $G$ and $\phi\in\Char(H)$, 
    then $\chi\in\Irr(G)$ 
    is such that $\langle\chi|_H,\phi\rangle_H\ne 0$.
\end{proposition}

\begin{proof}
    Assume that $\Irr(G)=\{\chi_1,\dots,\chi_k\}$. 
    If $L$ is the regular representation of $G$, then 
    \[
    \chi_L(g)=\begin{cases}
    |G| & \text{if $g=1$},\\
    0 & \text{otherwise}.
    \end{cases}
    \]
    Write $\chi_L=\sum_{i=1}^k\chi_i(1)\chi_i$. Since 
    \[
    0\ne \frac{|G|}{|H|}\phi(1)=\langle \chi_L|_H,\phi\rangle_H=\sum_{i=1}^k\chi_i(1)\langle\chi_i|_H,\phi\rangle_H,
    \]
    there exists $i\in\{1,\dots,k\}$ 
    such that $\langle\chi_i|_H,\phi\rangle_H\ne 0$. 
\end{proof}

\begin{proposition}
    Let $G$ be a group, $\chi\in\Irr(G)$ and $H$ be a subgroup of $G$. 
    If $\Irr(H)=\{\phi_1,\dots,\phi_l\}$, then 
    \[
    \chi|_H=\sum_{i=1}^ld_i\phi_i,
    \]
    where $\sum_{i=1}^l d_i^2\leq (G:H)$. Moreover, 
    \[
    \sum_{i=1}^l d_i^2=(G:H) 
    \Longleftrightarrow
    \chi(g)=0\text{ para todo $g\in G\setminus H$.}
    \]
\end{proposition}

\begin{proof}
    Since  
    \[
    \sum_{i=1}^ld_i^2=\langle\chi|_H,\chi|_H\rangle_H=\frac{1}{|H|}\sum_{h\in H}\chi(h)\overline{\chi(h)}.
    \]
    Since $\chi$ is irreducible, 
    \begin{align*}
        1=\langle\chi,\chi\rangle_G&=\frac{1}{|G|}\sum_{g\in G}\chi(g)\overline{\chi(g)}\\
        &=\frac{1}{|G|}\sum_{h\in H}\chi(h)\overline{\chi(h)}
        +\frac{1}{|G|}\sum_{g\in G\setminus H}\chi(g)\overline{\chi(g)}\\
        &=\frac{|H|}{|G|}\sum_{i=1}^l d_i^2+\frac{1}{|G|}\sum_{g\in G\setminus H}\chi(g)\overline{\chi(g)}.
    \end{align*}
    Since $\sum_{g\in G\setminus H}\chi(g)\overline{\chi(g)}\geq0$, it follows that
    $\sum_{i=1}^ld_i^2\leq(G:H)$. Moreover, 
    the equality holds if and only if $\sum_{g\in G\setminus H}\chi(g)\overline{\chi(g)}=0$, 
    that is, if and only if $\chi(g)=0$ for all $g\in G\setminus H$.
\end{proof}

\index{Bimódulo}
Discutiremos ahora la inducción de módulos. Para eso, repasaremos algunas nociones básicas sobre
\textbf{bimódulos} y \textbf{producto tensorial de bimódulos}. 
Si $R$ y $S$ son anillos, un grupo abeliano $M$ se dirá un $(R,S)$-bimódulo si 
$M$ es un $R$-módulo a izquierda, $M$ es un $S$-módulo a derecha y además
\[
r\cdot (m\cdot s)=(r\cdot m)\cdot s
\]
para todo $r\in R$, $s\in S$ y $m\in M$. 

\begin{examples}\
\begin{enumerate}
    \item Un $R$-módulo a izquierda es un $(R,\Z)$-bimódulo.
    \item Un $S$-módulo a derecha es un $(\Z,S)$-bimódulo.
    \item Todo anillo $R$ es un $(R,R)$-bimódulo.
%    \item Si $R$ es un anillo conmutativo...
\end{enumerate}
\end{examples}

\begin{example}
Si $M$ es un $(R,S)$-bimódulo y $N$ es un $R$-módulo, entonces el conjunto 
$\Hom_R(M,N)$ de morfismos de $R$-módulos $M\to N$ es un 
$S$-módulo con 
\[
(s\cdot \varphi)(m)=\varphi(m\cdot s),\quad s\in S,\,\varphi\in\Hom_R(M,N),\,m\in M.
\]
\end{example}

Sean $M$ un $(R,S)$-bimódulo, $N$ un $S$-módulo y $U$ un $R$-módulo. 
Diremos que una función $f\colon M\times N\to U$ 
es \textbf{balanceada} si 
\begin{align*}
    &f(m_1+m_2,n)=f(m_1,n)+f(m_2,n),\\
    &f(m,n_1+n_2)=f(m,n_1)+f(m,n_2),\\
    &f(m\cdot s,n)=f(m,s\cdot n),\\
    &f(r\cdot m,n)=r\cdot f(m,n)
\end{align*}
para todo $m,m_1,m_2\in M$, $n,n_1,n_2\in N$, $r\in R$ y $s\in S$. 

\begin{example}
Si $M$ es un $R$-módulo, la función $f\colon R\times M\to M$, $(r,m)\mapsto r\cdot m$, es balanceada. 
\end{example}

\index{Producto tensorial!de bimódulos}
Sean $M$ un $(R,S)$-bimódulo, $N$ un $S$-módulo y $U$ un $R$-módulo. 
Se define el \textbf{producto tensorial} $M\otimes_S N$ es un $R$-módulo provisto con una función balanceada 
$\eta\colon M\times N\to M\otimes_S N$ que cumple con la siguiente propiedad universal: 
\begin{quote}
Si $f\colon M\times N\to U$ es una función balanceada, entonces
existe un único morfismo de $R$-módulos $\alpha\colon M\otimes_S N\to U$ tal que $f=\alpha\circ\eta$. 
\end{quote}
Notación: $m\otimes n=\eta(m,n)$ para $m\in M$ y $n\in N$.
El producto tensorial existe y puede demostrarse que es único salvo isomorfismos. Más precisamente, $M\otimes_S N$
se define como el $R$-módulo generado por
el conjunto $\{m\otimes n:m\in M,\,n\in N\}$, donde los $m\otimes n$ satisfacen 
las siguientes identidades:
\begin{align}
    &(m+m_1)\otimes n=m\otimes n+m_1\otimes n &&\text{$m,m_1\in M$, $n\in N$},\\
    &m\otimes(n+n_1)=m\otimes n+m\otimes n_1 &&\text{$m\in M$, $n,n_1\in N$},\\
    &(ms)\otimes n=m\otimes (sn) &&\text{$m\in M$, $n\in N$, $s\in S$},\\
    &(rm)\otimes n=r(m\otimes n) &&\text{$m\in M$, $n\in N$, $r\in R$}.
\end{align}
Un elemento arbitrario de $M\otimes_S N$ es una suma finita
de la forma 
$\sum_{i=1}^k m_i\otimes n_i$,
donde $m_1,\dots,m_k\in M$ y $n_1,\dots,n_k\in N$, y no necesariamente un tensor elemental $m\otimes n$. 

\begin{example}
$M\simeq R\otimes_R M$ como $R$-módulos. Como la función $R\times M\to M$, $(r,m)\mapsto r\cdot m$, es balanceada, 
induce un morfismo $R\otimes_R M\to M$, $r\otimes m\mapsto r\cdot m$ con inversa $M\to R\otimes_R M$, $m\mapsto 1\otimes m$. 
\end{example}

\begin{example}
Si $M_1,\dots,M_k$ son $(R,S)$-bimódulos y $N$ es un $S$-módulo, entonces
\[
(M_1\oplus\cdots\oplus M_k)\otimes_S N\simeq (M_1\otimes_S N)\oplus\cdots\oplus (M_k\otimes_S N).
\]
\end{example}

Algunos ejercicios:

\begin{exercise}
    Demuestre que $M\otimes_RN\simeq N\otimes_{R^{\op}}M$.
\end{exercise}

\begin{exercise}
    Demuestre que $\Z/n\otimes_{\Z}\Q=\{0\}$.
\end{exercise}

\begin{exercise}
    Sean $M$ un $(R,S)$-bimódulo y $N$ un $(S,T)$-bimódulo. 
    Demuestre que $M\otimes_SN$ es un $(R,T)$-bimódulo 
    con $r(m\otimes n)t=(rm)\otimes (nt)$, 
    donde $m\in M$, $n\in N$, $r\in R$, $t\in T$.
\end{exercise}

\begin{exercise}
    Demuestre que $(M\otimes_R N)\otimes_RT\simeq M\otimes_R (N\otimes_RT)$.
\end{exercise}

\begin{exercise}
    Enuncie y demuestre la asociatividad del producto tensorial de bimódulos. 
\end{exercise}

% Atiyah-Mac Donald
% https://math.stackexchange.com/questions/2586211/associativity-of-tensor-products

Si $G$ es un grupo finito, $H$ es un subgrupo de $G$
y $V$ es un $K[H]$-módulo, entonces 
$K[G]$ es un $(K[G],K[H])$-bimódulo.

\begin{definition}
\index{Módulo!inducido}
Sea $G$ un grupo finito y sea 
$H$ un subgrupo de $G$. 
Si $V$ es un $K[H]$-módulo de $G$, 
se define el $K[G]$-módulo \textbf{inducido} de $V$ 
como
\[
\Ind_H^GV=K[G]\otimes_{K[H]}V.
\]
\end{definition}

\index{Transversal}
Si $H$ es un subgrupo de $G$, un \textbf{transversal} (a izquierda) 
de $H$ en $G$ es un subconjunto $T$ de $G$ que contiene exactamente un elemento de cada coclase (a izquierda) 
de $H$ en $G$. 

\begin{example}
Si $G=\Sym_3$ y $H=\{\id,(12)\}$, entonces
$T=\{\id,(123),(23)\}$ es un transversal de $H$ en $G$. Podemos descomponer 
a $G$ como
\[
G=\{\id,(12)\}\cup \{(123),(13)\}\cup\{(132),(23)\}=\bigcup_{t\in T}tH.
\]
Como cada $g\in G$ se escribe en forma única como $g=th$ para $t\in T$ y $h\in H$, podemos 
definir una transformación lineal 
$\varphi\colon K[G]\to K[H]\oplus K[H]\oplus K[H]=|T|K[H]$, que para $g=th$ devuelve $h$ en el lugar que corresponde a $t\in T$, es decir
\begin{align*}
\id&\mapsto (\id,0,0), && (12)\mapsto ((12),0,0), && (123)\mapsto (0,\id,0),\\
(23)&\mapsto (0,0,\id), && (13)\mapsto (0,(12),0), && (132)\mapsto (0,0,(12)).
\end{align*}
Por ejemplo, 
\[
\varphi( 5(12)-3(123)+7\id )=(7\id+5(12),-3\id,0).
\]
Es importante observar que $\varphi$ es un isomorfismo de $K[H]$-módulos (a derecha). 
\end{example}

La observación hecha en el ejemplo anterior es la clave del siguiente resultado.

\begin{proposition}
Sea $G$ un grupo finito y sea 
$H$ un subgrupo de $G$. Si $V$ es un $K[H]$-módulo de $G$, entonces 
\[
    \Ind_H^G(V)=\bigoplus_{t\in T}t\otimes V,
\]
donde $T$ es un transversal de $H$ en $G$ y $t\otimes V=\{t\otimes v:v\in V\}$. En particular, 
$\dim\Ind_H^GV=(G:H)\dim V$.
\end{proposition}

\begin{proof}
Descomponemos a $G$ como unión disjunta de coclases de $H$ con el transversal $T$, es decir
\[
G=\bigcup_{t\in T}tH.
\]
Cada $g\in G$ se escribe entonces unívocamente como $g=th$ con $t\in T$ y $h\in H$. Tal como 
hicimos en el ejemplo anterior, esto nos permite obtener un isomorfismo 
$\varphi\colon K[G]\to |T|K[H]$ de $K[H]$-módulos (a derecha), donde $\varphi(g)$ es $h$ en el sumando que corresponde a $t\in T$
y es cero en el resto de los sumandos. Luego
\[
\Ind_H^GV=K[G]\otimes_{K[H]}V\simeq (|T|K[H])\otimes_{K[H]}V\simeq |T|(K[H]\otimes_{K[H]}V)\simeq |T|V
\]
como $K[H]$-módulos. En particular, $\dim\Ind_H^GV=|T|\dim V$. 

Si escribimos $g=th$ con $t\in T$ y $h\in H$, entonces $g\otimes v=(th)\otimes v=t\otimes h\cdot v\in t\otimes V$. 
Luego $K[G]\otimes_{K[H]}V\subseteq \oplus_{t\in T}t\otimes V$. La otra inclusión es trivial. Por definición, 
la suma sobre $t\in T$ de los $t\otimes V$ es directa. 
\end{proof}

\begin{theorem}[Reciprocidad de Frobenius]
\index{Teorema!de reciprocidad de Frobenius}
Sea $G$ un grupo finito y $H$ un subgrupo de $G$. 
Si $U$ es un $K[G]$-módulo y $V$ es un $K[H]$-módulo, entonces
\[
\Hom_{K[H]}(V,\Res_H^GU)\simeq \Hom_{K[G]}(\Ind_H^GV,U)
\]
como espacios vectoriales.
\end{theorem}

\begin{proof}
Si $\varphi\in\Hom_{K[H]}(V,\Res_H^GU)$, sea 
\[
f_{\varphi}\colon K[G]\times V\to U,
\quad
(g,v)\mapsto g\cdot\varphi(v).
\]
Veamos que $f_{\varphi}$ es balanceada. Un cálculo directo muestra que
\begin{align*}
    &f_{\varphi}(g+g_1,v)=f_{\varphi}(g,v)+f_{\varphi}(g_1,v),&&
    f_{\varphi}(g,v+w)=f_{\varphi}(g,v)+f_{\varphi}(g,w).
\end{align*}
Como $\varphi$ es morfismo de $K[H]$-módulos,
\begin{align*}
    &f_{\varphi}(gh,v)=(gh)\cdot\varphi(v)
    =g\cdot (h\cdot \varphi(v))
    =g\cdot (h\cdot\varphi(v))
    =g\cdot \varphi(h\cdot v)=f_{\varphi}(g,h\cdot v)
\end{align*}
para todo $g\in G$, $h\in H$ y $v\in V$. Por último,
\begin{align*}
    &f_{\varphi}(gg_1,v)=(gg_1)\cdot\varphi(v)=g\cdot(g_1\cdot\varphi(v))=g\cdot f_{\varphi}(g_1,v)
\end{align*}
para todo $g,g_1\in G$ y $v\in V$. Para cada $\varphi\in\Hom_{K[H]}(V,\Res_H^GU)$ tenemos 
entonces un $\Gamma(\varphi)\in\Hom_{K[G]}(\Ind_H^GV,U)$ tal que
$\Gamma(\varphi)(g\otimes v)=g\cdot\varphi(v)$. 
Tenemos así definida una función 
\[
\Gamma\colon \Hom_{K[H]}(V,\Res_H^GU)\to\Hom_{K[G]}(\Ind_H^GV,U),
\quad
\varphi\mapsto\Gamma(\varphi).
\]

La función $\Gamma$ es lineal e inyectiva, ambas afirmaciones fáciles de verificar. 

Es también sobreyectiva, pues si $\theta\in\Hom_{K[H]}(\Ind_H^GV,U)$, entonces
la función $\varphi(v)=\theta(1\otimes v)$ es tal que $\varphi\in\Hom_{K[H]}(V,\Res_H^GU)$ y 
cumple 
\[
\Gamma(\varphi)(g\otimes v)=g\cdot\varphi(v)=g\cdot\theta(1\otimes v)=\theta(g\otimes v).\qedhere
\]
\end{proof}

Supongamos ahora que $K=\C$. 

Sea $H$ un subgrupo de $G$. Si $U$ es un $\C[G]$-módulo con caracter $\chi$, el caracter de $\Res_H^GU$ se denota por $\chi|_H$ y vale que 
que $\chi|_H(1)=\chi(1)$. Si $V$ es un $\C[H]$-módulo con 
caracter $\phi$, el módulo $\Ind_H^GV$ tiene caracter $\phi^G$ y vale que $\phi^G(1)=(G:H)\phi(1)$. 
\begin{align*}
\langle \phi,\chi|_H\rangle_H 
&=\dim\Hom_{\C[H]}(V,\Res_H^GU)
=\dim\Hom_{\C[G]}(\Ind_H^GV,U)
=\langle\phi^G,\chi\rangle_G,
\end{align*}
donde $\langle \alpha,\beta\rangle_X=\sum_{x\in X}\alpha(x)\overline{\beta(x)}$ denota el producto 
interno del espacio de funciones $X\to\C$. 

\begin{definition}
Si $\Irr(G)=\{\chi_1,\dots,\chi_k\}$ e $\Irr(H)=\{\phi_1,\dots,\phi_l\}$, se define
la \textbf{matriz de inducción--restricción} como la matriz $(c_{ij})\in\C^{l\times k}$, donde
\[
c_{ij}=\langle \phi_i^G,\chi_j\rangle_G=\langle\phi_i,\chi_j|_H\rangle_H.
\]
\end{definition}

La fila $i$-ésima de la matriz de inducción--restricción da la multiplicidad con que el caracter $\chi_j$ aparece
en la descomposición de $\phi_i^G$. La columna $j$-ésima da la multiplicidad con que el caracter $\phi_i$ aparece 
en la descomposición de $\chi_j|H$.

\begin{example}
Sea $G=\Sym_3$. 
La tabla de caracteres de $G$ es 
	\begin{center}
		\begin{tabular}{|c|rrr|}
			\hline
			& $1$ & $3$ & $2$\tabularnewline
			& $1$ & $(12)$ & $(123)$ \tabularnewline
			\hline 
			$\chi_{1}$ & $1$ & $1$ & $1$\tabularnewline
			$\chi_{2}$ & $1$ & $-1$ & $1$ \tabularnewline
			$\chi_{3}$ & $2$ & $0$ & $-1$ \tabularnewline
			\hline
		\end{tabular}
	\end{center}
La tabla de caracteres del subgrupo 
$H=\{\id,(12)\}$ es 
\begin{center}
\begin{tabular}{|c|rr|}
\hline 
& $1$ & $1$ \tabularnewline
& $\id$ & $(12)$ \tabularnewline
\hline 
$\phi_{1}$ & $1$ & $1$ \tabularnewline
$\phi_{2}$ & $1$ & $-1$\tabularnewline
\hline
\end{tabular}
\end{center}
A simple vista vemos que $\chi_1|_H=\phi_1$, $\chi_2|_H=\phi_2$ y que $\chi_3|_H=\phi_1+\phi_2$. 
La matriz de inducción--restricción es entonces
\[
\begin{pmatrix}
1 & 0 & 1\\
0 & 1 & 1
\end{pmatrix}.
\]
Observemos que además $\phi_1^G=\chi_1+\chi_3$ y que $\phi_2^G=\chi_2+\chi_3$. 
\end{example}

Veamos cómo calcular explícitamente caracteres inducidos. 

\begin{proposition}
Sea $H$ un subgrupo de $G$ y sea $V$ es un $\C[H]$-módulo con caracter $\chi$. Si 
$T$ es un trasversal de $H$ en $G$, entonces
\[
\chi^G(g)=\sum_{\substack{t\in T\\t^{-1}gt\in H}}\chi(t^{-1}gt)
\]
para todo $g\in G$. 
\end{proposition}

\begin{proof}
    Sabemos que $\Ind_H^GV=\oplus_{t\in T}t\otimes V$. 
    Supongamos que $T=\{t_1,\dots,t_m\}$ 
    y sea $\{v_1,\dots,v_n\}$ una base de $V$. 
    Entonces $\{t_i\otimes v_k:1\leq i\leq m,\,1\leq k\leq n\}$ es 
    una base de $\Ind_H^GV$ y la acción
    de $g$ en $\Ind_H^GV$ está dada por
    \[
    \rho^G(g)=\begin{cases}
    \rho(t_j^{-1}gt_i) & \text{si $t_j^{-1}gt_i\in H$},\\
    0 & \text{en otro caso}.
    \end{cases}
    \]
    En efecto, si $gt_i=t_jh$ para $h\in H$ y ciertos $i,j$, entonces 
    \[
    g\cdot (t_i\otimes v_k)=gt_i\otimes v_k=t_jh\otimes v_k=t_j\otimes h\cdot v_k
    \]
    y además $gt_i=t_jh$ si y sólo si $t_j^{-1}gt_i=h\in H$. Se concluye entones
    que $g$ actúa como $t^{-1}gt$ en $V$ en caso en que $t^{-1}gt\in H$ y 
    como la transformación nula en otro caso. 
\end{proof}

\begin{corollary}
\label{cor:induccion}
    Sea $H$ un subgrupo de $G$ 
    y sea $V$ es un $\C[H]$-módulo con caracter $\chi$.
    Si $g\in G$, entonces
    \[
    \chi^G(g)=\frac{1}{|H|}\sum_{\substack{x\in G\\x^{-1}gx\in H}}\chi(x^{-1}gx).
    \]
\end{corollary}

\begin{proof}
    Sea $T$ un transversal de $H$ en $G$. Si $x\in G$, escribimos $x=th$ para $t\in T$ y $h\in H$. 
    Como $x^{-1}gx=h^{-1}(t^{-1}gt)h$, entonces $x^{-1}gx\in H\Longleftrightarrow t^{-1}gt\in H$ y además, en ese caso, 
    $\chi(x^{-1}gx)=\chi(t^{-1}gt)$ pues $\chi$ es una función de clases. Eso implica que existen $|H|$ elementos $x\in G$ 
    tales que $x^{-1}gx\in H$. Para esos $x$, se tiene $\chi(x^{-1}gx)=\chi(t^{-1}gt)$, lo que implica 
    el corolario. 
\end{proof}


\subsection*{Mackey's irreducibility criterion}

It is not at all clear that 
induction of an irreducible character will produce an irreducible character. In fact, 
inducing the trivial character of the trivial subgroup to the whole group produces the 
regular representation, which in general is not irreducible. Mackey found a criterion 
that describes when an induced character is irreducible. See \cite[\S8.3]{MR2867444}. 

\subsection*{McKay's conjecture}

Prove McKay's conjecture \ref{conjecture:McKay} for all sporadic simple groups. 
This was first proved by Wilson in \cite{MR1643110}. 
Note that
for some ``small" sporadic simple groups this can be done
with the script presented in \S\ref{McKay}. However, 
for several sporadic simple groups a different approach is needed. One needs
to know the structure of normalizers. 
% http://www.math.rwth-aachen.de/~Thomas.Breuer/ctblocks/doc/overview.html

\subsection*{Ore's conjecture}

Prove Ore's conjecture \ref{conjecture:Ore} for alternating simple groups,
see for example \cite{MR40298}. It is also interesting to prove the conjecture
for other "small" simple groups such as $\PSL(3,2)$.  

\subsection*{An elementary proof of Brauer--Fowler theorem}

We need to find a subgroup of index $\leq 2n^2$. 
Let $X$ be the conjugacy class of $x$. For $g\in G$ let
\[
J(g)=\{z\in X:zgz^{-1}=g^{-1}\}.
\]
We claim that $|J(g)|\leq|C_G(g)|$. The map $J(g)\to C_G(g)$, $z\mapsto gz$, 
is well-defined,~as 
\[
(gz)g(gz)^{-1}=g(xgx^{-1})g^{-1}=g^{-1}\in C_G(g).
\]
It is injective, as $gz=gz_1$ implies $z=z_1$.

Let $\{(g,z)\in G\times X:zgz^{-1}=g^{-1}\}$.  
Since $X\times X\to J$, $(y,z)\mapsto (yz,z)$, 
is well-defined (since $z(yz)z^{-1}=zy=(yz)^{-1}$) and
it is trivially injective, 
\[
|X|^2\leq |J|=\sum_{(g,z)\in J}1\leq\sum_{g\in G}|J(g)|
=\sum_{g\in G}|C_G(g)|=k|G|,
\]
where $k$ is the number of conjugacy classes of $G$, 
as $(g,z)\in J$ if and only if $z\in J(g)$. Thus $|G|\leq kn^2$, as
\[
\left(\frac{|G|}{|C_G(x)|}\right)^2=|X|^2=\frac{|G|}{n^2}\leq k|G|.
\]

\begin{claim}
    There exists a conjugacy class with $\leq 2n^2$ elements.
\end{claim}

Assume that the claim is not true. Let
$C_1,\dots,C_k$ be the conjugacy classes of $G$, where 
$C_1=\{1\}$ and $|C_i|>2n^2$ for all $i\in\{2,\dots,k\}$. Then
\[
|G|=1+\sum_{i=2}^k|C_i|>1+\sum_{i=2}^kn^2=1+(k-1)2n^2\geq |G|,
\]
a contradiction. 

\begin{claim}
    There exists a subgroup $H$ of $G$ such that
    $(G:H)\leq 2n^2$.
\end{claim}

Let $C$ be a conjugacy class of $G$ such that 
$|C|\leq 2n^2$. Let $g\in C$.  
Then $H=C_G(g)$ is a subgroup of $G$ such that
$(G:H)\leq 2n^2$. 
This finishes the proof of the Brauer--Fowler theorem. 


\subsection*{Hirsh's theorem}

In \cite{MR36755} Hirsch found a generalization of Burnside's Theorem \ref{thm:Burnside_mod16}.  
If $G$ is a finite group and $d$ is the greatest common divisor of all 
the numbers $p^2-1$, where the $p$'s are prime divisors of $|G|$ and $r$ the number of conjugate sets in $G$. Then 
\[
|G|\equiv\begin{cases} 
    r\bmod 2d &\text{if $|G|$ odd,}\\
    r\bmod 3 & \text{if $|G|$ even and $\gcd(|G|,3)=1$.}
    \end{cases}
\]
The proof is elementary, does not use character theory. Is it possible
to prove Hirsch's theorem using characters?

\subsection*{Hurwitz's theorem}
\label{Hurwitz}

\index{Fibonacci identity}
\index{Euler identity}
\index{Hamilton identity}
We know that $x^2y^2=(xy)^2$ holds for all $x,y\in\C$. Fibonnaci
found the identity
\[
	(x_1^2+x_2^2)(y_1^2+y_2^2)=(x_1y_1-x_2y_2)^2+(x_1y_2-x_2y_1)^2.
\]
Euler and Hamilton, independently, found 
a similar identity:
\[
	(x_1^2+x_2^2+x_3^2+x_4^2)(y_1^2+y_2^2+y_3^2+y_4^2)=z_1^2+z_2^2+z_3^2+z_4^2,
\]
where
\begin{equation}
\label{eq:Hamilton}
\begin{aligned}
	 z_1&=x_1y_1-x_2y_2-x_3y_3-x_4y_4,\\
	 z_2&=x_1y_2+x_2y_1-x_3y_3-x_4y_4,\\
	 z_3&=x_1y_3+x_3y_1-x_2y_4+x_4y_2,\\ 
	 z_4&=-x_1y_4+x_4y_1x_2y_3-x_3y_2.
\end{aligned}
\end{equation}
Cayley found a similar identity for sums of eight squares. 
Are there other identities of this type? Hurwitz' proved that this is not the case. 
We present Eckmann's proof of Hurwitz' theorem. The proof uses character theory.

\begin{lemma}
\label{lem:grupo}
	Let $n>2$ be an even number. If 
	there exists a group $G$ with generators
	$\epsilon,x_1,\dots,x_{n-1}$ and relations 
	\[
		x_1^2=\cdots=x_{n-1}^2=\epsilon\ne1,\quad
		\epsilon^2=1,\quad
		[x_i,x_j]=\epsilon\quad\text{if}\quad i\ne j,
	\]
	then the following statements hold:
	\begin{enumerate}
		\item $|G|=2^n$.
		\item $[G,G]=\{1,\epsilon\}$. In particular, $G$ 
		    has exactly $2^{n-1}$ degree-one representations. 
		\item If $g\not\in Z(G)$, then the conjugacy class of $g$ is $\{g,\epsilon g\}$.
		\item $Z(G)=\{1,\epsilon,x_1\cdots x_{n-1},\epsilon x_1\cdots x_{n-1}\}$. 
		\item $G$ has $2^{n-1}+2$ conjugacy classes.
		\item $G$ has two irreducible representations of degree $2^{\frac{n-2}{2}}>1$. 
	\end{enumerate}
\end{lemma}

\begin{proof}
    Let us prove 1) and 2). Note that $\epsilon\in Z(G)$, as
    $\epsilon=x_i^2$ for all 
	$i\in\{1,\dots,n-1\}$. Since $n-1>2$, $[x_1,x_2]=\epsilon$. Hence 
	$\epsilon\in [G,G]$. Moreover, $G/\langle\epsilon\rangle$ is abelian. Thus 
	$[G,G]=\langle \epsilon\rangle$. Since $G/[G,G]$ is elementary 
	abelian of order 
	$2^{n-1}$, it follows that 
	$|G|=2^n$. 

	We now prove 3). Let $g\in G\setminus Z(G)$ and 
	$x\in G$ be such that $[x,g]\ne 1$. Then $[x,g]=\epsilon$ and 
	$xgx^{-1}=\epsilon g$. 

	To prove 4) let $g\in G$. Write
	\[
		g=\epsilon^{a_0}x_1^{a_1}\cdots x_{n-1}^{a_{n-1}},
	\]
	where $a_j\in\{0,1\}$ for all $j\in\{1,\dots,n-1\}$. 
	If $g\in Z(G)$, then $gx_i=x_ig$ for all $i\in\{1,\dots,n-1\}$. Hence 
	$g\in Z(G)$ if and only if 
	\[
		\epsilon^{a_0}x_1^{a_1}\cdots x_{n-1}^{a_{n-1}}=x_i(\epsilon^{a_0}x_1^{a_1}\cdots x_{n-1}^{a_{n-1}})x_i^{-1}.
	\]
	Since $x_ix_j^{a_j}x_i=\epsilon^{a_j}x_j^{a_j}$ 
	whenever $i\ne j$ y $\epsilon\in Z(G)$, the elmeent $g$ is 
	central if and only if 
	\[
		\sum_{\substack{j=1\\j\ne i}}^{n-1}a_j\equiv 0\bmod 2
	\]
	for all $i\in\{1,\dots,n-1\}$. In particular, 
	\[
	\sum_{j\ne i}a_j\equiv \sum_{j\ne k}a_j
	\]
	for all $k\ne i$. Therefore $a_i\equiv a_k\bmod 2$ for all 
	$i,k\in\{1,\dots,n-1\}$. Thus $a_1=\cdots=a_{n-1}$ and  
	$Z(G)=\{1,x_1\cdots x_{n-1},\epsilon,\epsilon x_1\cdots
	x_{n-1}\}$. 
	
    To prove 5) we use the class equation:
    \[
		2^n=|G|=|Z(G)|+\sum_{i=1}^N2=4+2N. 
	\]
	It follows that $G$ has $N+4=2^{n-1}+2$ conjugacy classes.
	
	Finally we prove 6). 
	Since $G$ 
	has exactly $2^{n-1}$ degree-one representations (because 
	$|G/[G,G]|=2^{n-1}$) and 
	has $2^{n-1}+2$ conjugacy classes, 
	it follows from 
	\[
		2^n=|G|=\underbrace{1+\cdots+1}_{2^{n-1}}+f_1^2+f_2^2=2^{n-1}+f_1^2+f_2^2,
	\]
	that $G$ has two irreducible representations
	of degrees $f_1=f_2=2^{\frac{n-2}{2}}>1$. 
\end{proof}

\begin{example}
	The formulas~\eqref{eq:Hamilton} give a representation for the
	group $G$ of the previous lemma. Write each $z_i$ as 
	$z_i=\sum_{k=1}^4a_{1k}(x_1,\dots,x_4)y_k$. Let $A$ be a matrix
	such that 
	$A_{ij}=a_{ij}(x_1,\dots,x_4)$, that is 
	\[
		A=\begin{pmatrix}
			x_1 & -x_2 & -x_3 & -x_4\\
			x_2 & x_1 & -x_4 & x_3\\
			x_3 & x_4 & x_1 & -x_2\\
			x_4 & -x_3 & x_2 & x_1
		\end{pmatrix}
	\]
	The matrix $A$ can be written as $A=A_1x_1+A_2x_2+A_3x_3+A_4x_4$, where
	\begin{align*}
		&A_1=\begin{pmatrix}
		1\\
		&1\\
		&&1\\
		&&&1\\
		\end{pmatrix},
		&&
		A_2=\begin{pmatrix}
			& -1\\
			1 \\
			&&&-1\\
			&&1
		\end{pmatrix},
		&&
		A_3=\begin{pmatrix}
			&& -1 \\
			&&&1 & \\
			1\\
			&-1
		  \end{pmatrix},
		  &&
		  A_4=\begin{pmatrix}
			&&&-1\\
			&&-1\\
			&1\\
			1
		\end{pmatrix}.
	\end{align*}
	For $i\in\{1,\dots,4\}$ let $B_i=A_iA_4^T$. Then
	$B_i=-B_i^T$ and  $B_i^2=-I$ 
	for all $i\in\{1,2,3\}$. Moreover, $B_iB_j=-B_jB_i$ for all $i,j\in\{1,2,3\}$ and
	$i\ne j$.  
	The group generated by $\{B_1,B_2,B_3\}$ has $2^3$ element, all of them
	of the form
	\[
		\pm B_1^{k_1}B_2^{k_2}B_3^{k_3}
	\]
	for $k_j\in\{0,1\}$. 
    The map 
	$G\to\langle B_1,B_2,B_3\rangle$,
	\[
		x_1\mapsto B_1,\quad
		x_2\mapsto B_2,\quad
		x_3\mapsto B_3 
	\]
	extends to a group isomomorphism. 
\end{example}

\begin{theorem}[Hurwitz]
	\index{Hurwitz' theorem}
	If there is an identity of the form 
	\begin{equation}
		\label{eq:Hurwitz}
		(x_1^2+\cdots+x_n^2)(y_1^2+\cdots+y_n^2)=z_1^2+\cdots+z_n^2,
	\end{equation}
	where the $x_j$'s and the $y_j$'s are real (or complex) numbers and
	each $z_k$ is a bilinear function in the $x_j$'s and the $y_j$'s, then 
	$n\in\{1,2,4,8\}$.
\end{theorem}

\begin{proof}
    We work over complex numbers. 
	Without loss of generality we may assume that $n>2$.  For 
	$i\in\{1,\dots,n\}$ let  
	\[
		z_i=\sum_{k=1}^n a_{ik}(x_1,\dots,x_n)y_k,
	\]
	where the $a_{ik}$'s are linear functions. Then 
	\[
		z_i^2=\sum_{k,l=1}^na_{ik}(x_1,\dots,x_n)a_{il}(x_1,\dots,x_n)y_ky_l
	\]
	for all $i\in\{1,\dots,n\}$. Using these expressions for each $z_i$
	in~\eqref{eq:Hurwitz} and comparing coefficients, 
	\begin{equation}
		\label{eq:delta}
		\sum_{i=1}^n a_{ik}(x_1,\dots,x_n)a_{il}(x_1,\dots,x_n)=\delta_{k,l}(x_1^2+\cdots+x_n^2),
	\end{equation}
	where $\delta_{k,l}$ is the usual Kronecker's map. Let 
	$A$ be the $n\times n$ matrix given by 
	\[
	A_{ij}=a_{ij}(x_1,\dots,x_n).
	\]
	Then 
	\begin{equation}
		\label{eq:AAT}
		AA^T=(x_1^2+\cdots+x_n^2)I,
	\end{equation}
	where $I$ denotes the $n\times n$ identity matrix, 
	as 
	\[
		(AA^T)_{kl}=\sum_{i=1}^na_{ki}(x_1,\dots,x_n)a_{li}(x_1,\dots,x_n)=\delta_{kl}(x_1^2+\cdots+x_n^2)
	\]
	by~\eqref{eq:delta}. Since each $a_{ki}(x_1,\dots,x_n)$ is a linear function, 
	there exist $\alpha_{ij1},\dots,a_{ijn}\in\C$ such that 
	\[
		a_{ij}(x_1,\dots,x_n)=\alpha_{ij1}x_1+\cdots+\alpha_{ijn}x_n.
	\]
	Write 
	\[
		A=A_1x_1+\cdots+A_nx_n,
	\]
	where each $A_k$ is the matrix $(A_k)_{ij}=\alpha_{ijk}$. 
	The formula~\eqref{eq:AAT} becomes
	\[
		\sum_{i=1}^n\sum_{j=1}^nA_iA_j^Tx_ix_j=(x_1^2+\cdots+x_n^2)I.
	\]
	Thus 
	\begin{equation}
		\label{eq:condiciones}
		A_iA_j^T+A_jA_i^T=0\quad i\ne j,\quad
		A_iA_i^T=I.
	\end{equation}
	We need $n$ complex square matrices of size $n\times n$
	satisfying~\eqref{eq:condiciones}. For $i\in\{1,\dots,n\}$ let  
	$B_i=A_n^TA_i$. Then~\eqref{eq:condiciones} turn into  
	\[
		B_iB_j^T+B_jB_i^T=0\quad i\ne j,\quad
		B_iB_i^T=I,\quad
		B_n=I.
	\]
	Set $j=n$ in the first family of equations to obtain $B_i=-B_i^T$ for all 
	$i\in\{1,\dots,n-1\}$. It follows that 
	\begin{equation}
	\label{eq:representation}
	\begin{aligned}
	    &B_i^2=-I && \text{for all $i\in\{1,\dots,n-1\}$},\\
	    &[B_i,B_j]=-I && \text{for all $i,j\in\{1,\dots,n-1\}$.}
	\end{aligned}
	\end{equation}
    
    \begin{claim}
        $n$ is even. 
    \end{claim}
    
	Computing the determinant of 
	$B_iB_j=-B_jB_i$ we obtain that 
	\[
	1=\det(B_iB_j)=(-1)^n\det(B_jB_i)=(-1)^n.
	\]
	Hence $n$ is even. 

	\begin{claim}
	    The group 
	    $G$ of the lemma admits a faithful
	    representation $\rho\colon G\to\GL_n(\C)$. 
	\end{claim}
	
	By \eqref{eq:representation}, there is a well-defined 
	injective group homomorphism $\rho$ such that 
	$x_i\mapsto B_i$ for all $i\in\{1,\dots,n-1\}$ and 
	$\epsilon\mapsto -I$. 
	
	\begin{claim}
	    $2^{\frac{n-2}{2}}$ divides $n$.
	\end{claim}
	
	Since $\epsilon\in[G,G]$ by Lemma \ref{lem:grupo}, 
	every one-dimensional representation satisfies $\epsilon\mapsto 1$.
	This implies that $\rho$ cannot have degree-one sub representations. 
	In fact, 
	if $W=\langle w\rangle$ is $G$-invariant subspace of $\C^n$, 
	then $\psi=\rho|_W\colon G\to\GL(W)\simeq\C^\times$ 
	is a representation. In particular, 
	\[
	-w=-Iw=\psi_{\epsilon}(w)=\psi_{[x_i,x_j]}(w)
	=\psi_{x_i}\psi_{x_j}\psi_{{x_i}}^{-1}\psi_{{x_j}}^{-1}(w)=w, 
	\]
	a contradiction. 
	
	This means that the $\C[G]$-module $\C^n$ 
	decomposes as $\C^n\simeq aS\oplus bT$,
	where $a$ and $b$ are integers and 
	$S$ and $T$ are simple $\C[G]$-modules of dimension
	$2^{\frac{n-2}{2}}$. In particular, 
	\[
	n=\dim V=\dim(aS\oplus bT)=(a+b)2^{\frac{n-2}{2}}.
	\]
	
	To finish the proof of the theorem write $n=2^ab$ 
	for $a\geq1$ and $b$ an odd integer. 
	Since $\frac{n-2}{2}$ divides $n$, 
	\[
	2^{\frac{n}{2}-1}=2^{\frac{n-2}{2}}\leq n=2^ab. 
	\]
	Thus $\frac{n}{2}-1\leq a$ and hence $2^a\leq n\leq 2(a+1)$. 
	It follows that $n\in\{4,8\}$.  
\end{proof}

We now present an application, see
\cite{MR1534187} for more information. 

\begin{theorem}
	Let $V$ be a real vector space (with an inner product) 
	such that $\dim
	V=n\geq3$. If there exists a bilinear function 
	$V\times V\to\R$, $(v,w)\mapsto v\times
	w$, such that $v\times w$ is orthogonal both 
	to $v$ and $w$ and 
	\[
		\|v\times w\|^2=\|v\|^2\|w\|^2-\langle v,w\rangle^2,
	\]
	where $\|v\|^2=\langle v,v\rangle$, then $n\in\{3,7\}$. 
\end{theorem}

\begin{proof}
	Let $W=V\oplus\R$ with the inner product  
	\[
		\langle (v_1,r_1),(v_2,r_2)\rangle = \langle v_1,v_2\rangle+r_1r_2.
	\]
	Note that
	\begin{align*}
		\langle v_1\times &v_2+r_1v_2+r_2v_1,v_1\times v_2+r_1v_2+r_2v_1\rangle\\
		&=\|v_1\times v_2\|^2+r_1^2\|v_2\|^2+2r_1r_2\langle v_1,v_2\rangle+r_2^2\|v_1\|^2.
	\end{align*}
	Thus  
	\begin{align*}
		(\|v_1\|^2+r_1^2)&(\|v_2\|^2+r_2)\\
		&= \|v_1\|^2\|v_2\|^2+r_2^2\|v_1\|^2+r_1^2\|v_2\|^2+r_1^2r_2^2\\
		&=\|v_1\times v_2+r_1v_1+r_2v_2\|^2-2r_1r_2\langle v_1,v_2\rangle+\langle v_1,v_2\rangle^2+r_1^2r_2^2\\
		&=\|v_1\times v_2+r_1v_1+r_2v_2\|^2+(\langle v_1,v_2\rangle-r_1r_2)^2\\
		&=z_1^2+\cdots+z_{n+1}^2,
	\end{align*}
	where the $z_k$'s are bilinear functions in $(v_1,r_1)$ and $(v_2,r_2)$. 
	By Hurwitz' theorem, 
	$n+1\in\{4,8\}$. Hence $n\in\{3,7\}$.
\end{proof}

In the theorem, if $\dim V=3$, we obtain the usual cross product. 
If $\dim V=7$, let 
\[
	W=\{(v,k,w):v,w\in V,k\in\R\}
\]
with the inner product 
\[
	\langle (v_1,k_1,w_1),(v_2,k_2,w_2)\rangle = \langle v_1,v_2\rangle+k_1k_2+\langle w_1,w_2\rangle.
\]
It is an exercise to show that 
\begin{multline*}
	(v_1,k_1,w_1)\times (v_2,k_2,w_2)\\
	=(k_1w_2-k_2w_1+v_1\times v_2-w_1\times w_2,
	\\-\langle v_1,w_2\rangle+\langle v_2,w_1\rangle, 
	k_2v_1-k_1v_2-v_1\times w_2-w_1\times v_2)
\end{multline*}
satisfies the properties of the theorem. 

\subsection*{Poincar\'e--Birkhoff--Witt theorem}

There are several proofs of the
Poincar\'e--Birkhoff--Witt theorem \ref{thm:PBW}, see for example 
\cite[\S17.4]{MR499562} or \cite[Theorem 2.17]{MR938524}. 
Bergman's proof based on the diamond lemma 
appears in \cite{MR506890}. 

\subsection*{Weyl's theorem}

Weyl's theorem states that every finite-dimensional module over
a semisimple Lie algebra is completely irreducible. 
See \cite[Theorem 17.4]{MR2218355} for a proof. 

\subsection*{Irreducible representations of $U_q(\sl(2,\C))$}

Let $q\in\C\setminus\{0,1,-1\}$. 
Let $U_q(\sl(2))$ be the (complex) algebra generated by 
variables $E$, $F$, $K$ and $K^{-1}$ with relations
\begin{align*}
    &KK^{-1}=K^{-1}K=1,
    &&
    KEK^{-1}=q^2E,\\
    &
    KFK^{-1}=q^{-2}F,
    &&
    [E,F]=\frac{1}{(q-q^{-1})}(K-K^{-1}).
\end{align*}

This algebra is a \emph{deformation} of the enveloping algebra
of $\sl(2,\C)$. The goal is to study the 
representation theory of $U_q(\sl(2))$. This splits into
two cases, depending on whether $q$ is a root of one or not. 
Finite-dimensional simple $U_q(\sl(2))$-modules are studied 
in \cite[VI]{MR1321145}. In particular, if 
$q$ is not a root of one, finite-dimensional simple $U_q(\sl(2))$-modules
are classified in \cite[Theorem VI.3.5]{MR1321145}. 

\subsection*{Semisimple modules of $U_q(\sl(2,\C))$}

Prove that if $q$ is not a root of one, any finite-dimensional
$U_q(\sl(2))$-module is semisimple. 
See \cite[Theorem VII.2.2]{MR1321145}. 