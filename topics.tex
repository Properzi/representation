\chapter*{Topics for final projects}

\pagestyle{plain}
\fancyhf{}
\fancyhead[LE,RO]{Representation theory of groups}
\fancyhead[RE,LO]{Some topics}
\fancyfoot[CE,CO]{\leftmark}
\fancyfoot[LE,RO]{\thepage}

\addcontentsline{toc}{chapter}{Some topics for a final project}

\subsection*{Staircase groups}

This topic describes a situation similar to that of \S\ref{Kolchin}, but
more general. See \cite[Chapter 5]{MR1369573}.

\subsection*{Kegel--Wielandt's theorem}

Prove Kegel--Wielandt's theorem. The theorem states that if a finite group 
$G$ factorizes as $G=AB$ with $A$ and $B$ nilpotent subgroups, then $G$ 
is solvable. For the proof see 
\cite[Theorem 2.13]{MR1211633}. 

\subsection*{The Drinfeld double of a finite group}

See \cite[Chapter IX]{MR1321145} and 
\cite[Chapter 8]{MR3752618}.

\subsection*{Ito's theorem}

Ito's theorem generalize Frobenius' theorem
(Theorem \ref{thm:Frobenius_chi(1)})  
and Schur's theorem (Theorem \ref{thm:Schur_chi(1)}). 
The theorem states that if $\chi$ is an irreducible character
of a finite group $G$, then $\chi(1)$ divides 
$(G:A)$ for every normal abelian subgroup $A$ of $G$. 
See \cite[\S8.1]{MR0450380}. 

\subsection*{Characters of $\GL_2(q)$ and $\SL_2(q)$}

One possible topic is the character table of $\GL_2(q)$, see
\cite[\S5.2]{MR2867444}. Alternatively, one can 
present the character table of the group $\SL_2(p)$  
following Humphreys's paper \cite{MR364478}. 
The character theory of $\SL_2(q)$ appears in 
\cite[\S5.2]{MR2867444}, see 
\cite[Chapter 20]{MR1650707} for details. 

\subsection*{Representations of the symmetric group}

See for example \cite[\S10]{MR2867444} and 
\cite{MR1153249}. 

\subsection*{Random walks on finite groups}

The goal is to construct the character table or 
the irreducible representations of the symmetric group. 
The topic has connections with combinatorics and applications 
to voting and card shuffling. 
See \cite[4]{MR1153249} and \cite[\S11]{MR2867444}.

\subsection*{Fourier analysis on finite groups}

See \cite[\S5]{MR2867444} for a very elementary approach and some
basic applications. Other applications 
appear in \cite{MR1695775}.

\subsection*{McKay's conjecture}

Prove McKay's conjecture \ref{conjecture:McKay} for all sporadic simple groups. 
This was first proved by Wilson in \cite{MR1643110}. 
Note that
for some ``small" sporadic simple groups this can be done
with the script presented in \S\ref{McKay}. However, 
for several sporadic simple groups a different approach is needed. One needs
to know the structure of normalizers. 
% http://www.math.rwth-aachen.de/~Thomas.Breuer/ctblocks/doc/overview.html

\subsection*{Ore's conjecture}

Prove Ore's conjecture \ref{conjecture:Ore} for alternating simple groups,
see for example \cite{MR40298}. It is also interesting to prove the conjecture
for other "small" simple groups such as $\PSL(3,2)$.  

\subsection*{Hirsh's theorem}

