\chapter*{Topics for final projects}

\pagestyle{plain}
\fancyhf{}
\fancyhead[LE,RO]{Representation theory of groups}
\fancyhead[RE,LO]{Some topics}
\fancyfoot[CE,CO]{\leftmark}
\fancyfoot[LE,RO]{\thepage}

\addcontentsline{toc}{chapter}{Some topics for a final project}

In this chapter we collect some topics for a final presentation. 

\subsection*{Staircase groups}

This topic describes a situation similar to that of \S\ref{Kolchin}, but
more general. See \cite[Chapter 5]{MR1369573}.

\subsection*{Solvable and nilpotent groups}

The character table of a finite group
detects solvability and nilpotency of groups, see
\cite[Chapter 6]{MR1369573}.

\subsection*{Kegel--Wielandt theorem}

Prove Kegel--Wielandt theorem. The theorem states that if a finite group 
$G$ factorizes as $G=AB$ with $A$ and $B$ nilpotent subgroups, then $G$ 
is solvable. For the proof see 
\cite[Theorem 2.13]{MR1211633}. 

\subsection*{The Drinfeld double of a finite group}

See \cite[Chapter IX]{MR1321145} and 
\cite[Chapter 8]{MR3752618}.

\subsection*{Ito's theorem}

Ito's theorem generalize Frobenius' theorem
(Theorem \ref{thm:Frobenius_chi(1)})  
and Schur's theorem (Theorem \ref{thm:Schur_chi(1)}). 
The theorem states that if $\chi$ is an irreducible character
of a finite group $G$, then $\chi(1)$ divides 
$(G:A)$ for every normal abelian subgroup $A$ of $G$. 
See \cite[\S8.1]{MR0450380}. 

\subsection*{Characters of $\GL_2(q)$ and $\SL_2(q)$}

One possible topic is the character table of $\GL_2(q)$, see
\cite[\S5.2]{MR2867444}. Alternatively, one can 
present the character table of the group $\SL_2(p)$  
following Humphreys's paper \cite{MR364478}. 
The character theory of $\SL_2(q)$ appears in 
\cite[\S5.2]{MR2867444}, see 
\cite[Chapter 20]{MR1650707} for details. 

\subsection*{Representations of the symmetric group}

See for example \cite[\S10]{MR2867444} and 
\cite{MR1153249}. 

\subsection*{Random walks on finite groups}

The goal is to construct the character table or 
the irreducible representations of the symmetric group. 
The topic has connections with combinatorics and applications 
to voting and card shuffling. 
See \cite[4]{MR1153249} and \cite[\S11]{MR2867444}.

\subsection*{Fourier analysis on finite groups}

See \cite[\S5]{MR2867444} for a very elementary approach and some
basic applications. Other applications 
appear in \cite{MR1695775}.

\subsection*{McKay's conjecture}

Prove McKay's conjecture \ref{conjecture:McKay} for all sporadic simple groups. 
This was first proved by Wilson in \cite{MR1643110}. 
Note that
for some ``small" sporadic simple groups this can be done
with the script presented in \S\ref{McKay}. However, 
for several sporadic simple groups a different approach is needed. One needs
to know the structure of normalizers. 
% http://www.math.rwth-aachen.de/~Thomas.Breuer/ctblocks/doc/overview.html

\subsection*{Ore's conjecture}

Prove Ore's conjecture \ref{conjecture:Ore} for alternating simple groups,
see for example \cite{MR40298}. It is also interesting to prove the conjecture
for other "small" simple groups such as $\PSL(3,2)$.  

\subsection*{An elementary proof of Brauer--Fowler theorem}

We need to find a subgroup of index $\leq 2n^2$. 
Let $X$ be the conjugacy class of $x$. For $g\in G$ let
\[
J(g)=\{z\in X:zgz^{-1}=g^{-1}\}.
\]
We claim that $|J(g)|\leq|C_G(g)|$. The map $J(g)\to C_G(g)$, $z\mapsto gz$, 
is well-defined,~as 
\[
(gz)g(gz)^{-1}=g(xgx^{-1})g^{-1}=g^{-1}\in C_G(g).
\]
It is injective, as $gz=gz_1$ implies $z=z_1$.

Let $\{(g,z)\in G\times X:zgz^{-1}=g^{-1}\}$.  
Since $X\times X\to J$, $(y,z)\mapsto (yz,z)$, 
is well-defined (since $z(yz)z^{-1}=zy=(yz)^{-1}$) and
it is trivially injective, 
\[
|X|^2\leq |J|=\sum_{(g,z)\in J}1\leq\sum_{g\in G}|J(g)|
=\sum_{g\in G}|C_G(g)|=k|G|,
\]
where $k$ is the number of conjugacy classes of $G$, 
as $(g,z)\in J$ if and only if $z\in J(g)$. Thus $|G|\leq kn^2$, as
\[
\left(\frac{|G|}{|C_G(x)|}\right)^2=|X|^2=\frac{|G|}{n^2}\leq k|G|.
\]

\begin{claim}
    There exists a conjugacy class with $\leq 2n^2$ elements.
\end{claim}

Assume that the claim is not true. Let
$C_1,\dots,C_k$ be the conjugacy classes of $G$, where 
$C_1=\{1\}$ and $|C_i|>2n^2$ for all $i\in\{2,\dots,k\}$. Then
\[
|G|=1+\sum_{i=2}^k|C_i|>1+\sum_{i=2}^kn^2=1+(k-1)2n^2\geq |G|,
\]
a contradiction. 

\begin{claim}
    There exists a subgroup $H$ of $G$ such that
    $(G:H)\leq 2n^2$.
\end{claim}

Let $C$ be a conjugacy class of $G$ such that 
$|C|\leq 2n^2$. Let $g\in C$.  
Then $H=C_G(g)$ is a subgroup of $G$ such that
$(G:H)\leq 2n^2$. 
This finishes the proof of the Brauer--Fowler theorem. 


\subsection*{Hirsh's theorem}

In \cite{MR36755} Hirsch found a generalization of Burnside's Theorem \ref{thm:Burnside_mod16}.  
If $G$ is a finite group and $d$ is the greatest common divisor of all 
the numbers $p^2-1$, where the $p$'s are prime divisors of $|G|$ and $r$ the number of conjugate sets in $G$. Then 
\[
|G|\equiv\begin{cases} 
    r\bmod 2d &\text{if $|G|$ odd,}\\
    r\bmod 3 & \text{if $|G|$ even and $\gcd(|G|,3)=1$.}
    \end{cases}
\]
The proof is elementary, does not use character theory. Is it possible
to prove Hirsch's theorem using characters?

\subsection*{Hurwitz's theorem}
\label{Hurwitz}

\index{Fibonacci identity}
\index{Euler identity}
\index{Hamilton identity}
We know that $x^2y^2=(xy)^2$ holds for all $x,y\in\C$. Fibonnaci
found the identity
\[
	(x_1^2+x_2^2)(y_1^2+y_2^2)=(x_1y_1-x_2y_2)^2+(x_1y_2-x_2y_1)^2.
\]
Euler and Hamilton, independently, found 
a similar identity:
\[
	(x_1^2+x_2^2+x_3^2+x_4^2)(y_1^2+y_2^2+y_3^2+y_4^2)=z_1^2+z_2^2+z_3^2+z_4^2,
\]
where
\begin{equation}
\label{eq:Hamilton}
\begin{aligned}
	& z_1=x_1y_1-x_2y_2-x_3y_3-x_4y_4, && 
	z_2=x_2y_1+x_1y_2-x_4y_3+x_3y_4,\\
	&z_3=x_3y_1+x_4y_2+x_1y_3-x_2y_4, && 
	z_4=x_4y_1-x_3y_2+x_2y_3-x_1y_4.
\end{aligned}
\end{equation}
Cayley found a similar identity for sums of eight squares. 
Are there other identities of this type? Hurwitz' proved that this is not the case. 
We present Eckmann's proof of Hurwitz' theorem. The proof uses character theory.

\begin{lemma}
	Let $n>2$ be an even number. If 
	there exists a group $G$ with generators
	$\epsilon,x_1,\dots,x_{n-1}$ and relations 
	\[
		x_1^2=\cdots=x_{n-1}^2=\epsilon\ne1,\quad
		\epsilon^2=1,\quad
		[x_i,x_j]=\epsilon\quad\text{if}\quad i\ne j,
	\]
	then the following statements hold:
	\begin{enumerate}
		\item $|G|=2^n$.
		\item $[G,G]=\{1,\epsilon\}$.
		\item If $g\not\in Z(G)$, then the conjugacy class of $g$ is $\{g,\epsilon g\}$.
		\item $Z(G)=\{1,\epsilon,x_1\cdots x_{n-1},\epsilon x_1\cdots x_{n-1}\}$. 
		\item $G$ has $2^{n-1}+2$ conjugacy classes.
	\end{enumerate}
\end{lemma}

\begin{proof}
%	Primero observemos que todo elemento de $G$ puede escribirse como
%	\[
%		\epsilon^{k_0}x_1^{k_1}\cdots x_{n-1}^{k_{n-1}}
%	\]
%	para algunos $k_0,k_1,\dots,k_{n-1}\in\{0,1\}$. 
	Primero demostramos las dos primeras afirmaciones. 
	Observemos que $\epsilon\in Z(G)$ pues $\epsilon=x_i^2$ para todo
	$i\in\{1,\dots,n-1\}$. Como $n-1>2$, $[x_1,x_2]=\epsilon$ y luego
	$\epsilon\in [G,G]$. Además $G/\langle\epsilon\rangle$ es abeliano y luego
	$[G,G]=\langle \epsilon\rangle$. Como $G/[G,G]$ es elemental abeliano de
	orden $2^{n-1}$, se sigue que $|G|=2^n$. 

	Demostremos ahora la tercera afirmación. Sea $g\in G\setminus Z(G)$ y sea
	$x\in G$ tal que $[x,g]\ne 1$. Entonces $[x,g]=\epsilon$ y luego
	$xgx^{-1}=\epsilon g$. 

	Demostremos la cuarta afirmación. Sea $g\in G$ y escribamos
	\[
		g=\epsilon^{a_0}x_1^{a_1}\cdots x_{n-1}^{a_{n-1}},
	\]
	donde $a_j\in\{0,1\}$ para todo $j\in\{1,\dots,n-1\}$. Si $g\in Z(G)$ entonces $gx_i=x_ig$ para todo $i\in\{1,\dots,n-1\}$. Luego
	$g\in Z(G)$ si y sólo si 
	\[
		\epsilon^{a_0}x_1^{a_1}\cdots x_{n-1}^{a_{n-1}}=x_i(\epsilon^{a_0}x_1^{a_1}\cdots x_{n-1}^{a_{n-1}})x_i^{-1}.
	\]
	Como $x_ix_j^{a_j}x_i=\epsilon^{a_j}x_j^{a_j}$ si $i\ne j$ y $\epsilon\in Z(G)$, el elemento $g$ es central si y sólo si 
	\[
		\sum_{\substack{j=1\\j\ne i}}^{n-1}a_j\equiv 0\bmod 2
	\]
	para todo $i\in\{1,\dots,n-1\}$. En particular, 
	\[
	\sum_{j\ne i}a_j\equiv \sum_{j\ne k}a_j
	\]
	para todo $k\ne i$, y en consecuencia $a_i\equiv a_k\bmod 2$ para todo
	$i,k\in\{1,\dots,n-1\}$. Luego $a_1=\cdots=a_{n-1}$ y entonces 
	$Z(G)=\{1,x_1\cdots x_{n-1},\epsilon,\epsilon x_1\cdots
	x_{n-1}\}$. 
	
	La última afirmación es entonces consecuencia de la ecuación de clases. Como
	\[
		2^n=|G|=|Z(G)|+\sum_{i=1}^N2=4+2N,
	\]
	se concluye que $G$ tiene $N+4=2^{n-1}+2$ clases de conjugación.
\end{proof}

\begin{example}
	Las fórmulas~\eqref{eq:Hamilton} dan una representación del grupo $G$ del
	lema anterior. Escribamos a cada $z_i$ como
	$z_i=\sum_{k=1}^4a_{1k}(x_1,\dots,x_4)y_k$. Sea $A$ la matriz tal que
	$A_{ij}=a_{ij}(x_1,\dots,x_4)$, es decir
	\[
		A=\begin{pmatrix}
			x_1 & -x_2 & -x_3 & -x_4\\
			x_2 & x_1 & -x_4 & x_3\\
			x_3 & x_4 & x_1 & -x_2\\
			x_4 & -x_3 & x_2 & -x_1
		\end{pmatrix}
	\]
	La matriz $A$ puede escribirse como $A=A_1x_1+A_2x_2+A_3x_3+A_4x_4$, donde $A_1=I$ y 
	\begin{align*}
		A_2=\begin{pmatrix}
			& -1\\
			1 \\
			&&&-1\\
			&&1
		\end{pmatrix},
		&&
		A_3=\begin{pmatrix}
			&& -1 \\
			&&&1 & \\
			1\\
			&-1
		  \end{pmatrix},
		  &&
		  A_4=\begin{pmatrix}
			&&&-1\\
			&&-1\\
			&1\\
			1
		\end{pmatrix}.
	\end{align*}
	Para cada $i\in\{1,\dots,4\}$ sea $B_i=A_4^TA_i$. Entonces $B_i=-B_i^T$ y $B_i^2=-I$ 
	para todo $i$ y además $B_iB_j=-B_jB_i$ para todo $i\ne j$. 
	El grupo generado por $B_1,B_2,B_3$ está formado por elementos de la forma
	\[
		\pm B_1^{k_1}B_2^{k_2}B_3^{k_3}
	\]
	para $k_j\in\{0,1\}$ y luego tiene orden $2^3$. 
	La función 
	$G\to\langle B_1,B_2,B_3\rangle$,
	\[
		x_1\mapsto B_1,\quad
		x_2\mapsto B_2,\quad
		x_3\mapsto B_3 
	\]
	se extiende entonces a un isomorfismo de grupos.
\end{example}

\begin{theorem}[Hurwitz]
	\index{Teorema!de Hurwitz}
	Si existe una identidad 
	\begin{equation}
		\label{eq:Hurwitz}
		(x_1^2+\cdots+x_n^2)(y_1^2+\cdots+y_n^2)=z_1^2+\cdots+z_n^2,
	\end{equation}
	donde los $x_j$ y los $y_j$ son números complejos y las $z_k$ son funciones
	bilineales en los $x_j$ y los $y_j$, entonces $n\in\{1,2,4,8\}$.
\end{theorem}

\begin{proof}
	Sin pérdida de generalidad podemos suponer que $n>2$.  Para cada
	$i\in\{1,\dots,n\}$ escribimos 
	\[
		z_i=\sum_{k=1}^n a_{ik}(x_1,\dots,x_n)y_k,
	\]
	donde las $a_{ik}$ son funciones lineales. Entonces
	\[
		z_i^2=\sum_{k,l=1}^na_{ik}(x_1,\dots,x_n)a_{il}(x_1,\dots,x_n)y_ky_l
	\]
	para todo $i\in\{1,\dots,n\}$.  Si usamos estas expresiones para los $z_i$
	en~\eqref{eq:Hurwitz} y comparamos coeficientes obtenemos
	\begin{equation}
		\label{eq:delta}
		\sum_{i=1}^n a_{ik}(x_1,\dots,x_n)a_{il}(x_1,\dots,x_n)=\delta_{k,l}(x_1^2+\cdots+x_n^2),
	\end{equation}
	donde $\delta_{k,l}$ es la función delta de Kronecker. Escribamos esta
	última expresión matricialmente. Para eso, 
	sea $A$ la matriz de $n\times n$ dada por
	\[
	A_{ij}=a_{ij}(x_1,\dots,x_n).
	\]
	Entonces 
	\begin{equation}
		\label{eq:AAT}
		AA^T=(x_1^2+\cdots+x_n^2)I,
	\end{equation}
	donde $I$ es la matriz identidad de $n\times n$ pues 
	\[
		(AA^T)_{kl}=\sum_{i=1}^na_{ki}(x_1,\dots,x_n)a_{li}(x_1,\dots,x_n)=\delta_{kl}(x_1^2+\cdots+x_n^2)
	\]
	por la fórmula~\eqref{eq:delta}. Como cada $a_{ki}(x_1,\dots,x_n)$ es una función lineal, 
	existen escalares $\alpha_{ij1},\dots,a_{ijn}\in\C$ tales que
	\[
		a_{ij}(x_1,\dots,x_n)=\alpha_{ij1}x_1+\cdots+\alpha_{ijn}x_n.
	\]
	Podemos escribir entonces 
	\[
		A=A_1x_1+\cdots+A_nx_n,
	\]
	donde cada $A_k$ es la matriz $(A_k)_{ij}=\alpha_{ijk}$. La fórmula~\eqref{eq:AAT} queda entonces
	\[
		\sum_{i=1}^n\sum_{j=1}^nA_iA_j^Tx_ix_j=(x_1^2+\cdots+x_n^2)I.
	\]
	Luego 
	\begin{equation}
		\label{eq:condiciones}
		A_iA_j^T+A_jA_i^T=0\quad i\ne j,\quad
		A_iA_i^T=I.
	\end{equation}
	Queremos entonces encontrar $n$ matrices complejas de $n\times n$ que
	cumplan las condiciones~\eqref{eq:condiciones}. Para cada $i\in\{1,\dots,n\}$ sea 
	$B_i=A_n^TA_i$. Entonces~\eqref{eq:condiciones} queda ahora 
	\[
		B_iB_j^T+B_jB_i^T=0\quad i\ne j,\quad
		B_iB_i^T=I,\quad
		B_n=I.
	\]
	Al poner $j=n$ en la primera ecuación obtenemos que $B_i=-B_i^T$ vale para
	todo $i\in\{1,\dots,n-1\}$ y luego $B_iB_j=-B_jB_i$ para todo
	$i,j\in\{1,\dots,n-1\}$. 

	Afirmamos que $n$ es par. De hecho, al calcular el determinante de
	$B_iB_j=-B_jB_i$ obtenemos $\det(B_iB_j)=(-1)^n\det(B_jB_i)$ y luego $n$ es
	par pues $1=(-1)^n$.

	
	Si existe una solución a~\ref{eq:Hurwitz}, entonces se tiene una
	representación fiel del grupo $G$ del lema anterior (y en particular, este
	grupo existe). Como $G/[G,G]$ tiene orden $2^{n-1}$, $G$ admite $2^{n-1}$
	representaciones de grado uno. Como $G$ tiene $2^{n-1}+2$ clases de
	conjugación, $G$ admite dos representaciones irreducibles de grados $f_1>1$
	y $f_2>1$ respectivamente. Además 
	\[
		2^n=|G|=\underbrace{1+\cdots+1}_{2^{n-1}}+f_1^2+f_2^2=2^{n-1}+f_1^2+f_2^2
	\]
	implica que $f_1=f_2=2^{\frac{n-2}{2}}>1$. Nuestra representación de
	$G$ no contiene subrepresentaciones de grado uno (pues en esta
	representación $\epsilon$ debería representarse como $-I$ y en las
	representaciones de grado uno $\epsilon$ es trivial porque
	$\epsilon\in[G,G]$). Luego $2^{\frac{n-2}{2}}$ divide a $n$. Al escribir
	$n=2^ab$ con $a\geq 1$ y $b$ un número impar, tenemos que
	$\frac{n-2}{2}\leq a$ y luego $n\in\{4,8\}$ pues $2^a\leq n\leq 2a+2$. 
\end{proof}

Veamos una aplicación. 

\begin{theorem}
	Sea $V$ un espacio vectorial real con producto interno tal que $\dim
	V=n\geq3$. Si existe una función $V\times V\to\R$, $(v,w)\mapsto v\times
	w$, bilineal tal que $v\times w$ es ortogonal a $v$ y a $w$ y además 
	\[
		\|v\times w\|^2=\|v\|^2\|w\|^2-\langle v,w\rangle^2,
	\]
	donde $\|v\|^2=\langle v,v\rangle$, entonces $n\in\{3,7\}$. 
\end{theorem}

\begin{proof}
	Sea $W=V\oplus\R$ con el producto escalar 
	\[
		\langle (v_1,r_1),(v_2,r_2)\rangle = \langle v_1,v_2\rangle+r_1r_2.
	\]
	Primero observemos que
	\begin{align*}
		\langle v_1\times &v_2+r_1v_2+r_2v_1,v_1\times v_2+r_1v_2+r_2v_1\rangle\\
		&=\|v_1\times v_2\|^2+r_1^2\|v_2\|^2+2r_1r_2\langle v_1,v_2\rangle+r_2^2\|v_1\|^2.
	\end{align*}
	Luego 
	\begin{align*}
		(\|v_1\|^2+r_1^2)&(\|v_2\|^2+r_2)\\
		&= \|v_1\|^2\|v_2\|^2+r_2^2\|v_1\|^2+r_1^2\|v_2\|^2+r_1^2r_2^2\\
		&=\|v_1\times v_2+r_1v_1+r_2v_2\|^2-2r_1r_2\langle v_1,v_2\rangle+\langle v_1,v_2\rangle^2+r_1^2r_2^2\\
		&=\|v_1\times v_2+r_1v_1+r_2v_2\|^2+(\langle v_1,v_2\rangle-r_1r_2)^2\\
		&=z_1^2+\cdots+z_{n+1}^2,
	\end{align*}
	donde las $z_k$ son funciones bilineales en $(v_1,r_1)$ y $(v_2,r_2)$. El teorema de Hurwitz implica
	entonces que $n+1\in\{4,8\}$ y luego $n\in\{3,7\}$.
\end{proof}

Si en el teorema anterior $\dim V=3$, el resultado nos da el producto vectorial usual. Si en cambio $\dim V=7$, 
sea 
\[
	W=\{(v,k,w):v,w\in V,k\in\R\}
\]
con el producto interno dado por
\[
	\langle (v_1,k_1,w_1),(v_2,k_2,w_2)\rangle = \langle v_1,v_2\rangle+k_1k_2+\langle w_1,w_2\rangle.
\]
Queda como ejercicio demostrar que la operación
\begin{multline*}
	(v_1,k_1,w_1)\times (v_2,k_2,w_2)\\
	=(k_1w_2-k_2w_1+v_1\times v_2-w_1\times w_2,
	\\-\langle v_1,w_2\rangle+\langle v_2,w_1\rangle, 
	k_2v_1-k_1v_2-v_1\times w_2-w_1\times v_2)
\end{multline*}
cumple las propiedades del teorema.

\subsection*{Poincar\'e--Birkhoff--Witt theorem}

\subsection*{Weyl's theorem}

\subsection*{Irreducible representations of $U_q(\sl(2,\C))$}

Let $q\in\C\setminus\{0,1,-1\}$. 
Let $U_q(\sl(2))$ be the (complex) algebra generated by 
variables $E$, $F$, $K$ and $K^{-1}$ with relations
\begin{align*}
    &KK^{-1}=K^{-1}K=1,
    &&
    KEK^{-1}=q^2E,\\
    &
    KFK^{-1}=q^{-2}F,
    &&
    [E,F]=\frac{1}{(q-q^{-1})}(K-K^{-1}).
\end{align*}

Study the representation theory of $U_q(\sl(2))$. This splits into
two cases, depending on whether $q$ is a root of one or not. 
Finite-dimensional simple $U_q(\sl(2))$-modules are studied 
in \cite[VI]{MR1321145}. In particular, if 
$q$ is not a root of one, finite-dimensional simple $U_q(\sl(2))$-modules
are classified in \cite[Theorem VI.3.5]{MR1321145}. 

\subsection*{Semisimple modules of $U_q(\sl(2,\C))$}

Prove that if $q$ is not a root of one, any finite-dimensional
$U_q(\sl(2))$-module is semisimple. 
See \cite[Theorem VII.2.2]{MR1321145}. 