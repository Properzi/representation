\chapter{}

\topic{Some theorems of Burnside}

For $n\geq1$ let $\{e_1,\dots,e_n\}$ be the standard basis of $\C^n$.  
The \textbf{natural representation} of $\Sym_n$ is 
$\rho\colon\Sym_n\to\GL_n(C)$, $\sigma\mapsto\rho_{\sigma}$, 
where $\rho_\sigma(e_j)=e_{\sigma(j)}$ for all $j\in\{1,\dots,n\}$. 
The matrix of $\rho_\sigma$ in the standard basis is  
\begin{equation}
    \label{eq:Sn_natural}
    (\rho_\sigma)_{ij}=\begin{cases}
      1 & \text{if $i=\sigma(j)$},\\
      0 & \text{otherwise}.
    \end{cases}
\end{equation}

\begin{lemma}
	\label{lem:permutaciones}
	For $n\geq1$ let $\rho\colon\Sym_n\to\GL_n(C)$ be the natural 
	representation of the symmetric group. 
	If $A\in\C^{n\times n}$ and $\sigma\in\Sym_n$, then
	\[
		A_{ij}=(\rho_{\sigma}A)_{\sigma(i)j}=(A\rho_{\sigma})_{i\sigma^{-1}(j)}
	\]
    for all $i,j\in\{1,\dots,n\}$.
\end{lemma}

\begin{proof}
	With~\eqref{eq:Sn_natural} we compute:
	\[
		(A\rho_{\sigma})_{ij}=\sum_{k=1}^n A_{ik}(\rho_{\sigma})_{kj}=A_{i\sigma(j)},
		\quad
		(\rho_\sigma A)_{ij}=\sum_{k=1}^n (\rho_\sigma)_{ik}A_{kj}=A_{\sigma^{-1}(i)j}.\qedhere
	\]
\end{proof}

\begin{definition}
  \index{Real!character}
  Let $G$ be a finite group. A character $\chi$ of $G$ is said to be
  \textbf{real} if
  $\chi=\overline{\chi}$, that is $\chi(g)\in\R$ for all $g\in G$. 
\end{definition}

\begin{exercise}
	\label{xca:chi_irreducible}
	Let $G$ be a finite group. If $\chi\in\Irr(G)$, then 
	$\overline{\chi}$ is irreducible.
\end{exercise}

\begin{definition}
  \index{Real!conjugacy class}
  Let $G$ be a group. A conjugacy class $C$ of $G$ is said to be
  \textbf{real} if for every $g\in C$ one has $g^{-1}\in C$. 
\end{definition}

We use the following notation: if $G$ is a group and $C=\{xgx^{-1}:x\in G\}$ is a conjugacy class of  
$G$, then $C^{-1}=\{xg^{-1}x^{-1}:x\in G\}$.  

\begin{theorem}[Burnside]
    \index{Burnside's!theorem}
    Let $G$ be a finite group. The number of real conjugacy classes 
    equals the number of real irreducible characters. 
\end{theorem}

\begin{proof}
  Let $C_1,\dots,C_r$ be the conjugacy classes of $G$ and  
  let $\chi_1,\dots,\chi_r$ be the irreducible characters of $G$. 
  Let $\alpha,\beta\in\Sym_r$ be such that $\overline{\chi_i}=\chi_{\alpha(i)}$ and 
  $C_i^{-1}=C_{\beta(i)}$ for all $i\in\{1,\dots,r\}$. Note that $\chi_i$
  is real if and only if $\alpha(i)=i$ and that $C_i$ is real if and only if 
  $\beta(i)=i$. The number $n$ of fixed points of $\alpha$ is equal to the number
  of real irreducible characters of $G$, and the number $m$ of fixed points of $\beta$ is equal
  to the number of real classes. 
  Let $\rho\colon\Sym_r\to\GL_r(\C)$ be the natural representation of $\Sym_r$, with character $\chi_\rho$.
  Then $\chi_\rho(\alpha)=n$ and $\chi_\rho(\beta)=m$. We claim that 
  $\trace\rho_\alpha=\trace\rho_\beta$. 
  Let $X\in\GL(r,\C)$ be the character matrix of $G$. 
  By Lemma~\ref{lem:permutaciones} 
  and the fact that $\overline{\chi(g)}=\chi(g^{-1})$ for all $g\in G$, 
  \[
	\rho_\alpha X=\overline{X}=X\rho_\beta.
  \]
  Since $X$ is invertible, $\rho_{\alpha}=X\rho_{\beta}X^{-1}$. Thus 
  \[
    n=\chi_{\rho}(\alpha)=\trace\rho_{\alpha}=\trace\rho_{\beta}=\chi_{\rho}(\beta)=m.\qedhere
  \]
\end{proof}

\begin{corollary}
  \label{corollary:|G|impar}
  Let $G$ be a finite group. Then $|G|$ is odd if and only if
  the only real $\chi\in\Irr(G)$ is the trivial character. 
\end{corollary}

\begin{proof}
    We first prove $\impliedby$. If $|G|$ is even, there exists 
    $g\in G$ of order two (Cauchy's theorem). The conjugacy class of $g$ 
    is real. 

    We now prove $\implies$. Assume that $G$ has a non-trivial 
    real conjugacy class $C$. Let $g\in C$. We claim that 
    $G$ has an element of even order. Let $h\in G$ be such that
    $hgh^{-1}=g^{-1}$. Then $h^2\in C_G(g)$, as $h^2gh^{-2}=g$. 
    If $h\in\langle h^2\rangle\subseteq C_G(g)$, then $g$ has 
    even order, as $g^{-1}=g$. If $h\not\in\langle h^2\rangle$, then 
    $h^2$ does not generate $\langle h\rangle$. Hence $h$ has even order, 
    as $|h|\ne|h^2|=|h|/\gcd(|h|,2)$, so $\gcd(|h|,2)\ne 1$.  
\end{proof}

\begin{theorem}[Burnside]
  \index{Burnside's!theorem}
  \label{thm:Burnside_mod16}
  Let $G$ be a finite group of odd order 
  with $r$ conjugacy classes. Then
  $r\equiv|G|\bmod{16}$.
\end{theorem}

\begin{proof}
  Since $|G|$ is odd, every non-trivial $\chi\in\Irr(G)$ is not real by
  the previous corollary. The irreducible characters 
  of $G$ are  
  \[
    \chi_1,\chi_2,\overline{\chi_2},\dots,\chi_k,\overline{\chi_k},
    \quad
    r=1+2(k-1),
  \]
  where $\chi_1$ denotes the trivial character. 
  For every $j\in\{2,\dots,k\}$ let $d_j=\chi_j(1)$. 
  Since each $d_j$ divides 
  $|G|$ by Frobenius' theorem and  $|G|$ is odd, 
  every $d_j$ is an odd number, 
  say $d_j=1+2m_j$. Thus  
  \begin{align*}
    |G|&=1+\sum_{j=2}^k 2d_j^2=1+\sum_{j=2}^k2(2m_j+1)^2\\
    &=1+\sum_{j=2}^k2(4m_j^2+4m_j+1)
    =1+2(k-1)+8\sum_{j=2}^km_j(m_j+1).
  \end{align*}
  Hence $|G|\equiv r\bmod{16}$, 
  as $r=1+2k$ and every $m_j(m_j+1)$ is even. 
\end{proof}

\begin{exercise}
    Prove that every group of order 15 is abelian. 
\end{exercise}



\topic{Solvable groups and Burnside's theorem}

\index{Derived series}
For a group $G$ let 
$G^{(0)}=G$ and 
$G^{(i+1)}=[G^{(i)},G^{(i)}]$ for $i\geq0$.
The \textbf{derived series} of $G$ is the sequence
\[
G=G^{(0)}\supseteq G^{(1)}\supseteq G^{(2)}\supseteq\cdots
\]
Each $G^{(i)}$ is a characteristic subgroup of $G$. We say that 
$G$ is \textbf{solvable} if $G^{(n)}=\{1\}$ for some $n$.  

\begin{example}
	Abelian groups are solvable. 
\end{example}

\begin{example}
	The group $\SL_2(3)$ is solvable, as the derived series is 
	\[
	\SL_2(3)\supseteq Q_8\supseteq C_4\supseteq C_2\supseteq \{1\}.
	\]
	Here is the what the computer says:
\begin{lstlisting}
gap> IsSolvable(SL(2,3));
true
gap> List(DerivedSeries(SL(2,3)),StructureDescription);
[ "SL(2,3)", "Q8", "C2", "1" ]
\end{lstlisting}
\end{example}

\begin{example}
	Non-abelian simple groups cannot be solvable. 
\end{example}

\begin{exercise}
	\label{xca:solvable}
	Let $G$ be a group. Prove the following statements:
	\begin{enumerate}
		\item A subgroup $H$ of $G$ is solvable, when $G$ is solvable.
		\item Let $K$ be a normal subgroup of $G$. 
		    Then $G$ is solvable if and only if $K$ and $G/K$ are solvable.
	\end{enumerate}
\end{exercise}

\begin{example}
	For $n\geq5$ the group $\Alt_n$ is not solvable. It follows that 
	$\Sym_n$ is not solvable for $n\geq5$. 
\end{example}

\begin{exercise}
\label{xca:pgroups_solvable}
	Let $p$ be a prime number. Prove that 
	finite $p$-groups are solvable.
\end{exercise}

\begin{theorem}[Burnside]
	\index{Burnside's theorem}
	\label{thm:Burnside_auxiliar}
	Let $G$ be a finite group. If $\phi\colon G\to\GL_n(\C)$ is a representation
	with character $\chi$ and $C$ is a conjugacy class of $G$ such that 
	$\gcd(|C|,n)=1$, then for every $g\in C$ either 
	$\chi(g)=0$ or $\phi_g$ is a scalar matrix. 
\end{theorem}

We need a lemma.

\begin{lemma}
	\label{lem:4Burnside}
	Let $\epsilon_1,\dots,\epsilon_n$ be roots of one such that 
	$(\epsilon_1+\cdots+\epsilon_n)/n\in\A$. Then either 
	$\epsilon_1=\cdots=\epsilon_n$ or 
	$\epsilon_1+\cdots+\epsilon_n=0$.
\end{lemma}

\begin{proof}
	Let $\alpha=(\epsilon_1+\cdots+\epsilon_n)/n$.
	If the $\epsilon_j$s are not all equal, then $\|\alpha\|<1$. Moreover, 
	$\|\beta\|<1$ for every algebraic conjugate $\beta$ of $\alpha$. Since the product 
	of the algebraic conjugates of $\alpha$ is an integer of absolute value 
	$<1$, it follows that it is zero. 
\end{proof}

Now we prove the theorem.

\begin{proof}[Proof of Theorem \ref{thm:Burnside_auxiliar}]
	Let $\epsilon_1,\dots,\epsilon_n$ be the eigenvalues of $\phi_g$. By assumption, 
	$\gcd(|C|,n)=1$, there exist $a,b\in\Z$ such that $a|C|+bn=1$. Since 
	$|C|\chi(g)/n\in\A$, after multiplying by $\chi(g)/n$ we obtain that  
	\[
		a|C|\frac{\chi(g)}{n}+b\chi(g)=\frac{\chi(g)}{n}=\frac{1}{n}(\epsilon_1+\cdots+\epsilon_n)\in\A.
	\]
	The previous lemma implies that there are two cases to consider: 
	either $\epsilon_1=\cdots=\epsilon_n$ or $\epsilon_1+\cdots+\epsilon_n=0$. In the first
	case, since $\phi_g$ is diagonalizable, $\phi_g$ is a scalar matrix. 
	In the second case, $\chi(g)=0$.
\end{proof}

\begin{theorem}[Burnside]
	\index{Burnside's theorem}
	Let $p$ be a prime number. If $G$ is a finite group and 
	$C$ is a conjugacy class of $G$ with $p^k>1$ elements, then $G$ 
	is not simple.
\end{theorem}

\begin{proof}
	Let $g\in C\setminus\{1\}$. Column orthogonality implies that 
	\begin{equation}
	\label{eq:Burnside}
	\begin{aligned}
		0&=\sum_{\chi\in\Irr(G)}\chi(1)\chi(g)\\
		&=\sum_{p\mid\chi(1)}\chi(1)\chi(g)+\sum_{p\nmid\chi(1):\chi\ne\chi_1}\chi(1)\chi(g)+1,
	\end{aligned}
	\end{equation}
	where the one corresponds to the trivial representation of
	$G$.
	
	Look at this equation modulo $p$. If $\chi(g)=0$ for all
	$\chi\in\Irr(G)$
	such that $\chi\ne\chi_1$ and $p\nmid\chi(1)$, then
	\[
	-\frac{1}{p}=\sum\frac{\chi(1)}{p}\chi(g)\in\A\cap\Q=\Z,
	\]
	where the sum is taken over all non-trivial irreducibles
	of $G$ of degree divisible by $p$, a contradiction. Hence 
	there exists an irreducible non-trivial representation 
	$\phi$ with character $\chi$ such that $p$ does not divide
	$\chi(1)$ and $\chi(g)\ne0$. By the previous theorem, 
	$\phi_g$ is a scalar matrix. If $\phi$ is faithful, then 
	$g$ is a non-trivial central element, a contradiction since 
    $|C|>1$. If $\phi$ is not faithful, then 
    $G$ is not simple (because 
	$\ker\phi$ is a non-trivial proper normal subgroup of $G$). 
\end{proof}

\begin{theorem}[Burnside]
  \index{Burnside's theorem}
  Let $p$ and $q$ be prime numbers. If $G$ has order $p^aq^b$, then $G$ is solvable.
\end{theorem}

\begin{proof}
	If $G$ is abelian, then it is solvable.
	Suppose now $G$ is non-abelian.
	Let us assume that the theorem is not true. Let $G$ be a group
	of minimal order $p^aq^b$
	that is not solvable. Since $|G|$ is minimal, $G$ is a non-abelian simple group.
	By the previous theorem, 
	$G$ has no conjugacy classes of size $p^k$ nor 
	conjugacy classes of size $q^l$ with $k,l\geq1$. The size
	of every conjugacy class of $G$ is one or divisible by $pq$. 
	Note that, since $G$ is a non-abelian simple group,
	the center of $G$ is trivial.
	Thus there is only one conjugacy class of size one.
	By the class
	equation,
	\[
		|G|=1+\sum_{C:|C|>1}|C|\equiv 1 \bmod pq,
	\]
	where the sum is taken over all conjugacy classes 
	with more than one element, a contradiction.
\end{proof}

Some generalizations of Burnside's theorem. 

\begin{theorem}[Kegel--Wielandt]
    \index{Kegel--Wielandt's theorem}
    \label{thm:KegelWielandt}
    If $G$ is a finite group and there are nilpotent subgroups 
    $A$ and $B$ of $G$ such that 
    $G=AB$, then $G$ is solvable.
\end{theorem}

See~\cite[Theorem 2.4.3]{MR1211633} for the proof.


Another generalization of Burnside's theorem
is based on \emph{word maps}. A word map
of a group $G$ is a map 
\[
G^k\to G,\quad 
(x_1,\dots,x_k)\mapsto w(x_1,\dots,x_k)
\]
for some word $w(x_1,\dots,x_k)$ of the free group $F_k$ of rank $k$. 
Some word maps are surjective in certain families of groups. For example, 
Ore's conjecture is precisely the surjectivity of the word map
$(x,y)\mapsto [x,y]=xyx^{-1}y^{-1}$ in every finite non-abelian simple 
group. 

\begin{theorem}[Guralnick--Liebeck--O'Brien--Shalev--Tiep]
    Let $a,b\geq0$, $p$ and $q$ be prime numbers and $N=p^aq^b$. The map 
    $(x,y)\mapsto x^Ny^N$ is surjective in every finite simple group. 
\end{theorem}

The proof appears in~\cite{MR3827208}. 

The theorem implies Burnside's theorem. Let $G$ be a group of order
$N=p^aq^b$. Assume that $G$ is not solvable. 
Fix a composition series of $G$. There is a non-abelian factor $S$ 
of order that divides $N$. Since 
$S$ is simple non-abelian and $s^N=1$, it follows that the word map
$(x,y)\mapsto x^Ny^N$ has trivial image in $S$, a contradiction 
to the theorem. 

\topic{Feit--Thompson theorem}

\begin{theorem}[Feit--Thompson]
    \index{Feit--Thompson theorem}
    Groups of odd order are solvable. 
\end{theorem}

The proof of Feit--Thompson theorem is extremely hard. 
It occupies a full volume of the 
\emph{Pacific Journal of Mathematics}~\cite{MR166261}. 
A formal verification of the proof 
(based on the computer software Coq) 
was announced in~\cite{MR3111271}.  

Back in the day it was believed that if a certain divisibility 
conjecture is true, 
the proof of Feit--Thompson theorem 
could be simplified. 

\begin{conjecture}[Feit--Thompson]
\index{Feit--Thompson conjecture}
    There are no prime numbers $p$ and $q$ such that
    $\frac {p^{q}-1}{p-1}$ divides $\frac{q^{p} - 1}{q - 1}$. 
\end{conjecture}

The conjecture remains open. However, now we know that 
proving the conjecture will not simplify further
the proof of Feit--Thompson theorem. 

In 2012 Le proved that the conjecture is true for $q=3$, see 
\cite{MR2900154}. 


In~\cite{MR297686} 
Stephens proved that a certain stronger version of the conjecture 
does not hold, as the integers 
$\frac {p^{q}-1}{p-1}$ and $\frac{q^{p} - 1}{q - 1}$ 
could have common factors. In fact, if $p=17$ and $q=3313$, 
then 
\[
\gcd\left(\frac {p^{q}-1}{p-1},\frac{q^{p} - 1}{q - 1}\right)=112643.
\]
Nowadays we can check this easily in almost every desktop computer:
\begin{lstlisting}
gap> Gcd((17^3313-1)/16,(3313^17-1)/3312);
112643
\end{lstlisting}
No other counterexamples have been found to Stephen’s stronger version of the conjecture.

\section{Hurwitz' theorem}

\subsection*{Hurwitz's theorem}
\label{Hurwitz}

\index{Fibonacci identity}
\index{Euler identity}
\index{Hamilton identity}
We know that $x^2y^2=(xy)^2$ holds for all $x,y\in\C$. Fibonnaci
found the identity
\[
	(x_1^2+x_2^2)(y_1^2+y_2^2)=(x_1y_1-x_2y_2)^2+(x_1y_2-x_2y_1)^2.
\]
Euler and Hamilton, independently, found 
a similar identity:
\[
	(x_1^2+x_2^2+x_3^2+x_4^2)(y_1^2+y_2^2+y_3^2+y_4^2)=z_1^2+z_2^2+z_3^2+z_4^2,
\]
where
\begin{equation}
\label{eq:Hamilton}
\begin{aligned}
	 z_1&=x_1y_1-x_2y_2-x_3y_3-x_4y_4,\\
	 z_2&=x_1y_2+x_2y_1-x_3y_3-x_4y_4,\\
	 z_3&=x_1y_3+x_3y_1-x_2y_4+x_4y_2,\\ 
	 z_4&=-x_1y_4+x_4y_1x_2y_3-x_3y_2.
\end{aligned}
\end{equation}
Cayley found a similar identity for sums of eight squares. 
Are there other identities of this type? Hurwitz' proved that this is not the case. 
We present Eckmann's proof of Hurwitz' theorem. The proof uses character theory.

\begin{lemma}
\label{lem:grupo}
	Let $n>2$ be an even number. If 
	there exists a group $G$ with generators
	$\epsilon,x_1,\dots,x_{n-1}$ and relations 
	\[
		x_1^2=\cdots=x_{n-1}^2=\epsilon\ne1,\quad
		\epsilon^2=1,\quad
		[x_i,x_j]=\epsilon\quad\text{if}\quad i\ne j,
	\]
	then the following statements hold:
	\begin{enumerate}
		\item $|G|=2^n$.
		\item $[G,G]=\{1,\epsilon\}$. In particular, $G$ 
		    has exactly $2^{n-1}$ degree-one representations. 
		\item If $g\not\in Z(G)$, then the conjugacy class of $g$ is $\{g,\epsilon g\}$.
		\item $Z(G)=\{1,\epsilon,x_1\cdots x_{n-1},\epsilon x_1\cdots x_{n-1}\}$. 
		\item $G$ has $2^{n-1}+2$ conjugacy classes.
		\item $G$ has two irreducible representations of degree $2^{\frac{n-2}{2}}>1$. 
	\end{enumerate}
\end{lemma}

\begin{proof}
    Let us prove 1) and 2). Note that $\epsilon\in Z(G)$, as
    $\epsilon=x_i^2$ for all 
	$i\in\{1,\dots,n-1\}$. Since $n-1>2$, $[x_1,x_2]=\epsilon$. Hence 
	$\epsilon\in [G,G]$. Moreover, $G/\langle\epsilon\rangle$ is abelian. Thus 
	$[G,G]=\langle \epsilon\rangle$. Since $G/[G,G]$ is elementary 
	abelian of order 
	$2^{n-1}$, it follows that 
	$|G|=2^n$. 

	We now prove 3). Let $g\in G\setminus Z(G)$ and 
	$x\in G$ be such that $[x,g]\ne 1$. Then $[x,g]=\epsilon$ and 
	$xgx^{-1}=\epsilon g$. 

	To prove 4) let $g\in G$. Write
	\[
		g=\epsilon^{a_0}x_1^{a_1}\cdots x_{n-1}^{a_{n-1}},
	\]
	where $a_j\in\{0,1\}$ for all $j\in\{1,\dots,n-1\}$. 
	If $g\in Z(G)$, then $gx_i=x_ig$ for all $i\in\{1,\dots,n-1\}$. Hence 
	$g\in Z(G)$ if and only if 
	\[
		\epsilon^{a_0}x_1^{a_1}\cdots x_{n-1}^{a_{n-1}}=x_i(\epsilon^{a_0}x_1^{a_1}\cdots x_{n-1}^{a_{n-1}})x_i^{-1}.
	\]
	Since $x_ix_j^{a_j}x_i=\epsilon^{a_j}x_j^{a_j}$ 
	whenever $i\ne j$ y $\epsilon\in Z(G)$, the elmeent $g$ is 
	central if and only if 
	\[
		\sum_{\substack{j=1\\j\ne i}}^{n-1}a_j\equiv 0\bmod 2
	\]
	for all $i\in\{1,\dots,n-1\}$. In particular, 
	\[
	\sum_{j\ne i}a_j\equiv \sum_{j\ne k}a_j
	\]
	for all $k\ne i$. Therefore $a_i\equiv a_k\bmod 2$ for all 
	$i,k\in\{1,\dots,n-1\}$. Thus $a_1=\cdots=a_{n-1}$ and  
	$Z(G)=\{1,x_1\cdots x_{n-1},\epsilon,\epsilon x_1\cdots
	x_{n-1}\}$. 
	
    To prove 5) we use the class equation:
    \[
		2^n=|G|=|Z(G)|+\sum_{i=1}^N2=4+2N. 
	\]
	It follows that $G$ has $N+4=2^{n-1}+2$ conjugacy classes.
	
	Finally we prove 6). 
	Since $G$ 
	has exactly $2^{n-1}$ degree-one representations (because 
	$|G/[G,G]|=2^{n-1}$) and 
	has $2^{n-1}+2$ conjugacy classes, 
	it follows from 
	\[
		2^n=|G|=\underbrace{1+\cdots+1}_{2^{n-1}}+f_1^2+f_2^2=2^{n-1}+f_1^2+f_2^2,
	\]
	that $G$ has two irreducible representations
	of degrees $f_1=f_2=2^{\frac{n-2}{2}}>1$. 
\end{proof}

\begin{example}
	The formulas~\eqref{eq:Hamilton} give a representation for the
	group $G$ of the previous lemma. Write each $z_i$ as 
	$z_i=\sum_{k=1}^4a_{1k}(x_1,\dots,x_4)y_k$. Let $A$ be a matrix
	such that 
	$A_{ij}=a_{ij}(x_1,\dots,x_4)$, that is 
	\[
		A=\begin{pmatrix}
			x_1 & -x_2 & -x_3 & -x_4\\
			x_2 & x_1 & -x_4 & x_3\\
			x_3 & x_4 & x_1 & -x_2\\
			x_4 & -x_3 & x_2 & x_1
		\end{pmatrix}
	\]
	The matrix $A$ can be written as $A=A_1x_1+A_2x_2+A_3x_3+A_4x_4$, where
	\begin{align*}
		&A_1=\begin{pmatrix}
		1\\
		&1\\
		&&1\\
		&&&1\\
		\end{pmatrix},
		&&
		A_2=\begin{pmatrix}
			& -1\\
			1 \\
			&&&-1\\
			&&1
		\end{pmatrix},
		&&
		A_3=\begin{pmatrix}
			&& -1 \\
			&&&1 & \\
			1\\
			&-1
		  \end{pmatrix},
		  &&
		  A_4=\begin{pmatrix}
			&&&-1\\
			&&-1\\
			&1\\
			1
		\end{pmatrix}.
	\end{align*}
	For $i\in\{1,\dots,4\}$ let $B_i=A_iA_4^T$. Then
	$B_i=-B_i^T$ and  $B_i^2=-I$ 
	for all $i\in\{1,2,3\}$. Moreover, $B_iB_j=-B_jB_i$ for all $i,j\in\{1,2,3\}$ and
	$i\ne j$.  
	The group generated by $\{B_1,B_2,B_3\}$ has $2^3$ element, all of them
	of the form
	\[
		\pm B_1^{k_1}B_2^{k_2}B_3^{k_3}
	\]
	for $k_j\in\{0,1\}$. 
    The map 
	$G\to\langle B_1,B_2,B_3\rangle$,
	\[
		x_1\mapsto B_1,\quad
		x_2\mapsto B_2,\quad
		x_3\mapsto B_3 
	\]
	extends to a group isomomorphism. 
\end{example}

\begin{theorem}[Hurwitz]
	\index{Hurwitz' theorem}
	If there is an identity of the form 
	\begin{equation}
		\label{eq:Hurwitz}
		(x_1^2+\cdots+x_n^2)(y_1^2+\cdots+y_n^2)=z_1^2+\cdots+z_n^2,
	\end{equation}
	where the $x_j$'s and the $y_j$'s are real (or complex) numbers and
	each $z_k$ is a bilinear function in the $x_j$'s and the $y_j$'s, then 
	$n\in\{1,2,4,8\}$.
\end{theorem}

\begin{proof}
    We work over complex numbers. 
	Without loss of generality we may assume that $n>2$.  For 
	$i\in\{1,\dots,n\}$ let  
	\[
		z_i=\sum_{k=1}^n a_{ik}(x_1,\dots,x_n)y_k,
	\]
	where the $a_{ik}$'s are linear functions. Then 
	\[
		z_i^2=\sum_{k,l=1}^na_{ik}(x_1,\dots,x_n)a_{il}(x_1,\dots,x_n)y_ky_l
	\]
	for all $i\in\{1,\dots,n\}$. Using these expressions for each $z_i$
	in~\eqref{eq:Hurwitz} and comparing coefficients, 
	\begin{equation}
		\label{eq:delta}
		\sum_{i=1}^n a_{ik}(x_1,\dots,x_n)a_{il}(x_1,\dots,x_n)=\delta_{k,l}(x_1^2+\cdots+x_n^2),
	\end{equation}
	where $\delta_{k,l}$ is the usual Kronecker's map. Let 
	$A$ be the $n\times n$ matrix given by 
	\[
	A_{ij}=a_{ij}(x_1,\dots,x_n).
	\]
	Then 
	\begin{equation}
		\label{eq:AAT}
		AA^T=(x_1^2+\cdots+x_n^2)I,
	\end{equation}
	where $I$ denotes the $n\times n$ identity matrix, 
	as 
	\[
		(AA^T)_{kl}=\sum_{i=1}^na_{ki}(x_1,\dots,x_n)a_{li}(x_1,\dots,x_n)=\delta_{kl}(x_1^2+\cdots+x_n^2)
	\]
	by~\eqref{eq:delta}. Since each $a_{ki}(x_1,\dots,x_n)$ is a linear function, 
	there exist $\alpha_{ij1},\dots,a_{ijn}\in\C$ such that 
	\[
		a_{ij}(x_1,\dots,x_n)=\alpha_{ij1}x_1+\cdots+\alpha_{ijn}x_n.
	\]
	Write 
	\[
		A=A_1x_1+\cdots+A_nx_n,
	\]
	where each $A_k$ is the matrix $(A_k)_{ij}=\alpha_{ijk}$. 
	The formula~\eqref{eq:AAT} becomes
	\[
		\sum_{i=1}^n\sum_{j=1}^nA_iA_j^Tx_ix_j=(x_1^2+\cdots+x_n^2)I.
	\]
	Thus 
	\begin{equation}
		\label{eq:condiciones}
		A_iA_j^T+A_jA_i^T=0\quad i\ne j,\quad
		A_iA_i^T=I.
	\end{equation}
	We need $n$ complex square matrices of size $n\times n$
	satisfying~\eqref{eq:condiciones}. For $i\in\{1,\dots,n\}$ let  
	$B_i=A_n^TA_i$. Then~\eqref{eq:condiciones} turn into  
	\[
		B_iB_j^T+B_jB_i^T=0\quad i\ne j,\quad
		B_iB_i^T=I,\quad
		B_n=I.
	\]
	Set $j=n$ in the first family of equations to obtain $B_i=-B_i^T$ for all 
	$i\in\{1,\dots,n-1\}$. It follows that 
	\begin{equation}
	\label{eq:representation}
	\begin{aligned}
	    &B_i^2=-I && \text{for all $i\in\{1,\dots,n-1\}$},\\
	    &[B_i,B_j]=-I && \text{for all $i,j\in\{1,\dots,n-1\}$.}
	\end{aligned}
	\end{equation}
    
    \begin{claim}
        $n$ is even. 
    \end{claim}
    
	Computing the determinant of 
	$B_iB_j=-B_jB_i$ we obtain that 
	\[
	1=\det(B_iB_j)=(-1)^n\det(B_jB_i)=(-1)^n.
	\]
	Hence $n$ is even. 

	\begin{claim}
	    The group 
	    $G$ of the lemma admits a faithful
	    representation $\rho\colon G\to\GL_n(\C)$. 
	\end{claim}
	
	By \eqref{eq:representation}, there is a well-defined 
	injective group homomorphism $\rho$ such that 
	$x_i\mapsto B_i$ for all $i\in\{1,\dots,n-1\}$ and 
	$\epsilon\mapsto -I$. 
	
	\begin{claim}
	    $2^{\frac{n-2}{2}}$ divides $n$.
	\end{claim}
	
	Since $\epsilon\in[G,G]$ by Lemma \ref{lem:grupo}, 
	every one-dimensional representation satisfies $\epsilon\mapsto 1$.
	This implies that $\rho$ cannot have degree-one sub representations. 
	In fact, 
	if $W=\langle w\rangle$ is $G$-invariant subspace of $\C^n$, 
	then $\psi=\rho|_W\colon G\to\GL(W)\simeq\C^\times$ 
	is a representation. In particular, 
	\[
	-w=-Iw=\psi_{\epsilon}(w)=\psi_{[x_i,x_j]}(w)
	=\psi_{x_i}\psi_{x_j}\psi_{{x_i}}^{-1}\psi_{{x_j}}^{-1}(w)=w, 
	\]
	a contradiction. 
	
	This means that the $\C[G]$-module $\C^n$ 
	decomposes as $\C^n\simeq aS\oplus bT$,
	where $a$ and $b$ are integers and 
	$S$ and $T$ are simple $\C[G]$-modules of dimension
	$2^{\frac{n-2}{2}}$. In particular, 
	\[
	n=\dim V=\dim(aS\oplus bT)=(a+b)2^{\frac{n-2}{2}}.
	\]
	
	To finish the proof of the theorem write $n=2^ab$ 
	for $a\geq1$ and $b$ an odd integer. 
	Since $\frac{n-2}{2}$ divides $n$, 
	\[
	2^{\frac{n}{2}-1}=2^{\frac{n-2}{2}}\leq n=2^ab. 
	\]
	Thus $\frac{n}{2}-1\leq a$ and hence $2^a\leq n\leq 2(a+1)$. 
	It follows that $n\in\{4,8\}$.  
\end{proof}

We now present an application, see
\cite{MR1534187} for more information. 

\begin{theorem}
	Let $V$ be a real vector space (with an inner product) 
	such that $\dim
	V=n\geq3$. If there exists a bilinear function 
	$V\times V\to\R$, $(v,w)\mapsto v\times
	w$, such that $v\times w$ is orthogonal both 
	to $v$ and $w$ and 
	\[
		\|v\times w\|^2=\|v\|^2\|w\|^2-\langle v,w\rangle^2,
	\]
	where $\|v\|^2=\langle v,v\rangle$, then $n\in\{3,7\}$. 
\end{theorem}

\begin{proof}
	Let $W=V\oplus\R$ with the inner product  
	\[
		\langle (v_1,r_1),(v_2,r_2)\rangle = \langle v_1,v_2\rangle+r_1r_2.
	\]
	Note that
	\begin{align*}
		\langle v_1\times &v_2+r_1v_2+r_2v_1,v_1\times v_2+r_1v_2+r_2v_1\rangle\\
		&=\|v_1\times v_2\|^2+r_1^2\|v_2\|^2+2r_1r_2\langle v_1,v_2\rangle+r_2^2\|v_1\|^2.
	\end{align*}
	Thus  
	\begin{align*}
		(\|v_1\|^2+r_1^2)&(\|v_2\|^2+r_2)\\
		&= \|v_1\|^2\|v_2\|^2+r_2^2\|v_1\|^2+r_1^2\|v_2\|^2+r_1^2r_2^2\\
		&=\|v_1\times v_2+r_1v_1+r_2v_2\|^2-2r_1r_2\langle v_1,v_2\rangle+\langle v_1,v_2\rangle^2+r_1^2r_2^2\\
		&=\|v_1\times v_2+r_1v_1+r_2v_2\|^2+(\langle v_1,v_2\rangle-r_1r_2)^2\\
		&=z_1^2+\cdots+z_{n+1}^2,
	\end{align*}
	where the $z_k$'s are bilinear functions in $(v_1,r_1)$ and $(v_2,r_2)$. 
	By Hurwitz' theorem, 
	$n+1\in\{4,8\}$. Hence $n\in\{3,7\}$.
\end{proof}

In the theorem, if $\dim V=3$, we obtain the usual cross product. 
If $\dim V=7$, let 
\[
	W=\{(v,k,w):v,w\in V,k\in\R\}
\]
with the inner product 
\[
	\langle (v_1,k_1,w_1),(v_2,k_2,w_2)\rangle = \langle v_1,v_2\rangle+k_1k_2+\langle w_1,w_2\rangle.
\]
It is an exercise to show that 
\begin{multline*}
	(v_1,k_1,w_1)\times (v_2,k_2,w_2)\\
	=(k_1w_2-k_2w_1+v_1\times v_2-w_1\times w_2,
	\\-\langle v_1,w_2\rangle+\langle v_2,w_1\rangle, 
	k_2v_1-k_1v_2-v_1\times w_2-w_1\times v_2)
\end{multline*}
satisfies the properties of the theorem. 
