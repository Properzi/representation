\section{Lecture: Week 10}

\subsection{Some theorems of Burnside}

For $n\geq1$ let $\{e_1,\dots,e_n\}$ be the standard basis of $\C^n$.  
The \emph{natural representation} of $\Sym_n$ is 
$\rho\colon\Sym_n\to\GL_n(\C)$, $\sigma\mapsto\rho_{\sigma}$, 
where $\rho_\sigma(e_j)=e_{\sigma(j)}$ for all $j\in\{1,\dots,n\}$. 
The matrix of $\rho_\sigma$ in the standard basis is  
\begin{equation}
    \label{eq:Sn_natural}
    (\rho_\sigma)_{ij}=\begin{cases}
      1 & \text{if $i=\sigma(j)$},\\
      0 & \text{otherwise}.
    \end{cases}
\end{equation}

\begin{lemma}
	\label{lem:permutaciones}
	For $n\geq1$ let $\rho\colon\Sym_n\to\GL_n(C)$ be the natural 
	representation of the symmetric group. 
	If $A\in\C^{n\times n}$ and $\sigma\in\Sym_n$, then
	\[
		A_{ij}=(\rho_{\sigma}A)_{\sigma(i)j}=(A\rho_{\sigma})_{i\sigma^{-1}(j)}
	\]
    for all $i,j\in\{1,\dots,n\}$.
\end{lemma}

\begin{proof}
	With~\eqref{eq:Sn_natural} we compute:
	\[
		(A\rho_{\sigma})_{ij}=\sum_{k=1}^n A_{ik}(\rho_{\sigma})_{kj}=A_{i\sigma(j)},
		\quad
		(\rho_\sigma A)_{ij}=\sum_{k=1}^n (\rho_\sigma)_{ik}A_{kj}=A_{\sigma^{-1}(i)j}.\qedhere
	\]
\end{proof}

\begin{definition}
  \index{Real!character}
  Let $G$ be a finite group. A character $\chi$ of $G$ is said to be
  \emph{real} if
  $\chi=\overline{\chi}$, that is $\chi(g)\in\R$ for all $g\in G$. 
\end{definition}

\begin{exercise}
	\label{xca:chi_irreducible}
	Let $G$ be a finite group. If $\chi\in\Irr(G)$, then 
	$\overline{\chi}$ is irreducible.
\end{exercise}

\begin{definition}
  \index{Real!conjugacy class}
  Let $G$ be a group. A conjugacy class $C$ of $G$ is said to be
  \emph{real} if for every $g\in C$ one has $g^{-1}\in C$. 
\end{definition}

We use the following notation: if $G$ is a group and $C=\{xgx^{-1}:x\in G\}$ is a conjugacy class of  
$G$, then $C^{-1}=\{xg^{-1}x^{-1}:x\in G\}$.  

\begin{theorem}[Burnside]
    \index{Burnside's!theorem}
    Let $G$ be a finite group. The number of real conjugacy classes 
    equals the number of real irreducible characters. 
\end{theorem}

\begin{proof}
  Let $C_1,\dots,C_r$ be the conjugacy classes of $G$ and  
  let $\chi_1,\dots,\chi_r$ be the irreducible characters of $G$. 
  Let $\alpha,\beta\in\Sym_r$ be such that $\overline{\chi_i}=\chi_{\alpha(i)}$ and 
  $C_i^{-1}=C_{\beta(i)}$ for all $i\in\{1,\dots,r\}$. Note that $\chi_i$
  is real if and only if $\alpha(i)=i$ and that $C_i$ is real if and only if 
  $\beta(i)=i$. The number $n$ of fixed points of $\alpha$ is equal to the number
  of real irreducible characters of $G$, and the number $m$ of fixed points of $\beta$ is equal
  to the number of real classes. 
  Let $\rho\colon\Sym_r\to\GL_r(\C)$ be the natural representation of $\Sym_r$, with character $\chi_\rho$.
  Then $\chi_\rho(\alpha)=n$ and $\chi_\rho(\beta)=m$. We claim that 
  $\trace\rho_\alpha=\trace\rho_\beta$. 
  Let $X\in\GL(r,\C)$ be the character matrix of $G$. 
  By Lemma~\ref{lem:permutaciones} 
  and the fact that $\overline{\chi(g)}=\chi(g^{-1})$ for all $g\in G$, 
  \[
	\rho_\alpha X=\overline{X}=X\rho_\beta.
  \]
  Since $X$ is invertible, $\rho_{\alpha}=X\rho_{\beta}X^{-1}$. Thus 
  \[
    n=\chi_{\rho}(\alpha)=\trace\rho_{\alpha}=\trace\rho_{\beta}=\chi_{\rho}(\beta)=m.\qedhere
  \]
\end{proof}

\begin{corollary}
  \label{corollary:|G|impar}
  Let $G$ be a finite group. Then $|G|$ is odd if and only if
  the only real $\chi\in\Irr(G)$ is the trivial character. 
\end{corollary}

\begin{proof}
    We first prove $\impliedby$. If $|G|$ is even, there exists 
    $g\in G$ of order two (Cauchy's theorem). The conjugacy class of $g$ 
    is real. 

    We now prove $\implies$. Assume that $G$ has a non-trivial 
    real conjugacy class $C$. Let $g\in C$. We claim that 
    $G$ has an element of even order. Let $h\in G$ be such that
    $hgh^{-1}=g^{-1}$. Then $h^2\in C_G(g)$, as $h^2gh^{-2}=g$. 
    If $h\in\langle h^2\rangle\subseteq C_G(g)$, then $g$ has 
    even order, as $g^{-1}=g$. If $h\not\in\langle h^2\rangle$, then 
    $h^2$ does not generate $\langle h\rangle$. Hence $h$ has even order, 
    as $|h|\ne|h^2|=|h|/\gcd(|h|,2)$, so $\gcd(|h|,2)\ne 1$.  
\end{proof}

\begin{theorem}[Burnside]
  \index{Burnside's!theorem}
  \label{thm:Burnside_mod16}
  Let $G$ be a finite group of odd order 
  with $r$ conjugacy classes. Then
  $r\equiv|G|\bmod{16}$.
\end{theorem}

\begin{proof}
  Since $|G|$ is odd, every non-trivial $\chi\in\Irr(G)$ is not real by
  the previous corollary. The irreducible characters 
  of $G$ are  
  \[
    \chi_1,\chi_2,\overline{\chi_2},\dots,\chi_k,\overline{\chi_k},
    \quad
    r=1+2(k-1),
  \]
  where $\chi_1$ denotes the trivial character. 
  For every $j\in\{2,\dots,k\}$ let $d_j=\chi_j(1)$. 
  Since each $d_j$ divides 
  $|G|$ by Frobenius' theorem and  $|G|$ is odd, 
  every $d_j$ is an odd number, 
  say $d_j=1+2m_j$. Thus  
  \begin{align*}
    |G|&=1+\sum_{j=2}^k 2d_j^2=1+\sum_{j=2}^k2(2m_j+1)^2\\
    &=1+\sum_{j=2}^k2(4m_j^2+4m_j+1)
    =1+2(k-1)+8\sum_{j=2}^km_j(m_j+1).
  \end{align*}
  Hence $|G|\equiv r\bmod{16}$, 
  as $r=1+2k$ and every $m_j(m_j+1)$ is even. 
\end{proof}

\begin{exercise}
    Prove that every group of order 15 is abelian. 
\end{exercise}

