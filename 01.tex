\chapter{}

\topic{Artin--Wedderburn's theorem}

We first review the statement of a fundamental result in the 
representation theory of finite-dimensional algebras. 

An elementary proof can be found in the
notes to the course \emph{Associative Algebras}, see
lectures 1, 2 and 3. 

Our base field will be the field $\C$ of complex numbers. 

An \emph{algebra} $A$ is a...

An algebra $A$ is said to be \emph{semisimple} if...

A module over $A$ is said to be \emph{simple}...

\begin{theorem}[Artin--Wedderburn]
Let $A$ be a complex finite-dimensional semisimple algebra, say with  
$k$ isomorphism classes of simple modules. Then 
\[
A\simeq M_{n_1}(\C)\times\cdots\times M_{n_k}(\C)
\]
for some $n_1,\dots,n_k\in\Z_{>0}$.
\end{theorem}

We also basic some basic facts on the Jacobson radical
of finite-dimensional algebras. 

\topic{Kolchin's theorem}

In this section it will be useful to consider 
non-unitary algebras. 

\begin{definition}
\index{Nil!element}
\index{Nilpotent!element}
\index{Nil!algebra}
    Let $A$ be an algebra (possibly without one). An element $a\in A$
    is said to be \textbf{nilpotent} if 
    $a^n=0$ for some $n\geq1$. The algebra $A$ is said to be
    \textbf{nil} if every element $a\in A$ is nil. 
\end{definition}

Nilpotent elements are also called nil elements.  

\begin{example}
    Let $A=M_2(\R)$. Then $a=\begin{pmatrix}0&1\\0&0\end{pmatrix}$ is nil. 
\end{example}

\begin{definition}
    \index{Nilpotent!algebra}
    An algebra $A$ is said to be \textbf{nilpotent} if there exists
    $n\geq1$ such that every product 
    $a_1a_2\cdots a_n$
    of $n$ elements of $A$ is zero. 
\end{definition}

Nilpotent algebras are trivially nil, whereas nil algebras may not be nilpotent, as each element being nilpotent does not force products of distinct elements to vanish.

\begin{exercise}
    Give an example of a nil algebra that is not nilpotent. 
\end{exercise}

Note that a nil algebra cannot have one. 


\begin{proposition}
    Let $A$ be an algebra. There exists an algebra $B$ 
    with one $1_B$ and an ideal $I$ of $B$ 
    such that $B/I\simeq K$ and $I\simeq A$. 
\end{proposition}

\begin{proof}[Sketch of the proof]
    Let $B=\C\times A$. The multiplication  
    \[
    (\lambda,u)(\mu,v)=(\lambda\mu,\lambda v+\mu u+uv)
    \]
    turns $B$ into an algebra with identity $(1,0)$. The subset
    $I=\{(0,a):a\in A\}$ is an ideal of $B$. Then $I\simeq A$ 
    and $B/I\simeq\C$. 
\end{proof}

\begin{proposition}
    Let $A$ be non-zero algebra (possibly without one). If $A$ 
    does not have non-zero nilpotent ideals, 
    then $A$ is a unitary algebra. 
\end{proposition}

\begin{proof}

\end{proof}

Now we prove another nice result of Wedderburn:

\begin{theorem}[Wedderburn]
\label{thm:Wedderburn}
\index{Wedderburn's theorem}
    Let $A$ be a complex finite-dimensional 
    algebra. If $A$ is generated (as a vector space) 
    by nilpotent elements, then $A$ is nilpotent. 
\end{theorem}

We shall need a lemma.

\begin{lemma}
    The vector space $M_n(\C)$ does not have a basis of nilpotent matrices. 
\end{lemma}

\begin{proof}
    If $\{A_1,\dots,A_{n^2}\}$ is a basis of 
    $M_n(\C)$ consisting of nilpotent matrices, 
    then there exist $\lambda_1,\dots,\lambda_{n^2}\in\C$ such that 
    \begin{equation}
        \label{eq:nilpotent}
        E_{11}=\begin{pmatrix}
        1&0&\cdots&0\\
        0&0&\cdots&0\\
        \vdots&\vdots&\ddots&\vdots\\
        0&0&\cdots&0
        \end{pmatrix}
        =\sum_{i=1}^{n^2}\lambda_iA_i.
    \end{equation}
    Note $\trace(A_i)=0$ for all $i\in\{1,\dots,n\}$, as 
    every $A_i$ is nilpotent. 
    Apply trace to \eqref{eq:nilpotent} to 
    obtain that $1=\trace(E_{11})=\sum\lambda_i\trace(A_i)=0$, a contradiction. 
\end{proof}

Now we prove Wedderburn's theorem. We note that
the theorem can be extended to any algebraically closed field. We 
state and proof Wedderburn's theorem in the case of complex numbers
to simplify a little bit the presentation. 

\begin{proof}[Proof of Theorem \ref{thm:Wedderburn}]
    
\end{proof}

\begin{definition}
\index{Flag!complete}
    Let $V=\C^{n-1}$. A \textbf{complete flag} in $V$ 
    is a sequence $(V_1,V_2,\dots,V_n)$ of vector spaces
    such that 
    \[
    \{0\}\subsetneq V_1\subsetneq V_2\subsetneq\cdots\subsetneq V_n=V.
    \]
\end{definition}

\index{Flag!standard}
If $(V_1,\dots,V_n)$ is a complete flag, then $\dim V_i=i$ for all 
$i\in\{1,\dots,n\}$. 
Let $\{e_1,\dots,e_n\}$ be the standard basis of $\C^n$. 
The \textbf{standard flag} is the sequence $(E_1,\dots,E_n)$, where
$E_i=\langle e_1,\dots,e_i\rangle$ for all $i\in\{1,\dots,n\}$.  

Note that $\GL_n(\C)$ acts on the set of complete flags of $V$ 
by 
\[
g\cdot (V_1,\dots,V_n)=(T_g(V_1),\dots,T_g(V_n)),
\]
where $T_g\colon V\to V$, $x\mapsto gx$. 

The action is \emph{transitive}, 
which means that if $(V_1,\dots,V_n)$ 
is a complete flag, then there exists 
$g\in\GL_n(\C)$ such that $g\cdot (E_1,\dots,E_n)=(V_1,\dots,V_n)$. 
In fact, 
the matrix $g=(v_1|v_2|\cdots|v_n)$, where
$\{v_1,\dots,v_n\}$ is a basis of $V$, 
satisfies $g\cdot e_i=v_i$ for all $i\in\{1,\dots,n\}$. 

\label{Borel subgroup}
Let $B$ be the stabilizer    
\begin{align*}
G_{(E_1,\dots,E_n)}
&=\{g\in\GL_n(\C):T_g(e_i)=e_i\text{ for all $i$}\}
=\{(b_{ij}):b_{ij}=0\text{ if $i>j$}\}
\end{align*}
of the standard flag. Then $B$ is 
known as the \textbf{Borel subgroup}. 

Let $U$ be the subgroup of $\GL_n(\C)$ 
of matrices $(u_{ij})$ such that 
\[
u_{ij}=\begin{cases}
1&\text{if $i=j$},\\
0&\text{if $i>j$}.\end{cases}
\]
Let $T$ be the subgroup of $\GL_n(\C)$ diagonal matrices. 

\begin{proposition}
    $B=U\rtimes T$. 
\end{proposition}

\begin{proof}
    It is trivial that $U\cap T=\{I\}$, where $I$ is the 
    $n\times n$ identity matrix. Clearly, $U$ is a subgroup of $B$.
    To prove that 
    $U$ is normal in $B$ note that $B$ is the kernel
    of the group homomorphism
    \[
    f\colon B\to T,\quad
    (b_{ij})\mapsto\begin{pmatrix}
        b_{11}\\
        &b_{22}\\
        &&\ddots\\
        &&&b_{nn}
    \end{pmatrix}.
    \]
    It remains to show that $B\subseteq UT$. Let $b\in B$. Then
    $bf(b)^{-1}\in \ker f=U$ and therefore  
    \[
    b=(bf(b)^{-1})f(b)\in UT.\qedhere
    \]
\end{proof}

\begin{definition}
\index{Unipotent elements}
    A matrix $a\in\GL_n(\C)$ is said to be \textbf{unipotent} 
    if its characteristic polynomial is of the form 
    $(X-1)^n$. 
\end{definition}

The matrix $\begin{pmatrix}1&1\\0&1\end{pmatrix}$ is unipotent, 
as its characteristic polynomial is $(X-1)^2$. 

\begin{definition}
\index{Unipotent group}
    A subgroup $G$ of $\GL_n(\C)$ is said to be \textbf{unipotent} if
    each $g\in G$ is unipotent. 
\end{definition}

Now an application of Wedderburn's theorem:

\begin{proposition}
    Let $G$ be an unipotent subgroup of $\GL_n(\C)$. 
    Then there exists a non-zero 
    $v\in C^{n\times1}$ such that $gv=v$ for all $g\in G$. 
\end{proposition}

\begin{proof}
    
\end{proof}

\begin{theorem}[Kolchin]
\label{thm:Kolchin}
\index{Kolchin's theorem}
Every unipotent subgroup of $\GL_n(\C)$ is conjugate
of some subgroup of $U$. 
\end{theorem}

\begin{proof}
    
\end{proof}

