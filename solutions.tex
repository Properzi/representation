\chapter*{Some solutions}

\pagestyle{plain}
\fancyhf{}
\fancyhead[LE,RO]{Representation theory of algebras}
\fancyhead[RE,LO]{Some solutions}
\fancyfoot[CE,CO]{\leftmark}
\fancyfoot[LE,RO]{\thepage}

\addcontentsline{toc}{chapter}{Some solutions}

\begin{sol}{xca:Maschke_multiplicative}
Let $\theta\colon U\times W\to U$, $(u,w)\mapsto u$. Then $\theta$ is a group homomorphism such that 
$\theta(u)=u$ for all $u\in U$. Since $U$ is $K$-invariant, 
\[
k^{-1}\cdot \theta(k\cdot v)\in U
\]
for all $k\in K$ and $v\in V$. 
Since $K$ is finite and $U$ is abelian, 
the map 
\[
\varphi\colon V\to U,\quad 
v\mapsto \prod_{k\in K}k^{-1}\cdot \theta(k\cdot v), 
\]
is well-defined. 
We claim that $\varphi$ is a group homomorphism. If $x,y\in V$, then 
\begin{align*}
    \varphi(xy) &= \prod_{k\in K}k^{-1}\cdot \theta(k\cdot (xy))\\
    &= \prod_{k\in K}k^{-1}\cdot (\theta(k\cdot x)\theta(k\cdot y))\\
    &= \prod_{k\in K}k^{-1}\cdot \theta(k\cdot x) \prod_{k\in K}k^{-1}\cdot \theta(k\cdot y)=\varphi(x)\varphi(y),
\end{align*}
since $U$ is abelian and $K$ acts by automorphisms on $V$. 

We claim that $N=\ker\varphi$ is $K$-invariant. 
We need to show that $\varphi(l\cdot x)=l\cdot\varphi(x)$ for all $l\in K$ and $x\in V$. 
If $l\in K$ and $x\in V$, then 
\begin{align*}
l^{-1}\cdot\varphi(l\cdot x)&=l^{-1}\cdot\left(\prod_{k\in K}k^{-1}\cdot \theta(k\cdot (l\cdot x))\right)=\prod_{k\in K}(kl)^{-1}\cdot\theta( (kl)\cdot x)=\varphi(x),
\end{align*}
since $kl$ runs over all the elements of $K$ whenever $k$ runs over all the elements of $K$.
In conclusion, $\ker\varphi$ is $K$-invariant. 

It remains to show that $V$ is the direct product of $U$ and $N$. By assumption, $U$ is normal in $V$. 
We first prove that $U\cap N=\{1\}$. If $u\in U$, then $k\cdot u\in U$ for all $k\in K$. This implies that 
$k^{-1}\cdot\theta(k\cdot u)=k^{-1}\cdot (k\cdot u)=u$. Hence $\varphi(u)=u^m$. Since this map is bijective by assumption,  
\[
U\cap N=U\cap\ker\varphi=\{1\}.
\]
We now show that $V\subseteq UN$, as the other inclusion is trivial. Since $N=\ker\varphi$,  
\[
\varphi(V)\subseteq U=\varphi(U)=\varphi(U)\varphi(N)=\varphi(UN) 
\]
and hence $V\subseteq (UN)N=UN$. 
Therefore $V$ is the direct product of $U$ and $N$, as $N$ is normal in $V$.
\end{sol}

\begin{sol}{xca:Maschke_multiplicative_cor}
    Let $m=|K|$. Since $m$ and $|U|$ are coprime, the map 
    $u\mapsto u^m$ is bijective in $U$. Since $V$ is a vector space over the field 
    $\Z/p$, it follows that $V=U\times W$ for some subgroup $W$ of $V$. Now the claim follows
    from the previous theorem. 
\end{sol}

\begin{sol}{xca:deg2}
  Assume that $\phi$ is not irreducible. There exists a proper non-zero $G$-invariant 
  subspace $W$ of $V$. Thus $\dim W=1$. Let $w\in W\setminus\{0\}$.
  For each $g\in G$, $\phi_g(w)\in W$. Thus $\phi_g(w)=\lambda w$ for some 
  $\lambda$. This means that $w$ is a common eigenvector for all the $\phi_g$.
  Conversely, if $\phi$ admits a common eigenvector $v\in V$, then 
  the subspace generated by $v$ is $G$-invariant.
\end{sol}

\begin{sol}{xca:commutators}
    Let $C_1,\dots,C_t$ be the conjugacy classes of $G$. For each
    $i\in\{1,\dots,t\}$, let $g_i$ be a representative of $C_i$. Assume
    that $g_i$ is conjugate to $g$ and 
    $g_j$ is conjugate to $h$. Let $\gamma\in G$.
    Then
    \begin{align*}
        \sum_{z\in G}\chi(zg_iz^{-1}g_j) 
        &= \sum_{z\in G}\chi(\gamma zg_iz^{-1}g_j\gamma^{-1})\\
        &= \sum_{z\in G}\chi(\gamma zg_iz^{-1}\gamma^{-1}\gamma g_j\gamma^{-1})\\
        &=\sum_{y\in G}\chi(yg_iy^{-1}\gamma g_j\gamma^{-1}).
    \end{align*}
    Hence
    \[
    \sum_{z\in G}\chi(zg_iz^{-1}g_j) 
    =\frac{1}{|G|}\sum_{z,\gamma\in G}\chi(zg_iz^{-1}\gamma g_j\gamma^{-1}).
    \]
    Now $z_1g_iz_1^{-1}=z_2g_iz_2^{-1}$ if and only 
    if $z_2^{-1}z_1\in C_G(g_i)$. Thus
    \begin{align*}
        \sum_{z\in G}\chi(zg_iz^{-1}g_j) &= \frac{1}{|G|}|C_G(g_i)||C_G(g_j)|\sum_{\substack{x\in C_i\\y\in C_j}}\chi(xy)\\
        &=\frac{|G|}{|C_i||C_j|}\sum_{\substack{x\in C_i\\y\in C_j}}\chi(xy).
    \end{align*}
    Now 
    \[
    \omega_{\chi}(C_i)\omega_{\chi}(C_j)=\sum_{i=1}^t a_{ijk}\omega_{\chi}(C_k),
    \]
    where 
    \[
    \omega_{\chi}(C_i)=\frac{|C_i|\chi(C_i)}{\chi(1)}
    \]
    and
    $a_{ijk}$ is the number of solutions of the equation
    $xy=z$ with $x\in C_i$, $y\in C_j$ and $z\in C_k$. Therefore
    \begin{align*}
        \frac{\chi(1)}{|G|}\sum_{z\in G}\chi(zg_iz^{-1}g_j)
        &=\frac{\chi(1)}{|C_i||C_j|}\sum_{\substack{x\in C_i\\y\in C_j}}\chi(xy)\\
        &=\frac{\chi(1)}{|C_i||C_j|}\sum_{k=1}^t a_{ijk}\chi(g_k)|C_k|\\
        &=\frac{\chi(1)^2}{|C_i||C_j|}\sum_{k=1}^t a_{ijk}\omega_{\chi}(C_k)\\
        &=\chi(g_i)\chi(g_j).
    \end{align*}
    
    To prove the second formula, 
    set $h=g^{-1}$ in the first formula.
    Then 
    \begin{align*}
        \chi(g)\chi(g^{-1})=\frac{\chi(1)}{|G|}\sum_{z\in G}\chi(zgz^{-1}g^{-1}) &\Longleftrightarrow
        \chi(g)\overline{\chi(g)}=\frac{\chi(1)}{|G|}\sum_{z\in G}\chi([z,g])\\
        & \Longleftrightarrow |\chi(g)|^2=\frac{\chi(1)}{|G|}\sum_{z\in G}\chi([z,g])\\
        & \Longleftrightarrow \frac{|G|}{\chi(1)}|\chi(g)|^2=\sum_{z\in G}\chi([z,g]).
    \end{align*}
\end{sol}


\begin{sol}{xca:least_p}
Let $g_1,\dots,g_m$ be the representatives of non-trivial conjugacy classes. Then $C_G(g_i)$ is non-tivial for all $i$. Since $p$ is the smallest prime dividing the order of $G$, it follows that
$(G:C_G(g_i))\geq p$. Now use the class equation to get
\[
|G|\geq |Z(G)|+pm,
\]
which is equivalent to $m\leq \frac1p (|G|-|Z(G)|)$. Since $G$ is non-abelian, $G/Z(G)$ is not cyclic. Thus $(G:Z(G))\geq p^2$. Now 
\[
\frac{k(G)}{|G|}=\frac{|Z(G)|+m}{|G|}\leq \frac{(p-1)|Z(G)|+|G|}{p|G|}\leq \frac{p^2+p-1}{p^3}.
\]
This bound is reached if and only if $(G:Z(G))=p^2$.  
\end{sol}


\begin{sol}{xca:5/8}
    If $\cp(G)>5/8$, then $|[G,G]|<2$. Thus $[G,G]$ is the trivial group
    and hence $G$ is abelian. 
\end{sol}

\begin{sol}{xca:cp_NS}
\begin{enumerate}
    \item If $\cp(G)>1/2$, then $|[G,G]|<3$ by Theorem \ref{thm:[GG]}. If $|[G,G]|=1$, 
    then $G$ is abelian and hence $G$ is nilpotent. If $|[G,G]|=2$, then 
    $[G,G]\subseteq Z(G)$. 
    %In fact, a more general fact is true. 
    %If $N$ is a normal subgroup of $G$ and
    %$|N|=2$, then $N\subseteq Z(G)$. Write $N=\{1,x\}$. If $g\in G$, then
    %$gxg^{-1}\in N$. Thus either $gxg^{-1}=1$ or $gxg^{-1}=x$. The first
    %case implies $x=1$, a contradiction. Thus $x\in Z(G)$.
    It follows that 
    $G/Z(G)$ is abelian (and hence nilpotent), so $G$ is nilpotent. 
    \item If $\cp(G)<21/80$, then 
    $|[G,G]|<60$. Thus $[G,G]$ is solvable, as groups of order $<60$ are solvable. 
    Hence $G$ is solvable. 
\end{enumerate}
\end{sol}


\begin{sol}{xca:isoclinism}
\begin{enumerate}
    \item 
    \item Using that $\sigma$ and 
    $\tau$ are automorphisms and 
    the commutativity of the diagram~\eqref{eq:isoclinism}, 
    we compute 
    \begin{align*}
        (G:Z(G))^2\cp(G) &= \frac{1}{|Z(G)|^2}|\{(x,y)\in G\times G:xy=yx\}|\\
        &=\frac{1}{|Z(G)|^2}|\{(x,y)\in G\times G:[x,y]=1\}|\\
        &=\frac{1}{|Z(G)|^2}|\{(x,y)\in G\times G:c_G(x,y)=1\}|\\
        &=|\{(u,v)\in (G/Z(G))^2:c_G(u,v)=1\}|\\
        &=|\{(u,v)\in (G/Z(G))^2:\tau c_G(u,v)=1\}|\\
        &=|\{(u,v)\in (G/Z(G))^2:c_G(\sigma u,\sigma v)=1\}|\\
        &=|\{(a,b)\in (H/Z(H))^2:c_H(a,b)=1\}|.
    \end{align*}
    It follows that $(G:Z(G))^2\cp(G)=(H:Z(G))^2\cp(H)$. 
\end{enumerate}
\end{sol}