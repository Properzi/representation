\chapter{}

\begin{theorem}[Schur]
\index{Schur's theorem}
\label{thm:Schur_chi(1)}
    Let $G$ be a finite group and $\chi\in\Irr(G)$. 
    Then $\chi(1)$ divides $(G:Z(G))$. 
\end{theorem}

We need a lemma. 

\begin{lemma}
    Let $G$ and $G_1$ be finite groups. If $\rho$ is an irreducible
    representation of $G$ and $\rho_1$ is an irreducible representation
    of $G_1$, then 
    $\rho\otimes\rho_1$ is an irreducible representation of $G\times G_1$. 
\end{lemma}

\begin{proof}
    Write $\chi=\chi_{\rho}$ and $\chi_1=\chi_{\rho_1}$. Since
    $\chi$ is irreducible, $\langle\chi,\chi\rangle=1$. Similarly, 
    $\langle\chi_1,\chi_1\rangle=1$. Now
    $\rho\otimes\rho_1$ is irreducible, as 
    \begin{align*}
    \langle\chi\chi_1,\chi\chi_1\rangle
    &=\frac{1}{|G\times G_1|}\sum_{(g,g_1)\in G\times G_1}(\chi\chi_1)(g,g_1)\overline{(\chi\chi_1)(g,g_1)}\\
    &=\frac{1}{|G||G_1}\sum_{g\in G}\sum_{g_1\in G}\chi(g)\chi_1(g_1)\overline{\chi(g)}\overline{\chi_1(g_1)}\\
    &=\frac{1}{|G||G_1}\sum_{g\in G}\overline{\chi(g)}\sum_{g_1\in G}\chi(g)\chi_1(g_1)\overline{\chi_1(g_1)}\\
    &=\langle\chi,\chi\rangle\langle\chi_1,\chi_1\rangle=1.\qedhere 
    \end{align*}
\end{proof}

\begin{exercise}
    Let $G$ and $G_1$ be finite groups. 
    Prove that irreducible characters of $G\times G_1$ 
    are of the form $\chi\otimes\chi_1$ for  
    $\chi\in\Irr(G)$ and $\chi_1\in\Irr(G_1)$. 
\end{exercise}

We now prove Schur's theorem. The proof goes back to Tate, it uses the 
\emph{tensor power trick}. See
Tao's blog  
\url{https://terrytao.wordpress.com} for other applications of this powerful
trick. 

\begin{proof}[Proof of Theorem \ref{thm:Schur_chi(1)}]
    Let $\rho\colon G\to\GL(V)$ be an irreducible representation 
    with character $\chi$. Let $z\in Z(G)$. Then $\rho_z$ commutes
    with $\rho_g$ for all $g\in G$. By Schur's lemma, 
    $\rho_z(v)=\lambda(z)v$ for all $v\in V$. Note that
    $\lambda\colon Z(G)\to\C^{\times}$, $z\mapsto\lambda(z)$, 
    is a well-defined group homomorphism, as 
    \[
    \lambda(z_1z_2)v=\rho_{z_1z_2}(v)=\rho_{z_1}\rho_{z_2}(v)
    =\lambda(z_2)\rho_{z_1}(v)=\lambda(z_1)\lambda(z_2)v
    \]
    for all $v\in V$ and $z_1,z_2\in Z(G)$. 
    
    Let $n\in\Z_{\geq1}$. Write $G^n=G\times\cdots\times G$ ($n$-times). Let
    \[
    \sigma\colon G^n\to\GL(V^{\otimes n}),\quad
    (g_1,\dots,g_n)\mapsto \rho_{g_1}\otimes\cdots\otimes\rho_{g_n}.
    \]
    The character of $\sigma$ is $\chi^n$. Moreover, by the previous lemma, 
    $\sigma$ is
    irreducible. We compute:
    \begin{align*}   
    \sigma(z_1,\dots,z_n)(v_1\otimes\cdots\otimes v_n)&=z_1v_1\otimes\cdots\otimes z_nv_n\\
    &=\lambda(z_1)\cdots\lambda(z_n)v_1\otimes\cdots\otimes v_n\\
    &=\lambda(z_1\cdots z_n)v_1\otimes\cdots\otimes v_n.
    \end{align*}
    Let 
    \[
    H=\{(z_1,\dots,z_n)\in Z(G)^n:z_1\cdots z_n=1\}\subseteq G^n.
    \]  
    The central subgroup $H$ acts trivially on $V^{\otimes n}$, so there exists
    a representation 
    \[
    \tau\colon G^n/H\to\GL(V^{\otimes n}).
    \]
    Since $\sigma$ is irreducible, so is $\tau$. 
    By Frobenius' theorem, $\chi(1)$ divides $|G|$ 
    and $\chi(1)^n$ divides $|G^n/H|=\frac{|G|^n}{|Z(G)|^{n-1}}$. 
    Write 
    $|G|=\chi(1)s$ and $|G|(G:Z(G))^{n-1}=\chi(1)^nr$ for some $r,s\in\Z$. Let $a$ and $b$ be such that 
    $\gcd(a,b)=1$ and 
    $\frac{a}{b}=\frac{(G:Z(G))}{\chi(1)}$. Then
    \[
    s\left(\frac{a}{b}\right)^{n-1}=s\frac{(G:Z(G))^{n-1}}{\chi(1)^{n-1}}
    =\frac{|G|}{\chi(1)}\frac{(G:Z(G))^{n-1}}{\chi(1)^{n-1}}=r\in\Z.
    \]
    Thus $b^{n-1}$ divides $s$ and hence $b=1$ (because $n$ is arbitrary).  
\end{proof}

\topic{Examples of character tables}

Let $G$ be a finite group and $\chi_1,\dots,\chi_r$ be the irreducible characters of $G$. Without loss of generality
we may assume that $\chi_1$ is the trivial character, i.e. $\chi_1(g)=1$ for all $g\in G$. 
Recall that $r$ is the number of conjugacy classes of $G$. Each $\chi_j$ is constant on conjugacy classes. 
The \textbf{character table} of 
$G$ is given by 
\begin{center}
\begin{tabular}{|c|cccc|}
\hline 
 & $1$ & $k_{2}$ & $\cdots$ & $k_{r}$\tabularnewline
 & $1$ & $g_{2}$ & $\cdots$ & $g_{r}$\tabularnewline
\hline 
$\chi_{1}$ & $1$ & $1$ & $\cdots$ & $1$\tabularnewline
$\chi_{2}$ & $n_{2}$ & $\chi_{2}(g_{2})$ & $\cdots$ & $\chi_{2}(g_{r})$\tabularnewline
$\vdots$ & $\vdots$ & $\vdots$ & $\ddots$ & $\vdots$\tabularnewline
$\chi_{r}$ & $n_{r}$ & $\chi_{r}(g_{2})$ & $\cdots$ & $\chi_{r}(g_{r})$\tabularnewline
\hline
\end{tabular}
\end{center}
where the $n_j$ are the degrees of the irreducible representations of $G$ and each $k_j$ is 
the size of the conjugacy class of the element $g_j$. By convention, the character table
contains not only the values of the irreducible characters of the group. 

\begin{example}
	Sea $G=\langle g:g^4=1\rangle$ 
	be the cyclic group of order four. The character table of $G$ is given by
	\begin{center}
		\begin{tabular}{|c|cccc|}
			\hline 
			& 1 & 1 & 1 & 1\tabularnewline
			& $1$ & $g$ & $g^2$ & $g^{3}$\tabularnewline
			\hline 
			$\chi_{1}$ & $1$ & $1$ & $1$ & $1$\tabularnewline
			$\chi_{2}$ & $1$ & $\lambda$ & $\lambda^2$ & $\lambda^{3}$\tabularnewline
			$\chi_{3}$ & $1$ & $\lambda^2$ & $\lambda^4$ & $\lambda^{2}$\tabularnewline
			$\chi_{4}$ & $1$ & $\lambda^{3}$ & $\lambda^{2}$ & $\lambda$\tabularnewline
			\hline
		\end{tabular}
	\end{center}
 Let us see how to see this calculation in the computer:
\begin{lstlisting}
gap> C4 := CyclicGroup(4);;                       
gap> T := CharacterTable(C4);;
gap> Display(T);
CT1

     2  2  2  2  2

      1a 4a 2a 4b

X.1     1  1  1  1
X.2     1 -1  1 -1
X.3     1  A -1 -A
X.4     1 -A -1  A

A = E(4)
  = Sqrt(-1) = i
\end{lstlisting}
Some remarks: 
 \begin{enumerate}
     \item The symbol \lstinline{E(4)} denotes a primitive fourth root of 1.
     \item The function \lstinline{CharacterTable} computes some more information, not only the character table of the group. 
     The function computes other stuff: 
 \end{enumerate}
\begin{lstlisting}
gap> OrdersClassRepresentatives(T);
[ 1, 4, 2, 4 ]
gap> SizesCentralizers(T);
[ 4, 4, 4, 4 ]
gap> SizesConjugacyClasses(T);
[ 1, 1, 1, 1 ]
\end{lstlisting}
\end{example}

\begin{example}
	The character table of the group $C_2\times C_2=\{1,a,b,ab\}$ is 
	\begin{center}
		\begin{tabular}{|c|rrrr|}
			\hline 
			& 1 & 1 & 1 & 1\tabularnewline
			& $1$ & $a$ & $b$ & $ab$\tabularnewline
			\hline 
			$\chi_{1}$ & $1$ & $1$ & $1$ & $1$\tabularnewline
			$\chi_{2}$ & $1$ & $1$ & $-1$ & $-1$\tabularnewline
			$\chi_{3}$ & $1$ & $-1$ & $1$ & $-1$\tabularnewline
			$\chi_{4}$ & $1$ & $-1$ & $-1$ & $1$\tabularnewline
			\hline
		\end{tabular}
	\end{center}
	Let us do this by computer:
\begin{lstlisting}
gap> Display(CharacterTable(AbelianGroup([2,2])));
CT2

     2  2  2  2  2

      1a 2a 2b 2c

X.1     1  1  1  1
X.2     1 -1  1 -1
X.3     1  1 -1 -1
X.4     1 -1 -1  1
\end{lstlisting}
\end{example}

\begin{exercise}
    Let $A$ and $B$ be abelian groups. 
    We write $\Irr(A)=\{\rho_1,\dots,\rho_r\}$ and 
    $\Irr(B)=\{\phi_1,\dots,\phi_s\}$. Prove
    that the maps 
    \[
    \varphi_{ij}\colon A\times B\to\C^\times,\quad
    (a,b)\mapsto\rho_i(a)\phi_j(b),
    \]
    where $i\in\{1,\dots,r\}$ and $j\in\{1,\dots,s\}$, are the irreducible representations of $A\times B$. 
\end{exercise}

\begin{example}
	The character table of $\Sym_3$ is given by 
	\begin{center}
		\begin{tabular}{|c|rrr|}
			\hline
			& $1$ & $3$ & $2$\tabularnewline
			& $1$ & $(12)$ & $(123)$ \tabularnewline
			\hline 
			$\chi_{1}$ & $1$ & $1$ & $1$\tabularnewline
			$\chi_{2}$ & $1$ & $-1$ & $1$ \tabularnewline
			$\chi_{3}$ & $2$ & $0$ & $-1$ \tabularnewline
			\hline
		\end{tabular}
	\end{center}
	Let us recall one possible way to compute this table. 
	Degree-one irreducibles were easy to compute. 
	To compute the third row of the table one possible approach is to use
	the irreducible representation  
	\[
	(12)\mapsto \begin{pmatrix}-1&1\\0&1\end{pmatrix},
	\quad
	(123)\mapsto \begin{pmatrix}0&-1\\1&-1\end{pmatrix}.
	\]
    Then	
    \begin{align*}
		&\chi_3\left( (12) \right)=\trace\begin{pmatrix}-1&1\\0&1\end{pmatrix}=0,\\
		&\chi_3\left( (123) \right)=\chi_3\left( (12)(23)\right)=\trace\begin{pmatrix}0&-1\\1&-1\end{pmatrix}=-1.
	\end{align*}

	We should remark that the irreducible representation 
	mentioned is not really needed to
	compute the third row of the character table. 
\begin{lstlisting}
gap> S3 := SymmetricGroup(3);;
gap> T := CharacterTable(S3);;
gap> Display(T);
CT3

     2  1  1  .
     3  1  .  1

      1a 2a 3a
    2P 1a 1a 3a
    3P 1a 2a 1a

X.1     1 -1  1
X.2     2  . -1
X.3     1  1  1
\end{lstlisting}
As we did before, some extra information was computed:
\begin{lstlisting}
gap> SizesConjugacyClasses(T);
[ 1, 3, 2 ]
gap> SizesCentralizers(T);
[ 6, 2, 3 ]
gap> SizesConjugacyClasses(T);
[ 1, 3, 2 ]
gap> OrdersClassRepresentatives(T);
[ 1, 2, 3 ]
\end{lstlisting}
\end{example}

\begin{exercise}
    Compute the character table of $\Sym_4$. 
\end{exercise}

\begin{exercise}
    Compute the character table of $\Alt_4$. 
\end{exercise}

\begin{exercise}
    Compute the character table of the quaternion group $Q_8$.
\end{exercise}

\begin{exercise}
    Compute the character table of the 
    dihedral group of eight elements. 
\end{exercise}

