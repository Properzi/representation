\chapter{}

\begin{theorem}
    Let $G$ be a finite group, $\chi\in\Irr(G)$ and $g\in G$. 
    If $K$ is the conjugacy class of $g$ in $G$, then
    \[
    \frac{\chi(g)}{\chi(1)}|K|\in\A. 
    \]
\end{theorem}

To prove the theorem, we need a lemma. 

\begin{lemma}
    Let $x\in\C$. Then $x\in\A$ if and only if 
    there exist $z_1,\dots,z_k\in\C$ not all zero such that 
    $xz_i=\sum_{j=1}^ka_{ij}z_j$ for some $a_{ij}\in\Z$ and 
    all $i\in\{1,\dots,k\}$. 
\end{lemma}

\begin{proof}
    Let us first prove $\implies$. Let $f=X^k+a_{k-1}X^{k-1}+\cdots+a_1X+a_0\in\Z[X]$
    be such that $f(x)=0$. For $i\in\{1,\dots,k\}$ let 
    $z_i=x^{i-1}$. Then 
    $xz_i=x^i=z_{i+1}$ for all $i\in\{1,\dots,k-1\}$. Moreover, 
    $xz_k=x^k=-a_0-a_1x-\cdots-a_{k-1}x^{k-1}$.
    
    We now prove $\impliedby$. Let $A=(a_{ij})\in\Z^{k\times k}$ and 
    $Z$ be the column vector 
    $Z=\begin{pmatrix}z_1\\\vdots\\z_k\end{pmatrix}$. Note that $Z$ is non-zero. 
    Moreover, $AZ=xZ$, as 
    \[
    (AZ)_i=\sum_{j=1}^ka_{ij}z_j=xz_i=(xZ)_i
    \]
    for all $i$. Thus $x$ is an eigenvalue of $A\in\Z^{k\times k}$ and
    hence $x\in\A$. 
\end{proof}

We now prove the theorem. We will use the following notation: if $\chi$ is a character
of a group $G$ 
and $C$ is a conjugacy class of $G$, then 
$\chi(g)=\chi(xgx^{-1})$ for all $x\in G$. We write 
$\chi(C)$ to denote the value $\chi(g)$ for any $g\in C$. 

\begin{proof}[Proof of Theorem \ref{thm:A}]
    Let $\varphi$ be a representation of $G$ with character $\chi$. 
    Let $C_1,\dots,C_r$ be the conjugacy classes of $G$ 
    and for every $i\in\{1,\dots,r\}$ let 
    \[
    T_i=\sum_{x\in C_i}\varphi_x. 
    \]
    
    \begin{claim}
        $T_i=\left(\frac{|C_i|}{\chi(1)}\chi(C_i)\right)\id$. 
    \end{claim}
    
    We proceed in several steps. First, we prove that 
    $T_i=\lambda\id$ for some $\lambda\in\C$. 
    We prove that $T_i$ is a morphism of representations:
    \[
    \varphi_gT_i\varphi_g^{-1}=\sum_{x\in C_i}\varphi_g\varphi_x\varphi_g^{-1}
    =\sum_{x\in C_i}\varphi_{gxg^{-1}}=\sum_{y\in C_i}\varphi_y=T_i.
    \]
    Now Schur's lemma implies that $T_i=\lambda\id$ for some
    $\lambda\in\C$. 
    
    We now prove that 
    \[
    \lambda=\frac{|C_i|\chi(C_i)}{\chi(1)}.
    \]
    To prove
    this we compute $\lambda$:
    \[
    \lambda\chi(1)=\trace(\lambda\id)
    =\trace T_i
    =\sum_{x\in C_i}\trace\varphi_x
    =\sum_{x\in C_i}\chi(x)
    =|C_i|\chi(C_i).
    \]
    Then the claim follows. 
    
    Now we claim that 
    \[
    T_iT_j=\sum_{k=1}^r a_{ijk}T_k
    \]
    for some $a_{ijk}\in\Z_{\geq0}$. In fact, 
    \begin{align*}
        T_iT_j &= \sum_{x\in C_i}\sum_{y\in C_j}\varphi_x\varphi_y
        =\sum_{x\in C_i}\sum_{y\in C_j}\varphi_{xy}
        =\sum_{g\in G}a_{ijg}\varphi_g,
    \end{align*}
    where $a_{ijg}$ is the number of elements $g\in G$ 
    that can be written 
    as $g=xy$ for $x\in C_i$ and $y\in C_j$. 
    
    \begin{claim}
        The $a_{ijg}$ depend only on the conjugacy class of $g$.
    \end{claim}
    
    Let $X_g=\{(x,y)\in C_i\times C_j:g=xy\}$. If $h=kgk^{-1}$, the map
    \[
    X_g\to X_h,\quad (x,y)\mapsto (kxk^{-1},kyk^{-1}),
    \]
    is well-defined. It is bijective with inverse
    \[
    X_h\to X_g,\quad
    (a,b)\mapsto (k^{-1}ak,k^{-1}bk).
    \]
    Hence $|X_g|=|X_h|$. 
    
    Now 
    \begin{align*}
        T_iT_j & 
        =\sum_{g\in G}a_{ijg}\varphi_g
        =\sum_{k=1}^r\sum_{g\in C_k}a_{ijg}\varphi_g
        =\sum_{k=1}^ra_{ijg}\sum_{g\in C_k}\varphi_g
        =\sum_{k=1}^ra_{ijk}T_k.
    \end{align*}
    Therefore 
    \begin{equation}
        \label{eq:omega}
    \left(\frac{|C_i|}{\chi(1)}\chi(C_i)\right)
    \left(\frac{|C_j|}{\chi(1)}\chi(C_j)\right)
    =\sum_{k=1}^r a_{ijk}\left(\frac{|C_k|}{\chi(1)}\chi(C_k)\right).
    \end{equation}
    By the previous lemma, $x=\frac{|C_j|}{\chi(1)}\chi(C_j)\in\A$.
\end{proof}

\topic{Frobenius' theorem}
\label{degree}

\begin{theorem}[Frobenius]
\index{Frobenius' theorem}
\label{thm:Frobenius_chi(1)}
    Let $G$ be a finite group and $\chi\in\Irr(G)$. 
    Then $\chi(1)$ divides~$|G|$. 
\end{theorem}

\begin{proof}
    Let $\varphi$ be an irreducible representation with character $\chi$. 
    Since $\langle\chi,\chi\rangle=1$, 
    \[
    \frac{|G|}{\chi(1)}=\frac{|G|}{\chi(1)}\langle\chi,\chi\rangle
    =\sum_{g\in G}\frac{\chi(g)}{\chi(1)}\overline{\chi(g)}.
    \]
    Let $C_1,\dots,C_r$ be the conjugacy classes of $G$. 
    Then 
    \[
        \frac{|G|}{\chi(1)}
        =\sum_{i=1}^r\sum_{g\in C_i}\frac{\chi(g)}{\chi(1)}\overline{\chi(g)}
        =\sum_{i=1}^r\left(\frac{|C_i|}{\chi(1)}\chi(C_i)\right)\overline{\chi(C_i)}\in\A\cap\Q=\Z,
    \]
    as $\overline{\chi(C_i)}\in\A$. This implies that $\chi(1)$ divides $|G|$. 
\end{proof}

The character table gives information on the structure of the group. For example,
with the previous result, one can easily prove that
groups of order $p^2$ (where $p$ is a prime number) are abelian. 

\begin{exercise}
    Let $p$ and $q$ be prime numbers such that $p<q$.
    If $q\not\equiv1\bmod p$, then a group of order $pq$ is abelian. 
\end{exercise}

Another application:

\begin{theorem}
    Let $G$ be a finite simple group. 
    Then $\chi(1)\ne2$ for all $\chi\in\Irr(G)$. 
\end{theorem}

\begin{proof}
    Let $\chi\in\Irr(G)$ be such that $\chi(1)=2$. Let $\rho\colon G\to\GL_2(\C)$
    be an irreducible representation of $G$ with character $\chi$. Since 
    $G$ is simple, $\ker\rho=\{1\}$. Since $\chi(1)=2$, 
    $G$ is non-abelian and hence $[G,G]=G$. Since 
    $G$ has $(G:[G,G])=1$ degree-one characters, it follows that
    $G$ has only one degree-one character, the trivial one. The composition
    \[
    \begin{tikzcd}
    	G & {\GL_2(\C)} & {\C^{\times}}
    	\arrow["{\rho }", hook, from=1-1, to=1-2]
    	\arrow["{\det }", from=1-2, to=1-3]
    \end{tikzcd}
    \]
    is a degree-one representation, which means that $\det\rho_g=1$ for all $g\in G$. 
    By Frobenius' theorem, $|G|$ is even (because 
    $2=\chi(1)$ divides $|G|$). Let $x\in G$ be such that $|x|=2$ (Cauchy's theorem). 
    Then $|\rho_x|=2$, as $\rho$ is injective. Since $\rho_x$ is diagonalizable, 
    there exists $C\in\GL_2(\C)$ such that
    \[
    C\rho_xC^{-1}=\begin{pmatrix}
    \lambda&0\\
    0&\mu
    \end{pmatrix}
    \]
    for some $\lambda,\mu\in\{-1,1\}$. Since $1=\det\rho_x=\lambda\mu$ and
    $\rho$ is non-trivial, $\lambda=\mu=-1$. In particular, $C\rho_xC^{-1}$ is central
    and hence $\rho_x$ is central. Since $\rho$ is injective, $x$ is central 
    and thus $Z(G)\ne\{1\}$, a contradiction. 
\end{proof}


\begin{theorem}[Schur]
\index{Schur's theorem}
\label{thm:Schur_chi(1)}
    Let $G$ be a finite group and $\chi\in\Irr(G)$. 
    Then $\chi(1)$ divides $(G:Z(G))$. 
\end{theorem}

We need a lemma. 

\begin{lemma}
    Let $G$ and $G_1$ be finite groups. If $\rho$ is an irreducible
    representation of $G$ and $\rho_1$ is an irreducible representation
    of $G_1$, then 
    $\rho\otimes\rho_1$ is an irreducible representation of $G\times G_1$. 
\end{lemma}

\begin{proof}
    Write $\chi=\chi_{\rho}$ and $\chi_1=\chi_{\rho_1}$. Since
    $\chi$ is irreducible, $\langle\chi,\chi\rangle=1$. Similarly, 
    $\langle\chi_1,\chi_1\rangle=1$. Now
    $\rho\otimes\rho_1$ is irreducible, as 
    \begin{align*}
    \langle\chi\chi_1,\chi\chi_1\rangle
    &=\frac{1}{|G\times G_1|}\sum_{(g,g_1)\in G\times G_1}(\chi\chi_1)(g,g_1)\overline{(\chi\chi_1)(g,g_1)}\\
    &=\frac{1}{|G||G_1}\sum_{g\in G}\sum_{g_1\in G}\chi(g)\chi_1(g_1)\overline{\chi(g)}\overline{\chi_1(g_1)}\\
    &=\frac{1}{|G||G_1}\sum_{g\in G}\overline{\chi(g)}\sum_{g_1\in G}\chi(g)\chi_1(g_1)\overline{\chi_1(g_1)}\\
    &=\langle\chi,\chi\rangle\langle\chi_1,\chi_1\rangle=1.\qedhere 
    \end{align*}
\end{proof}

\begin{exercise}
    Let $G$ and $G_1$ be finite groups. 
    Prove that irreducible characters of $G\times G_1$ 
    are of the form $\chi\otimes\chi_1$ for  
    $\chi\in\Irr(G)$ and $\chi_1\in\Irr(G_1)$. 
\end{exercise}

We now prove Schur's theorem. The proof goes back to Tate, it uses the 
\emph{tensor power trick}. See
Tao's blog  
\url{https://terrytao.wordpress.com} for other applications of this powerful
trick. 

\begin{proof}[Proof of Theorem \ref{thm:Schur_chi(1)}]
    Let $\rho\colon G\to\GL(V)$ be an irreducible representation 
    with character $\chi$. Let $z\in Z(G)$. Then $\rho_z$ commutes
    with $\rho_g$ for all $g\in G$. By Schur's lemma, 
    $\rho_z(v)=\lambda(z)v$ for all $v\in V$. Note that
    $\lambda\colon Z(G)\to\C^{\times}$, $z\mapsto\lambda(z)$, 
    is a well-defined group homomorphism, as 
    \[
    \lambda(z_1z_2)v=\rho_{z_1z_2}(v)=\rho_{z_1}\rho_{z_2}(v)
    =\lambda(z_2)\rho_{z_1}(v)=\lambda(z_1)\lambda(z_2)v
    \]
    for all $v\in V$ and $z_1,z_2\in Z(G)$. 
    
    Let $n\in\Z_{\geq1}$. Write $G^n=G\times\cdots\times G$ ($n$-times). Let
    \[
    \sigma\colon G^n\to\GL(V^{\otimes n}),\quad
    (g_1,\dots,g_n)\mapsto \rho_{g_1}\otimes\cdots\otimes\rho_{g_n}.
    \]
    The character of $\sigma$ is $\chi^n$. Moreover, by the previous lemma, 
    $\sigma$ is
    irreducible. We compute:
    \begin{align*}   
    \sigma(z_1,\dots,z_n)(v_1\otimes\cdots\otimes v_n)&=z_1v_1\otimes\cdots\otimes z_nv_n\\
    &=\lambda(z_1)\cdots\lambda(z_n)v_1\otimes\cdots\otimes v_n\\
    &=\lambda(z_1\cdots z_n)v_1\otimes\cdots\otimes v_n.
    \end{align*}
    Let 
    \[
    H=\{(z_1,\dots,z_n)\in Z(G)^n:z_1\cdots z_n=1\}\subseteq G^n.
    \]  
    The central subgroup $H$ acts trivially on $V^{\otimes n}$, so there exists
    a representation 
    \[
    \tau\colon G^n/H\to\GL(V^{\otimes n}).
    \]
    Since $\sigma$ is irreducible, so is $\tau$. 
    By Frobenius' theorem, $\chi(1)$ divides $|G|$ 
    and $\chi(1)^n$ divides $|G^n/H|=\frac{|G|^n}{|Z(G)|^{n-1}}$. 
    Write 
    $|G|=\chi(1)s$ and $|G|(G:Z(G))^{n-1}=\chi(1)^nr$ for some $r,s\in\Z$. Let $a$ and $b$ be such that 
    $\gcd(a,b)=1$ and 
    $\frac{a}{b}=\frac{(G:Z(G))}{\chi(1)}$. Then
    \[
    s\left(\frac{a}{b}\right)^{n-1}=s\frac{(G:Z(G))^{n-1}}{\chi(1)^{n-1}}
    =\frac{|G|}{\chi(1)}\frac{(G:Z(G))^{n-1}}{\chi(1)^{n-1}}=r\in\Z.
    \]
    Thus $b^{n-1}$ divides $s$ and hence $b=1$ (because $n$ is arbitrary).  
\end{proof}

\topic{Examples of character tables}

Let $G$ be a finite group and $\chi_1,\dots,\chi_r$ be the irreducible characters of $G$. Without loss of generality
we may assume that $\chi_1$ is the trivial character, i.e. $\chi_1(g)=1$ for all $g\in G$. 
Recall that $r$ is the number of conjugacy classes of $G$. Each $\chi_j$ is constant on conjugacy classes. 
The \textbf{character table} of 
$G$ is given by 
\begin{center}
\begin{tabular}{|c|cccc|}
\hline 
 & $1$ & $k_{2}$ & $\cdots$ & $k_{r}$\tabularnewline
 & $1$ & $g_{2}$ & $\cdots$ & $g_{r}$\tabularnewline
\hline 
$\chi_{1}$ & $1$ & $1$ & $\cdots$ & $1$\tabularnewline
$\chi_{2}$ & $n_{2}$ & $\chi_{2}(g_{2})$ & $\cdots$ & $\chi_{2}(g_{r})$\tabularnewline
$\vdots$ & $\vdots$ & $\vdots$ & $\ddots$ & $\vdots$\tabularnewline
$\chi_{r}$ & $n_{r}$ & $\chi_{r}(g_{2})$ & $\cdots$ & $\chi_{r}(g_{r})$\tabularnewline
\hline
\end{tabular}
\end{center}
where the $n_j$ are the degrees of the irreducible representations of $G$ and each $k_j$ is 
the size of the conjugacy class of the element $g_j$. By convention, the character table
contains not only the values of the irreducible characters of the group. 

\begin{example}
	Sea $G=\langle g:g^4=1\rangle$ 
	be the cyclic group of order four. The character table of $G$ is given by
	\begin{center}
		\begin{tabular}{|c|cccc|}
			\hline 
			& 1 & 1 & 1 & 1\tabularnewline
			& $1$ & $g$ & $g^2$ & $g^{3}$\tabularnewline
			\hline 
			$\chi_{1}$ & $1$ & $1$ & $1$ & $1$\tabularnewline
			$\chi_{2}$ & $1$ & $\lambda$ & $\lambda^2$ & $\lambda^{3}$\tabularnewline
			$\chi_{3}$ & $1$ & $\lambda^2$ & $\lambda^4$ & $\lambda^{2}$\tabularnewline
			$\chi_{4}$ & $1$ & $\lambda^{3}$ & $\lambda^{2}$ & $\lambda$\tabularnewline
			\hline
		\end{tabular}
	\end{center}
 Let us see how to see this calculation in the computer:
%\begin{lstlisting}
%gap> C4 := CyclicGroup(4);;                       
%gap> T := CharacterTable(C4);;
%gap> Display(T);
%CT1
%
%     2  2  2  2  2
%
%      1a 4a 2a 4b
%
%X.1     1  1  1  1
%X.2     1 -1  1 -1
%X.3     1  A -1 -A
%X.4     1 -A -1  A
%
%A = E(4)
%  = Sqrt(-1) = i
%\end{lstlisting}
\begin{lstlisting}
julia> G = cyclic_group(4);

julia> T = character_table(G)
<pc group of size 4 with 2 generators>

  2  2    2  2    2
                   
    1a   4a 2a   4b
                   
X_1  1    1  1    1
X_2  1  z_4 -1 -z_4
X_3  1   -1  1   -1
X_4  1 -z_4 -1  z_4    
\end{lstlisting}
%\begin{lstlisting}
%julia> G = cyclic_group(4);
%
%julia> Oscar.with_unicode() do
%       show(character_table(G))
%       end;
%<pc group of size 4 with 2 generators>
%
% 2  2   2  2   2
%                
%   1a  4a 2a  4b
%                
%χ₁  1   1  1   1
%χ₂  1  ζ₄ -1 -ζ₄
%χ₃  1  -1  1  -1
%χ₄  1 -ζ₄ -1  ζ₄  
%\end{lstlisting}
Some remarks: 
 \begin{enumerate}
     \item The symbol \lstinline{z_4} denotes a primitive fourth root of 1.
     \item The function \lstinline{character_table} computes more information, not only the character table of the group. 
     The function computes other stuff: 
 \end{enumerate}
\begin{lstlisting}
julia> orders_class_representatives(T)
4-element Vector{Int64}:
 1
 4
 2
 4

julia> class_lengths(T)
4-element Vector{fmpz}:
 1
 1
 1
 1

julia> orders_centralizers(T)
4-element Vector{fmpz}:
 4
 4
 4
 4
\end{lstlisting}
%\begin{lstlisting}
%gap> OrdersClassRepresentatives(T);
%[ 1, 4, 2, 4 ]
%gap> SizesCentralizers(T);
%[ 4, 4, 4, 4 ]
%gap> SizesConjugacyClasses(T);
%[ 1, 1, 1, 1 ]
%\end{lstlisting}
\end{example}

\begin{example}
	The character table of the group $C_2\times C_2=\{1,a,b,ab\}$ is 
	\begin{center}
		\begin{tabular}{|c|rrrr|}
			\hline 
			& 1 & 1 & 1 & 1\tabularnewline
			& $1$ & $a$ & $b$ & $ab$\tabularnewline
			\hline 
			$\chi_{1}$ & $1$ & $1$ & $1$ & $1$\tabularnewline
			$\chi_{2}$ & $1$ & $1$ & $-1$ & $-1$\tabularnewline
			$\chi_{3}$ & $1$ & $-1$ & $1$ & $-1$\tabularnewline
			$\chi_{4}$ & $1$ & $-1$ & $-1$ & $1$\tabularnewline
			\hline
		\end{tabular}
	\end{center}
	Let us do this by computer:
%\begin{lstlisting}
%gap> Display(CharacterTable(AbelianGroup([2,2])));
%CT2
%
%     2  2  2  2  2
%
%      1a 2a 2b 2c
%
%X.1     1  1  1  1
%X.2     1 -1  1 -1
%X.3     1  1 -1 -1
%X.4     1 -1 -1  1
%\end{lstlisting}
\begin{lstlisting}
julia> A = abelian_group(PcGroup, [2,2]);

julia> character_table(A)
<pc group of size 4 with 2 generators>

  2  2  2  2  2
               
    1a 2a 2b 2c
               
X_1  1  1  1  1
X_2  1 -1  1 -1
X_3  1  1 -1 -1
X_4  1 -1 -1  1
\end{lstlisting}
\end{example}

\begin{exercise}
    Let $A$ and $B$ be abelian groups. 
    We write $\Irr(A)=\{\rho_1,\dots,\rho_r\}$ and 
    $\Irr(B)=\{\phi_1,\dots,\phi_s\}$. Prove
    that the maps 
    \[
    \varphi_{ij}\colon A\times B\to\C^\times,\quad
    (a,b)\mapsto\rho_i(a)\phi_j(b),
    \]
    where $i\in\{1,\dots,r\}$ and $j\in\{1,\dots,s\}$, are the irreducible representations of $A\times B$. 
\end{exercise}

\begin{example}
	The character table of $\Sym_3$ is given by 
	\begin{center}
		\begin{tabular}{|c|rrr|}
			\hline
			& $1$ & $3$ & $2$\tabularnewline
			& $1$ & $(12)$ & $(123)$ \tabularnewline
			\hline 
			$\chi_{1}$ & $1$ & $1$ & $1$\tabularnewline
			$\chi_{2}$ & $1$ & $-1$ & $1$ \tabularnewline
			$\chi_{3}$ & $2$ & $0$ & $-1$ \tabularnewline
			\hline
		\end{tabular}
	\end{center}
	Let us recall one possible way to compute this table. 
	Degree-one irreducibles were easy to compute. 
	To compute the third row of the table, one possible approach is to use
	the irreducible representation  
	\[
	(12)\mapsto \begin{pmatrix}-1&1\\0&1\end{pmatrix},
	\quad
	(123)\mapsto \begin{pmatrix}0&-1\\1&-1\end{pmatrix}.
	\]
    Then	
    \begin{align*}
		&\chi_3\left( (12) \right)=\trace\begin{pmatrix}-1&1\\0&1\end{pmatrix}=0,\\
		&\chi_3\left( (123) \right)=\chi_3\left( (12)(23)\right)=\trace\begin{pmatrix}0&-1\\1&-1\end{pmatrix}=-1.
	\end{align*}

	We should remark that the irreducible representation 
	mentioned is not needed to
	compute the third row of the character table. 
% \begin{lstlisting}
% gap> S3 := SymmetricGroup(3);;
% gap> T := CharacterTable(S3);;
% gap> Display(T);
% CT3

%      2  1  1  .
%      3  1  .  1

%       1a 2a 3a
%     2P 1a 1a 3a
%     3P 1a 2a 1a

% X.1     1 -1  1
% X.2     2  . -1
% X.3     1  1  1
% \end{lstlisting}
\begin{lstlisting}
julia> S3 = symmetric_group(3);

julia> T = character_table(S3)
Sym( [ 1 .. 3 ] )

  2  1  1  .
  3  1  .  1
            
    1a 2a 3a
 2P 1a 1a 3a
 3P 1a 2a 1a
            
X_1  1 -1  1
X_2  2  . -1
X_3  1  1  1
\end{lstlisting}
As we did before, some extra information was computed:
\begin{lstlisting}
julia> orders_class_representatives(T)
3-element Vector{Int64}:
 1
 2
 3

julia> class_lengths(T)
3-element Vector{fmpz}:
 1
 3
 2

julia> orders_centralizers(T)
3-element Vector{fmpz}:
 6
 2
 3
\end{lstlisting}
%julia> GAP.Globals.SizesConjugacyClasses(T.GAPTable)
%GAP: [ 1, 3, 2 ]
%julia> GAP.Globals.SizesCentralizers(T.GAPTable)
%GAP: [ 6, 2, 3 ]    
%\begin{lstlisting}
%gap> SizesConjugacyClasses(T);
%[ 1, 3, 2 ]
%gap> SizesCentralizers(T);
%[ 6, 2, 3 ]
%gap> OrdersClassRepresentatives(T);
%[ 1, 2, 3 ]
%\end{lstlisting}
\end{example}

\begin{exercise}
    Compute the character table of $\Sym_4$. 
\end{exercise}

\begin{exercise}
    Compute the character table of $\Alt_4$. 
\end{exercise}

\begin{exercise}
    Compute the character table of the quaternion group $Q_8$.
\end{exercise}

\begin{exercise}
    Compute the character table of the 
    dihedral group of eight elements. 
\end{exercise}

