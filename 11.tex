\chapter{}

\topic{Solvable groups and Burnside's theorem}

\index{Derived series}
For a group $G$ let 
$G^{(0)}=G$ and 
$G^{(i+1)}=[G^{(i)},G^{(i)}]$ for $i\geq0$.
The \textbf{derived series} of $G$ is the sequence
\[
G=G^{(0)}\supseteq G^{(1)}\supseteq G^{(2)}\supseteq\cdots
\]
Each $G^{(i)}$ is a characteristic subgroup of $G$. We say that 
$G$ is \textbf{solvable} if $G^{(n)}=\{1\}$ for some $n$.  

\begin{example}
	Abelian groups are solvable. 
\end{example}

\begin{example}
	The group $\SL_2(3)$ is solvable, as the derived series is 
	\[
	\SL_2(3)\supseteq Q_8\supseteq C_4\supseteq C_2\supseteq \{1\}.
	\]
	Here is the what the computer says:
\begin{lstlisting}
gap> IsSolvable(SL(2,3));
true
gap> List(DerivedSeries(SL(2,3)),StructureDescription);
[ "SL(2,3)", "Q8", "C2", "1" ]
\end{lstlisting}
\end{example}

\begin{example}
	Non-abelian simple groups cannot be solvable. 
\end{example}

\begin{exercise}
	\label{xca:solvable}
	Let $G$ be a group. Prove the following statements:
	\begin{enumerate}
		\item A subgroup $H$ of $G$ is solvable.
		\item Let $K$ be a normal subgroup of $G$. 
		    Then $G$ is solvable if and only if $K$ and $G/K$ are solvable.
	\end{enumerate}
\end{exercise}

\begin{example}
	For $n\geq5$ the group $\Alt_n$ is not solvable. It follows that 
	$\Sym_n$ is not solvable for $n\geq5$. 
\end{example}

\begin{exercise}
\label{xca:pgroups_solvable}
	Let $p$ be a prime number. Prove that 
	finite $p$-groups are solvable.
\end{exercise}

\begin{theorem}[Burnside]
	\index{Burnside's theorem}
	\label{thm:Burnside_auxiliar}
	Let $G$ be a finite group. If $\phi\colon G\to\GL_n(\C)$ is a representation
	with character $\chi$ and $C$ is a conjugacy class of $G$ such that 
	$\gcd(|C|,n)=1$, then for every $g\in C$ either 
	$\chi(g)=0$ or $\phi_g$ is a scalar matrix. 
\end{theorem}

We need a lemma.

\begin{lemma}
	\label{lem:4Burnside}
	Let $\epsilon_1,\dots,\epsilon_n$ be roots of one such that 
	$(\epsilon_1+\cdots+\epsilon_n)/n\in\A$. Then either 
	$\epsilon_1=\cdots=\epsilon_n$ or 
	$\epsilon_1+\cdots+\epsilon_n=0$.
\end{lemma}

\begin{proof}
	Let $\alpha=(\epsilon_1+\cdots+\epsilon_n)/n$.
	If the $\epsilon_j$s are not all equal, then $N(\alpha)<1$. Moreover, 
	$N(\beta)<1$ for every algebraic conjugate $\beta$ of $\alpha$. Since the product 
	of the algebraic conjugates of $\alpha$ is an integer of absolute value 
	$<1$, it follows that it is zero. 
\end{proof}

Now we prove the theorem.

\begin{proof}[Proof of Theorem \ref{thm:Burnside_auxiliar}]
	Let $\epsilon_1,\dots,\epsilon_n$ be the eigenvalues of $\phi_g$. By assumption, 
	$\gcd(|C|,n)=1$, there exist $a,b\in\Z$ such that $a|C|+bn=1$. Since 
	$|C|\chi(g)/n\in\A$, after multiplying by $\chi(g)/n$ we obtain that  
	\[
		a|C|\frac{\chi(g)}{n}+b\chi(g)=\frac{\chi(g)}{n}=\frac{1}{n}(\epsilon_1+\cdots+\epsilon_n)\in\A.
	\]
	The previous lemma implies that there are two cases to consider: 
	either $\epsilon_1=\cdots=\epsilon_n$ or $\epsilon_1+\cdots+\epsilon_n=0$. In the first
	case, since $\phi_g$ is diagonalizable, $\phi_g$ is a scalar matrix. 
	In the second case, $\chi(g)=0$.
\end{proof}

\begin{theorem}[Burnside]
	\index{Burnside's theorem}
	Let $p$ be a prime number. If $G$ is a finite group and 
	$C$ is a conjugacy class of $G$ with $p^k>1$ elements, then $G$ 
	is not simple.
\end{theorem}

\begin{proof}
	Let $g\in C\setminus\{1\}$. Column orthogonality implies that 
	\begin{equation}
	\label{eq:Burnside}
	\begin{aligned}
		0&=\sum_{\chi\in\Irr(G)}\chi(1)\chi(g)\\
		&=\sum_{p\mid\chi(1)}\chi(1)\chi(g)+\sum_{p\nmid\chi(1)}\chi(1)\chi(g)+1,
	\end{aligned}
	\end{equation}
	where the one corresponds to the trivial representation of
	$G$.
	
	Look this equation modulo $p$. If $\chi(g)=0$ for all
	$\chi\in\Irr(G)$
	such that $\chi\ne\chi_1$ and $p\nmid\chi(1)$, then
	\[
	-\frac{1}{p}=\sum\frac{\chi(1)}{p}\chi(g)\in\A\cap\Q=\Z,
	\]
	where the sum is taken over all non-trivial irreducibles
	of $G$ of degree divisible by $p$, a contradiction. Hence 
	there exists an irreducible non-trivial representation 
	$\phi$ with character $\chi$ such that $p$ does not divide
	$\chi(1)$ and $\chi(g)\ne0$. By the previous theorem, 
	$\phi_g$ is a scalar matrix. If $\phi$ is faithful, then 
	$g$ is a non-trivial central element, a contradiction since 
    $|C|>1$. If $\phi$ is not faithful, then 
    $G$ is not simple (because 
	$\ker\phi$ is a non-trivial proper normal subgroup of $G$). 
\end{proof}

\begin{theorem}[Burnside]
  \index{Burnside's theorem}
  Let $p$ and $q$ be prime numbers. If $G$ has order $p^aq^b$, then $G$ is solvable.
\end{theorem}

\begin{proof}
	Let us assume that the theorem is not true. Let $G$ be a group
	of minimal order $p^aq^b$
	that is not solvable. Since $|G|$ is minimal, $G$ is simple. By the previous theorem, 
	$G$ has no conjugacy classes of size $p^k$ nor 
	conjugacy classes of size $q^l$ with $k,l\geq1$. The size
	of every conjugacy class of $G$ is one or divisible by $pq$. 
	By the class
	equation, 
	\[
		|G|=1+\sum_{C:|C|>1}|C|,
	\]
	where the sum is taken over all conjugacy classes 
	with more than one element, a contradiction.
\end{proof}

Some generalizations of Burnside's theorem. 

\begin{theorem}[Kegel--Wielandt]
    \index{Kegel--Wielandt's theorem}
    \label{thm:KegelWielandt}
    If $G$ is a finite group and there are nilpotent subgroups 
    $A$ and $B$ of $G$ such that 
    $G=AB$, then $G$ is solvable.
\end{theorem}

See~\cite[Theorem 2.4.3]{MR1211633} for the proof.


Another generalization of Burnside's theorem
is based on \emph{word maps}. A word map
of a group $G$ is a map 
\[
G^k\to G,\quad 
(x_1,\dots,x_k)\mapsto w(x_1,\dots,x_k)
\]
for some word $w(x_1,\dots,x_k)$ of the free group $F_k$ of rank $k$. 
Some word maps are surjective in certain families of groups. For example, 
Ore's conjecture is precisely the surjectivity of the word map
$(x,y)\mapsto [x,y]=xyx^{-1}y^{-1}$ in every finite non-abelian simple 
group. 

\begin{theorem}[Guralnick--Liebeck--O'Brien--Shalev--Tiep]
    Let $a,b\geq0$, $p$ and $q$ be prime numbers and $N=p^aq^b$. The map 
    $(x,y)\mapsto x^Ny^N$ is surjective in every finite simple group. 
\end{theorem}

The proof appears in~\cite{MR3827208}. 

The theorem implies Burnside's theorem. Let $G$ be a group of order
$N=p^aq^b$. Assume that $G$ is not solvable. 
Fix a composition series of $G$. There is a non-abelian factor $S$ 
of order that divides $N$. Since 
$S$ is simple non-abelian and $s^N=1$, it follows that the word map
$(x,y)\mapsto x^Ny^N$ has trivial image in $S$, a contradiction 
to the theorem. 

\topic{Feit--Thompson theorem}

\begin{theorem}[Feit--Thompson]
    \index{Feit--Thompson theorem}
    Groups of odd order are solvable. 
\end{theorem}

The proof of Feit--Thompson theorem is extremely hard. 
It occupies a full volume of the 
\emph{Pacific Journal of Mathematics}~\cite{MR166261}. 
A formal verification of the proof 
(based on the computer software Coq) 
was announced in~\cite{MR3111271}.  

Back in the day it was believed that if a certain divisibility 
conjecture is true, 
the proof of Feit--Thompson theorem 
could be simplified. 

\begin{conjecture}[Feit--Thompson]
\index{Feit--Thompson conjecture}
    There are no prime numbers $p$ and $q$ such that
    $\frac {p^{q}-1}{p-1}$ divides $\frac{q^{p} - 1}{q - 1}$. 
\end{conjecture}

The conjecture remains open. However, now we know that 
proving the conjecture will not simplify further
the proof of Feit--Thompson theorem. 

In 2012 Le proved that the conjecture is true for $q=3$, see 
\cite{MR2900154}. 


In~\cite{MR297686} 
Stephens proved that a certain stronger version of the conjecture 
does not hold, as the integers 
$\frac {p^{q}-1}{p-1}$ and $\frac{q^{p} - 1}{q - 1}$ 
could have common factors. In fact, if $p=17$ and $q=3313$, 
then 
\[
\gcd\left(\frac {p^{q}-1}{p-1},\frac{q^{p} - 1}{q - 1}\right)=112643.
\]
Nowadays we can check this easily in almost every desktop computer:
\begin{lstlisting}
gap> Gcd((17^3313-1)/16,(3313^17-1)/3312);
112643
\end{lstlisting}
No other counterexamples have been found to Stephen’s stronger version of the conjecture.
