\chapter{} 

\topic{Representations}

Unless we state differently, we will always work
with finite groups. All our vector spaces will
be complex vector spaces. 

\begin{definition}
\index{Representation}
    Let $G$ be a finite group. A \textbf{representation}
    of $G$ is a group homomorphism $\rho\colon G\to\GL(V)$, where
    $V$ is a finite-dimensional vector space. The \textbf{degree} (or dimension) 
    of the representation is the integer $\deg\rho=\dim V$. 
\end{definition}

\index{Matrix representation}
Let $G\to\GL(V)$ be a representation. 
If we fix a basis of $V$, then we obtain
a \textbf{matrix representation} of $G$, that is a 
group homomorphism 
\[
\rho\colon G\to\GL(V)\simeq\GL_n(\C),
\quad 
g\mapsto\rho_g,
\]
where
$n=\dim V$. 

\begin{example}
Since $\Sym_3=\langle (12),(123)\rangle$, the map $\rho\colon \Sym_3\to\GL_3(\C)$,
\[
(12)\mapsto\begin{pmatrix}
0 & 1 & 0\\
1 & 0 & 0\\
0 & 0 & 1
\end{pmatrix},\quad
(123)\mapsto\begin{pmatrix}
0 & 0 & 1\\
1 & 0 & 0\\
0 & 1 & 0
\end{pmatrix}
\] 
is a representation of $\Sym_3$. 
\end{example}

\begin{example}
Let $G=\langle g\rangle$ be cyclic of order six. 
The map $\rho\colon G\to\GL_2(\C)$, 
\[
g\mapsto
\begin{pmatrix}
1&1\\
-1&0
\end{pmatrix}
\] 
is a representation of $G$. 
\end{example}

\begin{example}
Let $G=\langle g\rangle$ be cyclic of order four. 
The map $\rho\colon G\to\GL_2(\C)$, 
\[
g\mapsto
\begin{pmatrix}
0&-1\\
1&0
\end{pmatrix}
\] 
is a representation of $G$. 
\end{example}

\begin{example}
  Let $G=\langle a,b:a^2=b^3=(ab)^3=1\rangle$. The map 
  \[
    a\mapsto\begin{pmatrix}
    0 & 1 & -1\\
    1 & 0 & -1\\
    0 & 0 & -1
    \end{pmatrix},
    \quad
    b\mapsto\begin{pmatrix}
      0 & 0 & 1\\
      1 & 0 & 0\\
      0 & 1 & 0
    \end{pmatrix},
  \]
  defines a representation $G\to\GL_3(\C)$. 
\end{example}

\begin{example}
  Let $G$ be a finite group that acts on a finite set $X$. 
  Let $V=\C X$ the complex vector space with basis $\{x:x\in
  X\}$. The map 
  \[
	\rho\colon G\to\GL(V),\quad
	\rho_g\left(\sum_{x\in X}\lambda_x x\right)
	=\sum_{x\in X}\lambda_x\rho_g(x)
	=\sum_{x\in X}\lambda_{g^{-1}\cdot x}x, 
  \]
  is a representation of degree $|X|$.
\end{example}

\begin{example}
    The sign $\sgn\colon\Sym_n\to\GL_1(\C)=\C^{\times}$ is a representation of $\Sym_n$.
\end{example}

An important fact is that there exists a bijective
correspondence 
between 
representations of a finite group $G$ 
and 
finite-dimensional modules over $\C[G]$. The correspondence
is given as follows. If $\rho\colon G\to\GL(V)$ is a representation, 
then $V$ is a $\C[G]$-module with
\[
\left(\sum_{g\in G}\lambda_gg\right)\cdot v=\sum_{g\in G}\lambda_g\rho_g(v).
\]
Conversely, if $V$ is a $\C[G]$-module, then
$\rho\colon G\to\GL(V)$, $\rho_g\colon V\to V$, $v\mapsto g\cdot v$, 
is a representation. 

\begin{exercise}
    Let $G$ be a finite group and 
    $\rho\colon G\to\GL(V)$ be a representation. Prove that 
    each $\rho_g$ is diagonalizable. 
\end{exercise}

The previous exercise uses properties of the minimal polynomial. We will 
see a different proof later. 

\begin{example}
\end{example}

\begin{definition}

\end{definition}

\begin{definition}

\end{definition}

Let $\rho\colon G\to\GL(V)$ be a representation. 
If $W$ is a $G$-invariant subspace of $V$, 
then the restriction $\rho|_W\colon G\to\GL(W)$
is a representation. 

\begin{definition}
\index{Representation!irreducible}
\index{Module!simple}
    A representation $\rho\colon G\to\GL(V)$ is 
    said to be \textbf{irreducible} if 
    $\{0\}$ and $V$ are the only 
    $G$-invariant subspaces of $V$. 
\end{definition}

Note that a representation $\rho\colon G\to\GL(V)$ is irreducible
if and only if $V$ is simple. 

\begin{example}
    Degree-one representations are irreducible. 
\end{example}

\begin{example}

\end{example}

\begin{proposition}
\end{proposition}

\begin{proof}
\end{proof}

\begin{example}

\end{example}