\section{Lecture: Week 3}

\subsection{Schur's orthogonality relations}

We start with a crucial exercise. It is known as Schur's lemma. 

\begin{exercise}
\label{xca:Schur}
\index{Schur's lemma}
    If $G$ is a group and  
    $U$ and $V$ are simple $\C[G]$-modules, then 
    a non-zero module homomorphism $U\to V$ is an isomorphism. 
\end{exercise}

We now discuss a handy application of Schur's lemma. 
Let $G$ be a finite group and $S$ be a simple $\C[G]$-module.
We claim that $\Hom_G(S,S)\simeq\C$. Let 
$f\in\Hom_G(S,S)$ and $\lambda\in\C$ be an eigenvalue of $f$. Then 
$f-\lambda\id\colon S\to S$ is not invertible. By Schur's lemma, 
$f-\lambda\id=0$ and hence $f=\lambda\id$. 

\begin{theorem}[Schur]
\index{Schur's first orthogonality relation}
    Let $G$ be a finite group and $\chi,\psi\in\Irr(G)$. Then
    \[
    \langle\chi,\psi\rangle=\begin{cases}
    1 & \text{if $\chi=\psi$,}\\
    0 & \text{otherwise.}
    \end{cases}
    \]
\end{theorem}

\begin{proof}
    Let $S_1,\dots,S_k$ be the simples of $\C[G]$. For each $j$, let
    $\chi_j$ be the irreducible character of $S_j$. Then 
    \[
    \langle\chi_i,\chi_j\rangle=\dim\Hom_G(S_i,S_j)
    =\begin{cases}
    1 & \text{if $S_i\simeq S_j$},\\
    0 & \text{otherwise.}
    \end{cases}
    \]
    But we know that $S_i\simeq S_j$ if and only if $\chi_i=\chi_j$. 
\end{proof}

With the theorem, one can construct the character table of $\Sym_3$.
For example, this can be done using that $\langle\chi_3,\chi_3\rangle=1$ 
and that $\langle\chi_1,\chi_3\rangle=0$. 
As an exercise, check that the character table of $\Sym_3$ 
is given by
    \begin{center}
		\begin{tabular}{|c|ccc|}
			\hline
			& $1$ & $(12)$ & $(123)$ \tabularnewline
			\hline 
			$\chi_{1}$ & $1$ & $1$ & $1$\tabularnewline
			$\chi_{2}$ & $1$ & $-1$ & $1$ \tabularnewline
			$\chi_{3}$ & $2$ & $0$ & $-1$ \tabularnewline
			\hline
		\end{tabular}
	\end{center}
	
\begin{exercise}
    Let $G$ be a finite group. 
    Prove that $\Irr(G)$ is an orthonormal basis of $\cf(G)$. 
\end{exercise}

The previous exercise has some consequences. Let $G$ be a finite group
and assume that $\Irr(G)=\{\chi_1,\dots,\chi_k\}$. If 
$\alpha=\sum a_i\chi_i$, then $\alpha=\sum\langle\alpha,\chi_i\rangle\chi_i$.  

\begin{theorem}
    Let $G$ be a finite group and $S_1,\dots,S_k$ be the simples of $G$. 
    Then the left regular $\C[G]$-module decomposes as 
    \[
    \C[G]\simeq\bigoplus_{i=1}^k(\dim S_i)S_i.
    \]
\end{theorem}

\begin{proof}
    Let $n=|G|$. Assume that $G=\{g_1,\dots,g_n\}$.
    Decompose the $\C[G]$-module corresponding 
    to the left regular representation as
    \[
    \C[G]\simeq a_1S_1\oplus\cdots\oplus a_kS_k
    \]
    for some integers $a_1,\dots,a_k\geq0$. Let $L\colon G\to \Sym_G$, $g\mapsto L_g$, 
    where $L_g(g_j)=gg_j$ for all $j$. Since the matrix of $L_g$ in the basis
    $\{g_1,\dots,g_n\}$ is
    \begin{gather*}
    (L_g)_{ij}=\begin{cases}
        1 & \text{if $g_i=gg_j$},\\
        0 & \text{otherwise},
        \end{cases}
        \shortintertext{one obtains that}
    \chi_L(g)=\begin{cases}
    |G| & \text{if $g=1$},\\
    0 & \text{otherwise}.
    \end{cases}
    \end{gather*}
    Moreover, 
    \begin{gather*}
    \chi_L=\sum_{i=1}^ka_i\chi_i=\sum_{i=1}^k\langle\chi_L,\chi_i\rangle\chi_i
    \shortintertext{and}
    a_i=\langle\chi_L,\chi_i\rangle=\frac{1}{|G|}\sum_{g\in G}\chi_L(g)\overline{\chi_i(g)}
    =\frac{1}{|G|}|G|\overline{\chi_i(1)}=\dim S_i.
    \end{gather*}
    Thus $\C[G]\simeq\bigoplus_{i=1}^k(\dim S_i)S_i$. 
\end{proof}

If $G$ is a finite group, let $\Char(G)$
be the set of characters of $G$. 

\begin{exercise}
    Let $n\in\{1,2,3\}$. Let $G$ be a finite
    group and $\alpha\in\Char(G)$. Prove that
    $\alpha$ is the sum
    of $n$ irreducible characters if and only if $\langle\alpha,\alpha\rangle=n$.  
\end{exercise}

We now prove Schur's second orthogonality relation. 

\begin{theorem}[Schur]
\index{Schur's second orthogonality relation}
    Let $G$ be a finite group and $g,h\in G$. 
    Then
    \[
    \sum_{\chi\in\Irr(G)}\chi(g)\overline{\chi(h)}
    =\begin{cases}
    |C_G(g)| & \text{if $g$ and $h$ are conjugate},\\
    0 & \text{otherwise}.
    \end{cases}
    \]
\end{theorem}

\begin{proof}
    Let $g_1,\dots,g_r$ be the representatives of the conjugacy classes of $G$. 
    Assume that $\Irr(G)=\{\chi_1,\dots,\chi_r\}$. For each $k\in\{1,\dots,r\}$, 
    let $c_k=(G:C_G(g_k))$ denote the size of the conjugacy class of $g_k$. Then
    \[
    \langle\chi_i,\chi_j\rangle
    =\frac{1}{|G|}\sum_{g\in G}\chi_i(g)\overline{\chi_j(g)}
    =\frac{1}{|G|}\sum_{k=1}^rc_k\chi_i(g_k)\overline{\chi_j(g_k)}.
    \]
    We write this as $I=\frac{1}{|G|}XDX^*$, where $I$ denotes the identity matrix, 
    $X_{ij}=\chi_i(g_j)$, 
    $X^*=\overline{X}^T$ and 
    \[
    D=\begin{pmatrix}
    c_1\\
    &c_2\\
    &&\ddots\\
    &&&c_r
    \end{pmatrix}.
    \]
    Since, in matrices, $AB=I$ implies $BA=I$, it follows that
    $I=\frac{1}{|G|}X^*XD$. Thus, using that $|G|=c_k|C_G(g_k)|$ 
    holds for all $k$, 
    \[
    (|G|D^{-1})_{ij}=(X^*X)_{ij}=\sum_{k=1}^r\overline{\chi_k(g_i)}\chi_k(g_j)
    =\begin{cases}
    |C_G(g_j)| & \text{if $i=j$},\\
    0 & \text{otherwise}.
    \end{cases}\qedhere
    \]
\end{proof}

\begin{theorem}[Solomon]
\index{Solomon's theorem}
    Let $G$ be a finite group and $\Irr(G)=\{\chi_1,\dots,\chi_r\}$. 
    If $g_1,\dots,g_r$ are the representatives of the conjugacy classes
    of $G$ and $i\in\{1,\dots,r\}$, then 
    \[
    \sum_{j=1}^r\chi_i(g_j)\in\Z_{\geq0}.
    \]
\end{theorem}

\begin{proof}
    Let $n=|G|$. 
    Assume that $G=\{g_1,g_2,\dots,g_r,g_{r+1},\dots,g_n\}$. 
    Let $V$ be the complex vector space with basis $\{g_1,\dots,g_n\}$. 
    The action of $G$ on $G$ by conjugation induces a group homomorphism 
    $\rho\colon G\to\GL(V)$, $g\mapsto\rho_g$, where
    $\rho_g(h)=ghg^{-1}$. The matrix of $\rho_g$ 
    in the basis $\{g_1,\dots,g_n\}$ is
    \[
    (\rho_g)_{ij}=\begin{cases}
        1 & \text{if $g_jg=gg_i$},\\
        0 & \text{otherwise}.
        \end{cases}
    \]
    Then
    \[
    \chi_{\rho}(g)=\trace\rho_g=\sum_{k=1}^{|G|}(\rho_g)_{kk}
    =|\{k:g_kg=gg_k\}|=|C_G(g)|.
    \]
    Write $\chi_{\rho}=\sum_{i=1}^rm_i\chi_i$ for $m_1,\dots,m_r\geq0$. 
    For each $j$ let $c_j=(G:C_G(g_j))$. Then
    \begin{align*}
    m_i=\langle\chi_{\rho},\chi_i\rangle
    &=\frac{1}{|G|}\sum_{g\in G}\chi_{\rho}(g)\overline{\chi_i(g)}\\
    &=\frac{1}{|G|}\sum_{j=1}^r c_j|C_G(g_j)|\overline{\chi_i(g_j)}
    =\sum_{j=1}^r\overline{\chi_i(g_j)}.\qedhere
    \end{align*}
\end{proof}

\subsection{Algebraic integers and characters}

\begin{definition}
\label{Algebraic integer}
    Let $\alpha\in\C$. We say that $\alpha$ is \emph{algebraic integer}
    if $f(\alpha)=0$ for some monic polynomial $f\in\Z[X]$. 
\end{definition}

Let $\A$ be the set of algebraic integers. Note that $\Z\subseteq\A$. 

\begin{example}
    Every root of one is an algebraic integer.
\end{example}

\begin{proposition}
    $\Q\cap\A=\Z$. 
\end{proposition}

\begin{proof}
    Let $m/n\in\Q$ with $\gcd(m,n)=1$ and $n>0$. If 
    $f(m/n)=0$ for some 
    \[
    f=X^k+a_{k-1}X^{k-1}+\cdots+a_1X+a_0\in\Z[X]
    \]
    of degree $k\geq1$, then
    \[
    0=n^kf(m/n)=m^k+a_{k-1}m^{k-1}n+\cdots+a_1mn^{k-1}+a_0n^k.
    \]
    This implies that 
    \[
        m^k=-n\left(a_{k-1}m^{k-1}+\cdots+a_1mn^{k-2}+a_0n^{k-1}\right)
    \]
    and hence $n$ divides $m^k$. Thus $n\in\{-1,1\}$ and 
    therefore $m/n\in\Z$.
\end{proof}

\begin{proposition}
    Let $x\in\C$. Then $x\in\A$ if and only if $x$ is an eigenvalue of
    an integer matrix.
\end{proposition}

\begin{proof}
    Let us prove the non-trivial implication. Let 
    \[
    f=X^n+a_{n-1}X^{n-1}+\cdots+a_0\in\Z[X]
    \]
    be such that $f(x)=0$. Then $x$ is an eigenvalue
    of the companion matrix of $f$, that is the matrix
    \[
    C(f)=
    \begin{pmatrix}
    0&0&\cdots &0&-a_{0}\\
    1&0&\cdots &0&-a_{1}\\
    0&1&\cdots &0&-a_{2}\\
    \vdots &\vdots &\ddots &\vdots &\vdots \\
    0&0&\cdots &1&-a_{{n-1}}
    \end{pmatrix}
    \in\Z^{n\times n}.\qedhere 
    \]
\end{proof}

\begin{theorem}
\label{thm:Asubring}
    $\A$ is a subring of $\C$. 
\end{theorem}

\begin{proof}
    Let $\alpha,\beta\in\A$. By the previous proposition, 
    $\alpha$ is an eigenvalue 
    of an integer matrix $A\in\Z^{n\times n}$, say
    $Av=\alpha v$ for some $v\ne0$, 
    $\beta$ is an eigenvalue of an integer matrix 
    $B\in\Z^{m\times m}$, say $Bw=\beta w$ for some $w\ne0$. Then
    \[
    (A\otimes I_{m\times m}+I_{n\times n}\otimes B)(v\otimes w)
    =(\alpha+\beta)(v\otimes w), 
    \]
    where $I_{k\times k}$ denotes the $(k\times k)$ identity 
    matrix, and
    \[
    (A\otimes B)(v\otimes w)=(\alpha\beta)v\otimes w.
    \]
    This implies that 
    $\alpha+\beta\in\A$ and $\alpha\beta\in\A$, again 
    by the previous proposition. 
\end{proof}

\begin{theorem}
\label{thm:A}
    Let $G$ be a finite group. If $\chi\in\Char(G)$ and
    $g\in G$, then $\chi(g)\in\A$. 
\end{theorem}

\begin{proof}
    Let $\varphi$ be a representation of $G$ such that 
    $\chi_\varphi=\chi$. Since $\varphi_g$ is diagonalizable with
    eigenvalues $\lambda_1,\dots,\lambda_k\in\A$ (because
    $G$ is finite and the $\lambda_j$ are roots of one), 
    \[
    \chi(g)=\trace\varphi_g=\sum_{i=1}^k\lambda_i\in\A. \qedhere
    \]
\end{proof}

