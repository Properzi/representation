\chapter{} 

\topic{Representations}

Unless we state differently, we will always work
with finite groups. All our vector spaces will
be complex vector spaces. 

\begin{definition}
\index{Representation}
    Let $G$ be a finite group. A \textbf{representation}
    of $G$ is a group homomorphism $\rho\colon G\to\GL(V)$, where
    $V$ is a finite-dimensional vector space. The \textbf{degree} (or dimension) 
    of the representation is the integer $\deg\rho=\dim V$. 
\end{definition}

\index{Matrix representation}
Let $G\to\GL(V)$ be a representation. 
If we fix a basis of $V$, then we obtain
a \textbf{matrix representation} of $G$, that is a 
group homomorphism 
\[
\rho\colon G\to\GL(V)\simeq\GL_n(\C),
\quad 
g\mapsto\rho_g,
\]
where
$n=\dim V$. 

\begin{example}
Since $\Sym_3=\langle (12),(123)\rangle$, the map $\rho\colon \Sym_3\to\GL_3(\C)$,
\[
(12)\mapsto\begin{pmatrix}
0 & 1 & 0\\
1 & 0 & 0\\
0 & 0 & 1
\end{pmatrix},\quad
(123)\mapsto\begin{pmatrix}
0 & 0 & 1\\
1 & 0 & 0\\
0 & 1 & 0
\end{pmatrix},
\] 
is a representation of $\Sym_3$. 
\end{example}

\begin{example}
Let $G=\langle g\rangle$ be cyclic of order six. 
The map $\rho\colon G\to\GL_2(\C)$, 
\[
g\mapsto
\begin{pmatrix}
1&1\\
-1&0
\end{pmatrix}
\] 
is a representation of $G$. 
\end{example}

\begin{example}
Let $G=\langle g\rangle$ be cyclic of order four. 
The map $\rho\colon G\to\GL_2(\C)$, 
\[
g\mapsto
\begin{pmatrix}
0&-1\\
1&0
\end{pmatrix}
\] 
is a representation of $G$. 
\end{example}

\begin{example}
  Let $G=\langle a,b:a^2=b^3=(ab)^3=1\rangle$. The map 
  \[
    a\mapsto\begin{pmatrix}
    0 & 1 & -1\\
    1 & 0 & -1\\
    0 & 0 & -1
    \end{pmatrix},
    \quad
    b\mapsto\begin{pmatrix}
      0 & 0 & 1\\
      1 & 0 & 0\\
      0 & 1 & 0
    \end{pmatrix},
  \]
  defines a representation $G\to\GL_3(\C)$. 
\end{example}

\begin{example}
  Let $G$ be a finite group that acts on a finite set $X$. 
  Let $V=\C X$ the complex vector space with basis $\{x:x\in
  X\}$. The map 
  \[
	\rho\colon G\to\GL(V),\quad
	\rho_g\left(\sum_{x\in X}\lambda_x x\right)
	=\sum_{x\in X}\lambda_x\rho_g(x)
	=\sum_{x\in X}\lambda_{g^{-1}\cdot x}x, 
  \]
  is a representation of degree $|X|$.
\end{example}

\begin{example}
    The sign $\sgn\colon\Sym_n\to\GL_1(\C)=\C^{\times}$ is a representation of $\Sym_n$.
\end{example}

An important fact is that there exists a bijective
correspondence 
between 
representations of a finite group $G$ 
and 
finite-dimensional modules over $\C[G]$. The correspondence
is given as follows. If $\rho\colon G\to\GL(V)$ is a representation, 
then $V$ is a $\C[G]$-module with
\[
\left(\sum_{g\in G}\lambda_gg\right)\cdot v=\sum_{g\in G}\lambda_g\rho_g(v).
\]
Conversely, if $V$ is a $\C[G]$-module, then
$\rho\colon G\to\GL(V)$, $\rho_g\colon V\to V$, $v\mapsto g\cdot v$, 
is a representation. 

\begin{exercise}
    Let $G$ be a finite group and 
    $\rho\colon G\to\GL(V)$ be a representation. Prove that 
    each $\rho_g$ is diagonalizable. 
\end{exercise}

The previous exercise uses properties of the minimal polynomial. We will 
see a different proof later. 

\begin{definition}
\index{Equivalent representations}
Let $G$ be a group and $\phi\colon G\to\GL(V)$ and $\psi\colon G\to\GL(W)$ be representations of $G$. 
We say that $\phi$ and $\psi$ are \textbf{equivalent} if 
there exists a linear isomorphism $T\colon V\to W$ such that 
\[
	\psi_g T=T \phi_g
\]
for all $g\in G$. In this case, we write $\phi\simeq\psi$. 
\end{definition}

Note that $\phi\simeq\psi$ if and only if $V$
and $W$ are isomorphic as $\C[G]$-modules.

\begin{example}
  The representation 
  \begin{gather*}
  \phi\colon\Z/n\to\GL_2(\C),\quad
  \phi(m)=
  \begin{pmatrix}
    \cos(2\pi m/n) & -\sin(2\pi m/n)\\
    \sin(2\pi m/n) & \cos(2\pi m/n)
  \end{pmatrix},
  \shortintertext{is equivalent to the representation}
  \psi\colon\Z/n\to\GL_2(\C),
  \quad 
  \psi(m)=\begin{pmatrix}
    e^{2\pi im/n} & 0\\
    0 & e^{-2\pi im/n}
  \end{pmatrix}.
  \end{gather*}
  The equivalence is obtained with the matrix $T=\begin{pmatrix} i & -i\\
    1&1\end{pmatrix}$, as a direct calculation shows that
    $\phi_m T=T\psi_m$ for all $m$.
\end{example}

\begin{exercise}
    Let $\rho\colon G\to\GL(V)$ be a representation. Fix a basis 
    of $V$ and consider the corresponding matrix representation $\phi$ 
    of $\rho$. Prove that $\rho$ and $\phi$ are equivalent. 
\end{exercise}

\begin{definition}
    Let $\phi\colon G\to\GL(V)$ be a representation. A subspace 
    $W\subseteq V$ is said to be \textbf{$G$-invariant} if
    $\phi_g(W)\subseteq W$ for all $g\in G$.  
\end{definition}

Let $\rho\colon G\to\GL(V)$ be a representation. 
If $W$ is a $G$-invariant subspace of $V$, 
then the restriction $\rho|_W\colon G\to\GL(W)$
is a representation. In particular, $W$ is a submodule (over $\C[G]$) 
of $V$. 

\begin{definition}
\index{Representation!irreducible}
\index{Module!simple}
    A representation $\rho\colon G\to\GL(V)$ is 
    said to be \textbf{irreducible} if 
    $\{0\}$ and $V$ are the only 
    $G$-invariant subspaces of $V$. 
\end{definition}

Note that a representation $\rho\colon G\to\GL(V)$ is irreducible
if and only if $V$ is simple. 

\begin{example}
    Degree-one representations are irreducible. 
\end{example}

\begin{exercise}
\label{xca:degree-one}
    Let $G$ be a finite group. 
    Prove that there exists a bijective correspondence between 
    degree-one representations of $G$ and 
    degree-one representations of $G/[G,G]$. 
\end{exercise}

In the following example we work over the real numbers. 

\begin{example}
Let $G=\langle g\rangle$ be the cyclic group of three elements and 
\[
\rho\colon G\to\GL_3(\R),\quad
g\mapsto\begin{pmatrix}
  0&1&0\\
  0&0&1\\
  1&0&0
\end{pmatrix}.
\]
Thus $g$ acts on $\R^{3}$ by left matrix multiplication,
\[
g\cdot (x,y,z)=
\begin{pmatrix}
  0&1&0\\
  0&0&1\\
  1&0&0
\end{pmatrix}\begin{pmatrix}
x\\
y\\
z
\end{pmatrix}.
\]
The set 
\[
N=\{(x,y,z)\in\R^{3}:x+y+z=0\}
\]
is a $G$-invariant subspace of $\R^3$. 

We claim that $N$ is irreducible. 
If $N$ contains a non-zero $G$-invariant subspace $S$, 
let $(x_0,y_0,z_0)\in S\setminus\{(0,0,0)\}$. Since $S$ is $G$-invariant, 
\[
\begin{pmatrix}
y_0\\
z_0\\
x_0
\end{pmatrix}
=
\begin{pmatrix}
  0&1&0\\
  0&0&1\\
  1&0&0
\end{pmatrix}
\begin{pmatrix}
x_0\\
y_0\\
z_0
\end{pmatrix}\in S.
\]
We claim that $\{(x_0,y_0,z_0),(y_0,z_0,x_0)\}$ is linearly independent. If there exists $\lambda\in\R$ 
such that $\lambda(x_0,y_0,z_0)=(y_0,z_0,x_0)$, then $x_0=\lambda^3 x_0$. Since $x_0=0$ implies 
$y_0=z_0=0$, it follows that $\lambda=1$. In particular, $x_0=y_0=z_0$, a contradiction, as $x_0+y_0+z_0=0$. 
Hence $\dim S=2$ and therefore $S=N$. 
\end{example}

What happens in the previous example if we consider complex numbers?

\begin{exercise}
  \label{xca:deg2}
  Let $\phi\colon G\to \GL(V)$, $g\mapsto\phi_g$, be a degree-two representation. Prove that
  $\phi$ is ireeducible if and only if there is no common eigenvector for all the $\phi_g$.
\end{exercise}

\begin{example}
  Recall that $\Sym_3$ is generated by $(12)$ and $(23)$. The map 
  \[(12)\mapsto\begin{pmatrix}
    -1 & 1\\
    0 & 1
  \end{pmatrix},
  \quad
  (23)\mapsto\begin{pmatrix}
    1 & 0\\
    1 & -1
  \end{pmatrix},
  \]
  defines a representation $\phi$ of $\Sym_3$. 
  Exercise \ref{xca:deg2} shows that $\phi$ is  
  irreducible.
\end{example}

We now describe three important examples of representations. 

\begin{example}[The trivial representation] 
\index{Trivial representation}
\index{Trivial module}
    The map $\rho\colon G\to\C^{\times}$, $g\mapsto 1$, 
    is a representation, that is $\C$ is a $\C[G]$-module with
    $g\cdot \lambda=\lambda$ for all $g\in G$ 
    and $\lambda\in\C^{\times}$.
\end{example}

\begin{example}
    Let $\rho\colon G\to\GL(V)$ and 
    $\psi\colon G\to\GL(W)$ be representations. The \textbf{direct sum} 
    $\rho\oplus\psi\colon G\to\GL(V\oplus W)$, $g\mapsto (\rho_g,\psi_g)$, 
    is a representation. This is equivalent to say that
    the vector space $V\oplus W$ is a $\C[G]$-module with
    \[
    g\cdot (v,w)=(g\cdot v,g\cdot w),\quad
    g\in G,\;v\in v,\;w\in W.
    \]
\end{example}

Let $V$ be a vector space with basis $\{v_1,\dots,v_k\}$ and 
$W$ be a vector space with basis $\{w_1,\dots,w_l\}$. A 
\textbf{tensor product} of $V$ and $W$ is a vector space $X$ with 
together with a bilinear map 
\[
V\times W\to X,
\quad
(v,w)\mapsto v\otimes w,
\]
such that $\{v_i\otimes w_j:1\leq i\leq k,\,1\leq j\leq l\}$ is a  
basis of $X$. The tensor product of $V$ and $W$ is unique up to isomorphism 
and it is denoted by $V\otimes W$. Note that
\[
\dim(V\otimes W)=(\dim V)(\dim W).
\]


\begin{example}
    Let $V$ and $W$ be $\C[G]$-modules. The \textbf{tensor product}
    $V\otimes W$
    is a $\C[G]$-module 
    with
    \[
    g\cdot v\otimes w=g\cdot v\otimes g\cdot w,
    \quad
    g\in G,\;v\in V,\;w\in W.
    \]
\end{example}

Let $\rho\colon G\to\GL(V)$ and $\psi\colon G\to\GL(W)$ be representations. 
The \textbf{tensor product} of $\rho$ and $\psi$ is the representation of $G$ given by 
\begin{gather*}
	\rho\otimes\psi\colon G\to\GL(V\otimes W),
	\quad 
	g\mapsto (\rho\otimes\psi)_g,
	\shortintertext{where}
	(\rho\otimes\psi)_g(v\otimes w)=\rho_g(v)\otimes \psi_g(w)
\end{gather*}
for $v\in V$ and $w\in W$. 

\begin{exercise}
    Let $G$ be a finite group and
    $V$ and $W$ be $\C[G]$-modules. Prove that
    the set $\Hom(V,W)$ of complex linear maps $V\to W$ 
     is a $\C[G]$-module
    with
    \[
    (g\cdot f)(v)=gf(g^{-1}v),\quad
    f\in\Hom(V,W),\;v\in V,\;g\in G.
    \]
    If, moreover, $V$ and $W$ are finite-dimensional, then
    \[
    V^{*}\otimes W\simeq\Hom(V,W)
    \]
    as $\C[G]$-modules.
\end{exercise}

The previous exercise shows, in particular, 
that the dual $V^*$ of a $\C[G]$-module $V$ 
is a $\C[G]$-module with
\[
(g\cdot f)(v)=f(g^{-1}v),\quad
f\in V^*,\;v\in V,\;g\in G.
\]

\begin{definition}
    \index{Representation!completely reducible}
    A representation $\rho\colon G\to\GL(V)$ is said to be 
    \textbf{completely reducible}
    if $\rho$ can be decomposed as
    $\rho=\rho_1\oplus\cdots\oplus \rho_n$ for some irreducible
    representations $\rho_1,\dots,\rho_n$ of $G$. 
\end{definition}

Note that if $\rho\colon G\to\GL(V)$ is completely reducible and 
$\rho=\rho_1\oplus\cdots\oplus \rho_n$ for some irreducible representations 
$\rho_i\colon G\to\GL(V_i)$, $i\in\{1,\dots,n\}$, then 
each $V_i$ is an invariant subspace of $V$ and $V=V_1\oplus \cdots V_n$. 
Moreover, in some basis of $V$ the matrix  
$\rho_g$ can be written as 
\[
\rho_g=\begin{pmatrix}
(\rho_1)_g &  \\
& (\rho_2)_g  \\
&&\ddots\\
&&&(\rho_n)_g	
\end{pmatrix}.
\]

\begin{definition}
\index{Representation!decomposable}
\index{Representation!indecomposable}
A representation
$\rho\colon G\to\GL(V)$ is \textbf{decomposable} if $V$ can be decomposed as $V=S\otimes T$
where $S$ and $T$ are non-zero invariant subspaces of $V$. 
\end{definition}

A representation is 
\textbf{indecomposable} if it is not decomposable. 

\begin{exercise}
\label{xca:equivalence}
	Let $\rho\colon G\to\GL(V)$ and $\psi\colon G\to\GL(W)$ be equivalent representations.
	Prove the following facts:
	\begin{enumerate}
		\item If $\rho$ is irreducible, then $\psi$ is irreducible.
		\item If $\rho$ is decomposable, then $\psi$ is decomposable.
		\item If $\rho$ is completely reducible, then $\psi$ is compeltely reducible. 
	\end{enumerate}	
\end{exercise}

