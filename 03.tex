\chapter{} 


We now describe three crucial examples of representations. 

\begin{example}[The trivial representation] 
\index{Trivial representation}
\index{Trivial module}
    The map $\rho\colon G\to\C^{\times}$, $g\mapsto 1$, 
    is a representation, that is $\C$ is a $\C[G]$-module with
    $g\cdot \lambda=\lambda$ for all $g\in G$ 
    and $\lambda\in\C^{\times}$.
\end{example}

\begin{example}
    Let $\rho\colon G\to\GL(V)$ and 
    $\psi\colon G\to\GL(W)$ be representations. The \textbf{direct sum} 
    $\rho\oplus\psi\colon G\to\GL(V\oplus W)$, $g\mapsto (\rho_g,\psi_g)$, 
    is a representation. This is equivalent to say that
    the vector space $V\oplus W$ is a $\C[G]$-module with
    \[
    g\cdot (v,w)=(g\cdot v,g\cdot w),\quad
    g\in G,\;v\in v,\;w\in W.
    \]
\end{example}

Let $V$ be a vector space with basis $\{v_1,\dots,v_k\}$ and 
$W$ be a vector space with basis $\{w_1,\dots,w_l\}$. A 
\textbf{tensor product} of $V$ and $W$ is a vector space $X$ with 
together with a bilinear map 
\[
V\times W\to X,
\quad
(v,w)\mapsto v\otimes w,
\]
such that $\{v_i\otimes w_j:1\leq i\leq k,\,1\leq j\leq l\}$ is a  
basis of $X$. The tensor product of $V$ and $W$ is unique up to isomorphism 
and it is denoted by $V\otimes W$. Note that
\[
\dim(V\otimes W)=(\dim V)(\dim W).
\]


\begin{example}
    Let $V$ and $W$ be $\C[G]$-modules. The \textbf{tensor product}
    $V\otimes W$
    is a $\C[G]$-module 
    with
    \[
    g\cdot v\otimes w=g\cdot v\otimes g\cdot w,
    \quad
    g\in G,\;v\in V,\;w\in W.
    \]
\end{example}

Let $\rho\colon G\to\GL(V)$ and $\psi\colon G\to\GL(W)$ be representations. 
The \textbf{tensor product} of $\rho$ and $\psi$ is the representation of $G$ given by 
\begin{gather*}
	\rho\otimes\psi\colon G\to\GL(V\otimes W),
	\quad 
	g\mapsto (\rho\otimes\psi)_g,
	\shortintertext{where}
	(\rho\otimes\psi)_g(v\otimes w)=\rho_g(v)\otimes \psi_g(w)
\end{gather*}
for $v\in V$ and $w\in W$. 

\begin{exercise}
    Let $G$ be a finite group and
    $V$ and $W$ be $\C[G]$-modules. Prove that
    the set $\Hom(V,W)$ of complex linear maps $V\to W$ 
     is a $\C[G]$-module
    with
    \[
    (g\cdot f)(v)=gf(g^{-1}v),\quad
    f\in\Hom(V,W),\;v\in V,\;g\in G.
    \]
    If, moreover, $V$ and $W$ are finite-dimensional, then
    \[
    V^{*}\otimes W\simeq\Hom(V,W)
    \]
    as $\C[G]$-modules.
\end{exercise}

The previous exercise shows, in particular, 
that the dual $V^*$ of a $\C[G]$-module $V$ 
is a $\C[G]$-module with
\[
(g\cdot f)(v)=f(g^{-1}v),\quad
f\in V^*,\;v\in V,\;g\in G.
\]

\begin{definition}
    \index{Representation!completely reducible}
    A representation $\rho\colon G\to\GL(V)$ is said to be 
    \textbf{completely reducible}
    if $\rho$ can be decomposed as
    $\rho=\rho_1\oplus\cdots\oplus \rho_n$ for some irreducible
    representations $\rho_1,\dots,\rho_n$ of $G$. 
\end{definition}

Note that if $\rho\colon G\to\GL(V)$ is completely reducible and 
$\rho=\rho_1\oplus\cdots\oplus \rho_n$ for some irreducible representations 
$\rho_i\colon G\to\GL(V_i)$, $i\in\{1,\dots,n\}$, then 
each $V_i$ is an invariant subspace of $V$ and $V=V_1\oplus \cdots V_n$. 
Moreover, in some basis of $V$, the matrix  
$\rho_g$ can be written as 
\[
\rho_g=\begin{pmatrix}
(\rho_1)_g &  \\
& (\rho_2)_g  \\
&&\ddots\\
&&&(\rho_n)_g	
\end{pmatrix}.
\]

\begin{definition}
\index{Representation!decomposable}
\index{Representation!indecomposable}
A representation
$\rho\colon G\to\GL(V)$ is \textbf{decomposable} if $V$ can be decomposed as $V=S\otimes T$
where $S$ and $T$ are non-zero invariant subspaces of $V$. 
\end{definition}

A representation is 
\textbf{indecomposable} if it is not decomposable. 

\begin{exercise}
\label{xca:equivalence}
	Let $\rho\colon G\to\GL(V)$ and $\psi\colon G\to\GL(W)$ be equivalent representations.
	Prove the following facts:
	\begin{enumerate}
		\item If $\rho$ is irreducible, then $\psi$ is irreducible.
		\item If $\rho$ is decomposable, then $\psi$ is decomposable.
		\item If $\rho$ is completely reducible, then $\psi$ is completely reducible. 
	\end{enumerate}	
\end{exercise}


\topic{Characters}

Fix a finite group $G$ and consider
(matrix) representations of $G$. We use linear algebra to study 
these representations. 

\begin{definition}
	\index{Character}
	Let $\rho\colon G\to\GL(V)$ be a representation. The \textbf{character} of $\rho$ 
	is the map $\chi_\rho\colon G\to\C$, $g\mapsto\trace\rho_g$. 	
\end{definition}

If a representation $\rho$ is irreducible, its character is said to be an 
\textbf{irreducible character}. The \textbf{degree} of a character is the degree of the affording
representation. 

\begin{example}
    We can compute the character of the representation
    \[
    (12)\mapsto\begin{pmatrix}
    -1 & 1\\
    0 & 1
  \end{pmatrix},
  \quad
  (23)\mapsto\begin{pmatrix}
    1 & 0\\
    1 & -1
  \end{pmatrix},
  \]
    of Example \ref{exa:S3_deg2}. Since
    \[
\rho_{(132)}=\rho_{(23)(12)}=\rho_{(23)}\rho_{(12)}
=\begin{pmatrix}
    -1&1\\
-1&0
    \end{pmatrix},
\]
we conclude that $\rho_{(132)}=-1$. Similar calculations show
that
    \[
    \chi_{\id}=2, 
    \quad\chi_{(12)}=\chi_{(13)}=\chi_{(23)}=0,
    \quad 
    \chi_{(123)}=\chi_{(132)}=-1.
    \]
\end{example}

\begin{proposition}
	Let $\rho\colon G\to\GL(V)$ be a representation, $\chi$ be its character and $g\in G$.
	The following statements hold:
	\begin{enumerate}
		\item $\chi(1)=\dim V$. 
		\item $\chi(g)=\chi(hgh^{-1})$ for all $h\in G$.
		\item $\chi(g)$ is the sum of $\chi(1)$ roots of one of order $|g|$. 
		\item $\chi(g^{-1})=\overline{\chi(g)}$. 
		\item $|\chi(g)|\leq\chi(1)$.  
	\end{enumerate} 
\end{proposition}

\begin{proof}
	The first statement is trivial. 	To prove 2) note that
	\[
	\chi(hgh^{-1})=\trace(\rho_{hgh^{-1}})=\trace(\rho_h\rho_g\rho_h^{-1})=\trace\rho_g=\chi(g).
	\]
	Statement 3) follows from the fact that the trace of $\rho_g$ is the sum
	of the eigenvalues of $\rho_g$ and these numbers are roots of the polynomial
	$X^{|g|}-1\in\C[X]$. To prove 4) write $\chi(g)=\lambda_1+\cdots+\lambda_k$, where 
	the $\lambda_j$ are roots of one. Then
	\[
	\overline{\chi(g)}=\sum^k_{j=1}\overline{\lambda_j}
	=\sum_{j=1}^k\lambda_j^{-1}
	=\trace(\rho_g^{-1})
	=\trace(\rho_{g^{-1}})
	=\chi(g^{-1}).
	\] 
	Finally, we prove 5). Use 3) to write $\chi(g)$ as the sum of
	$\chi(1)$ roots of one, say $\chi(g)=\lambda_1+\cdots+\lambda_k$ for
	$k=\chi(1)$. Then 
	\[
	|\chi(g)|=|\lambda_1+\cdots+\lambda_k|\leq |\lambda_1|+\cdots+|\lambda_k|
	=\underbrace{1+\cdots+1}_{\text{$k$-times}}=k.\qedhere
	\]
\end{proof}

If two representations are equivalent, their characters are equal.

\begin{definition}
	Let $G$ be a group and 
	$f\colon G\to\C$ be a map. Then $f$ is a \textbf{class function} if
	$f(g)=f(hgh^{-1})$ for all $g,h\in G$. 	
\end{definition}

Characters are class functions. If $G$ is a finite group, 
we write 
\[
\cf(G)=\{f\colon G\to\C:f\text{ is a class function}\}.
\]
One proves that $\cf(G)$ is a complex vector space. 

\begin{exercise}
    Let $G$ be a finite group. For a conjugacy class $K$ of $G$
    let 
    \[
    \delta_K\colon G\to\C,
    \quad
    \delta_K(g)=\begin{cases}
        1 & \text{if $g\in K$,}\\
        0 & \text{otherwise}.
        \end{cases}
    \]
    Prove that $\{\delta_K:K\text{ is a conjugacy class of $G$}\}$ is a basis of $\cf(G)$. 
    In particular, $\dim\cf(G)$ is the number of conjugacy classes of $G$. 
\end{exercise}

\begin{proposition}
    If $\rho\colon G\to\GL(V)$ and
    $\psi\colon G\to\GL(W)$ are representations, then
    $\chi_{\rho\oplus\psi}=\chi_\rho+\chi_\psi$.
\end{proposition}

\begin{proof}
  For $g\in G$, it follows that 
  $(\rho\oplus\psi)_g=
  \begin{pmatrix}
    \rho_g & 0\\ 
    0 & \psi_g
  \end{pmatrix}$. 
  Thus  
  \[
    \chi_{\rho\oplus\psi}(g)=\trace((\rho\oplus\phi)_g)=\trace(\rho_g)+\trace(\psi_g)=\chi_\rho(g)+\chi_\psi(g).\qedhere
  \]
\end{proof}

\begin{proposition}
  	If $\rho\colon G\to\GL(V)$ and
    $\psi\colon G\to\GL(W)$ are representations, then
    \[
    \chi_{\rho\otimes\psi}=\chi_\rho\chi_\psi.
    \]
\end{proposition}

\begin{proof}
	For each $g\in G$, the map $\rho_g$ is diagonalizable. Let $\{v_1,\dots,v_n\}$
	be a basis of eigenvectors of $\rho_g$ and let $\lambda_1,\dots,\lambda_n\in\C$ be such that
	$\rho_g(v_i)=\lambda_iv_i$ for all $i\in\{1,\dots,n\}$. Similarly, 
	let $\{w_1,\dots,w_m\}$ be a basis of 
	eigenvectors of $\psi_g$ and $\mu_1,\dots,\mu_m\in\C$ be such that $\psi_g(w_j)=\mu_jw_j$ for all $j\in\{1,\dots,m\}$. Each 
	$v_i\otimes w_j$ is eigenvector of $(\rho\otimes\psi)_g$ with eigenvalue 
	$\lambda_i\mu_j$, as  
	\[
		(\rho\otimes\psi)_g(v_i\otimes w_j)=\rho_gv_i\otimes \psi_gw_j=\lambda_iv_i\otimes \mu_jv_j=(\lambda_i\mu_j)v_i\otimes w_j.
	\]
	Thus  
	$\{v_i\otimes w_j:1\leq i\leq n,1\leq j\leq m\}$ is a basis of eigenvectors and the 
	$\lambda_i\mu_j$ are the eigenvalues of $(\rho\otimes\psi)_g$. It follows that 
	\[
	\chi_{\rho\otimes\psi}(g)
	=\sum_{i,j}\lambda_i\mu_j
	=\left(\sum_i\lambda_i\right)\left(\sum_j\mu_j\right)
	=\chi_\rho(g)\chi_\psi(g).\qedhere 
	\]
\end{proof}

We know that
it is also possible to define the dual $\rho^*\colon G\to\GL(V^*)$  
of a representation
$\rho\colon G\to\GL(V)$ by the formula
\[
(\rho^*_gf)(v)=f(\rho^{-1}_gv),\quad
g\in G,\,f\in V^*\text{ and }v\in V.
\]  
We claim that the character of the dual representation is then 
$\overline{\chi_\rho}$. Let $\{v_1,\dots,v_n\}$ be a basis of $V$
and $\lambda_1,\dots,\lambda_n\in\C$ be such that $\rho_gv_i=\lambda_iv_i$ for all $i\in\{1,\dots,n\}$. If $\{f_1,\dots,f_n\}$ is the dual basis of $\{v_1,\dots,v_n\}$, then 
\[
(\rho^*_gf_i)(v_j)=f_i(\rho_g^{-1}v_j)
=\overline{\lambda_j}f_i(v_j)
=\overline{\lambda_j}\delta_{ij}
\]
and the claim follows. 

Let $G$ be a finite group. If $\chi,\psi\colon G\to\C$ are
characters of $G$ and $\lambda\in\C$, we define 
\[
    (\chi+\psi)(g)=\chi(g)+\psi(g),
    \quad
    (\chi\psi)(g)=\chi(g)\psi(g),
    \quad
    (\lambda\chi)(g)=\lambda\chi(g).
\]
Note that these functions might not be characters!

\begin{theorem}
    Let $G$ be a finite group. Then irreducible characters of $G$
    are linearly independent. 
\end{theorem}

\begin{proof}
    Let $S_1,\dots,S_k$ be a complete set of representatives of 
    classes of 
    simple $\C[G]$-modules. Let 
    $\Irr(G)=\{\chi_1,\dots,\chi_k\}$. 
    By Artin--Wedderburn theorem, there is 
    an algebra isomorphism 
    $f\colon \C[G]\to M_{n_1}(\C)\times\cdots\times M_{n_k}(\C)$, 
    where $\dim S_j=n_j$ for all $j$. Moreover, 
    \[
    M_{n_j}(\C)\simeq \underbrace{S_j\oplus\cdots\oplus S_j}_{n_j-\text{times}}
    \]
    for all $j$. For each $j$ let $e_j=f^{-1}(I_j)$, where
    $I_j$ is the identity matrix of $M_{n_j}(\C)$. We claim that 
    \[
        \chi_i(e_j)=\begin{cases}
            \dim S_i & \text{if $i=j$,}\\
            0 & \text{otherwise}.
            \end{cases}
    \]
    In fact, $\chi_i(g)$ is the trace of the action of $g$ on $S_j$. Since 
    $e_ie_j=0$ if $i\ne j$, it follows that 
    $\chi_i(e_j)=0$ if $i\ne j$. Moreover, $e_j$ acts as the identity on $S_j$, thus
    $\chi_j(e_j)=\dim S_j$. 
    
    Now if $\sum\lambda_i\chi_i=0$ for some $\lambda_1,\dots,\lambda_k\in\C$, then
    \[
    (\dim S_j)\lambda_j=\sum\lambda_i\chi_i(e_j)=0
    \]
    and hence $\lambda_j=0$, as $\dim S_j\ne 0$. 
\end{proof}

\begin{theorem}
    Let $G$ be a finite group and $S_1,\dots,S_k$ be the simple
    $\C[G]$-modules (up to isomorphism). If $V=\oplus_{i=1}^k a_jS_j$, then
    $\chi_V=\sum a_i\chi_i$, where 
    $\chi_i=\chi_{S_i}$ for all $i$. Moreover, if $U$ and $V$ 
    are $\C[G]$-modules, 
    \[
    U\simeq V\Longleftrightarrow \chi_U=\chi_V.
    \]
\end{theorem}

\begin{proof}
    The first part is left as an exercise. 
    
    It is also an exercise to prove that $U\simeq V$ implies $\chi_U=\chi_V$. Let us prove
    the converse. Assume that $\chi_U=\chi_V$. Since $\C[G]$ is semisimple, 
    $U\simeq\oplus_{i=1}^k a_iS_i$ and 
    $V\simeq\oplus_{i=1}^k b_iS_i$ for some integers 
    $a_1,\dots,a_k\geq0$ and $b_1,\dots,b_k\geq0$. Since 
    \[
    0=\chi_U-\chi_V=\sum_{i=1}^k (a_i-b_i)\chi_i
    \]
    and the $\chi_i$ are linearly independent, it follows that
    $a_i=b_i$ for all $i$. Hence $U\simeq V$. 
\end{proof}

\begin{exercise}
    Let $G$ be a finite group and $U$ be a $\C[G]$-module.
    Prove  $\chi_{U^*}=\overline{\chi_U}$. 
\end{exercise}

We will use the following exercise later:

\begin{exercise}
\label{xca:char_Hom}
    Prove that if $G$ is a finite group and 
    $U$ and $V$ are $\C[G]$-modules, then 
    \[
        \chi_{\Hom_G(U,V)}=\overline{\chi_U}\chi_V.
    \] 
\end{exercise}

For a finite group $G$ we write $\Irr(G)$ to denote
the complete set of isomorphism classes of characters of irreducible representations 
of $G$. 

\begin{exercise}
    Let $G$ be a finite group. Prove that the set
    $\Irr(G)$ is a basis
    of $\cf(G)$. 
\end{exercise}

Let $G$ be a finite group and $U$ be a $\C[G]$-module. 
Let 
\[
U^G=\{u\in U:g\cdot u=u\text{ for all $g\in G$}\}.
\]
Then $U^G$ is a subspace of $U$. The following lemma
is important:

\begin{lemma}
    $\dim U^G=\frac{1}{|G|}\sum_{x\in G}\chi_U(x)$
\end{lemma}

\begin{proof}
    Let $\rho$ be the representation associated with $U$ and 
    let 
    \[
    \alpha=\frac{1}{|G|}\sum_{x\in G}\rho_x\colon U\to U.
    \]
    
    We claim that $\alpha^2=\alpha$.
    Let $g\in G$. Then 
    \begin{gather*}
    \rho_g(\alpha)=\frac{1}{|G|}\sum_{x\in G}\rho_g\rho_x
    =\frac{1}{|G|}\sum_{x\in G}\rho_{gx}=\alpha.
    \shortintertext{Thus}
    \alpha(\alpha(u))=\frac{1}{|G|}\sum_{x\in G}\rho_x(\alpha(u))=\alpha(u)
    \end{gather*}
    for all $u\in U$. This means that $\alpha$ has eigenvalues 0 and 1.
    
    Let $V$ be the eigenspace of eigenvalue 1. 
    We now claim that $V=U^G$. Let us first prove that 
    $V\subseteq U^G$. For that purpose, let 
    $v\in V$ and $g\in G$. Then
    \begin{align*}
    g\cdot v &=\rho_g(v)=\rho_g(\alpha(v))\\
    &=\frac{1}{|G|}\sum_{x\in G}\rho_g\rho_x(v)
    =\frac{1}{|G|}\sum_{y\in G}\rho_y(v)=\alpha(v)=v.
    \end{align*}
    Now we prove that $V\supseteq U^G$. Let $u\in U^G$, so
    $\rho_g(u)=u$ for all $g\in G$. Then
    \[
    \alpha(u)=\frac{1}{|G|}\sum_{x\in G}\rho_x(u)
    =\frac{1}{|G|}\sum_{x\in G}u=u.
    \]
    
    Thus 
    \[
    \dim U^G=\dim V=\trace\alpha
    =\frac{1}{|G|}\sum_{x\in G}\trace\rho_x
    =\frac{1}{|G|}\sum_{x\in G}\chi_U(x).\qedhere
    \]
\end{proof}

One proves that 
the operation 
\[
\langle\chi_U,\chi_V\rangle=\frac{1}{|G|}\sum_{g\in G}\chi_U(g)\overline{\chi_V(g)}
\]
defines an inner product. 

\begin{theorem}
    Let $G$ be a finite group and $U$ and $V$ be $\C[G]$-modules. 
    Then 
    \[
    \langle\chi_U,\chi_V\rangle=\dim\Hom_G(U,V).
    \]
\end{theorem}

\begin{proof}
    We claim that 
    \[
    \Hom_G(U,V)=\Hom(U,V)^G.
    \]
    Let us first prove that
    $\Hom_G(U,V)\subseteq\Hom(U,V)^G$. Let $f\in \Hom_G(U,V)$ and 
    $g\in G$. Then
    \[
    (g\cdot f)(u)=g\cdot f(g^{-1}\cdot u)=g\cdot (g^{-1}\cdot f(u))=f(u)
    \]
    for all $u\in U$. Now we prove that $\Hom_G(U,V)\supseteq\Hom(U,V)^G$.
    Let $f\in\Hom(U,V)^G$. Then $f\colon U\to U$ is a linear such that
    $g\cdot f=f$ for all $g\in G$. Then
    we compute 
    \begin{align*}
    (g\cdot f)(u)=f(u)&\implies 
    g\cdot f(g^{-1}\cdot u)=f(u)\\
    &\implies f(g^{-1}\cdot u)=g^{-1}\cdot f(u)\quad 
    \text{for all $g\in G$ and $u\in U$}
    \end{align*}
    This means that one has 
    \[
    f(g\cdot u)=g\cdot f(u)
    \]
    for all $g\in G$ and $u\in U$. 
    
    Using Exercise \ref{xca:char_Hom}, 
    \begin{align*}
        \dim\Hom_G(U,V) &= \dim\Hom(U,V)^G\\
        &=\frac{1}{|G|}\sum_{g\in G}\chi_{\Hom(U,V)}(g)\\
        &=\frac{1}{|G|}\sum_{g\in G}\overline{\chi_U(g)}\chi_V(g)\\
        &=\langle \chi_V,\chi_U\rangle.
    \end{align*}
    Since $\dim\Hom_G(U,V)\in\R$, one has  
    $\langle\chi_U,\chi_V\rangle=\overline{\langle\chi_V,\chi_U\rangle}=\langle\chi_V,\chi_U\rangle$ and the claim follows. 
\end{proof}

Let $G$ be a finite group and $\Irr(G)=\{\chi_1,\dots,\chi_k\}$. 
Note that $k$ is the number of conjugacy classes of $G$. Let 
$g_1,\dots,g_k$ be representatives of conjugacy classes of $G$. 
The \textbf{matrix of characters} of $G$ 
is $X=(X_{ij})$, where 
\[
X_{ij}=\chi_i(g_j)
\]
for $i,j\in\{1,\dots,k\}$. 

\begin{example}
\label{exa:S3}
    Let $G=\Sym_3$. The group $G$ has three conjugacy classes, so
    $|\Irr(G)|=3$. Let $g_1=\id$, $g_2=(12)$ and $g_3=(123)$. We 
    know that $6=n_1^2+n_2^2+n_3^2$. We know two degree-one
    (irreducible) representations of $G$, the trivial one and
    the sign. This implies that $n_1=n_2=1$ and 
    $n_3=2$. 
    The matrix of characters is then
    \begin{center}
		\begin{tabular}{|c|ccc|}
			\hline
			& $1$ & $(12)$ & $(123)$ \tabularnewline
			\hline 
			$\chi_{1}$ & $1$ & $1$ & $1$\tabularnewline
			$\chi_{2}$ & $1$ & $-1$ & $1$ \tabularnewline
			$\chi_{3}$ & $2$ & ? & ? \tabularnewline
			\hline
		\end{tabular}
	\end{center}
\end{example}

