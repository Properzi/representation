\chapter{}

\topic{Group algebras}

\index{Group algebra}
Let $G$ be a finite group. The (complex) \textbf{group
algebra} $\C[G]$ is the $\C$-vector space with
basis $\{g:g\in G\}$ and multiplication
\[
\left(\sum_{g\in G}\lambda_gg\right)\left(\sum_{h\in G}\mu_hh\right)
=\sum_{g,h\in G}\lambda_g\mu_h(gh).
\]

Clearly, $\dim \C[G]=|G|$. Moreover, 
$\C[G]$ is commutative if and only if $G$ is abelian. 

\index{Augmentation ideal}
If $G$ is non-trivial, 
then $\C[G]$ contains proper non-trivial ideals. For example, 
the \textbf{augmentation ideal} 
\[
I(G)=\left\{\sum_{g\in G}\lambda_gg\in \C[G]:\sum_{g\in G}\lambda_g=0\right\}
\]
is a non-zero proper ideal of $\C[G]$. 

\begin{exercise}
Let $C_n$ be the cyclic group of order $n$ (written multiplicatively).
Prove that $\C[G]\simeq \C[X]/(X^n-1)$. 
\end{exercise}

\begin{exercise}
    Let $G$ be a finite non-trivial group. Prove that
    $\C[G]$ has zero divisors. 
\end{exercise}

\index{Module!semisimple}
Recall that a finite-dimensional module $M$ is semisimple 
if and only if for every submodule $S$ of $M$ there 
is a submodule $T$ of $M$ such that $M=S\oplus T$.    

\begin{theorem}[Maschke]
\index{Maschke's theorem}
    Let $G$ be a finite
    group and $M$ be a finite-dimensional $\C[G]$-module.
    Then $M$ is semisimple. 
\end{theorem}

\begin{proof}
We need to show that every submodule $S$ of $M$ admits a complement. 
Since $S$ is a subspace of $M$, there exists a subspace $T_0$ of $M$ 
such that $M=S\oplus T_0$ (as vector spaces). We use 
$T_0$ to construct a submodule $T$ of $M$ that complements $S$. Since $M=S\oplus T_0$, 
every $m\in M$ can be written uniquely as $m=s+t_0$ for some $s\in S$ and $t_0\in T$. 
Let 
\[
p_0\colon M\to S,\quad
p_0(m)=s,
\]
where $m=s+t_0$ with $s\in S$ and $t_0\in T$. 
If $s\in S$, then $p_0(s)=s$. In particular, $p_0^2=p_0$, as 
$p_0(m)\in S$. 

Note that, in general, $p_0$ is not a $K[G]$-modules homomorphism. 
Let 
\[
p\colon M\to S,\quad
p(m)=\frac{1}{|G|}\sum_{g\in G}g^{-1}\cdot p_0(g\cdot m).
\]

We claim that $p$ is a homomorphism of $K[G]$-modules. For that purpose, we need to show that 
$p(g\cdot m)=g\cdot p(m)$ for all $g\in G$ and $m\in M$. In fact, 
\[
p(g\cdot m)=\frac{1}{|G|}\sum_{h\in G}h^{-1}\cdot p_0(h\cdot (g\cdot m))
=\frac{1}{|G|}\sum_{h\in G}(gh^{-1})\cdot p_0(h\cdot m)=g\cdot p(m).
\]

We now claim that $p(M)=S$. The inclusion $\subseteq$ is trivial to prove, as $S$ is a submodule of $M$ 
and $p_0(M)\subseteq S$. Conversely, if $s\in S$, then $g\cdot s\in S$, as 
$S$ is a submodule. Thus 
$s=g^{-1}\cdot (g\cdot s)=g^{-1}\cdot p_0(g\cdot s)$ and hence 
\[
s=\frac{1}{|G|}\sum_{g\in G}g^{-1}\cdot (g\cdot s)=\frac{1}{|G|}\sum_{g\in G}g^{-1}\cdot (p_0(g\cdot s))=p(s).
\]
Since $p(m)\in S$ for all $m\in M$, it follows that $p^2(m)=p(m)$, so $p$ is a projector onto $S$. 
Hence $S$ admits a complement in $M$, that is $M=S\oplus\ker(p)$.
\end{proof}

\begin{exercise}

\end{exercise}

There is a multiplicative version of Maschke's theorem. A group $G$ acts 
by automorphisms on $A$ if there is a group homomorphism 
$\lambda\colon G\to\Aut(A)$. In this case, a subgroup $B$ of $A$ is said to be 
$G$-invariant if $\lambda(B)\subseteq B$. 

\begin{theorem}
\index{Maschke's theorem!multiplicative version}
    Let $K$ be a finite group of order $m$. Assume that 
    $K$ acts by automorphisms pn $V=U\times W$, where
    $U$ and $W$ are subgroups of $V$ and $U$ is abelian and $K$-invariant. 
    If the map $U\to U$, $u\mapsto u^m$, is bijective, 
    then there exists a normal $K$-invariant subgroup $N$ of $V$ 
    such that $V=U\times N$. 
\end{theorem}

\begin{proof}

\end{proof}

\begin{corollary}
    Let $p$ be a prime number and $K$ be a finite
    group with order not divisible by $p$. Let $V$ be
    a $p$-elementary abelian group. Assume that $K$ acts
    by automorphism on $V$. If $U$ be a $K$-invariant subgroup of $V$, 
    then there exists a $K$-invariant subgroup $N$ of $V$ 
    such that $V=U\times N$. 
\end{corollary}

\begin{proof}

\end{proof}

If $G$ is a finite group, 
then $\C[G]$ is semisimple. By Artin--Wedderburn's theorem, 
\[
\C[G]\simeq\prod_{i=1}^r M_{n_i}(\C),
\]
where $r$ is the number of isomorphism classes of simple modules of $\C[G]$. Moreover, 
$|G|=\dim\C[G]=\sum_{i=1}^r n_i^2$. By convention, 
we always assume that $n_1=1$. 
This corresponds, of course, to the \textbf{trivial module}. 

\begin{theorem}
    Let $G$ be a finite group. The number of simple 
    modules of $\C[G]$ coincides with the number of conjugacy classes of $G$. 
\end{theorem}

\begin{proof}

\end{proof}

\begin{exercise}
    Let $G$ be a finite group of order $n$ with $k$ conjugacy classes.
    Let $m=(G:[G,G])$. Prove that $m+3m\geq4k$. 
\end{exercise}

\begin{exercise}
    Prove that $\C[C^4]\simeq\C^4$. 
\end{exercise}

For $n\geq1$ let $\Sym_n$ denote the symmetric group in $n$ letters. 

\begin{example}
    The group $\Sym_3$ has three conjugacy classes:
    $\{\id\}$, $\{(12),(13),(23)\}$ and $\{(123),(132)\}$. 
    Since $6=1^2+a^2+b^2$, it follows that 
    $\C[G]\simeq\C\times\C\times M_2(\C)$. 
\end{example}    
