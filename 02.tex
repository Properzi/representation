\chapter{}

\topic{Group algebras}

\index{Group algebra}
Let $G$ be a finite group. The (complex) \textbf{group
algebra} $\C[G]$ is the $\C$-vector space with
basis $\{g:g\in G\}$ and multiplication
\[
\left(\sum_{g\in G}\lambda_gg\right)\left(\sum_{h\in G}\mu_hh\right)
=\sum_{g,h\in G}\lambda_g\mu_h(gh).
\]

Clearly, $\dim \C[G]=|G|$. Moreover, 
$\C[G]$ is commutative if and only if $G$ is abelian. 

\index{Augmentation ideal}
If $G$ is non-trivial, 
then $\C[G]$ contains proper non-trivial ideals. For example, 
the \textbf{augmentation ideal} 
\[
I(G)=\left\{\sum_{g\in G}\lambda_gg\in \C[G]:\sum_{g\in G}\lambda_g=0\right\}
\]
is a non-zero proper ideal of $\C[G]$. 

\begin{exercise}
Let $C_n$ be the cyclic group of order $n$ (written multiplicatively).
Prove that $\C[G]\simeq \C[X]/(X^n-1)$. 
\end{exercise}

\begin{exercise}
    Let $G$ be a finite non-trivial group. Prove that
    $\C[G]$ has zero divisors. 
\end{exercise}

\index{Module!semisimple}
Recall that a finite-dimensional module $M$ is semisimple 
if and only if for every submodule $S$ of $M$ there 
is a submodule $T$ of $M$ such that $M=S\oplus T$.    

\begin{theorem}[Maschke]
\index{Maschke's theorem}
    Let $G$ be a finite
    group and $M$ be a finite-dimensional $\C[G]$-module.
    Then $M$ is semisimple. 
\end{theorem}

\begin{proof}
We need to show that every submodule $S$ of $M$ admits a complement. 
Since $S$ is a subspace of $M$, there exists a subspace $T_0$ of $M$ 
such that $M=S\oplus T_0$ (as vector spaces). We use 
$T_0$ to construct a submodule $T$ of $M$ that complements $S$. Since $M=S\oplus T_0$, 
every $m\in M$ can be written uniquely as $m=s+t_0$ for some $s\in S$ and $t_0\in T$. 
Let 
\[
p_0\colon M\to S,\quad
p_0(m)=s,
\]
where $m=s+t_0$ with $s\in S$ and $t_0\in T$. 
If $s\in S$, then $p_0(s)=s$. In particular, $p_0^2=p_0$, as 
$p_0(m)\in S$. 

Note that, in general, $p_0$ is not a $K[G]$-modules homomorphism. 
Let 
\[
p\colon M\to S,\quad
p(m)=\frac{1}{|G|}\sum_{g\in G}g^{-1}\cdot p_0(g\cdot m).
\]

We claim that $p$ is a homomorphism of $K[G]$-modules. For that purpose, we need to show that 
$p(g\cdot m)=g\cdot p(m)$ for all $g\in G$ and $m\in M$. In fact, 
\[
p(g\cdot m)=\frac{1}{|G|}\sum_{h\in G}h^{-1}\cdot p_0(h\cdot (g\cdot m))
=\frac{1}{|G|}\sum_{h\in G}(gh^{-1})\cdot p_0(h\cdot m)=g\cdot p(m).
\]

We now claim that $p(M)=S$. The inclusion $\subseteq$ is trivial to prove, as $S$ is a submodule of $M$ 
and $p_0(M)\subseteq S$. Conversely, if $s\in S$, then $g\cdot s\in S$, as 
$S$ is a submodule. Thus 
$s=g^{-1}\cdot (g\cdot s)=g^{-1}\cdot p_0(g\cdot s)$ and hence 
\[
s=\frac{1}{|G|}\sum_{g\in G}g^{-1}\cdot (g\cdot s)=\frac{1}{|G|}\sum_{g\in G}g^{-1}\cdot (p_0(g\cdot s))=p(s).
\]
Since $p(m)\in S$ for all $m\in M$, it follows that $p^2(m)=p(m)$, so $p$ is a projector onto $S$. 
Hence $S$ admits a complement in $M$, that is $M=S\oplus\ker(p)$.
\end{proof}

\begin{exercise}
Let $G=\langle g\rangle$ be the cyclic group 
of order four and $\rho_g=\begin{pmatrix}
0&-1\\
1&0\end{pmatrix}$. 
Let $M=\R^{2\times 1}$ as an $\C[G]$-module with 
\[
g\cdot\begin{pmatrix}u\\v\end{pmatrix}
=\begin{pmatrix}-v\\u\end{pmatrix}.
\]
Prove that $M$ is a semisimple $\C[G]$-module that is not simple.  
\end{exercise}

\begin{exercise}
Let $G=\langle g\rangle$ be the cyclic group 
of order four and $\rho_g=\begin{pmatrix}
0&-1\\
1&0\end{pmatrix}$. 
Let $M=\R^{2\times 1}$ as an $\R[G]$-module with 
\[
g\cdot\begin{pmatrix}u\\v\end{pmatrix}
=\begin{pmatrix}-v\\u\end{pmatrix}.
\]
Prove that $M$ is a simple $\R[G]$-module. 
\end{exercise}

There is a multiplicative version of Maschke's theorem. A group $G$ acts 
by automorphisms on $A$ if there is a group homomorphism 
$\lambda\colon G\to\Aut(A)$. In this case, a subgroup $B$ of $A$ is said to be 
$G$-invariant if $\lambda(B)\subseteq B$. 

\begin{theorem}
\index{Maschke's theorem!multiplicative version}
    Let $K$ be a finite group of order $m$. Assume that 
    $K$ acts by automorphisms pn $V=U\times W$, where
    $U$ and $W$ are subgroups of $V$ and $U$ is abelian and $K$-invariant. 
    If the map $U\to U$, $u\mapsto u^m$, is bijective, 
    then there exists a normal $K$-invariant subgroup $N$ of $V$ 
    such that $V=U\times N$. 
\end{theorem}

\begin{proof}
Let $\theta\colon U\times W\to U$, $(u,w)\mapsto u$. Then $\theta$ is a group homomorphism such that 
$\theta(u)=u$ for all $u\in U$. Since $U$ is $K$-invariant, 
\[
k^{-1}\cdot \theta(k\cdot v)\in U
\]
for all $k\in K$ and $v\in V$. 
Since $K$ is finite and $U$ is abelian, 
the map 
\[
\varphi\colon V\to U,\quad 
v\mapsto \prod_{k\in K}k^{-1}\cdot \theta(k\cdot v), 
\]
is well-defined. 
We claim that $\varphi$ is a group homomorphism. If $x,y\in V$, then 
\begin{align*}
    \varphi(xy) &= \prod_{k\in K}k^{-1}\cdot \theta(k\cdot (xy))\\
    &= \prod_{k\in K}k^{-1}\cdot (\theta(k\cdot x)\theta(k\cdot y))\\
    &= \prod_{k\in K}k^{-1}\cdot \theta(k\cdot x) \prod_{k\in K}k^{-1}\cdot \theta(k\cdot y)=\varphi(x)\varphi(y),
\end{align*}
since $U$ is abelian and $K$ acts by automorphisms on $V$. 

We claim that $N=\ker\varphi$ is $K$-invariant. 
We need to show that $\varphi(l\cdot x)=l\cdot\varphi(x)$ for all $l\in K$ and $x\in V$. 
If $l\in K$ and $x\in V$, then 
\begin{align*}
l^{-1}\cdot\varphi(l\cdot x)&=l^{-1}\cdot\left(\prod_{k\in K}k^{-1}\cdot \theta(k\cdot (l\cdot x))\right)=\prod_{k\in K}(kl)^{-1}\cdot\theta( (kl)\cdot x)=\varphi(x),
\end{align*}
since $kl$ runs over all the elements of $K$ whenever $k$ runs over all the elements of $K$.
In conclusion, $\ker\varphi$ is $K$-invariant. 

It remains to show that $V$ is the direct product of $U$ and $N$. By assumption, $U$ is normal in $V$. 
We first prove that $U\cap N=\{1\}$. If $u\in U$, then $k\cdot u\in U$ for all $k\in K$. This implies that 
$k^{-1}\cdot\theta(k\cdot u)=k^{-1}\cdot (k\cdot u)=u$. Hence $\varphi(u)=u^m$. Since this map is bijective by assumption,  
\[
U\cap N=U\cap\ker\varphi=\{1\}.
\]
We now show that $V\subseteq UN$, as the other inclusion is trivial. Since $N=\ker\varphi$,  
\[
\varphi(V)\subseteq U=\varphi(U)=\varphi(U)\varphi(N)=\varphi(UN) 
\]
and hence $V\subseteq (UN)N=UN$. 
Therefore $V$ is the direct product of $U$ and $N$, as $N$ is normal in $V$.
\end{proof}

\begin{corollary}
    Let $p$ be a prime number and $K$ be a finite
    group with order not divisible by $p$. Let $V$ be
    a $p$-elementary abelian group. Assume that $K$ acts
    by automorphism on $V$. If $U$ be a $K$-invariant subgroup of $V$, 
    then there exists a $K$-invariant subgroup $N$ of $V$ 
    such that $V=U\times N$. 
\end{corollary}

\begin{proof}
    Let $m=|K|$. Since $m$ and $|U|$ are coprime, the map 
    $u\mapsto u^m$ is bijective in $U$. Since $V$ is a vector space over the field 
    $\Z/p$, it follows that $V=U\times W$ for some subgroup $W$ of $V$. Now the claim follows
    from the previous theorem. 
\end{proof}

If $G$ is a finite group, 
then $\C[G]$ is semisimple. By Artin--Wedderburn theorem, 
\[
\C[G]\simeq\prod_{i=1}^r M_{n_i}(\C),
\]
where $r$ is the number of isomorphism classes of simple modules of $\C[G]$. Moreover, 
$|G|=\dim\C[G]=\sum_{i=1}^r n_i^2$. By convention, 
we always assume that $n_1=1$. 
This corresponds, of course, to the \textbf{trivial module}. 

\begin{theorem}
    Let $G$ be a finite group. The number of simple 
    modules of $\C[G]$ coincides with the number of conjugacy classes of $G$. 
\end{theorem}

\begin{proof}
    By Artin--Wedderburn theorem $\C[G]\simeq\prod_{i=1}^rM_{n_i}(\C)$. Thus 
    \[
		Z(\C[G])\simeq\prod_{i=1}^rZ(M_{n_i}(\C))\simeq\C^r.
	\]
	In particular, $\dim Z(\C[G])=r$. If $\alpha=\sum_{g\in
	G}\lambda_gg\in Z(\C[G])$, then $h^{-1}\alpha h=\alpha$ for all $h\in
	G$. Thus 
	\[
		\sum_{g\in G}\lambda_{hgh^{-1}}g=
		\sum_{g\in g}\lambda_g h^{-1}gh=\sum_{g\in G}\lambda_gg
	\]
	and hence $\lambda_{g}=\lambda_{hgh^{-1}}$ for all $g,h\in G$. A basis for 
	$Z(\C[G])$ is given by elements of the form 
	\[
		\sum_{g\in K}g,
	\]
	where $K$ is a conjugacy class of $G$. Therefore $\dim Z(\C[G])$ is equal to 
	the number of conjugacy classes of $G$.
\end{proof}

\begin{exercise}
    Let $G$ be a finite group of order $n$ with $k$ conjugacy classes.
    Let $m=(G:[G,G])$. Prove that $m+3m\geq4k$. 
\end{exercise}

For $n\in\Z_{\geq2}$ we write $C_n$ to denote the (multiplicative) cyclic group of order $n$. 

\begin{exercise}
    Prove that $\C[C^4]\simeq\C^4$. 
\end{exercise}

For $n\geq1$ let $\Sym_n$ denote the symmetric group in $n$ letters. 

\begin{example}
    The group $\Sym_3$ has three conjugacy classes:
    $\{\id\}$, $\{(12),(13),(23)\}$ and $\{(123),(132)\}$. 
    Since $6=1^2+a^2+b^2$, it follows that 
    $\C[G]\simeq\C\times\C\times M_2(\C)$. 
\end{example}    
