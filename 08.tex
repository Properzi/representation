\section{Lecture: Week 8}

\subsection{The correspondence theorem}

Let $N$ be a normal subgroup of $G$ 
and 
\[
\pi\colon G\to G/N,\quad 
g\mapsto gN,
\]
be the canonical map. 
If $\widetilde{\rho}\colon G/N\to\GL(V)$ 
is a representation of $G/N$ with 
character
$\widetilde{\chi}$, the composition 
$\rho=\widetilde{\rho}\pi\colon G\to \GL(V)$, $\rho(g)=\widetilde{\rho}(gN)$, 
is a representation of $G$. 
Thus
\[
\chi(g)=\trace{\rho_g}=\trace(\widetilde{\rho}_{gN})=\widetilde{\chi}(gN).
\]
In particular, $\chi(1)=\widetilde{\chi}(1)$. The character $\chi$ 
is the \emph{lifting} to $G$ of the character 
$\widetilde{\chi}$ of $G/N$. 

\begin{proposition}
If $\chi\in\Char(G)$, then 
\[
\ker\chi=\{g\in G:\chi(g)=\chi(1)\}
\]
is a normal subgroup of $G$. 
\end{proposition}

\begin{proof}
Let $\rho\colon G\to\GL_n(\C)$ be a representation with character $\chi$. Then 
$\ker\rho\subseteq\ker\chi$, as $\rho_g=\id$ implies 
$\chi(g)=\trace(\rho_g)=n=\chi(1)$. We claim that  
$\ker\chi\subseteq\ker\rho$. If $g\in G$ is such that $\chi(g)=\chi(1)$, since 
$\rho_g$ is diagonalizable, there exist eigenvalues $\lambda_1,\dots,\lambda_n\in\C$ such that
\[
n=\chi(1)=\chi(g)=\sum_{i=1}^n\lambda_i.
\]
Since each $\lambda_i$ is a root of one,  
$\lambda_1=\cdots=\lambda_n=1$. Hence $\rho_g=\id$. 
\end{proof}

\index{Kernel!of a character}
If $\chi$ is a character, the subgroup $\ker\chi$ 
is the \emph{kernel} of $\chi$. 

\begin{theorem}[Correspondence theorem]
\index{Correspondence theorem!for characters}
\label{thm:correspondence}
Let $N$ be a normal subgroup of a finite group $G$. There exists
a bijective correspondence 
\[
\Char(G/N) \longleftrightarrow \{\chi\in\Char(G): 
N\subseteq\ker\chi\}
\]
that maps irreducible characters to irreducible characters.
\end{theorem}

\begin{proof}
If $\widetilde{\chi}\in\Char(G/N)$, let $\chi$ be the lifting of $\widetilde{\chi}$ to $G$. If $n\in N$, 
then
\[
\chi(n)=\widetilde{\chi}(nN)=\widetilde{\chi}(N)=\chi(1)
\]
and thus $N\subseteq\ker\chi$. 

If $\chi\in\Char(G)$ is such that $N\subseteq\ker\chi$, let $\rho\colon G\to\GL(V)$ be a representation
with character $\chi$. 
Let $\widetilde{\rho}\colon G/N\to\GL(V)$, $gN\mapsto \rho(g)$. We claim that $\widetilde{\rho}$
is well-defined: 
\[
gN=hN\Longleftrightarrow h^{-1}g\in N\Longrightarrow\rho(h^{-1}g)=\id\Longleftrightarrow \rho(h)=\rho(g).
\]
Moreover, $\widetilde{\rho}$ is a representation, as 
\[
\widetilde{\rho}((gN)(hN))=\widetilde{\rho}(ghN)=\rho(gh)=\rho(g)\rho(h)=\widetilde{\rho}(gN)\widetilde{\rho}(hN).
\]
If $\widetilde{\chi}$ is the character of $\widetilde{\rho}$, then 
$\widetilde{\chi}(gN)=\chi(g)$.

We now prove that $\chi$ is irreducible if and only if 
$\widetilde{\chi}$ is irreducible. If $U$ is a subspace of $V$, then 
\begin{align*}
\text{$U$ is $G$-invariant}
%&\Longleftrightarrow g\cdot U\subseteq U\text{ for all $g\in G$}\\
&\Longleftrightarrow \rho(g)(U)\subseteq U\text{ for all $g\in G$}\\
&\Longleftrightarrow \widetilde{\rho}(gN)(U)\subseteq U\text{ for all $g\in G$}.
\shortintertext{Thus}
\chi\text{ is irreducible }&\Longleftrightarrow
\rho\text{ is irreducible }\\
&\Longleftrightarrow\widetilde{\rho}\text{ is irreducible }\Longleftrightarrow
\widetilde{\chi}\text{ is irreducible }\qedhere.
\end{align*}
\end{proof}

\begin{example}
    Let $G=\Sym_4$ and $N=\{\id,(12)(34),(13)(24),(14)(23)\}$. We know that $N$ is normal in $G$ 
    and that $G/N=\langle a,b\rangle\simeq\Sym_3$, where 
    $a=(123)N$ and $b=(12)N$. 
    The character table of $G/N$ is 
    \begin{center}
		\begin{tabular}{|c|rrr|}
			\hline
			%& $1$ & $3$ & $2$\tabularnewline
			& $N$ & $(12)N$ & $(123)N$ \tabularnewline
			\hline 
			$\widetilde{\chi}_{1}$ & $1$ & $1$ & $1$\tabularnewline
			$\widetilde{\chi}_{2}$ & $1$ & $-1$ & $1$ \tabularnewline
			$\widetilde{\chi}_{3}$ & $2$ & $0$ & $-1$ \tabularnewline
			\hline
		\end{tabular}
	\end{center}
    For each $i\in\{1,2,3\}$ we compute the lifting $\chi_i$ to $G$ of the character  
    $\widetilde{\chi}_i$ of $G/N$. 
    Since $(12)(34)\in N$ and $(13)(1234)=(12)(34)\in N$, 
    \begin{align*}
        \chi( (12)(34) )=\widetilde{\chi}(N),\quad
        \chi( (1234) )=\widetilde{\chi}((13)N)=\widetilde{\chi}((12)N).
    \end{align*}
    Since the characters $\widetilde{\chi_i}$ are irreducibles, 
    the liftings $\chi_i$ are also irreducibles. With this process
    we obtain the following irreducible characters of $G$:
    	\begin{center}
		\begin{tabular}{|c|rrrrr|}
			\hline
			& $1$ & $(12)$ & $(123)$ & $(12)(34)$ & $(1234)$ \tabularnewline
			\hline 
			$\chi_{1}$ & $1$ & $1$ & $1$ & 1 & 1\tabularnewline
			$\chi_{2}$ & $1$ & $-1$ & $1$ & 1 & -1 \tabularnewline
			$\chi_{3}$ & $2$ & $0$ & $-1$ & 2 & 0\tabularnewline
			\hline
		\end{tabular}
	\end{center}
\end{example}

The character table of a group can be used to find the lattice 
of normal subgroups. In particular, the character table detects simple groups. 

\begin{lemma}
    Let $G$ be a finite group and 
    let $g,h\in G$. Then $g$ and $h$ 
    are conjugate if and only if 
    $\chi(g)=\chi(h)$ for all
    $\chi\in\Char(G)$. 
\end{lemma}

\begin{proof}
    If $g$ and $h$ are conjugate, then $\chi(g)=\chi(h)$, as characters are class functions
    of $G$.
    Conversely, if $\chi(g)=\chi(h)$ for all $\chi\in\Char(G)$, then 
    $f(g)=f(h)$ for all class function $f$ of $G$, 
    as characters $G$ generate the space of class functions of $G$. In particular, 
    $\delta(g)=\delta(h)$, where
    \[
    \delta(x)=\begin{cases}
    1 & \text{if $x$ and $g$ are conjugate},\\
    0 & \text{otherwise}.
    \end{cases}
    \]
    This implies that $g$ and $h$ are conjugate.
\end{proof}

As a consequence, we get that 
\begin{equation}
\label{eq:kernels}
\bigcap_{\chi\in\Irr(G)}\ker\chi=\{1\}.
\end{equation}
Indeed, if $g\in\ker\chi$ for all $\chi\in\Irr(G)$, then $g=1$ since 
the lemma implies that $g$ and $1$ are conjugate
because 
$\chi(g)=\chi(1)$ for all $\chi\in\Irr(G)$.

\begin{proposition}
\label{pro:normal}
    Let $G$ be a finite group. 
    If $N$ is a normal subgroup of $G$, 
    then there exist characters
    $\chi_1,\dots,\chi_k\in\Irr(G)$ 
    such that
    \[
    N=\bigcap_{i=1}^k\ker\chi_i.
    \]
\end{proposition}

\begin{proof}
    Apply the previous remark to the group $G/N$ to obtain that 
    \[
    \bigcap_{\widetilde{\chi}\in\Irr(G/N)}\ker\widetilde{\chi}=\{N\}.
    \]
    Assume that $\Irr(G/N)=\{\widetilde{\chi}_1,\dots,\widetilde{\chi}_k\}$. 
    We lift the irreducible characters of $G/N$ to $G$ 
    to obtain (some) irreducible characters $\chi_1,\dots,\chi_k$ 
    of $G$ such that 
    \[
    N\subseteq\ker\chi_1\cap\cdots\cap\ker\chi_k.
    \]
    If $g\in\ker\chi_i$ for all $i\in\{1,\dots,k\}$, then 
    \[
    \widetilde{\chi}_i(N)=\chi_i(1)=\chi_i(g)=\widetilde{\chi}_i(gN)
    \]
    for all $i\in\{1,\dots,k\}$. This implies that
    \[
    gN\in\bigcap_{i=1}^k\ker\widetilde{\chi}_i=\{N\},
    \]
    that is $g\in N$. 
\end{proof}

\index{Group!simple}
Recall that a non-trivial group is \emph{simple} if it contains no non-trivial normal 
proper subgroups. Examples of simple groups are cyclic groups of prime order
and the alternating groups $\Alt_n$ for $n\geq5$. 
As a corollary of Proposition \ref{pro:normal}, 
we can use the character table to detect simple groups.

\begin{proposition}
    Let $G$ be a finite group. Then $G$ is not simple if and only if 
    there exists a non-trivial irreducible character $\chi$ such that
    $\chi(g)=\chi(1)$ 
    for some $g\in G\setminus\{1\}$. 
\end{proposition}

\begin{proof}
    If $G$ is not simple, there exists a normal subgroup $N$ of $G$ such that
    $N\ne G$ and $N\ne\{1\}$. 
    By Proposition \ref{pro:normal}, there exist characters 
    $\chi_1,\dots,\chi_k\in\Irr(G)$
    such that 
    $N=\ker\chi_1\cap\cdots\cap\ker\chi_k$.
    In particular, there exists a non-trivial character
    $\chi_i$ such that $\ker\chi_i\ne\{1\}$. Thus 
    there exists $g\in G\setminus\{1\}$ such that
    $\chi_i(g)=\chi_i(1)$. 
    
    Assume now that there exists a non-trivial irreducible character $\chi$ 
    such that $\chi(g)=\chi(1)$ for some $g\in G\setminus\{1\}$. In particular, $g\in\ker\chi$ 
    and hence $\ker\chi\ne\{1\}$. Since $\chi$ is non-trivial, $\ker\chi\ne G$. 
    Thus $\ker\chi$ is a proper non-trivial normal subgroup of $G$.
\end{proof}

\begin{example}
\index{Mathieu's group $M_9$}
    If there exists a group $G$ with
    a character table 
    of the form
    \begin{center}
		\begin{tabular}{|c|rrrrrr|}
			\hline
			$\chi_{1}$ & 1 & 1 & 1 & 1 & 1 & 1\tabularnewline
			$\chi_{2}$ & 1 & 1 & 1 & -1 & 1 & -1 \tabularnewline
			$\chi_{3}$ & 1 & 1 & 1 & 1 & -1 & -1\tabularnewline
		    $\chi_{4}$ & 1 & 1 & 1 & -1 & -1 & 1\tabularnewline
			$\chi_{5}$ & 2 & -2 & 2 & 0 & 0 & 0\tabularnewline
			$\chi_{6}$ & 8 & 0 & -1 & 0 & 0 & 0\tabularnewline
			\hline
		\end{tabular}
	\end{center}
	then $G$ cannot be simple. Note that such a group $G$ would have order $\sum_{i=1}^6\chi_i(1)^2=72$. 
	Mathieu's group $M_{9}$ has precisely this character table! 
\end{example}

\begin{example}
    Let $\alpha=\frac{1}{2}(-1+\sqrt{7}i)$. 
    If there exists a group $G$ with a character table
    of the form
    \begin{center}
		\begin{tabular}{|c|rrrrrr|}
			\hline
			$\chi_{1}$ & 1 & 1 & 1 & 1 & 1 & 1\tabularnewline
			$\chi_{2}$ & 7 & -1 & -1 & 1 & 0 & 0 \tabularnewline
			$\chi_{3}$ & 8 & 0 & 0 & -1 & 1 & 1\tabularnewline
		    $\chi_{4}$ & 3 & -1 & 1 & 0 & $\alpha$ & $\overline{\alpha}$ \tabularnewline
			$\chi_{5}$ & 3 & -1 & 1 & 0 & $\overline{\alpha}$ & $\alpha$\tabularnewline
			$\chi_{6}$ & 6 & 2 & 0 & 0 & 0 & 0\tabularnewline
			\hline
		\end{tabular}
	\end{center}    
	then $G$ is simple. Note that such a group $G$ would have order 
	$\sum_{i=1}^6\chi_i(1)^2=168$. 
	The group  
	\[
	\PSL_2(7)=\SL_2(7)/Z(\SL_2(7))
	\]
	is a simple group that has precisely this character table!  
\end{example}

\subsection{Frobenius' groups}
\label{Frobenius}

If $p$ is a prime number, then
the units $(\Z/p)^{\times}$ 
of $\Z/p$ form a multiplicative group. Moreover, 
$(\Z/p)^{\times}$ 
is cyclic of order $p-1$. 

Let 
\[
G=\left\{\begin{pmatrix}
x & y\\
0 & 1
\end{pmatrix}
:x\in(\Z/p)^\times,\,y\in\Z/p\right\}.
\]
Then $G$ is a group with the usual matrix multiplication
and $|G|=p(p-1)$. 
Let $p$ and $q$ be prime numbers such that $q$ divides $p-1$, 
$z\in\Z$ be an element of multiplicative order $q$ modulo $p$ 
and 
\[
a=\begin{pmatrix}
1&1\\
0&1
\end{pmatrix},
\quad
b=\begin{pmatrix}
z&1\\
0&1
\end{pmatrix},
\quad
H=\langle a,b\rangle.
\]
A direct calculation shows that 
\begin{equation}
\label{eq:pq}
a^p=b^q=\begin{pmatrix}
1&0\\
0&1
\end{pmatrix},
\quad
bab^{-1}=\begin{pmatrix}
1&z\\
0&1
\end{pmatrix}
=a^z.
\end{equation}
Every element of $H$ is of the form $a^ib^j$ for $i\in\{0,\dots,p-1\}$ and  $j\in\{0,\dots,q-1\}$. 
Thus $|H|=pq$. Using~\eqref{eq:pq} we can compute 
the multiplication table of $G$. 

\begin{exercise}
    Let $p$ and $q$ be prime numbers such that $q$ divides $p-1$. Let
    $u,v\in\Z$ be elements of order $q$ modulo $p$. 
    Prove that 
    \[
    \langle a,b:a^p=b^q=1,bab^{-1}=a^u\rangle
    \simeq \langle a,b:a^p=b^q=1,bab^{-1}=a^v\rangle.
    \]
\end{exercise}

The group   
\[
F_{p,q}=\langle a,b:a^p=b^q=1,bab^{-1}=a^u\rangle,
\]
where $u\in\Z$ has order $q$ modulo $p$, 
is a particular case of a  
\emph{Frobenius group}. 

\begin{proposition}
\label{pro:Frobenius_pq}
    Let $p$ and $q$ be prime numbers such that $p>q$. Let  
    $G$ be a group of order $pq$. Then either $G$ is abelian or
    $q$ divides $p-1$ and 
    $G\simeq F_{p,q}$.
\end{proposition}

\begin{proof}
    Assume that $G$ is not abelian. By Sylow's theorems, 
    $q$ divides $p-1$ and there exists 
    a unique Sylow $p$-subgroup $P$ of $G$. Let $a,b\in G$ be such that 
    $P=\langle a\rangle\simeq\Z/p$ and $G/P=\langle bP\rangle\simeq\Z/q$. By Lagrange's theorem, 
    $G=\langle a,b\rangle$. We compute the order of $b^q$. Since 
    $G$ is not cyclic (because it is not abelian) and $b^q\in P$, 
    we conclude that $|b^q|=1$. 
    Since $P$ is normal in $G$, 
    $bab^{-1}\in P$ and hence $bab^{-1}=a^z$ for some $z\in\Z$. Therefore
    $b^qab^{-q}=a^{z^q}$. This implies that 
    $z^q\equiv1\bmod p$. The order of $z$ in $(\Z/p)^{\times}$ divides 
    $q$ and hence it is equal to $q$ (otherwise, $z=1$ and thus $bab^{-1}=a$, which implies
    that $G$ is abelian). In conclusion, 
    $G\simeq F_{p,q}$. 
\end{proof}

Using Proposition~\ref{pro:Frobenius_pq}, we can prove, for example, that every group of order 15 is abelian. We can also show that, up to isomorphism, $\Z/20$ and $F_{5,4}$ are the only groups of order 20.

\begin{definition}
  \index{Frobenius!complement}
  \index{Frobenius!kernel}
  \index{Frobenius!group}
  We say that a finite group $G$ is a 
  \emph{Frobenius group} if $G$ 
  has a non-trivial proper subgroup $H$ such that $H\cap
  xHx^{-1}=\{1\}$ for all $x\in G\setminus H$. In this case, the subgroup 
  $H$ is called a \emph{Frobenius complement}.
\end{definition}

\index{Malnormal subgroup}
A subgroup $H$ such that $gHg^{-1}\cap H=\{1\}$ for all 
$g\not\in H$ is called a \emph{malnormal} subgroup. 
Note that if $H$ is malnormal, then $N_G(H)=H$. 

\begin{exercise}
\label{xca:malnormal}
    Let $G$ be a group and $H$ be a subgroup of $G$. Prove that the following statements are equivalent: 
    \begin{enumerate}
        \item $H$ is malnormal. 
        \item The action of $H$ 
            on $G/H\setminus\{H\}$ by left multiplication is free. 
        \item Any $g\in G\setminus\{1\}$ has zero
            or one fixed point on $G/H$. 
    \end{enumerate}
\end{exercise}

For any group $G$, the subgroups $\{1\}$ and $G$ are malnormal in $G$. Moreover, they are the only subgroups of $G$ that are both normal and malnormal

\begin{exercise}
    Let $G$ be a group. Prove the following statements:
    \begin{enumerate}

        \item If $H$ is malnormal in $G$, then
        $gHg^{-1}$ is malnormal in $G$ for all $g\in G$. 
        \item If $H$ is malnormal in $G$ and 
        $K$ is malnormal in $H$, then $K$ is malnormal
        in $G$. 
        \item The intersection of malnormal
        subgroups is malnormal.
        \item If $H$ is malnormal in $G$ and 
        $S$ is a subgroup of $G$, then 
        $H\cap S$ is malnormal in $S$. 
        
    \end{enumerate}
\end{exercise}

\begin{example}
    Let 
    \[
    G=\left\{\begin{pmatrix}a&b\\0&1\end{pmatrix}:a\in\R^{\times},b\in\R\right\}\quad\text{and}\quad 
    H=\left\{\begin{pmatrix}a&0\\0&1\end{pmatrix}:a\in\R^{\times}\right\}\subseteq G. 
    \]
    Let $g=\begin{pmatrix}x&y\\0&1\end{pmatrix}\in G\setminus H$. Then $y\ne 0$. Since  
    \[
    g\begin{pmatrix}a&0\\0&1\end{pmatrix}g^{-1}
    =\begin{pmatrix}a&-ay+y\\0&1\end{pmatrix},
    \]
    it follows that the subgroup $H$ 
    is malnormal in $G$. 
\end{example}

\begin{exercise}
\label{xca:malnormal_center}
    Let $G$ be a group and $H$ be a non-trivial
    subgroup of $G$. Prove that if $Z(G)\ne\{1\}$, then
    $H$ is not malnormal in $G$. 
\end{exercise}

\begin{bonus}
\label{xca:malnormal_no2torsion}
    Let $G$ be a group with no 2-torsion 
    that contains a normal infinite cyclic group. Prove 
    that $G$ cannot contain a non-trivial proper malnormal subgroup. 
\end{bonus}


\begin{example}
    Let $G$ be a finite group 
    and $P\in\Syl_p(G)$ be such that $|P|=p$ and $N_G(P)=P$. Then $G$ is a Frobenius group
    with complement $P$. 
\end{example}

The previous example shows that 
$\Alt_4$ is a Frobenius group
with complement $\langle(123)\rangle$. Another situation
where the example applies is the dihedral
group 
\[
\D_{2n+1}=\langle r,s:r^{2n+1}=s^2=1,srs=r^{-1}\rangle
\]
of order $2(2n+1)$. It follows that
$\D_{2n+1}$ is a Frobenius
group with complement $\langle b\rangle$. 

\begin{theorem}[Frobenius]
  \label{thm:Frobenius}
  \index{Frobenius'!theorem}
  Let $G$ be a Frobenius group with complement $H$. Then
  \[
	N=\left( G\setminus\bigcup_{x\in G}xHx^{-1}\right)\cup\{1\}
  \]
  is a normal subgroup of $G$.
\end{theorem}

\begin{proof}
  For each $\chi\in\Irr(H)$, $\chi\ne1_H$, let 
  $\alpha=\chi-\chi(1)1_H\in\cf(H)$, where $1_H$ denotes the trivial character of $H$. 

  We claim that $\Res_H^G\Ind_H^G\alpha=\alpha$.
  First, $\Ind_H^G\alpha(1)=\alpha(1)=0$. If $h\in H\setminus\{1\}$, then 
  \[
    \Ind_H^G\alpha(h)=\frac{1}{|H|}\sum_{\substack{x\in G\\x^{-1}hx\in H}}\alpha(x^{-1}hx)
    =\frac{1}{|H|}\sum_{x\in H}\alpha(h)=\alpha(h),
  \]
  since, if $x\not\in H$, then $x^{-1}hx\in H$ implies that 
  $h\in H\cap xHx^{-1}=\{1\}$.

  By Frobenius' reciprocity, 
  \begin{equation}
    \label{eq:<a,a>=1+chi2}
    \langle\Ind_H^G\alpha,\Ind_H^G\alpha\rangle
    =\langle\alpha,\Res_H^G\Ind_H^G\alpha\rangle=\langle\alpha,\alpha\rangle
    =1+\chi(1)^2.
  \end{equation}
  Again, by Frobenius' reciprocity, 
  \[
  \langle\Ind_H^G\alpha,1_G\rangle
  =\langle\alpha,\Res_H^G1_G\rangle
  =\langle\alpha,1_H\rangle
  =\langle\chi-\chi(1)1_H,1_H\rangle
  =-\chi(1),
  \]
  where $1_G$ is the trivial character of $G$. If we write 
  \[
  \Ind_H^G\alpha=\sum_{\eta\in\Irr(G)}\langle\Ind_H^G\alpha,\eta\rangle\eta
  =\langle\Ind_H^G\alpha,1_G\rangle1_G+\underbrace{\sum_{\substack{1_G\ne\eta\\\eta\in\Irr(G)}}\langle\Ind_H^G\alpha,\eta\rangle\eta}_{\phi},
  \]
  then $\Ind_H^G\alpha=-\chi(1)1_G+\phi$, where $\phi$ is a linear combination of non-trivial 
  irreducible characters of $G$. We compute 
  \[
  1+\chi(1)^2=\langle\Ind_H^G\alpha,\Ind_H^G\alpha\rangle
  =\langle\phi-\chi(1)1_G,\phi-\chi(1)1_G\rangle
  =\langle\phi,\phi\rangle+\chi(1)^2
  \]
  and hence $\langle\phi,\phi\rangle=1$. 
  
  \begin{claim}
  If $\eta\in\Irr(G)$ is such that $\eta\ne 1_G$, then 
  $\langle\Ind_H^G\alpha,\eta\rangle\in\Z$. 
  \end{claim}
  
  By Frobenius' reciprocity, $\langle\Ind_H^G\alpha,\eta\rangle=\langle\alpha,\Res_H^G\eta\rangle$. 
  If we decompose $\Res_H^G\eta$ into irreducibles of $H$, say 
  \[
  \Res_H^G\eta=m_11_H+m_2\chi+m_3\theta_3+\cdots+m_t\theta_t
  \]
  for some $m_1,m_2,\dots,m_t\geq0$, 
  then, since 
  \begin{align*}
  \langle\alpha,1_H\rangle=\langle\chi-\chi(1)1_H,1_H\rangle=-\chi(1),
  &&\langle\alpha,\chi\rangle=\langle\chi-\chi(1)1_H,\chi\rangle=1,
  \end{align*}
  and 
  \[
  \langle\alpha,\theta_j\rangle=\langle\chi-\chi(1)1_H,\theta_j\rangle=0
  \]
  for all $j\in\{3,\dots,t\}$, we conclude that 
  \[
  \langle\Ind_H^G\alpha,\eta\rangle=-m_1\chi(1)+m_2\in\Z.
  \]
  
  \begin{claim}
  $\phi\in\Irr(G)$.
  \end{claim}
  
  Since $\langle\Ind_H^G\alpha,\eta\rangle\in\Z$ for all $\eta\in\Irr(G)$ such that 
  $\eta\ne 1_G$ and 
  \[
  1=\langle\phi,\phi\rangle
  =\sum_{\substack{\eta,\theta\in\Irr(G)\\\eta,\theta\ne1_G}}\langle\Ind_H^G\alpha,\eta\rangle\langle\Ind_H^G\alpha,\theta\rangle\langle\eta,\theta\rangle
  =\sum_{\substack{\eta\ne 1_G\\\eta\in\Irr(G)}}\langle\Ind_H^G\alpha,\eta\rangle^2,
  \]
  there is a unique $\eta\in\Irr(G)$ such that 
  $\langle\Ind_H^G\alpha,\eta\rangle^2=1$ and all the other products are zero, 
  that is 
  $\phi=\pm\eta$ for some $\eta\in\Irr(G)$. Since 
  \[
  \chi-\chi(1)1_H=\alpha=\Res_H^G\Ind_H^G\alpha=\Res_H^G(\phi-\chi(1)1_G)=\Res_H^G\phi-\chi(1)1_H,
  \]
  it follows that $\phi(1)=\Res_H^G\phi(1)=\chi(1)\in\Z_{\geq1}$. Thus $\phi\in\Irr(G)$. 

  \medskip
  We have proved that if $\chi\in\Irr(H)$ is such that $\chi\ne 1_H$, then 
  there exists $\phi_\chi\in\Irr(G)$ such that $\Res_H^G(\phi_\chi)=\chi$. 
  
  \medskip
  We prove that $N$ is equal to 
  \[
	M=\bigcap_{\substack{\chi\in\Irr(H)\\\chi\ne1_H}}\ker\phi_{\chi}.
  \]

  We first prove that $N\subseteq M$. 
  Let $n\in N\setminus\{1\}$ and $\chi\in\Irr(H)\setminus\{1_H\}$. Since $n$ 
  does not belong to a conjugate of 
  $H$, 
  \[
	\Ind_H^G\alpha(n)=\frac{1}{|H|}\sum_{\substack{x\in G\\x^{-1}nx\in H}}\alpha(x^{-1}nx)=0, 
  \]
  as $n\in N$ implies that the set $\{x\in G:x^{-1}nx\in H\}$ is empty. Since 
  \[
  0=\Ind_H^G\alpha(n)
  =\phi_{\chi}(n)-\chi(1)=\phi_{\chi}(n)-\phi_{\chi}(1),
  \]
  we conclude that $n\in\ker\phi_{\chi}$. 
  
  We now prove that $M\subseteq N$. 
  Let $h\in M\cap H$ and $\chi\in\Irr(H)\setminus\{1_H\}$. Then 
  \[
    \phi_{\chi}(h)-\chi(1)=\Ind_H^G\alpha(h)=\alpha(h)=\chi(h)-\chi(1),
  \]
  and $h\in\ker\chi$, as
  \[
    \chi(h)=\phi_{\chi}(h)=\phi_{\chi}(1)=\chi(1).
  \]
  Therefore 
  \[
  h\in\bigcap_{\chi\in\Irr(H)}\ker\chi=\{1\}.
  \]
  By~\eqref{eq:kernels}, the kernels
  of irreducible characters have trivial intersection. 
  We now prove that $M\cap
  xHx^{-1}=\{1\}$ for all $x\in G$. Let $x\in G$ and $m\in M\cap xHx^{-1}$. Since 
  $m=xhx^{-1}$ for some $h\in H$, $x^{-1}mx\in H\cap M=\{1\}$.  This implies that 
  $m=1$.
\end{proof}

There is no proof of Frobenius’ theorem that is  independent of character theory. Purely group-theoretic proofs exist in cases where the Frobenius complement has even order or is solvable. The Feit--Thompson theorem (which relies heavily on character theory and is significantly more difficult than Frobenius’ theorem) implies that these two cases cover all possibilities. 

In 2013, Terence Tao discovered an alternative proof of Frobenius’ theorem, though it resembles the original character-theoretic approach.

\begin{definition}
  \index{Frobenius!kernel}
  Let $G$ be a Frobenius group. The normal subgroup 
  $N$ of Frobenius' theorem is called the \emph{Frobenius kernel}. 
\end{definition}

\begin{corollary}
  Let $G$ be a Frobenius group with complement $H$. 
  Then there exists a normal subgroup $N$ of $G$ 
  such that 
  $G=HN$ and $H\cap N=\{1\}$.
\end{corollary}

\begin{proof}
  Frobenius' theorem yields the subgroup $N$. Since 
  $H\cap gHg^{-1}=\{1\}$ for all $g\in G\setminus H$, 
  it follows that 
  $N_G(H)=H$. It follows that $H$
  has $(G:H)$ conjugates. 
  Let 
  \[
  N=\left( G\setminus\bigcup_{x\in G}xHx^{-1}\right)\cup\{1\}.
  \]
  Then  
  $|N|=|G|-(G:H)(|H|-1)=(G:H)$.
  Since, moreover, $N\cap H=\{1\}$, we conclude that
  \[
  |HN|=|N||H|/|H\cap N|=|N||H|=|G|.
  \]
  Therefore $G=NH$.
\end{proof}

\subsection{The Cameron--Cohen theorem, II}

In this section, we use Frobenius’ theorem to strengthen the Cameron--Cohen theorem on derangements (Theorem~\ref{thm:CameronCohen}). To do so, we first require an alternative version of Frobenius’ theorem.

\begin{corollary}[Frobenius]
    \label{cor:Frobenius_combinatorio}
    \index{Frobenius'!theorem}
    Let $G$ be a group acting transitively on a finite set $X$. 
    Assume that each $g\in G\setminus\{1\}$ fixes 
    at most one element of 
     $X$. The set $N$ formed by the identity and the derangements 
     of $G$ is a normal subgroup of $G$.
\end{corollary}

\begin{proof}
  Let $x\in X$ and $H=G_x$. We claim that 
  if $g\in G\setminus H$, then $H\cap
  gHg^{-1}=\{1\}$. If $h\in H\cap gHg^{-1}$, then
  $h\cdot x=x$ and $(g^{-1}hg)\cdot
  x=x$. Since $g\cdot x\ne x$, $h$ fixes two elements of
  $X$. Thus 
  $h=1$, as every non-trivial element fixes at most one element of $X$. 

  By Theorem~\ref{thm:Frobenius}, 
  \[
    N=\left(G\setminus\bigcup_{g\in G}gHg^{-1}\right)\cup\{1\}
  \]
  is a subgroup of $G$. Let us compute the elements of $N$. If 
  $h\in\bigcup_{g\in G}gHg^{-1}$, then there exists  $g\in G$ such that $g^{-1}hg\in H$,
  that is $(g^{-1}hg)\cdot x=x$; equivalently, 
  $h\in G_{g\cdot x}$. Therefore, the 
  non-identity elements of $N$ are the elements of $G$
  moving every element of $X$.
\end{proof}

\begin{example}
  Let $F$ be a finite field and $G$ be the group of maps 
  $f\colon F\to F$ of the form 
  $f(x)=ax+b$, $a,b\in F$ with $a\ne0$. The group $G$ acts on 
  $F$ and every 
  $f\ne\id$ fixes at most one element of $F$, as 
  \[
	x=f(x)=ax+b\implies a\ne 1\text{ and } x=b/(1-a).
  \]
  In this case, $N=\{f:f(x)=x+b\,,b\in F\}$ 
  is a subgroup of $G$.
\end{example}

\begin{exercise}
    Prove that Theorem~\ref{thm:Frobenius} can be obtained from
    Corollary~\ref{cor:Frobenius_combinatorio}.
\end{exercise}


% Wielandt 8.5.4
% 8.5.6 para ver algo de grupos de permutaciones
% 7.1 para ejemplo H(q)
% 10.5.6 (Thompson) N es nilpotente, se usa 10.5.4 

\begin{theorem}[Cameron--Cohen]
    Let $G\leq\Sym_n$ be a transitive subgroup. 
    If $n$ is not the power of a prime number, then
    $c_0>\frac{1}{n}$. 
\end{theorem}

\begin{proof}
    Let us go back to the proof
    of Theorem~\ref{thm:CameronCohen}. Assume that 
    $c_0=1/n$. Then
    \[
    \frac{1}{|G|}\sum_{g\in G}(\chi(g)^2-(n+1)\chi(g)+n)=1
    \]
    and hence $\frac{1}{|G|}\sum_{g\in G}\chi(g)^2=2$. Moreover, 
    since 
    \[
    \frac{1}{|G|}\sum_{g\in G_0}(\chi(g)-1)(\chi(n)-n)
    +\frac{1}{|G|}\sum_{g\in G\setminus G_0}(\chi(g)-1)(\chi(g)-n)=1,
    \]
    it follows that 
    \[
    \sum_{g\in G\setminus G_0}(\chi(g)-1)(\chi(g)-n)=0.
    \]
    Hence $(\chi(g)-1)(\chi(g)-n)=0$
    for all $g\in G\setminus G_0$. 
    
    By Corollary~\ref{cor:Frobenius_combinatorio}, 
    the subset $N=G_0\cup\{\id\}$ is a normal subgroup of $G$. Moreover, $G=N\rtimes H$ for some 
    subgroup $H$ of $G$ of order $n$. Since 
    $n=|H|=|N|-1$, $H$ acts freely and transitively 
    on $N\setminus\{1\}$. 

    We claim that $N$ is a $p$-group for some prime number $p$. Let $n,m\in N\setminus\{1\}$. Since $H$ is transitive on $N\setminus\{1\}$, 
    there exists $h\in H$ such that $h\cdot n=m$. Then
    \[
    |n|=|h\cdot n|=|m|,
    \]
    since for each $h\in H$, the map 
    $x\mapsto h\cdot x$ is an automorphism of $N$. Thus every two elements of $N\setminus\{1\}$ have 
    the same order. Let $p$ be a prime divisor 
    of $|N|$. By Cauchy's theorem, there exists 
    $n\in N$ such that $|n|=p$. Since all non-trivial
    elements of $N$ have the same order, 
    $N$ is a $p$-group. Therefore 
    $n=|N|$ is a power of a prime.
\end{proof}

In his doctoral thesis Thompson proved the following result, conjectured
by Frobenius. 

\begin{theorem}[Thompson]
\index{Thompson's theorem}
    Let $G$ be a Frobenius group. If $N$ is the Frobenius kernel, then $N$ 
    is nilpotent.
\end{theorem}

See~\cite[Theorem 6.24]{MR2426855} for the proof.

\begin{exercise}
\label{xca:Frobenius_size20}
Let $G$ be the group of matrices 
of the form $\begin{pmatrix}a&b\\0&1\end{pmatrix}$ where $a,b\in\Z/5$ and $a\ne 0$. Then $|G|=20$. Let 
\[
    h=\begin{pmatrix}
        2\\
        &1
    \end{pmatrix},\quad 
    k=\begin{pmatrix}
        1&1\\
        &1
    \end{pmatrix}.
\]
A direct calculation shows that 
$h^4=1$, $k^5=1$ and $hkh^{-1}=k^2$. Let $H=\langle h\rangle$ 
and $K=\langle k\rangle$. Prove the following statements: 

\begin{enumerate}
    \item Prove that $G=K\rtimes H$.
    \item Find the conjugacy classes of $G$: 
\begin{center}
        \begin{tabular}{cccccc}
             Size & $1$ & $4$ & $5$ & $5$ & $5$\\
             \hline 
             Representative & $1$ & $k$ & $h$ & $h^2$ & $h^3$\\
        \end{tabular}
\end{center}
\item Prove that $G/K$ is cyclic of order four. 
\item Prove that $[G,G]=K$. 
\item Use Theorem~\ref{thm:correspondence} on $G/K$ 
    to find the degree-one characters of $G$. 
\item Let $\chi\in\Irr(K)$ be such that $\chi(k)=\exp(2\pi i/5)$. Prove that 
$\Ind_K^G\chi\in\Irr(G)$. 
\end{enumerate}
\end{exercise}

% \begin{center}   
%         \begin{tabular}{|c|ccccc|}
%              \hline
%              & $1$ & $k$ & $h$ & $h^2$ & $h^3$\\
%              \hline
%              $\chi_1$ & $1$ & $1$ & $1$ & $1$ & $1$\\
%              $\chi_2$ & $1$ & $1$ & $i$ & $-1$ & $-i$\\
%              $\chi_3$ & $1$ & $1$ & $-1$ & $1$ & $-1$\\
%              $\chi_4$ & $1$ & $1$ & $-i$ & $-1$ & $i$\\
%              $\chi_5$ & $4$ & $-1$ & $0$ & $0$ & $0$\\
%              \hline
%         \end{tabular}
%     \end{center} 


