\chapter{}

\topic{Brauer--Fowler theorem}

\index{Symmetric}
\index{Antisymmetric}
Let $\rho\colon G\to\GL(V)$ 
be a representation with character $\chi$. The $\C[G]$-module $V\otimes V$ 
has character $\chi^2$. Let 
$\{v_1,\dots,v_n\}$ be a basis of $V$ and 
\[
T\colon V\otimes V\to V\otimes V,\quad
v_i\otimes v_j\mapsto v_j\otimes v_i.
\]
It is an exercise to check that 
\[
T(v\otimes w)=w\otimes v
\]
for all 
$v,w\in V$. It follows that  
$T$ does not depend on the chosen basis. Note that
$T$ is a homomorphism of $\C[G]$-modules, as
\[
T(g\cdot (v\otimes w))=T((g\cdot v)\otimes (g\cdot w))=(g\cdot w)\otimes (g\cdot v)=g\cdot T(v\otimes w)
\]
for all $g\in G$ y $v,w\in V$. 
In particular, the \textbf{symmetric part} 
\begin{gather*}
S(V\otimes V)=\{x\in V\otimes V:T(x)=x\}
\shortintertext{and the \textbf{antisymmetric} part}
A(V\otimes V)=\{x\in V\otimes V:T(x)=-x\}
\end{gather*}
of $V\otimes V$ are both  
$\C[G]$-submodules of $V\otimes V$. 
The terminology is motivated by the following fact:
\[
V\otimes V=S(V\otimes V)\oplus A(V\otimes V).
\]
In fact, 
$S(V\otimes V)\cap A(V\otimes V)=\{0\}$, as   
$x\in S(V\otimes V)\cap A(V\otimes V)$ implies
$x=T(x)$ and $x=-T(x)$. Hence $x=0$. Moreover, 
$V\otimes V=S(V\otimes V)+ A(V\otimes V)$, as every $x\in V\otimes V$ can be written 
as 
\[
x=\frac12(x+T(x))+\frac12(x-T(x))
\]
with $\frac12(x+T(x))\in S(V\otimes V)$ and $\frac12(x-T(x))\in A(V\otimes V)$. 

We claim that $\{v_i\otimes v_j+v_j\otimes v_i:1\leq i\leq j\leq n\}$ is
a basis of $S(V\otimes V)$ 
and that  
\[
\{v_i\otimes v_j-v_j\otimes v_i:1\leq i<j\leq n\}
\]
is a basis of $A(V\otimes V)$. Since both sets are linearly independent, 
\[
\dim S(V\otimes V)\geq n(n+1)/2\text{ and }
\dim A(V\otimes V)\geq n(n-1)/2.
\]
Moreover, 
\[
n^2=\dim (V\otimes V)=\dim S(V\otimes V)+\dim A(V\otimes V),
\]
so it follows that
$\dim S(V\otimes V)=n(n+1)/2$ and $\dim A(V\otimes V)=n(n-1)/2$. 

\begin{proposition}
    Let $G$ be a finite group and
    $V$ be a finite-dimensional 
    $\C[G]$-module with character $\chi$. If $S(V\otimes V)$ 
    has character $\chi_S$ and $A(V\otimes V)$ has character
    $\chi_A$, then 
    \begin{align*}
        &\chi_S(g)=\frac12(\chi^2(g)+\chi(g^2)) && \text{and} &&
        \chi_A(g)=\frac12(\chi^2(g)-\chi(g^2)).
    \end{align*}
\end{proposition}

\begin{proof}
    Let $g\in G$ and $\rho\colon G\to\GL(V)$ be the representation
    associated with $V$, that is $\rho(g)(v)=\rho_g(v)=g\cdot v$. 
    Since $\rho_g$ is diagonalizable, let $\{e_1,\dots,e_n\}$ 
    be a basis of eigenvectors of $\rho_g$, say
    $g\cdot e_i=\lambda_ie_i$ with $\lambda_i\in\C$ for all $i\in\{1,\dots,n\}$. In particular, $\chi(g)=\sum_{i=1}^n\lambda_i$. 
    
    Since $\{e_i\otimes e_j-e_j\otimes e_i:1\leq i<j\leq n\}$ is a basis of
    $A(V\otimes V)$ and 
    \[
    g\cdot (e_i\otimes e_j-e_j\otimes e_i)=\lambda_i\lambda_j(e_i\otimes e_j-e_j\otimes e_i),
    \]
    it follows that
    $\chi_A(g)=\sum_{1\leq i<j\leq n}\lambda_i\lambda_j$. On the other hand,
    $g^2\cdot e_i=\lambda_i^2e_i$ for all $i$,
    $\chi(g^2)=\sum_{i=1}^n\lambda_i^2$. Thus 
    \[
    \chi^2(g)=\chi(g)^2=\sum_{i=1}^n\sum_{j=1}^n\lambda_i\lambda_j=2\sum_{1\leq i<j\leq n}\lambda_i\lambda_j+\sum_{i=1}^n\lambda_i^2=2\chi_A(g)+\chi(g^2).
    \]
    Since $V\otimes V=S(V\otimes V)\oplus A(V\otimes V)$, it follows that  
    $\chi^2(g)=\chi_S(g)+\chi_A(g)$, that is 
    $\chi_S(g)=\frac12(\chi^2(g)+\chi(g^2))$.
\end{proof}

\index{Involution}
An \textbf{involution} of a group is an element $x\ne 1$ such that $x^2=1$. 
It is possible to use the character table to count the number
of involutions.

\begin{proposition}
    If $G$ is a finite group with $t$ involutions, then
    \[
        1+t=\sum_{\chi\in\Irr(G)}\langle\chi_S-\chi_A,\chi_1\rangle\chi(1),
    \]
    where $\chi_1$ is 
    the trivial character of $G$.
\end{proposition}

\begin{proof}
    Assume that $\Irr(G)=\{\chi_1,\dots,\chi_k\}$.  
    For $x\in G$ let 
    \[
    \theta(x)=|\{y\in G:y^2=x\}|.
    \]
    Since $\theta$ is a class function, 
    $\theta$ is a linear combination of the $\chi_j$'s, say 
    \[
    \theta=\sum_{\chi\in\Irr(G)}\langle\theta,\chi\rangle\chi.
    \]
    For every $\chi\in\Irr(G)$ we compute: 
    \begin{align*}
        \langle\chi_S-\chi_A,\chi_1\rangle 
        &=\frac{1}{|G|}\sum_{g\in G}\chi(g^2)\\
        &=\frac{1}{|G|}\sum_{x\in G}\sum_{\substack{g\in G\\g^2=x}}\chi(g^2)
        =\frac{1}{|G|}\sum_{x\in G}\theta(x)\chi(x)=\langle\theta,\chi\rangle.
    \end{align*}
    Thus $\theta=\sum_{\chi\in\Irr(G)}\langle\chi_S-\chi_A,\chi_1\rangle\chi$. Now
    the claim follows after evaluating this expression in 
    $x=1$. 
\end{proof}

\index{Cauchy--Schwarz inequality}
Before proving the Brauer-Fowler theorem, we
need a lemma. We will use the Cauchy--Schwarz inequality: 
\[
x_1,\dots,x_n\in\R\implies
\sum x_i^2\geq\frac{1}{n}(\sum x_i)^2.
\]

\begin{lemma}
    Let $G$ be a finite group with $k$ conjugacy classes. 
    If $t$ is the number of involutions of $G$, then
    $t^2\leq (k-1)(|G|-1)$. 
\end{lemma}

\begin{proof}
    Assume that $\Irr(G)=\{\chi_1,\dots,\chi_k\}$, where $\chi_1$ is the
    trivial character of $G$. 
    If $\chi\in\Irr(G)$, then 
    \[
        \langle\chi^2,\chi_1\rangle=\frac{1}{|G|}\sum_{g\in G}\chi(g)\chi(g)=\langle\chi,\overline{\chi}\rangle=\begin{cases}
        1 & \text{if $\chi=\overline{\chi}$},\\
        0 & \text{otherwise}.
        \end{cases}
    \]
    Since $\chi^2=\chi_S+\chi_A$, if $\langle\chi^2,\chi_1\rangle=1$, then
    the trivial character either is part of $\chi_S$ or $\chi_A$, but not both. 
    Thus
    \[
    \langle\chi_S-\chi_A,\chi_1\rangle\in\{-1,1,0\}.
    \]
    
    We claim that 
    $t\leq\sum_{i=2}^k\chi_i(1)$. In fact, since 
    $|\langle\chi_S-\chi_A,\chi_1\rangle|\leq 1$, 
    \begin{align*}
        1+t=\theta(1)
        &=\left|\sum_{\chi\in\Irr(G)}\langle\chi_S-\chi_A,\chi_1\rangle\chi(1)\right|\\
        &\leq\sum_{\chi\in\Irr(G)}|\langle\chi_S-\chi_A,\chi_1\rangle|\chi(1)
        \leq\sum_{\chi\in\Irr(G)}\chi(1).
    \end{align*}
    It follows that $t\leq\sum_{i=2}^k\chi_i(1)$. 
    By the Cauchy--Schwarz inequality, 
    \[
        t^2\leq\left(\sum_{i=2}^k\chi_i(1)\right)^2
        \leq(k-1)\sum_{i=2}^k\chi(1)^2=(k-1)(|G|-1).\qedhere
    \]
\end{proof}

Now we prove the Brauer--Fowler theorem. 

\begin{theorem}[Brauer--Fowler]
    \index{Brauer--Fowler theorem}
    Let $G$ be a finite simple group and $x$ be an involution of $G$. If $|C_G(x)|=n$, then $|G|\leq (n^2)!$	
\end{theorem}

\begin{proof}
    If $G$ is abelian, the claim is trivial. Let $G$ be a finite non-abelian simple group.
    We first assume the existence of a proper subgroup $H$ of $G$ 
    such that 
    \[
    (G:H)\leq n^2.
    \]
    The group $G$ acts on $G/H$ 
    by left multiplication, so there is a group homomorphism 
    $\rho\colon G\to\Sym_{n^2}$. Since $G$ is simple, either 
    $\ker\rho=\{1\}$ or $\ker\rho=G$. If $\ker\rho=G$, then
    $\rho(g)(yH)=yH$ for all $g\in G$ and $y\in G$. 
    Hence $H=G$, a contradiction. Therefore $\rho$ is injective
    and hence $G$ is isomorphic to a subgroup of $\Sym_{n^2}$. 
    In particular, $|G|$ divides $(n^2)!$. 

    Let $m=(|G|-1)/t$, where $t$ is the number of involutions of $G$. 
    Since $|C_G(x)|=n$, the group $G$ has at least $|G|/n$ involutions (because
    the conjugacy class of $x$ has size $|G|/n$ and all its elements are involutions), 
    that is $t\geq |G|/n$. Hence 
    \[
    m=(|G|-1)/t<n.
    \]
    It is enough to show that
    $G$ contains a subgroup of index $\leq m^2$. 

    Let $C_1,\dots,C_k$ be the conjugacy classes of $G$, where $C_1=\{1\}$. 
    Since $G$ is simple and non-abelian, $|C_i|>1$ 
    for all $i\in\{2,\dots,k\}$. By the previous lemma, 
    \[
    t^2\leq(k-1)(|G|-1)\implies |G|-1=\frac{mt^2}{t}\leq\frac{(k-1)(|G|-1)^2}{t^2}=(k-1)m^2.
    \]
    If $|C_i|>m^2$ for all $i\in\{2,\dots,k\}$, then
    \[
    |G|-1=\sum_{i=2}^k|C_i|>(k-1)m^2,
    \]
    a contradiction. Thus there exists a non-trivial conjugacy class
    $C$ of $G$ such that $|C|\leq m^2$. If $g\in C$, then
    $C_G(g)$ is a proper subgroup of $G$ of index $|C|\leq m^2$.
\end{proof}

The bound of the Brauer--Fowler theorem is not essential.
What matters is the following consequence:

\begin{corollary}
    Let $n\geq 1$ be an integer. There are at most finitely many 
    finite simple groups with an involution with a centralizer of order $n$.
\end{corollary}

As an exercise, a simple application: 

\begin{exercise}
    If $G$ is a finite simple group and $x$ is an involution with
    centralizer of order two, then  
    $G\simeq\Z/2$. 
\end{exercise}

\subsection{A comment: An elementary proof of Brauer--Fowler theorem}

We need to find a subgroup of index $\leq 2n^2$. 
Let $X$ be the conjugacy class of $x$. For $g\in G$ let
\[
J(g)=\{z\in X:zgz^{-1}=g^{-1}\}.
\]
We claim that $|J(g)|\leq|C_G(g)|$. The map $J(g)\to C_G(g)$, $z\mapsto gz$, 
is well-defined,~as 
\[
(gz)g(gz)^{-1}=g(zgz^{-1})g^{-1}=g^{-1}\in C_G(g).
\]
It is injective, as $gz=gz_1$ implies $z=z_1$.

Let $J=\{(g,z)\in G\times X:zgz^{-1}=g^{-1}\}$.  
Since $X\times X\to J$, $(y,z)\mapsto (yz,z)$, 
is well-defined (since $z(yz)z^{-1}=zy=(yz)^{-1}$) and
it is trivially injective, 
\[
|X|^2\leq |J|=\sum_{(g,z)\in J}1\leq\sum_{g\in G}|J(g)|
\leq\sum_{g\in G}|C_G(g)|=k|G|,
\]
where $k$ is the number of conjugacy classes of $G$, 
as $(g,z)\in J$ if and only if $z\in J(g)$. Thus $|G|\leq kn^2$, as
\[
\left(\frac{|G|}{|C_G(x)|}\right)^2=|X|^2=\frac{|G|^2}{n^2}\leq k|G|.
\]

\begin{claim}
    There exists a non-trivial conjugacy class with $\leq 2n^2$ elements.
\end{claim}

Assume that the claim is not true. Let
$C_1,\dots,C_k$ be the conjugacy classes of $G$, where 
$C_1=\{1\}$ and $|C_i|>2n^2$ for all $i\in\{2,\dots,k\}$. Then
\[
|G|=1+\sum_{i=2}^k|C_i|>1+\sum_{i=2}^k2n^2=1+(k-1)2n^2\geq |G|,
\]
a contradiction. 

\begin{claim}
    There exists a subgroup $H$ of $G$ such that
    $(G:H)\leq 2n^2$.
\end{claim}

Let $C$ be a conjugacy class of $G$ such that 
$|C|\leq 2n^2$. Let $g\in C$.  
Then $H=C_G(g)$ is a subgroup of $G$ such that
$(G:H)\leq 2n^2$. 
This finishes the proof of the Brauer--Fowler theorem. 



\topic{Frobenius's reciprocity}

We now present a very quick version of Frobenius'
reciprocity theorem. We first 
define the restriction of class functions. 

\begin{definition}
    Let $G$ be a finite group and $f\colon G\to\C$ be
    a map. For a subgroup $H$ of $G$, the \textbf{restriction}
    of $f$ to $H$ is the map 
    $=\Res_H^G=f|_H\colon H\to\C$, $h\mapsto f(h)$. 
\end{definition}

\begin{exercise}
\label{xca:restriction}
    Let $G$ be a finite group. Prove that
    the map $\Res_H^G\colon\cf(G)\to\cf(H)$, $f\mapsto\Res_H^G(f)$, 
    is a well-defined linear map. 
\end{exercise}

We now define induction. Let $G$ be a finite group
and $H$ be a subgroup of $G$. If $f\colon H\to\C$ is a map, 
then 
\[
\dot{f}(x)=\begin{cases}
    f(x) & \text{if $x\in H$},\\
    0 & \text{otherwise}.
    \end{cases}
\]
It is an exercise to prove that
the map $f\mapsto\dot{f}$ is linear. 

\begin{definition}
    Let $G$ be a finite group and $H$ be a subgroup of $G$. Let
    $f\colon H\to\C$ be
    a map. The \textbf{induction}
    of $f$ to $G$ is the map 
    \begin{align*}
      g\mapsto\Ind_H^Gf(g)=\frac{1}{|H|}\sum_{x\in G}\dot{f}(x^{-1}gx).
    \end{align*}
\end{definition}

\begin{exercise}
\label{xca:induction}
    Let $G$ be a finite group. Prove that
    the map $\Ind_H^G\colon\cf(H)\to\cf(G)$, $f\mapsto\Ind_H^G(f)$, 
    is a well-defined linear map. 
\end{exercise}

\begin{theorem}[Frobenius' reciprocity theorem]
\index{Frobenius' reciprocity theorem}
    Let $G$ be a finite group and $H$ be a subgroup of $G$. 
    If $a\in\cf(H)$ and $b\in\cf(G)$, then
    \[
    \langle\Ind_H^Ga,b\rangle=\langle a,\Res_H^Gb\rangle.
    \]
\end{theorem}

\begin{proof}
    It follows from a direct calculation:
    \begin{equation}
    \label{eq:reciprocity}
    \begin{aligned}
        \langle\Ind_H^Ga,b\rangle 
        &= \frac{1}{|G|}\sum_{x\in G}\Ind_H^Ga(x)\overline{b(x)}
        = \frac{1}{|G|}\frac{1}{|H|}\sum_{x,y\in G}\dot{a}(y^{-1}xy)\overline{b(x)}.
    \end{aligned}
    \end{equation}
    Since 
    \[
    \dot{a}(y^{-1}xy)\ne 0\Longrightarrow
    y^{-1}xy\in H\Longleftrightarrow x\in yHy^{-1},
    \]
    setting $h=y^{-1}xy$ 
    we can write \eqref{eq:reciprocity} as 
    \begin{align*}
        \langle\Ind_H^Ga,b\rangle
        &=\frac{1}{|G|}\frac{1}{|H|}\sum_{x\in G}\sum_{h\in H}a(h)\overline{b(xhx^{-1})}\\
        &=\frac{1}{|G|}\frac{1}{|H|}\sum_{x\in G}\sum_{h\in H}a(h)\overline{b(h)}\\
        &=\frac{1}{|G|}\sum_{x\in G}\langle a,\Res_H^Gb\rangle.
    \end{align*}
    From this the claim follows. 
\end{proof}

% 8.1.4 de Steinberg
% Seccion 8.2 define la inducción de representaciones
% 8.2.1 Induce la trivial del trivial a todo el grupo o obtiene la regular
% 8.2.2 Induce la representación por permutaciones
% Define la matriz y mira el dihedral de orden 2n (yo tengo un caso particular)
% Hace el ejemplo de Q8
% En el teorema 8.2.5 prueba la inducción da una representación

