\chapter{}

\topic{Brauer–Fowler theorem}

\index{Symmetric}
\index{Antisymmetric}
Let $\rho\colon G\to\GL(V)$ 
be a representation with character $\chi$. The $\C[G]$-module $V\otimes V$ 
has character $\chi^2$. Let 
$\{v_1,\dots,v_n\}$ be a basis of $V$ and 
\[
T\colon V\to V,\quad
v_i\otimes v_j\mapsto v_j\otimes v_i.
\]
It is an exercise to check that $T(v\otimes w)=w\otimes v$ for all 
$v,w\in V$. It follows that  
$T$ does not depend on the chosen basis. Note that
$T$ is a homomorphism of $\C[G]$-modules, as
\[
T(g\cdot (v\otimes w))=T((g\cdot v)\otimes (g\cdot w))=(g\cdot w)\otimes (g\cdot v)=g\cdot T(w\otimes v)
\]
for all $g\in G$ y $v,w\in V$. 
In particular, the \textbf{symmetric part} 
\begin{gather*}
S(V\otimes V)=\{x\in V\otimes V:T(x)=x\}
\shortintertext{and the \textbf{antisymmetric part}}
A(V\otimes V)=\{x\in V\otimes V:T(x)=-x\}
\end{gather*}
of $V\otimes V$ are both  
$\C[G]$-submodules of $V\otimes V$. 
The terminology is motivated by the following fact:
\[
V\otimes V=S(V\otimes V)\oplus A(V\otimes V).
\]
In fact, 
$S(V\otimes V)\cap A(V\otimes V)=\{0\}$, as   
$x\in S(V\otimes V)\cap A(V\otimes V)$ implies
$x=T(x)$ and $x=-T(x)$. Hence $x=0$. Moreover, 
$V\otimes V=S(V\otimes V)+ A(V\otimes V)$, as every $x\in V\otimes V$ can be written 
as 
\[
x=\frac12(x+T(x))+\frac12(x-T(x))
\]
with $\frac12(x+T(x))\in S(V\otimes V)$ and $\frac12(x-T(x))\in A(V\otimes V)$. 

We claim that $\{v_i\otimes v_j+v_j\otimes v_i:1\leq i,j\leq n\}$ is
a basis of $S(V\otimes V)$ 
and that  
\[
\{v_i\otimes v_j-v_j\otimes v_i:1\leq i<j\leq n\}
\]
is a basis of $A(V\otimes V)$. Since both sets are linearly independent, 
\[
\dim S(V\otimes V)\geq n(n+1)/2\text{ and }
\dim A(V\otimes V)\geq n(n-1)/2.
\]
Moreover, 
\[
n^2=\dim (V\otimes V)=\dim S(V\otimes V)+\dim A(V\otimes V),
\]
so it follows that
$\dim S(V\otimes V)=n(n+1)/2$ and $\dim A(V\otimes V)=n(n-1)/2$. 

\begin{proposition}
    Sea $G$ un grupo finito y  
    sea $V$ un $\C[G]$-módulo de dimensión finita con caracter $\chi$. Si el módulo $S(V\otimes V)$ 
    tiene caracter $\chi_S$ y el módulo $A(V\otimes V)$ tiene caracter $\chi_A$, entonces 
    \begin{align*}
        &\chi_S(g)=\frac12(\chi^2(g)+\chi(g^2)),\\
        &\chi_A(g)=\frac12(\chi^2(g)-\chi(g^2)).
    \end{align*}
\end{proposition}

\begin{proof}
    Sea $g\in G$. Sea $\rho\colon G\to\GL(V)$ la representación asociada al módulo $V$, es decir $\rho(g)(v)=\rho_g(v)=g\cdot v$. 
    Sabemos que $\rho_g$ es diagonalizable. Sea $\{e_1,\dots,e_n\}$ una base de autovectores de $\rho_g$, digamos
    $g\cdot e_i=\lambda_ie_i$ con $\lambda_i\in\C$ para $i\in\{1,\dots,n\}$. En particular, $\chi(g)=\sum_{i=1}^n\lambda_i$. 
    
    Como $\{e_i\otimes e_j-e_j\otimes e_i:1\leq i<j\leq n\}$ es base de $A(V\otimes V)$ y además 
    \[
    g\cdot (e_i\otimes e_j-e_j\otimes e_i)=\lambda_i\lambda_j(e_i\otimes e_j-e_j\otimes e_i),
    \]
    tenemos $\chi_A(g)=\sum_{1\leq i<j\leq n}\lambda_i\lambda_j$. Por otro lado, como $g^2\cdot e_i=\lambda_i^2e_i$ para todo $i$,
    $\chi(g^2)=\sum_{i=1}^n\lambda_i^2$. Luego
    \[
    \chi^2(g)=\chi(g)^2=\sum_{i=1}^n\sum_{j=1}^n\lambda_i\lambda_j=2\sum_{1\leq i<j\leq n}\lambda_i\lambda_j+\sum_{i=1}^n\lambda_i^2=2\chi_A(g)+\chi(g^2).
    \]
    Como además $V\otimes V=S(V\otimes V)\oplus A(V\otimes V)$, se tiene 
    $\chi^2(g)=\chi_S(g)+\chi_A(g)$, es decir 
    $\chi_S(g)=\frac12(\chi^2(g)+\chi(g^2))$.
\end{proof}

\index{Involución}
Una \textbf{involución} en un grupo es un elemento $x\ne 1$ tal que $x^2=1$. 
Es posible la cantidad de involuciones 
con la tabla de caracteres:

\begin{proposition}
Si $G$ es un grupo finito con $t$ involuciones, entonces 
\[
1+t=\sum_{\chi\in\Irr(G)}\langle\chi_S-\chi_A,\chi_1\rangle\chi(1).
\]
\end{proposition}

\begin{proof}
Supongamos que $\Irr(G)=\{\chi_1,\dots,\chi_k\}$, donde $\chi_1$ es el caracter trivial de $G$. 
Para $x\in G$ sea 
\[
\theta(x)=|\{y\in G:y^2=x\}|.
\]
Como $\theta$ es una función de clases
$\theta$ puede escribirse como combinación lineal de los $\chi_j$, digamos
\[
\theta=\sum_{\chi\in\Irr(G)}\langle\theta,\chi\rangle\chi.
\]
Calculamos
\begin{align*}
    \langle\chi_S-\chi_A,\chi_1\rangle 
    &=\frac{1}{|G|}\sum_{g\in G}\chi(g^2)\\
    &=\frac{1}{|G|}\sum_{x\in G}\sum_{\substack{g\in G\\g^2=x}}\chi(g^2)
    =\frac{1}{|G|}\sum_{x\in G}\theta(x)\chi(x)=\langle\theta,\chi\rangle.
\end{align*}
Luego $\theta(x)=\sum_{\chi\in\Irr(G)}\langle\chi_S-\chi_A,\chi_1\rangle\chi$ y el resultado se obtiene
al evaluar esta expresión en $x=1$. 
\end{proof}

\index{Desigualdad!de Cauchy--Schwartz}
Necesitamos un lema:
% Recordemos la desigualdad de Cauchy--Schwartz. Si $x_1,\dots,x_n\in\R$, entonces
% $\sum x_i^2\geq\frac{1}{n}(\sum x_i)^2$. 

\begin{lemma}
Sea $G$ un grupo finito con $k$ clases de conjugación. 
Si $t$ es la cantidad de involuciones de $G$, entonces 
$t^2\leq (k-1)(|G|-1)$. 
\end{lemma}

\begin{proof}
Supongamos que $\Irr(G)=\{\chi_1,\dots,\chi_k\}$, donde $\chi_1$ es el 
carácter trivial de $G$. 
Si $\chi\in\Irr(G)$, entonces
\[
\langle\chi^2,\chi_1\rangle=\frac{1}{|G|}\sum_{g\in G}\chi(g)\chi(g)=\langle\chi,\overline{\chi}\rangle=\begin{cases}
1 & \text{si $\chi=\overline{\chi}$},\\
0 & \text{en otro caso}.
\end{cases}
\]
Como $\chi^2=\chi_S+\chi_A$, si $\langle\chi^2,\chi_1\rangle=1$, entonces el caracter trivial 
o bien es $\chi_1$ es parte de $\chi_S$ o bien es parte de $\chi_A$, pero no de ambos. Esto implica que
\[
\langle\chi_S-\chi_A,\chi_1\rangle\in\{-1,1,0\}.
\]
Vamos a demostrar ahora que 
$t\leq\sum_{i=2}^k\chi_i(1)$. En efecto, 
como $|\langle\chi_S-\chi_A,\chi_1\rangle|\leq 1$, 
entonces 
\begin{align*}
1+t=\theta(1)
&=\left|\sum_{\chi\in\Irr(G)}\langle\chi_S-\chi_A,\chi_1\rangle\chi(1)\right|\\
&\leq\sum_{\chi\in\Irr(G)}|\langle\chi_S-\chi_A,\chi_1\rangle|\chi(1)
\leq\sum_{\chi\in\Irr(G)}\chi(1),
\end{align*}
de donde se obtiene inmediatamente que $t\leq\sum_{i=2}^k\chi_i(1)$. 
Si utilizamos ahora 
la desigualdad de Cauchy--Schwartz, 
\[
t^2\leq\left(\sum_{i=2}^k\chi_i(1)\right)^2
\leq(k-1)\sum_{i=2}^k\chi(1)^2=(k-1)(|G|-1).\qedhere
\]
\end{proof}

Ahora sí estamos en condiciones de dar la primera demostración del teorema de
Brauer--Fowler. 

\begin{theorem}[Brauer--Fowler]
\index{Teorema!de Brauer--Fowler}
Sea $G$ un grupo finito y simple y sea $x$ una involución. Si $|C_G(x)|=n$, entonces $|G|\leq (n^2)!$	
\end{theorem}

\begin{proof}
Supongamos primero que existe un subgrupo propio $H$ de $G$ tal que
$(G:H)\leq n^2$. En ese caso, hacemos actuar a $G$ en $G/H$ por multiplicación a izquierda 
y tenemos un morfismo de grupos $\rho\colon G\to\Sym_{n^2}$. Como $G$ es un grupo simple, 
$\ker\rho=\{1\}$ o bien $\ker\rho=G$. Si $\ker\rho=G$, entonces $\rho(g)(yH)=yH$ para todo
$g\in G$ e $y\in G$, lo que implica que $g\in H$, una contradicción. Luego $\rho$ es inyectiva
y entonces $G$ es isomorfo a un subgrupo de $\Sym_{n^2}$. En particular, $|G|$ divide a $(n^2)!$

Sea $m=(|G|-1)/t$. 
Como $|C_G(x)|=n$, el grupo $G$ tiene al menos $|G|/n$ involuciones (pues la clase de conjugación
de $x$ tiene tamaño $|G|/n$ y todos sus elementos son involuciones), es decir $t\geq |G|/n$. Luego
$m=(|G|-1)/t<n$. Basta demostrar entonces que $G$ contiene un subgrupo de índice $\leq m^2$. 

Sean $C_1,\dots,C_k$ las clases de conjugación de $G$, donde $C_1=\{1\}$. 
Como $G$ es simple, $|C_i|>1$ 
para todo $i\in\{2,\dots,k\}$. Notar que 
\[
|G|-1\leq\frac{(k-1)(|G|-1)^2}{t^2}\Longleftrightarrow t^2\leq(k-1)(|G|-1),
\]
que vale gracias al lema anterior. 
Si $|C_i|>m$ para todo $i\in\{2,\dots,k\}$, entonces, como
\[
|G|-1\leq\frac{(k-1)(|G|-1)^2}{t^2}=(k-1)m^2,
\]
tendríamos 
\[
|G|-1=\sum_{i=2}^k|C_i|>(k-1)m^2,
\]
una contradicción. Luego existe una clase de conjugación $C$ de $G$ tal que $|C|\leq m^2$. Si $g\in C$, entonces
$C_G(g)$ es un subgrupo de $G$ de índice $|C|\leq m^2$.
\end{proof}

The bound of Brauer--Fowler's is not important.

\begin{corollary}
    Let $n\geq 1$ be an integer. There are at most finitely many 
    finite simple groups with an involution with a centralizer of order $n$.
\end{corollary}

As an exercise, a simple applications: 

\begin{exercise}
    If $G$ is a finite simple group and $x$ is an involution with
    centralizer of order two, then  
    $G\simeq\Z/2$. 
\end{exercise}

\topic{Induction and restriction}

\topic{Frobenius' theorem}

\topic{Some theorems of Burnside}

For $n\geq1$ let $\{e_1,\dots,e_n\}$ be the standard basis of $\C^n$.  
The \textbf{natural representation} of $\Sym_n$ is 
$\rho\colon\Sym_n\to\GL_n(C)$, $\sigma\mapsto\rho_{\sigma}$, 
where $\rho_\sigma(e_j)=e_{\sigma(j)}$ for all $j\in\{1,\dots,n\}$. 
The matrix of $\rho_\sigma$ in the standard basis is  
\begin{equation}
    \label{eq:Sn_natural}
    (\rho_\sigma)_{ij}=\begin{cases}
      1 & \text{if $i=\sigma(j)$},\\
      0 & \text{otherwise}.
    \end{cases}
\end{equation}

\begin{lemma}
	\label{lem:permutaciones}
	For $n\geq1$ let $\rho\colon\Sym_n\to\GL_n(C)$ be the natural 
	representation of the symmetric group. 
	If $A\in\C^{n\times n}$ and $\sigma\in\Sym_n$, then
	\[
		A_{ij}=(\rho_{\sigma}A)_{\sigma(i)j}=(A\rho_{\sigma})_{i\sigma^{-1}(j)}
	\]
    for all $i,j\in\{1,\dots,n\}$.
\end{lemma}

\begin{proof}
	With~\eqref{eq:Sn_natural} we compute:
	\[
		(A\rho_{\sigma})_{ij}=\sum_{k=1}^n A_{ik}(\rho_{\sigma})_{kj}=A_{i\sigma(j)},
		\quad
		(\rho_\sigma A)_{ij}=\sum_{k=1}^n (\rho_\sigma)_{ik}A_{kj}=A_{\sigma^{-1}(i)j}.\qedhere
	\]
\end{proof}

\begin{definition}
  \index{Real!character}
  Let $G$ be a finite group. A character $\chi$ of $G$ is said to be
  \textbf{real} if
  $\chi=\overline{\chi}$, that is $\chi(g)\in\R$ for all $g\in G$. 
\end{definition}

\begin{exercise}
	\label{xca:chi_irreducible}
	Let $G$ be a finite group. If $\chi\in\Irr(G)$, then 
	$\overline{\chi}$ is irreducible.
\end{exercise}

\begin{definition}
  \index{Real!conjugacy class}
  Let $G$ be a group. A conjugacy class $C$ of $G$ is said to be
  \textbf{real} if for every $g\in C$ one has $g^{-1}\in C$. 
\end{definition}

We use the following notation: if $G$ is a group and $C=\{xgx^{-1}:x\in G\}$ is a conjugacy class of  
$G$, then $C^{-1}=\{xg^{-1}x^{-1}:x\in G\}$.  

\begin{theorem}[Burnside]
    \index{Burnside's!theorem}
    Let $G$ be a finite group. The number of real conjugacy classes is equal 
    to the number of real irreducible characters. 
\end{theorem}

\begin{proof}
  Let $C_1,\dots,C_r$ be the conjugacy classes of $G$ and  
  let $\chi_1,\dots,\chi_r$ be the irreducible characters of $G$. 
  Let $\alpha,\beta\in\Sym_r$ be such that $\overline{\chi_i}=\chi_{\alpha(i)}$ and 
  $C_i^{-1}=C_{\beta(i)}$ for all $i\in\{1,\dots,r\}$. Note that $\chi_i$
  is real if and only if $\alpha(i)=i$ and that $C_i$ is real if and only if 
  $\beta(i)=i$. The number $n$ of fixed points of $\alpha$ is equal to the number
  of irreducible characters of $G$ and the number $m$ of fixed points of $\beta$ is equal
  to the number of real classes. 
  Let $\rho\colon\Sym_r\to\GL(r,\C)$ be the natural representation of $\Sym_r$. Then
  $\chi_\rho(\alpha)=n$ and $\chi_\rho(\beta)=m$. We claim that 
  $\trace\rho_\alpha=\trace\rho_\beta$. 
  Let $X\in\GL(r,\C)$ be the character matrix of $G$. 
  By Lemma~\ref{lem:permutaciones}, 
  \[
	\rho_\alpha X=\overline{X}=X\rho_\beta.
  \]
  Since $X$ is invertible, $\rho_{\alpha}=X\rho_{\beta}X^{-1}$. Thus 
  \[
    n=\chi_{\rho}(\alpha)=\trace\rho_{\alpha}=\trace\rho_{\beta}=\chi_{\rho}(\beta)=m.\qedhere
  \]
\end{proof}

\begin{corollary}
  \label{corollary:|G|impar}
  Let $G$ be a finite group. Then $|G|$ is odd if and only if
  the only real $\chi\in\Irr(G)$ is the trivial character. 
\end{corollary}

\begin{proof}
    We first prove $\impliedby$. If $|G|$ is even, there exists 
    $g\in G$ of order two (Cauchy's theorem). The conjugacy class of $g$ 
    is real. 

    We now prove $\implies$. Assume that $G$ has a non-trivial 
    real conjugacy class $C$. Let $g\in C$. We claim that 
    $G$ has an element of even order. Let $h\in G$ be such that
    $hgh^{-1}=g^{-1}$. Then $h^2\in C_G(g)$, as $h^2gh^{-2}=g$. 
    If $h\in\langle h^2\rangle\subseteq C_G(g)$, then $g$ has 
    even order, as $g^{-1}=g$. If $h\not\in\langle h^2\rangle$, then 
    $h^2$ does not generate $\langle h\rangle$. Hence $h$ has odd order, 
    as $|h|\ne|h^2|=|h|/(|h|:2)$.  
\end{proof}

\begin{theorem}[Burnside]
  \index{Burnside's!theorem}
  \label{thm:Burnside_mod16}
  Let $G$ be a finite group of odd order 
  with $r$ conjugacy classes. Then
  $r\equiv|G|\bmod{16}$.
\end{theorem}

\begin{proof}
  Since $|G|$ is odd, every non-trivial $\chi\in\Irr(G)$ is not real by
  the previous corollary. The irreducible characters 
  of $G$ are then 
  \[
    \chi_1,\chi_2,\overline{\chi_2},\dots,\chi_k,\overline{\chi_k},
    \quad
    r=1+2k,
  \]
  where $\chi_1$ denotes the trivial character. 
  For every $j\in\{2,\dots,k\}$ let $d_j=\chi_j(1)$. 
  Since each $d_j$ divides 
  $|G|$ by Frobenius' theorem and  $|G|$ is odd, 
  every $d_j$ is an odd number, 
  say $d_j=1+2m_j$. Thus  
  \begin{align*}
    |G|&=1+\sum_{j=2}^k 2d_j^2=1+\sum_{j=2}^k2(2m_j+1)^2\\
    &=1+\sum_{j=2}^k2(4m_j^2+4m_j+1)
    =1+2k+8\sum_{j=2}^km_j(m_j+1).
  \end{align*}
  Hence $|G|\equiv r\bmod{16}$, 
  as $r=1+2k$ and every $m_j(m_j+1)$ is even. 
\end{proof}

\begin{exercise}
    Prove that every group of order 15 is abelian. 
\end{exercise}

\topic{Solvable groups and Burnside's theorem}

\index{Derived series}
For a group $G$ let 
$G^{(0)}=G$ and 
$G^{(i+1)}=[G^{(i)},G^{(i)}]$ for $i\geq0$.
The \textbf{derived series} of $G$ is the sequence
\[
G=G^{(0)}\supseteq G^{(1)}\supseteq G^{(2)}\supseteq\cdots
\]
Each $G^{(i)}$ is a characteristic subgroup of $G$. We say that 
$G$ is \textbf{solvable} if $G^{(n)}=\{1\}$ for some $n$.  

\begin{example}
	Abelian groups are solvable. 
\end{example}

\begin{example}
	The group $\SL_2(3)$ is solvable, as the derived series is 
	\[
	\SL_2(3)\supseteq Q_8\supseteq C_4\supseteq C_2\supseteq \{1\}.
	\]
	Here is the what the computer says:
\begin{lstlisting}
gap> IsSolvable(SL(2,3));
true
gap> List(DerivedSeries(SL(2,3)),StructureDescription);
[ "SL(2,3)", "Q8", "C2", "1" ]
\end{lstlisting}
\end{example}

\begin{example}
	Non-abelian simple groups cannot be solvable. 
\end{example}

\begin{exercise}
	\label{xca:solvable}
	Let $G$ be a group. Prove the following statements:
	\begin{enumerate}
		\item A subgroup $H$ of $G$ is solvable.
		\item Let $K$ be a normal subgroup of $G$. 
		    Then $G$ is solvable if and only if $K$ and $G/K$ are solvable.
	\end{enumerate}
\end{exercise}

\begin{example}
	For $n\geq5$ the group $\Alt_n$ is not solvable. It follows that 
	$\Sym_n$ is not solvable for $n\geq5$. 
\end{example}

\begin{exercise}
\label{xca:pgroups_solvable}
	Let $p$ be a prime number. Prove that 
	finite $p$-groups are solvable.
\end{exercise}

\begin{theorem}[Burnside]
	\index{Burnside's theorem}
	\label{thm:Burnside_auxiliar}
	Let $G$ be a finite group. If $\phi\colon G\to\GL_n(\C)$ is a representation
	with character $\chi$ and $C$ is a conjugacy class of $G$ such that 
	$\gcd(|C|,n)=1$, then for every $g\in C$ either 
	$\chi(g)=0$ or $\phi_g$ is a scalar matrix. 
\end{theorem}

We need a lemma.

\begin{lemma}
	\label{lem:4Burnside}
	Let $\epsilon_1,\dots,\epsilon_n$ be roots of one such that 
	$(\epsilon_1+\cdots+\epsilon_n)/n\in\A$. Then either 
	$\epsilon_1=\cdots=\epsilon_n$ or 
	$\epsilon_1+\cdots+\epsilon_n=0$.
\end{lemma}

\begin{proof}
	Let $\alpha=(\epsilon_1+\cdots+\epsilon_n)/n$.
	If the $\epsilon_j$s are not all equal, then $N(\alpha)<1$. Moreover, 
	$N(\beta)<1$ for every algebraic conjugate $\beta$ of $\alpha$. Since the product 
	of the algebraic conjugates of $\alpha$ is an integer of absolute value 
	$<1$, it follows that it is zero. 
\end{proof}

Now we prove the theorem.

\begin{proof}[Proof of Theorem \ref{thm:Burnside_auxiliar}]
	Let $\epsilon_1,\dots,\epsilon_n$ be the eigenvalues of $\phi_g$. By assumption, 
	$\gcd(|C|,n)=1$, there exist $a,b\in\Z$ such that $a|C|+bn=1$. Since 
	$|C|\chi(g)/n\in\A$, after multiplying by $\chi(g)/n$ we obtain that  
	\[
		a|C|\frac{\chi(g)}{n}+b\chi(g)=\frac{\chi(g)}{n}=\frac{1}{n}(\epsilon_1+\cdots+\epsilon_n)\in\A.
	\]
	The previous lemma implies that there are two cases to consider: 
	either $\epsilon_1=\cdots=\epsilon_n$ or $\epsilon_1+\cdots+\epsilon_n=0$. In the first
	case, since $\phi_g$ is diagonalizable, $\phi_g$ is a scalar matrix. 
	In the second case, $\chi(g)=0$.
\end{proof}

\begin{theorem}[Burnside]
	\index{Burnside's theorem}
	Let $p$ be a prime number. If $G$ is a finite group and 
	$C$ is a conjugacy class of $G$ with $p^k>1$ elements, then $G$ 
	is not simple.
\end{theorem}

\begin{proof}
	Let $g\in C\setminus\{1\}$. Column orthogonality implies that 
	\begin{equation}
	\label{eq:Burnside}
	\begin{aligned}
		0&=\sum_{\chi\in\Irr(G)}\chi(1)\chi(g)\\
		&=\sum_{p\mid\chi(1)}\chi(1)\chi(g)+\sum_{p\nmid\chi(1)}\chi(1)\chi(g)+1,
	\end{aligned}
	\end{equation}
	where the one corresponds to the trivial representation of
	$G$.
	
	Look this equation modulo $p$. If $\chi(g)=0$ for all
	$\chi\in\Irr(G)$
	such that $\chi\ne\chi_1$ and $p\nmid\chi(1)$, then
	\[
	-\frac{1}{p}=\sum\frac{\chi(1)}{p}\chi(g)\in\A\cap\Q=\Z,
	\]
	where the sum is taken over all non-trivial irreducibles
	of $G$ of degree divisible by $p$, a contradiction. Hence 
	there exists an irreducible non-trivial representation 
	$\phi$ with character $\chi$ such that $p$ does not divide
	$\chi(1)$ and $\chi(g)\ne0$. By the previous theorem, 
	$\phi_g$ is a scalar matrix. If $\phi$ is faithful, then 
	$g$ is a non-trivial central element, a contradiction since 
    $|C|>1$. If $\phi$ is not faithful, then 
    $G$ is not simple (because 
	$\ker\phi$ is a non-trivial proper normal subgroup of $G$). 
\end{proof}

\begin{theorem}[Burnside]
  \index{Burnside's theorem}
  Let $p$ and $q$ be prime numbers. If $G$ has order $p^aq^b$, then $G$ is solvable.
\end{theorem}

\begin{proof}
	Let us assume that the theorem is not true. Let $G$ be a group
	of minimal order $p^aq^b$
	that is not solvable. Since $|G|$ is minimal, $G$ is simple. By the previous theorem, 
	$G$ has no conjugacy classes of size $p^k$ nor 
	conjugacy classes of size $q^l$ with $k,l\geq1$. The size
	of every conjugacy class of $G$ is one or divisible by $pq$. 
	By the class
	equation, 
	\[
		|G|=1+\sum_{C:|C|>1}|C|,
	\]
	where the sum is taken over all conjugacy classes 
	with more than one element, a contradiction.
\end{proof}

Some generalizations of Burnside's theorem. 

\begin{theorem}[Kegel--Wielandt]
    \index{Kegel--Wielandt's theorem}
    If $G$ is a finite group and there are nilpotent subgroups 
    $A$ and $B$ of $G$ such that 
    $G=AB$, then $G$ is solvable.
\end{theorem}

See~\cite[Theorem 2.4.3]{MR1211633} for the proof.


Another generalization of Burnside's theorem
is based on \emph{word maps}. A word map
of a group $G$ is a map 
\[
G^k\to G,\quad 
(x_1,\dots,x_k)\mapsto w(x_1,\dots,x_k)
\]
for some word $w(x_1,\dots,x_k)$ of the free group $F_k$ of rank $k$. 
Some word maps are surjective in certain families of groups. For example, 
Ore's conjecture is precisely the surjectivity of the word map
$(x,y)\mapsto [x,y]=xyx^{-1}y^{-1}$ in every finite non-abelian simple 
group. 

\begin{theorem}[Guralnick--Liebeck--O'Brien--Shalev--Tiep]
    Let $a,b\geq0$, $p$ and $q$ be prime numbers and $N=p^aq^b$. The map 
    $(x,y)\mapsto x^Ny^N$ is surjective in every finite simple group. 
\end{theorem}

The proof appears in~\cite{MR3827208}. 

The theorem implies Burnside's theorem. Let $G$ be a group of order
$N=p^aq^b$. Assume that $G$ is not solvable. 
Fix a composition series of $G$. There is a non-abelian factor $S$ 
of order that divides $N$. Since 
$S$ is simple non-abelian and $s^N=1$, it follows that the word map
$(x,y)\mapsto x^Ny^N$ has trivial image in $S$, a contradiction 
to the theorem. 

\topic{Feit--Thompson theorem}

\begin{theorem}[Feit--Thompson]
    \index{Feit--Thompson's theorem}
    Groups of odd order are solvable. 
\end{theorem}

The proof of Feit--Thompson theorem is extremely hard. 
It occupies a full volume of the 
\emph{Pacific Journal of Mathematics}~\cite{MR166261}. 
A formal verification of the proof 
(based on the computer software Coq) 
was announced in~\cite{MR3111271}.  

Back in the day it was believed that if a certain divisibility 
conjecture is true, 
the proof of Feit--Thompson 
could be simplified. 

\begin{conjecture}[Feit--Thompson]
\index{Feit--Thompson conjecture}
    There are no prime numbers $p$ and $q$ such that
    $\frac {p^{q}-1}{p-1}$ divides $\frac{q^{p} - 1}{q - 1}$. 
\end{conjecture}

The conjecture remains open. However, now we know that 
proving the conjecture will not simplify further
the proof of Feit--Thompson theorem. 

In 2012 Le proved that the conjecture is true for $q=3$, see 
\cite{MR2900154}. 


In~\cite{MR297686} 
Stephens proved that a certain stronger version of the conjecture 
does not hold, as the integers 
$\frac {p^{q}-1}{p-1}$ and $\frac{q^{p} - 1}{q - 1}$ 
could have common factors. In fact, if $p=17$ and $q=3313$, 
then 
\[
\gcd\left(\frac {p^{q}-1}{p-1},\frac{q^{p} - 1}{q - 1}\right)=112643.
\]
Nowadays we can check this easily in almost every desktop computer:
\begin{lstlisting}
gap> Gcd((17^3313-1)/16,(3313^17-1)/3312);
112643
\end{lstlisting}
No other counterexamples have been found to Stephen’s stronger version of the conjecture.
