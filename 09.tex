\chapter{}

\topic{Brauer--Fowler theorem}

\index{Symmetric}
\index{Antisymmetric}
Let $\rho\colon G\to\GL(V)$ 
be a representation with character $\chi$. The $\C[G]$-module $V\otimes V$ 
has character $\chi^2$. Let 
$\{v_1,\dots,v_n\}$ be a basis of $V$ and 
\[
T\colon V\to V,\quad
v_i\otimes v_j\mapsto v_j\otimes v_i.
\]
It is an exercise to check that $T(v\otimes w)=w\otimes v$ for all 
$v,w\in V$. It follows that  
$T$ does not depend on the chosen basis. Note that
$T$ is a homomorphism of $\C[G]$-modules, as
\[
T(g\cdot (v\otimes w))=T((g\cdot v)\otimes (g\cdot w))=(g\cdot w)\otimes (g\cdot v)=g\cdot T(w\otimes v)
\]
for all $g\in G$ y $v,w\in V$. 
In particular, the \textbf{symmetric part} 
\begin{gather*}
S(V\otimes V)=\{x\in V\otimes V:T(x)=x\}
\shortintertext{and the \textbf{antisymmetric} part}
A(V\otimes V)=\{x\in V\otimes V:T(x)=-x\}
\end{gather*}
of $V\otimes V$ are both  
$\C[G]$-submodules of $V\otimes V$. 
The terminology is motivated by the following fact:
\[
V\otimes V=S(V\otimes V)\oplus A(V\otimes V).
\]
In fact, 
$S(V\otimes V)\cap A(V\otimes V)=\{0\}$, as   
$x\in S(V\otimes V)\cap A(V\otimes V)$ implies
$x=T(x)$ and $x=-T(x)$. Hence $x=0$. Moreover, 
$V\otimes V=S(V\otimes V)+ A(V\otimes V)$, as every $x\in V\otimes V$ can be written 
as 
\[
x=\frac12(x+T(x))+\frac12(x-T(x))
\]
with $\frac12(x+T(x))\in S(V\otimes V)$ and $\frac12(x-T(x))\in A(V\otimes V)$. 

We claim that $\{v_i\otimes v_j+v_j\otimes v_i:1\leq i,j\leq n\}$ is
a basis of $S(V\otimes V)$ 
and that  
\[
\{v_i\otimes v_j-v_j\otimes v_i:1\leq i<j\leq n\}
\]
is a basis of $A(V\otimes V)$. Since both sets are linearly independent, 
\[
\dim S(V\otimes V)\geq n(n+1)/2\text{ and }
\dim A(V\otimes V)\geq n(n-1)/2.
\]
Moreover, 
\[
n^2=\dim (V\otimes V)=\dim S(V\otimes V)+\dim A(V\otimes V),
\]
so it follows that
$\dim S(V\otimes V)=n(n+1)/2$ and $\dim A(V\otimes V)=n(n-1)/2$. 

\begin{proposition}
    Let $G$ be a finite group and
    $V$ be a finite-dimensional 
    $\C[G]$-module with character $\chi$. If $S(V\otimes V)$ 
    has character $\chi_S$ and $A(V\otimes V)$ has character
    $\chi_A$, then 
    \begin{align*}
        &\chi_S(g)=\frac12(\chi^2(g)+\chi(g^2)) && \text{and} &&
        \chi_A(g)=\frac12(\chi^2(g)-\chi(g^2)).
    \end{align*}
\end{proposition}

\begin{proof}
    Let $g\in G$ and $\rho\colon G\to\GL(V)$ be the representation
    associated with $V$, that is $\rho(g)(v)=\rho_g(v)=g\cdot v$. 
    Since $\rho_g$ is diagonalizable, let $\{e_1,\dots,e_n\}$ 
    be a basis of eigenvectors of $\rho_g$, say
    $g\cdot e_i=\lambda_ie_i$ with $\lambda_i\in\C$ for all $i\in\{1,\dots,n\}$. In particular, $\chi(g)=\sum_{i=1}^n\lambda_i$. 
    
    Since $\{e_i\otimes e_j-e_j\otimes e_i:1\leq i<j\leq n\}$ is basis of
    $A(V\otimes V)$ and 
    \[
    g\cdot (e_i\otimes e_j-e_j\otimes e_i)=\lambda_i\lambda_j(e_i\otimes e_j-e_j\otimes e_i),
    \]
    it follows that
    $\chi_A(g)=\sum_{1\leq i<j\leq n}\lambda_i\lambda_j$. On the other hand,
    $g^2\cdot e_i=\lambda_i^2e_i$ for all $i$,
    $\chi(g^2)=\sum_{i=1}^n\lambda_i^2$. Thus 
    \[
    \chi^2(g)=\chi(g)^2=\sum_{i=1}^n\sum_{j=1}^n\lambda_i\lambda_j=2\sum_{1\leq i<j\leq n}\lambda_i\lambda_j+\sum_{i=1}^n\lambda_i^2=2\chi_A(g)+\chi(g^2).
    \]
    Since $V\otimes V=S(V\otimes V)\oplus A(V\otimes V)$, it follows that  
    $\chi^2(g)=\chi_S(g)+\chi_A(g)$, that is 
    $\chi_S(g)=\frac12(\chi^2(g)+\chi(g^2))$.
\end{proof}

\index{Involution}
An \textbf{involution} of a group is an element $x\ne 1$ such that $x^2=1$. 
It is possible to use the character table to count the number
of involutions.

\begin{proposition}
    If $G$ is a finite group with $t$ involutions, then
    \[
        1+t=\sum_{\chi\in\Irr(G)}\langle\chi_S-\chi_A,\chi_1\rangle\chi(1).
    \]
\end{proposition}

\begin{proof}
    Assume that $\Irr(G)=\{\chi_1,\dots,\chi_k\}$, where $\chi_1$ is 
    the trivial character of $G$. 
    For $x\in G$ let 
    \[
    \theta(x)=|\{y\in G:y^2=x\}|.
    \]
    Since $\theta$ is a class function, 
    $\theta$ is a linear combination of the $\chi_j$'s, say 
    \[
    \theta=\sum_{\chi\in\Irr(G)}\langle\theta,\chi\rangle\chi.
    \]
    We compute: 
    \begin{align*}
        \langle\chi_S-\chi_A,\chi_1\rangle 
        &=\frac{1}{|G|}\sum_{g\in G}\chi(g^2)\\
        &=\frac{1}{|G|}\sum_{x\in G}\sum_{\substack{g\in G\\g^2=x}}\chi(g^2)
        =\frac{1}{|G|}\sum_{x\in G}\theta(x)\chi(x)=\langle\theta,\chi\rangle.
    \end{align*}
    Thus $\theta(x)=\sum_{\chi\in\Irr(G)}\langle\chi_S-\chi_A,\chi_1\rangle\chi$. Now
    the claim follows after evaluating this expression in 
    $x=1$. 
\end{proof}

\index{Cauchy--Schwartz inequality}
Before proving the Brauer-Fowler theorem we
need a lemma. We will use the Cauchy--Schwartz inequality: 
\[
x_1,\dots,x_n\in\R\implies
\sum x_i^2\geq\frac{1}{n}(\sum x_i)^2.
\]

\begin{lemma}
    Let $G$ be a finite group with $k$ conjugacy classes. 
    If $t$ is the number of involutions of $G$, then
    $t^2\leq (k-1)(|G|-1)$. 
\end{lemma}

\begin{proof}
    Assume that $\Irr(G)=\{\chi_1,\dots,\chi_k\}$, where $\chi_1$ is the
    trivial character of $G$. 
    If $\chi\in\Irr(G)$, then 
    \[
        \langle\chi^2,\chi_1\rangle=\frac{1}{|G|}\sum_{g\in G}\chi(g)\chi(g)=\langle\chi,\overline{\chi}\rangle=\begin{cases}
        1 & \text{if $\chi=\overline{\chi}$},\\
        0 & \text{otherwise}.
        \end{cases}
    \]
    Since $\chi^2=\chi_S+\chi_A$, if $\langle\chi^2,\chi_1\rangle=1$, then
    the trivial character either is part of $\chi_S$ or $\chi_A$, but not both. 
    Thus
    \[
    \langle\chi_S-\chi_A,\chi_1\rangle\in\{-1,1,0\}.
    \]
    
    We claim that 
    $t\leq\sum_{i=2}^k\chi_i(1)$. In fact, since 
    $|\langle\chi_S-\chi_A,\chi_1\rangle|\leq 1$, 
    \begin{align*}
        1+t=\theta(1)
        &=\left|\sum_{\chi\in\Irr(G)}\langle\chi_S-\chi_A,\chi_1\rangle\chi(1)\right|\\
        &\leq\sum_{\chi\in\Irr(G)}|\langle\chi_S-\chi_A,\chi_1\rangle|\chi(1)
        \leq\sum_{\chi\in\Irr(G)}\chi(1).
    \end{align*}
    It follows that $t\leq\sum_{i=2}^k\chi_i(1)$. 
    By the Cauchy--Schwartz inequality, 
    \[
        t^2\leq\left(\sum_{i=2}^k\chi_i(1)\right)^2
        \leq(k-1)\sum_{i=2}^k\chi(1)^2=(k-1)(|G|-1).\qedhere
    \]
\end{proof}

Now we prove the Brauer--Fowler theorem. 

\begin{theorem}[Brauer--Fowler]
    \index{Brauer--Fowler theorem}
    Let $G$ be a finite simple group and $x$ be an involution of $G$. If $|C_G(x)|=n$, then $|G|\leq (n^2)!$	
\end{theorem}

\begin{proof}
    We first assume the existence of a proper subgroup $H$ of $G$ 
    such that 
    \[
    (G:H)\leq n^2.
    \]
    The group $G$ acts on $G/H$ 
    by left multiplication, so there is a group homomorpihsm 
    $\rho\colon G\to\Sym_{n^2}$. Since $G$ is simple, either 
    $\ker\rho=\{1\}$ or $\ker\rho=G$. If $\ker\rho=G$, then
    $\rho(g)(yH)=yH$ for all $g\in G$ and $y\in G$. 
    Hence $g\in H$, a contradiction. Therefore $\rho$ is injective
    and hence $G$ is isomorphic to a subgroup of $\Sym_{n^2}$. 
    In particular, $|G|$ divides $(n^2)!$. 

    Let $m=(|G|-1)/t$. 
    Since $|C_G(x)|=n$, the group $G$ has at least $|G|/n$ involutions (because
    the conjugacy class of $x$ has size $|G|/n$ and all its elements are involutions), 
    that is $t\geq |G|/n$. Hence 
    $m=(|G|-1)/t<n$. It is enough then to show that
    $G$ contains a subgroup of index $\leq m^2$. 

    Let $C_1,\dots,C_k$ be the conjugacy classes of $G$, where $C_1=\{1\}$. 
    Since $G$ is simple, $|C_i|>1$ 
    for all $i\in\{2,\dots,k\}$. By the previous lemma, 
    \[
    |G|-1\leq\frac{(k-1)(|G|-1)^2}{t^2}\Longleftrightarrow t^2\leq(k-1)(|G|-1).
    \]
    If $|C_i|>m$ for all $i\in\{2,\dots,k\}$, then, since 
    \[
    |G|-1\leq\frac{(k-1)(|G|-1)^2}{t^2}=(k-1)m^2,
    \]
    it follows that  
    \[
    |G|-1=\sum_{i=2}^k|C_i|>(k-1)m^2,
    \]
    a contradiction. Thus there exists a conjugacy class
    $C$ of $G$ such that $|C|\leq m^2$. If $g\in C$, then
    $C_G(g)$ is a subgroup of $G$ of index $|C|\leq m^2$.
\end{proof}

The bound of the Brauer--Fowler theorem is not important.
What matters is the following consequence:

\begin{corollary}
    Let $n\geq 1$ be an integer. There are at most finitely many 
    finite simple groups with an involution with a centralizer of order $n$.
\end{corollary}

As an exercise, a simple application: 

\begin{exercise}
    If $G$ is a finite simple group and $x$ is an involution with
    centralizer of order two, then  
    $G\simeq\Z/2$. 
\end{exercise}

\subsection{A comment: An elementary proof of Brauer--Fowler theorem}

We need to find a subgroup of index $\leq 2n^2$. 
Let $X$ be the conjugacy class of $x$. For $g\in G$ let
\[
J(g)=\{z\in X:zgz^{-1}=g^{-1}\}.
\]
We claim that $|J(g)|\leq|C_G(g)|$. The map $J(g)\to C_G(g)$, $z\mapsto gz$, 
is well-defined,~as 
\[
(gz)g(gz)^{-1}=g(xgx^{-1})g^{-1}=g^{-1}\in C_G(g).
\]
It is injective, as $gz=gz_1$ implies $z=z_1$.

Let $\{(g,z)\in G\times X:zgz^{-1}=g^{-1}\}$.  
Since $X\times X\to J$, $(y,z)\mapsto (yz,z)$, 
is well-defined (since $z(yz)z^{-1}=zy=(yz)^{-1}$) and
it is trivially injective, 
\[
|X|^2\leq |J|=\sum_{(g,z)\in J}1\leq\sum_{g\in G}|J(g)|
=\sum_{g\in G}|C_G(g)|=k|G|,
\]
where $k$ is the number of conjugacy classes of $G$, 
as $(g,z)\in J$ if and only if $z\in J(g)$. Thus $|G|\leq kn^2$, as
\[
\left(\frac{|G|}{|C_G(x)|}\right)^2=|X|^2=\frac{|G|}{n^2}\leq k|G|.
\]

\begin{claim}
    There exists a conjugacy class with $\leq 2n^2$ elements.
\end{claim}

Assume that the claim is not true. Let
$C_1,\dots,C_k$ be the conjugacy classes of $G$, where 
$C_1=\{1\}$ and $|C_i|>2n^2$ for all $i\in\{2,\dots,k\}$. Then
\[
|G|=1+\sum_{i=2}^k|C_i|>1+\sum_{i=2}^kn^2=1+(k-1)2n^2\geq |G|,
\]
a contradiction. 

\begin{claim}
    There exists a subgroup $H$ of $G$ such that
    $(G:H)\leq 2n^2$.
\end{claim}

Let $C$ be a conjugacy class of $G$ such that 
$|C|\leq 2n^2$. Let $g\in C$.  
Then $H=C_G(g)$ is a subgroup of $G$ such that
$(G:H)\leq 2n^2$. 
This finishes the proof of the Brauer--Fowler theorem. 


\topic{The correspondence theorem}

Let $N$ be a normal subgroup of $G$ 
and $\pi\colon G\to G/N$, $g\mapsto gN$, be the canonical map. 
If $\widetilde{\rho}\colon G/N\to\GL(V)$ 
is a representation of $G/N$ with 
character
$\widetilde{\chi}$, the composition 
$\rho=\widetilde{\rho}\pi\colon G\to \GL(V)$, $\rho(g)=\widetilde{\rho}(gN)$, 
is a representation of $G$. 
Thus
\[
\chi(g)=\trace{\rho_g}=\trace(\widetilde{\chi}(gN))=\widetilde{\chi}(gN).
\]
In particular, $\chi(1)=\widetilde{\chi}(1)$. The character $\chi$ 
is the \textbf{lifting} to $G$ of the character 
$\widetilde{\chi}$ of $G/N$. 

\begin{proposition}
If $\chi\in\Irr(G)$, then 
\[
\ker\chi=\{g\in G:\chi(g)=\chi(1)\}
\]
is a normal subgroup of $G$. 
\end{proposition}

\begin{proof}
Let $\rho\colon G\to\GL_n(\C)$ be a representation with character $\chi$. Then 
$\ker\rho\subseteq\ker\chi$, as $\rho_g=\id$ implies 
$\chi(g)=\trace(\rho_g)=n=\chi(1)$. We claim that  
$\ker\chi\subseteq\ker\rho$. If $g\in G$ is such that $\chi(g)=\chi(1)$, since 
$\rho_g$ is diagonalizable, there exist eigenvalues $\lambda_1,\dots,\lambda_n\in\C$ such that
\[
n=\chi(1)=\chi(g)=\sum_{i=1}^n\lambda_i.
\]
Since each $\lambda_i$ is a root of one,  
$\lambda_1=\cdots=\lambda_n=1$. Hence $\rho_g=\id$. 
\end{proof}

\index{Kernel!of a character}
If $\chi$ is an irreducible character, the subgroup $\ker\chi$ 
is the \textbf{kernel} of $\chi$. 

\begin{theorem}[Correspondence theorem]
\index{Correspondence theorem!for characters}
Let $N$ be a normal subgroup of a finite group $G$. There exists
a bijective correspondence 
\[
\Char(G/N) \longleftrightarrow \{\chi\in\Char(G): 
N\subseteq\ker\chi\}
\]
that maps irreducible characters to irreducible characters.
\end{theorem}

\begin{proof}
If $\widetilde{\chi}\in\Char(G/N)$, let $\chi$ be the lifting of $\widetilde{\chi}$ to $G$. If $n\in N$, 
then
\[
\chi(n)=\widetilde{\chi}(nN)=\widetilde{\chi}(N)=\chi(1)
\]
and thus $N\subseteq\ker\chi$. 

If $\chi\in\Char(G)$ is such that $N\subseteq\ker\chi$, let $\rho\colon G\to\GL(V)$ be a representationn
with character $\chi$. 
Let $\widetilde{\rho}\colon G/N\to\GL(V)$, $gN\mapsto \rho(g)$. We claim that $\widetilde{\rho}$
is well-defined: 
\[
gN=hN\Longleftrightarrow h^{-1}g\in N\Longleftrightarrow\rho(h^{-1}g)=\id\Longleftrightarrow \rho(h)=\rho(g).
\]
Moreover, $\widetilde{\rho}$ is a representation, as 
\[
\widetilde{\rho}((gN)(hN))=\widetilde{\rho}(ghN)=\rho(gh)=\rho(g)\rho(h)=\widetilde{\rho}(gN)\widetilde{\rho}(hN).
\]
If $\widetilde{\chi}$ is the character of $\widetilde{\rho}$, then 
$\widetilde{\chi}(gN)=\chi(g)$.

We now prove that $\chi$ is irreducible if and only if 
$\widetilde{\chi}$ is irreducible. If $U$ is a subspace of $V$, then 
\begin{align*}
\text{$U$ is invariant}
%&\Longleftrightarrow g\cdot U\subseteq U\text{ for all $g\in G$}\\
&\Longleftrightarrow \rho(g)(U)\subseteq U\text{ for all $g\in U$}\\
&\Longleftrightarrow \widetilde{\rho}(gN)(U)\subseteq U\text{ for all $g\in U$}.
\shortintertext{Thus}
\chi\text{ is irreducible }&\Longleftrightarrow
\rho\text{ is irreducible }\\
&\Longleftrightarrow\widetilde{\rho}\text{ is irreducible }\Longleftrightarrow
\widetilde{\chi}\text{ is irreducible }\qedhere.
\end{align*}
\end{proof}

\begin{example}
    Let $G=\Sym_4$ and $N=\{\id,(12)(34),(13)(24),(14)(23)\}$. We know that $N$ is normal in $G$ 
    and that $G/N=\langle a,b\rangle\simeq\Sym_3$, where 
    $a=(123)N$ and $b=(12)N$. 
    The character table of $G/N$ is 
    \begin{center}
		\begin{tabular}{|c|rrr|}
			\hline
			%& $1$ & $3$ & $2$\tabularnewline
			& $1$ & $(12)N$ & $(123)N$ \tabularnewline
			\hline 
			$\widetilde{\chi}_{1}$ & $1$ & $1$ & $1$\tabularnewline
			$\widetilde{\chi}_{2}$ & $1$ & $-1$ & $1$ \tabularnewline
			$\widetilde{\chi}_{3}$ & $2$ & $0$ & $-1$ \tabularnewline
			\hline
		\end{tabular}
	\end{center}
    For each $i\in\{1,2,3\}$ we compute the lifting $\chi_i$ to $G$ of the character  
    $\widetilde{\chi}_i$ of $G/N$. 
    Since $(12)(34)\in N$ and $(13)(1234)=(12)(34)\in N$, 
    \begin{align*}
        \chi( (12)(34) )=\widetilde{\chi}(N),\quad
        \chi( (1234) )=\widetilde{\chi}((13)N)=\widetilde{\chi}((12)N).
    \end{align*}
    Since the characters $\widetilde{\chi_i}$ are irreducibles, 
    the liftings $\chi_i$ are also irreducibles. With this process
    we obtain the following irreducible characters of $G$:
    	\begin{center}
		\begin{tabular}{|c|rrrrr|}
			\hline
			& $1$ & $(12)$ & $(123)$ & $(12)(34)$ & $(1234)$ \tabularnewline
			\hline 
			$\chi_{1}$ & $1$ & $1$ & $1$ & 1 & 1\tabularnewline
			$\chi_{2}$ & $1$ & $-1$ & $1$ & 1 & -1 \tabularnewline
			$\chi_{3}$ & $2$ & $0$ & $-1$ & 2 & 0\tabularnewline
			\hline
		\end{tabular}
	\end{center}
\end{example}

The character table of a group can be used to find the lattice 
of normal subgroups. In particular, the character table detect simple groups. 

\begin{lemma}
    Let $G$ be a finite group and 
    let $g,h\in G$. Then $g$ and $h$ 
    are conjugate if and only if 
    $\chi(g)=\chi(h)$ for all
    $\chi\in\Char(G)$. 
\end{lemma}

\begin{proof}
    If $g$ and $h$ are conjugate, then $\chi(g)=\chi(h)$, as characters are class functions
    of $G$.
    Conversely, if $\chi(g)=\chi(h)$ for all $\chi\in\Char(G)$, then 
    $f(g)=f(h)$ for all class function $f$ of $G$, 
    as characters $G$ generate the space of class functions of $G$. In particular, 
    $\delta(g)=\delta(h)$, where
    \[
    \delta(x)=\begin{cases}
    1 & \text{if $x$ and $g$ are conjugate},\\
    0 & \text{otherwise}.
    \end{cases}
    \]
    This implies that $g$ and $h$ are conjugate.
\end{proof}

As a consequence, we get that 
\begin{equation}
\label{eq:kernels}
\bigcap_{\chi\in\Irr(G)}\ker\chi=\{1\}.
\end{equation}
Indeed, if $g\in\ker\chi$ for all $\chi\in\Irr(G)$, then $g=1$ since 
the lemma implies that $g$ and $1$ are conjugate
because 
$\chi(g)=\chi(1)$ for all $\chi\in\Irr(G)$.

\begin{proposition}
\label{pro:normal}
    Let $G$ be a finite group. 
    If $N$ is a normal subgroup of $G$, 
    then there exist characters
    $\chi_1,\dots,\chi_k\in\Irr(G)$ 
    such that
    \[
    N=\bigcap_{i=1}^k\ker\chi_i.
    \]
\end{proposition}

\begin{proof}
    Apply the previous remark to the group $G/N$ to obtain that 
    \[
    \bigcap_{\widetilde{\chi}\in\Irr(G/N)}\ker\widetilde{\chi}=\{N\}.
    \]
    Assume that $\Irr(G/N)=\{\widetilde{\chi}_1,\dots,\widetilde{\chi}_k\}$. 
    We lift the irreducible characters of $G/N$ to $G$ 
    to obtain (some) irreducible characters $\chi_1,\dots,\chi_k$ 
    of $G$ such that 
    $N\subseteq\ker\chi_1\cap\cdots\cap\ker\chi_k$. 
    If $g\in\ker\chi_i$ for all $i\in\{1,\dots,k\}$, then 
    \[
    \widetilde{\chi}_i(N)=\chi_i(1)=\chi_i(g)=\widetilde{\chi}_i(gN)
    \]
    for all $i\in\{1,\dots,k\}$. This implies that
    \[
    gN\in\bigcap_{i=1}^k\ker\widetilde{\chi}_i=\{N\},
    \]
    that is $g\in N$. 
\end{proof}

\index{Group!simple}
Recall that a non-trivial group is \textbf{simple} if it contains no non-trivial normal 
proper subgroups. Examples of simple groups are cyclic groups of prime order
and the alternating groups $\Alt_n$ for $n\geq5$. 
As a corollary of Proposition \ref{pro:normal}, 
we can use the character table to detect simple groups.

\begin{proposition}
    Let $G$ be a finite group. Then $G$ is not simple if and only if 
    there exists a non-trivial irreducible character $\chi$ such that
    $\chi(g)=\chi(1)$ 
    for some $g\in G\setminus\{1\}$. 
\end{proposition}

\begin{proof}
    If $G$ is not simple, there exists a normal subgroup $N$ of $G$ such that
    $N\ne G$ and $N\ne\{1\}$. 
    By Proposition \ref{pro:normal}, there exist characters 
    $\chi_1,\dots,\chi_k\in\Irr(G)$
    such that 
    $N=\ker\chi_1\cap\cdots\cap\ker\chi_k$.
    In particular, there exists a non-trivial character
    $\chi_i$ such that $\ker\chi_i\ne\{1\}$. Thus 
    there exists $g\in G\setminus\{1\}$ such that
    $\chi_i(g)=\chi_i(1)$. 
    
    Assume now that there exists a non-trivial character $\chi$ 
    such that $\chi(g)=\chi(1)$ for some $g\in G\setminus\{1\}$. In particular, $g\in\ker\chi$ 
    and hence $\ker\chi\ne\{1\}$. Since $\chi$ is non-trivial, $\ker\chi\ne G$. 
    Thus $\ker\chi$ is a proper non-trivial normal subgroup of $G$.
\end{proof}

\begin{example}
\index{Mathieu's group $M_9$}
    If there exists a group $G$ with
    a character table 
    of the form
    \begin{center}
		\begin{tabular}{|c|rrrrrr|}
			\hline
			$\chi_{1}$ & 1 & 1 & 1 & 1 & 1 & 1\tabularnewline
			$\chi_{2}$ & 1 & 1 & 1 & -1 & 1 & -1 \tabularnewline
			$\chi_{3}$ & 1 & 1 & 1 & 1 & -1 & -1\tabularnewline
		    $\chi_{4}$ & 1 & 1 & 1 & -1 & -1 & 1\tabularnewline
			$\chi_{5}$ & 2 & -2 & 2 & 0 & 0 & 0\tabularnewline
			$\chi_{6}$ & 8 & 0 & -1 & 0 & 0 & 0\tabularnewline
			\hline
		\end{tabular}
	\end{center}
	then $G$ cannot be simple. Note that such a group $G$ would have order $\sum_{i=1}^6\chi_i(1)^2=72$. 
	Mathieu's group $M_{9}$ has precisely this character table! 
\end{example}

\begin{example}
    Let $\alpha=\frac{1}{2}(-1+\sqrt{7}i)$. 
    If there exists a group $G$ with a character table
    of the form
    \begin{center}
		\begin{tabular}{|c|rrrrrr|}
			\hline
			$\chi_{1}$ & 1 & 1 & 1 & 1 & 1 & 1\tabularnewline
			$\chi_{2}$ & 7 & -1 & -1 & 1 & 0 & 0 \tabularnewline
			$\chi_{3}$ & 8 & 0 & 0 & -1 & 1 & 1\tabularnewline
		    $\chi_{4}$ & 3 & -1 & 1 & 0 & $\alpha$ & $\overline{\alpha}$ \tabularnewline
			$\chi_{5}$ & 3 & -1 & 1 & 0 & $\overline{\alpha}$ & $\alpha$\tabularnewline
			$\chi_{6}$ & 6 & 2 & 0 & 0 & 0 & 0\tabularnewline
			\hline
		\end{tabular}
	\end{center}    
	then $G$ is simple. Note that such a group $G$ would have order 
	$\sum_{i=1}^6\chi_i(1)^2=168$. 
	The group  
	\[
	\PSL_2(7)=\SL_2(7)/Z(\SL_2(7))
	\]
	is a simple group that has precisely this character table!  
\end{example}


\topic{Frobenius's reciprocity}

We now present a very quick version of Frobenius'
reciprocity theorem. We first 
define restriction of class functions. 

\begin{definition}
    Let $G$ be a finite group and $f\colon G\to\C$ be
    a map. For a subgroup $H$ of $G$, the \textbf{restriction}
    of $f$ to $H$ is the map 
    $=\Res_H^G=f|_H\colon H\to\C$, $h\mapsto f(h)$. 
\end{definition}

\begin{exercise}
\label{xca:restriction}
    Let $G$ be a finite group. Prove that
    the map $\Res_H^G\colon\cf(G)\to\cf(H)$, $f\mapsto\Res_H^G(f)$, 
    is a well-defined linear map. 
\end{exercise}

We now define induction. Let $G$ be a finite group
and $H$ be a subgroup of $G$. If $f\colon G\to\C$ is a map, 
then 
\[
\dot{f}(x)=\begin{cases}
    f(x) & \text{if $x\in H$},\\
    0 & \text{otherwise}.
    \end{cases}
\]
It is an exercise to prove that
the map $f\mapsto\dot{f}$ is linear. 

\begin{definition}
    Let $G$ be a finite group and $f\colon G\to\C$ be
    a map. For a subgroup $H$ of $G$, the \textbf{induction}
    of $f$ to $H$ is the map 
    \begin{align*}
      g\mapsto\Ind_H^Gf(g)=\frac{1}{|H|}\sum_{x\in G}\dot{f}(x^{-1}gx).
    \end{align*}
\end{definition}

\begin{exercise}
\label{xca:induction}
    Let $G$ be a finite group. Prove that
    the map $\Ind_H^G\colon\cf(H)\to\cf(G)$, $f\mapsto\Ind_H^G(f)$, 
    is a well-defined linear map. 
\end{exercise}

\begin{theorem}[Frobenius' reciprocity theorem]
\index{Frobenius' reciprocity theorem}
    Let $G$ be a finite group and $H$ be a subgroup of $G$. 
    If $a\in\cf(H)$ and $b\in\cf(G)$, then
    \[
    \langle\Ind_H^Ga,b\rangle=\langle a,\Res_H^Gb\rangle.
    \]
\end{theorem}

\begin{proof}
    It follows from a direct calculation:
    \begin{equation}
    \label{eq:reciprocity}
    \begin{aligned}
        \langle\Ind_H^Ga,b\rangle 
        &= \frac{1}{|G|}\sum_{x\in G}\Ind_H^Ga(x)\overline{b(x)}
        = \frac{1}{|G|}\frac{1}{|H|}\sum_{x,y\in G}\dot{f}(y^{-1}xy)\overline{b(x)}.
    \end{aligned}
    \end{equation}
    Since 
    \[
    \dot{a}(y^{-1}xy)\ne 0\Longleftrightarrow
    y^{-1}xy\in H\Longleftrightarrow x\in yHy^{-1},
    \]
    setting $h=y^{-1}xy$ 
    we can write \eqref{eq:reciprocity} as 
    \begin{align*}
        \langle\Ind_H^Ga,b\rangle
        &=\frac{1}{|G|}\frac{1}{|H|}\sum_{x\in G}\sum_{h\in H}a(h)\overline{b(xhx^{-1})}\\
        &=\frac{1}{|G|}\frac{1}{|H|}\sum_{x\in G}\sum_{h\in H}a(h)\overline{b(h)}\\
        &=\frac{1}{|G|}\sum_{x\in G}\langle a,\Res_H^Gb\rangle.
    \end{align*}
    From this the claim follows. 
\end{proof}

% 8.1.4 de Steinberg
% Seccion 8.2 define la inducción de representaciones
% 8.2.1 Induce la trivial del trivial a todo el grupo o obtiene la regular
% 8.2.2 Induce la representación por permutaciones
% Define la matriz y mira el dihedral de orden 2n (yo tengo un caso particular)
% Hace el ejemplo de Q8
% En el teorema 8.2.5 prueba la inducción da una representación

\topic{Frobenius' groups}
\label{Frobenius}

If $p$ is a prime number, then
the units $(\Z/p)^{\times}$ 
of $\Z/p$ form a multiplicative group. Moreover, 
$(\Z/p)^{\times}$ 
is cyclic of order $p-1$. 

Let $p$ and $q$ be prime numbers such that $q$ divides $p-1$. Let 
\[
G=\left\{\begin{pmatrix}
x & y\\
0 & 1
\end{pmatrix}
:x\in(\Z/p)^\times,\,y\in\Z/p\right\}.
\]
Then $G$ is a group with the usual matrix multiplication
and $|G|=p(p-1)$. Let $z\in\Z$ be an element of order $q$ modulo $p$ 
and let 
\[
a=\begin{pmatrix}
1&1\\
0&1
\end{pmatrix},
\quad
b=\begin{pmatrix}
z&1\\
0&1
\end{pmatrix},
\quad
H=\langle a,b\rangle.
\]
A direct calculation shows that 
\begin{equation}
\label{eq:pq}
a^p=b^q=\begin{pmatrix}
1&0\\
0&1
\end{pmatrix},
\quad
bab^{-1}=\begin{pmatrix}
1&z\\
0&1
\end{pmatrix}
=a^z.
\end{equation}
Every element of $H$ is of the form $a^ib^j$ for $i\in\{0,\dots,p-1\}$ and  $j\in\{0,\dots,q-1\}$. 
Thus $|H|=pq$. Using~\eqref{eq:pq} we can compute 
the multiplication table of $G$. 

\begin{exercise}
    Let $p$ and $q$ be prime numbers such that $q$ divides $p-1$. Let
    $u,v\in\Z$ be elements of order $q$ modulo $p$. 
    Prove that 
    \[
    \langle a,b:a^p=b^q=1,bab=a^u\rangle
    \simeq \langle a,b:a^p=b^q=1,bab=a^v\rangle.
    \]
\end{exercise}

The group   
\[
F_{p,q}=\langle a,b:a^p=b^q=1,bab=a^u\rangle,
\]
where $u\in\Z$ has order $q$ modulo $p$, 
is a particular case of a  
\emph{Frobenius group}. 

\begin{proposition}
    Let $p$ and $q$ be prime numbers such that $p>q$. Let  
    $G$ be a group of order $pq$. Then either $G$ is abelian or
    $q$ divides $p-1$ and 
    $G\simeq F_{p,q}$.
\end{proposition}

\begin{proof}
    Assume that $G$ is not abelian. By Sylow's theorems, 
    $q$ divides $p-1$ and there exists 
    a unique Sylow $p$-subgroup $P$ of $G$. Let $a,b\in G$ be such that 
    $P=\langle a\rangle\simeq\Z/p$ and $G/P=\langle bP\rangle\simeq\Z/q$. By Lagrange's theorem, 
    $G=\langle a,b\rangle$. We compute the order of $b^q$. Since 
    $G$ is not cyclic (because it is not abelian) and $b^q\in P$, 
    we conclude that $|b^q|=q$. 
    Since $P$ is normal in $G$, 
    $bab^{-1}\in P$ and hence $bab^{-1}=a^z$ for some $z\in\Z$. Therefore
    $b^qab^{-q}=a^{z^q}$. This implies that 
    $z^q\equiv1\bmod p$. The order of $u$ in $(\Z/p)^{\times}$ divides 
    $q$ and hence it is equal to $q$ (otherwise, $u=1$ and thus $bab^{-1}=a$, which implies
    that $G$ is abelian). In conclusion, 
    $G\simeq F_{p,q}$. 
\end{proof}

With the proposition we prove, for example, that 
every group of order 15 is abelian. We can also prove
that up to isomorphism $\Z/20$ and  
$F_{5,4}$ are the only groups of order 20. 

\begin{definition}
  \index{Frobenius!complement}
  \index{Frobenius!kernel}
  \index{Frobenius!group}
  We say that a finite group $G$ is a 
  \textbf{Frobenius group} if $G$ 
  has a non-trivial proper subgroup $H$ such that $H\cap
  xHx^{-1}=\{1\}$ for all $x\in G\setminus H$. In this case, the subgroup 
  $H$ is called a \textbf{Frobenius complement}.
\end{definition}

\begin{theorem}[Frobenius]
  \label{thm:Frobenius}
  \index{Frobenius'!Theorem}
  Let $G$ be a Frobenius group with complement $H$. Then
  \[
	N=\left( G\setminus\bigcup_{x\in G}xHx^{-1}\right)\cup\{1\}
  \]
  is a normal subgroup of $G$.
\end{theorem}

\begin{proof}
  For each $\chi\in\Irr(H)$, $\chi\ne1_H$, let 
  $\alpha=\chi-\chi(1)1_H\in\cf(H)$, where $1_H$ denotes the trivial character of $H$. 

  We claim that $(\alpha^G)_H=\alpha$.
  First, $\alpha^G(1)=\alpha(1)=0$. If $h\in H\setminus\{1\}$, then, by Corollary~\ref{cor:induccion}, 
  \[
    \alpha^G(h)=\frac{1}{|H|}\sum_{\substack{x\in G\\x^{-1}hx\in H}}\alpha(x^{-1}hx)
    =\frac{1}{|H|}\sum_{x\in H}\alpha(h)=\alpha(h),
  \]
  since, if $x\not\in H$, then $x^{-1}hx\in H$ implies that 
  $h\in H\cap xHx^{-1}=\{1\}$.

  By Frobenius' reciprocity, 
  \begin{equation}
    \label{eq:<a,a>=1+chi2}
    \langle\alpha^G,\alpha^G\rangle
    =\langle\alpha,(\alpha^G)_H\rangle=\langle\alpha,\alpha\rangle
    =1+\chi(1)^2.
  \end{equation}
  Again, by Frobenius' reciprocity, 
  \[
  \langle\alpha^G,1_G\rangle
  =\langle\alpha,(1_G)_H\rangle
  =\langle\alpha,1_H\rangle
  =\langle\chi-\chi(1)1_H,1_H\rangle
  =-\chi(1),
  \]
  where $1_G$ is the trivial character of $G$. If we write 
  \[
  \alpha^G=\sum_{\eta\in\Irr(G)}\langle\alpha^G,\eta\rangle\eta
  =\langle\alpha^G,1_G\rangle1_G+\underbrace{\sum_{\substack{1_G\ne\eta\\\eta\in\Irr(G)}}\langle\alpha^G,\eta\rangle\eta}_{\phi},
  \]
  then $\alpha^G=-\chi(1)1_G+\phi$, where $\phi$ is a linear combination 
  (with integer coefficients) of non-trivial 
  irreducible characters of $G$. We compute 
  \[
  1+\chi(1)^2=\langle\alpha^G,\alpha^G\rangle
  =\langle\phi-\chi(1)1_G,\phi-\chi(1)1_G\rangle
  =\langle\phi,\phi\rangle+\chi(1)^2
  \]
  and hence $\langle\phi,\phi\rangle=1$. 
  
  \begin{claim}
  If $\eta\in\Irr(G)$ is such that $\eta\ne 1_G$, then 
  $\langle\alpha^G,\eta\rangle\in\Z$. 
  \end{claim}
  
  By Frobenius' reciprocity, $\langle\alpha^G,\eta\rangle=\langle\alpha,\eta_H\rangle$. 
  If we decompose $\eta_H$ into irreducibles of $H$, say 
  \[
  \eta_H=m_11_H+m_2\chi+m_3\theta_3+\cdots+m_t\theta_t
  \]
  for some $m_1,m_2,\dots,m_t\geq0$, 
  then, since 
  \begin{align*}
  \langle\alpha,1_H\rangle=\langle\chi-\chi(1)1_H,1_H\rangle=-\chi(1),
  &&\langle\alpha,\chi\rangle=\langle\chi-\chi(1)1_H,\chi\rangle=1,
  \end{align*}
  and 
  \[
  \langle\alpha,\theta_j\rangle=\langle\chi-\chi(1)1_H,\theta_j\rangle=0
  \]
  for all $j\in\{3,\dots,t\}$, we conclude that 
  \[
  \langle\alpha^G,\eta\rangle=-m_1\chi(1)+m_2\in\Z.
  \]
  
  \begin{claim}
  $\phi\in\Irr(G)$.
  \end{claim}
  
  Since $\langle\alpha^G,\eta\rangle\in\Z$ for all $\eta\in\Irr(G)$ such that 
  $\eta\ne 1_G$ and 
  \[
  1=\langle\phi,\phi\rangle
  =\sum_{\substack{\eta,\theta\in\Irr(G)\\\eta,\theta\ne1_G}}\langle\alpha^G,\eta\rangle\langle\alpha^G,\theta\rangle\langle\eta,\theta\rangle
  =\sum_{\substack{\eta\ne 1_G\\\eta\in\Irr(G)}}\langle\alpha^G,\eta\rangle^2,
  \]
  there is a unique $\eta\in\Irr(G)$ such that 
  $\langle\alpha^G,\eta\rangle^2=1$ and all the other products are zero, 
  that is 
  $\alpha^G=\pm\eta$ for some $\eta\in\Irr(G)$. Since 
  \[
  \chi-\chi(1)1_H=\alpha=(\alpha^G)_H=(\phi-\chi(1)1_G)_H=\phi_H-\chi(1)1_H,
  \]
  it follows that $\phi(1)=\phi_H(1)=\chi(1)\in\Z_{\geq1}$. Thus $\phi\in\Irr(G)$. 

  \medskip
  We have proved that if $\chi\in\Irr(H)$ is such that $\chi\ne 1_H$, then 
  there exists $\phi_\chi\in\Irr(G)$ such that $(\phi_\chi)_H=\chi$. 
  
  \medskip
  We prove that $N$ is equal to 
  \[
	M=\bigcap_{\substack{\chi\in\Irr(H)\\\chi\ne1_H}}\ker\phi_{\chi}.
  \]

  We first prove that $N\subseteq M$. 
  Let $n\in N\setminus\{1\}$ and $\chi\in\Irr(H)\setminus\{1_H\}$. Since $n$ 
  does not belong to a conjugate of 
  $H$, 
  \[
	\alpha^G(n)=\frac{1}{|H|}\sum_{\substack{x\in G\\x^{-1}nx\in H}}\chi(x^{-1}nx)=0, 
  \]
  as $n\in N$ implies that the set $\{x\in G:x^{-1}nx\in H\}$ is empty. Since 
  \[
  0=\alpha^G(n)
  =\phi_{\chi}(n)-\chi(1)=\phi_{\chi}(n)-\phi_{\chi}(1),
  \]
  we conclude that $n\in\ker\phi_{\chi}$. 
  
  We now prove that $M\subseteq N$. 
  Let $h\in M\cap H$ and $\chi\in\Irr(H)\setminus\{1_H\}$. Then 
  \[
    \phi_{\chi}(h)-\chi(1)=\alpha^G(h)=\alpha(h)=\chi(h)-\chi(1),
  \]
  and $h\in\ker\chi$, as
  \[
    \chi(h)=\phi_{\chi}(h)=\phi_{\chi}(1)=\chi(1).
  \]
  Therefore $h\in\cap_{\chi}\ker\chi=\{1\}$. By~\eqref{eq:kernels}, the kernels
  of irreducible characters have trivial intersection. 
  We now prove that $M\cap
  xHx^{-1}=\{1\}$ for all $x\in G$. Let $x\in G$ y $m\in M\cap xHx^{-1}$. Since 
  $m=xhx^{-1}$ for some $h\in H$, $x^{-1}mx\in H\cap M=\{1\}$.  This imples that 
  $m=1$.
\end{proof}

At the moment, there is no proof of Frobenius' theorem independent of character theory. 

\begin{definition}
  \index{Frobenius!kernel}
  Let $G$ be a Frobenius group. The normal subgroup 
  $N$ of Frobenius' theorem is called the \textbf{Frobenius kernel}. 
\end{definition}

\begin{corollary}
  Let $G$ be a Frobenius group with complement $H$. 
  Then there exists a normal subgroup $N$ of $G$ 
  such that 
  $G=HN$ and $H\cap N=\{1\}$.
\end{corollary}

\begin{proof}
  Frobenius' theorem yields the subgroup $N$. 
  Let us prove that $H\subseteq N_H(H)$. If $h\in
  H\setminus\{1\}$ and $g\in G$ are such that $ghg^{-1}\in H$, then $h\in
  g^{-1}Hg\cap H$ and hence $g\in H$. Since $H=N_G(H)$, the subgroup $H$
  has $(G:H)$ conjugates. Thus $|G|=|H||N|$, as 
  \[
    |N|=|G|-(G:H)(|H|-1)=(G:H).
  \]
  Since $N\cap H=\{1\}$, 
  \[
  |HN|=|N||H|/|H\cap N|=|N||H|=|G|
  \]
  and therefore $G=NH$.
\end{proof}

\begin{corollary}[Combinatorial Frobenius' theorem]
    \label{cor:Frobenius_combinatorio}
    \index{Frobenius'!theorem}
    Sea $X$ un conjunto finito y sea $G$ un grupo que actúa transitivamente en
    $X$. Supongamos que todo $g\in G\setminus\{1\}$ fija a lo sumo un punto de
     $X$. El conjunto $N$ formado por la identidad y las permutaciones que mueven
     todos los puntos de $X$ es un subgrupo de $G$.
\end{corollary}

\begin{proof}
  Sea $x\in X$ y sea $H=G_x$. Veamos que si $g\in G\setminus H$ entonces $H\cap
  gHg^{-1}=1$. Si $h\in H\cap gHg^{-1}$ entonces $h\cdot x=x$ y $g^{-1}hg\cdot
  x=x$. Como $g\cdot x\ne x$, entonces $h$ fija dos puntos de $X$. Esto implica
  que $h=1$ (pues todo elemento no trivial fija a lo sumo un punto de $X$). 

  Por el teorema~\ref{thm:Frobenius}, el conjunto
  \[
    N=\left(G\setminus\bigcup_{g\in G}gHg^{-1}\right)\cup\{1\}
  \]
  es un subgrupo de $G$. Veamos cómo son los elementos de $N$: Si
  $h\in\cup_{g\in G}gHg^{-1}$ entonces existe $g\in G$ tal que $g^{-1}hg\in H$,
  es decir $(g^{-1}hg)\cdot x=x$ o quivalentemente $h\in G_{g\cdot x}$. Luego,
  a excepción de la identidad, los elementos de $N$ son los elementos de $G$
  que mueven algún punto de $X$.
\end{proof}

\begin{example}
  Sea $F$ un cuerpo finito y sea $G$ el grupo de funciones $f\colon G\to G$ de
  la forma $f(x)=ax+b$, $a,b\in F$ con $a\ne0$. El grupo $G$ actúa en $F$ y toda
  $f\ne\id$ fija a lo sumo un punto de $F$ pues 
  \[
	x=f(x)=ax+b\implies x=1-(b/a).
  \]
  En este caso, $N=\{f:f(x)=x+b\,,b\in F\}$ que es
  un subgrupo de $G$.
\end{example}

\begin{exercise}
    Prove that Theorem~\ref{thm:Frobenius} can be obtained from
    Corollary~\ref{cor:Frobenius_combinatorio}.
\end{exercise}


% Wielandt 8.5.4
% 8.5.6 para ver algo de grupos de permutaciones
% 7.1 para ejemplo H(q)
% 10.5.6 (Thompson) N es nilpotente, se usa 10.5.4 

In his doctoral thesis Thompson proved the following result, conjectured
by Frobenius. 

\begin{theorem}[Thompson]
\index{Thompson's theorem}
    Let $G$ be a Frobenius group. If $N$ is the Frobenius kernel, then $N$ 
    is nilpotent.
\end{theorem}

See~\cite[Theorem 6.24]{MR2426855} for the proof.


