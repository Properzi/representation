\chapter{}

\topic{Representations of $\sl(2,\C)$}

We discuss a particular important Lie algebra: 
\[
\sl(2,\C)=\left\{\begin{pmatrix}
    a & b\\
    c & -a
    \end{pmatrix}:a,b,c\in\C\right\}.
\]
Note that 
$e=\begin{pmatrix}
        0&1\\
        0&0\end{pmatrix}$, $h=\begin{pmatrix}
        1&0\\
        0&-1\end{pmatrix}$ and $f=\begin{pmatrix}0&0\\1&0\end{pmatrix}$ 
        is an ordered basis for $\sl(2,\C)$. In this basis,
\[
[h,e]=2e,\quad
[h.f]=-2f,\quad
[e,f]=h.
\]

Let $V$ be an $\sl(2,\C)$-module. Note that we do not assume that $V$ is simple. 
For $\lambda\in\C$ an eigenvector of $h$ let
\[
V_{\lambda}=\{v\in V:h\cdot v=\lambda v\}
\]
be the \textbf{weight space} of $\lambda$. 
If $\lambda$ is not an eigenvector of $h$, 
we set $V_{\lambda}=\{0\}$. A \textbf{weight} of $V$  
is a scalar $\lambda$ such that $V_{\lambda}\ne\{0\}$. 

\begin{lemma}
    Let $V$ be an $\sl(2,\C)$-module and $v\in V_{\lambda}$. 
    \begin{enumerate}
        \item Either $e\cdot v=0$ or $e\cdot v$ is an eigenvector of $h$ with
            eigenvalue $\lambda+2$.
        \item Either $f\cdot v=0$ or $f\cdot v$ is an eigenvector of $h$ with
            eigenvalue $\lambda-2$.
    \end{enumerate} 
\end{lemma}

\begin{proof}
    We only prove 1):
    \[h\cdot(e\cdot v)=e\cdot(h\cdot v)+[h,e]\cdot v=e\cdot(\lambda v)+2e\cdot v=(\lambda+2)e\cdot v.\qedhere
    \]
\end{proof}

%The previous lemma allows us to describe the actions of 
%$e$, $f$ and $h$ on $V$ with a very nice picture:
%\[
%\begin{tikzcd}
%	{\{0\}} & {V_{-\lambda}} & {V_{-\lambda+2}} & \cdots & {V_{\lambda-2}} & {V_{\lambda}} & {\{0\}}
%	\arrow["f"', curve={height=12pt}, from=1-1, to=1-2]
%	\arrow["f"', curve={height=12pt}, from=1-2, to=1-3]
%	\arrow["f"', curve={height=12pt}, from=1-3, to=1-4]
%	\arrow["f"', curve={height=12pt}, from=1-4, to=1-5]
%	\arrow["f"', curve={height=12pt}, from=1-5, to=1-6]
%	\arrow["e", curve={height=-12pt}, from=1-6, to=1-7]
%	\arrow["e", curve={height=-12pt}, from=1-2, to=1-3]
%	\arrow["e", curve={height=-12pt}, from=1-3, to=1-4]
%	\arrow["e", curve={height=-12pt}, from=1-4, to=1-5]
%	\arrow["e", curve={height=-12pt}, from=1-5, to=1-6]
%	%\draw (3) to [out=330,in=300,looseness=8] (3);
%\end{tikzcd}
%\]

\begin{lemma}
\label{lem:maximal_weight}
    Let $V$ be a finite-dimensional $\sl(2,\C)$-module.
    There exists an eigenvector 
    $w\in V$ of $h$ such that $e\cdot w=0$. 
\end{lemma}

\begin{proof}
    The linear map $h\colon V\to V$ has at least one eigenvector 
    $v$ with eigenvalue $\lambda$. If the elements 
    $v,e\cdot v,e^2\cdot v,\dots$ are non-zero, they are linearly independent, as they 
    form a sequence of eigenvectors of $h$ with different eigenvalues.  
    As $\dim V<\infty$, it follows that there exists $k$ 
    such that $e^k\cdot v\ne 0$ and $e^{k+1}\cdot v=0$. 
    Let $w=e^k\cdot v\ne 0$. 
    Then
    $h\cdot w=(\lambda+2k)\cdot w$ and $e\cdot w=0$. 
\end{proof}

A vector $v\in V$ such that $V_{\lambda}\ne\{0\}$ and
$V_{\lambda+2}=\{0\}$ will be called a \textbf{highest weight vector} 
of weight $\lambda$. 

\begin{lemma}
\label{lem:basis}
    Let $V$ be a finite-dimensional simple $\sl(2,\C)$-module
    and let $w$ be a maximal vector of weight $\lambda$. Let 
    $k$ be such that $f^k\cdot w\ne 0$ and $f^{k+1}\cdot w=0$.
    Then $\{w,f\cdot w,\dots,f^{k}\cdot w\}$ is a basis of $V$.
    Moreover, $\lambda=k$.
\end{lemma}

\begin{proof}
    The elements $w,f\cdot w,\dots,f^{k}\cdot w$ 
    are linearly independent, as they are eigenvectors of $h$ with 
    different eigenvalues. Since $V$ is simple, it is enough 
    to prove that 
    the non-zero subspace $W=\langle w,f\cdot w,\dots,f^k\cdot w\rangle$ 
    is a submodule of $V$. This subspace 
    is invariant under the action of 
    $h$ and $f$. In fact, one easily proves by induction that 
    \[
    (hf^j)\cdot w=(\lambda-2j)f^jw
    \]
    for all $j\geq0$. 
    Let us prove that $W$ is invariant 
    under the action of $e$, that is $e\cdot W\subseteq W$. 
    We claim that 
    \[
    (ef^{j})\cdot w=j(\lambda-j+1)f^{j-1}w\in W
    \]
    for all $j$. We proceed by induction on $j$. Note that 
    the case $j=0$ is trivial, as $e\cdot w=0$. 
    The case $j=1$ is easy:
    \[
    (ef)\cdot w=(h+fe)\cdot w=h\cdot w+f\cdot (e\cdot w)=\lambda w.
    \]
    If the claim holds for some $j$, by using 
    the inductive hypothesis, 
    \begin{align*}
        e\cdot(f^{j+1}\cdot w)&=(ef)\cdot (f^{j}\cdot w)\\
        &=(fe+h)\cdot (f^{j}\cdot w)\\
        &=h\cdot (f^j\cdot w)+j(\lambda-j+1)(f^j\cdot w)\\
        &=(j+1)(\lambda-j)(f^{j}\cdot w).
    \end{align*}
    
    We now compute $\lambda$. 
    The matrix of $h$ with respect to the basis 
    $\{w,f\cdot w,\dots,f^k\cdot w\}$ 
    is diagonal with trace 
    \[
    \lambda+(\lambda-2)+\cdots+(\lambda-2k)=(k+1)(\lambda-k).
    \]
    Since $[e,f]=h$ has trace zero, it follows that $\lambda=k$. 
\end{proof}

We now summarize what we know about 
simple $\sl(2,\C)$-modules.

\begin{theorem}
    Let $V$ be a finite-dimensional simple $\sl(2,\C)$-module.
    Then 
    $V$ is the direct sum of one-dimensional weight spaces
    \begin{equation}
    \label{eq:sl2_decomposition}
        V=V_{\lambda}\oplus V_{\lambda-2}\oplus\cdots\oplus V_{-\lambda+2}\oplus V_{-\lambda}.
    \end{equation}
    In particular, $\dim V=\lambda+1$. 
    The set $\{w,f\cdot w,\dots,f^\lambda\cdot w\}$ 
    is a basis of $V$ and 
    \begin{equation}
        \label{eq:sl2_module}
        \begin{aligned}
            &h\cdot (f^j\cdot w)=(\lambda-2j)f^j\cdot w,\\
            &f\cdot (f^j\cdot w)=f^{j+1}\cdot w,\\
            &e\cdot (f^j\cdot w)=j(\lambda-j+1)f^{j-1}\cdot w.
        \end{aligned}
    \end{equation}
\end{theorem}

\begin{proof}
    By Lemma \ref{lem:maximal_weight}, there exists an eigenvector $w$ of $h$ 
    with eigenvalue $\lambda$ such that $e\cdot w=0$. By Lemma \ref{lem:basis}, 
    the set 
    $\{w,f\cdot w,\dots,f^k\cdot w\}$ is a basis of $V$ and
    the formulas of \eqref{eq:sl2_module} follow. For $j\in\{0,\dots,k\}$
    the complex vector space generated by $f^j\cdot w$ 
    is a one-dimensional weight space of weight $\lambda-2j$. Thus \eqref{eq:sl2_decomposition}
    follows. 
\end{proof}

\begin{exercise}
    Prove that two 
    finite-dimensional $\sl(2,\C)$ generated by
    highest weight vectors of the same weight are isomorphic. 
\end{exercise}

Consider the polynomial ring $\C[X,Y]$ in two commuting variables
$X$ and $Y$. Let $V_d$ be the subspace of homogeneous polynomials
of degree $d$. Then 
\[
\dim V_d=\begin{cases}
    1 & \text{if $d=0$},\\
    d+1 & \text{otherwise},
    \end{cases}
\]
as a basis of $V_d$ is given 
by $\{X^d,X^{d-1}Y,X^{d-2}Y^2,\dots,XY^{d-1},Y^d\}$. 

\begin{exercise}
    Prove that $\varphi\colon\sl(2,\C)\to\gl(V_d)$, 
    \begin{align}
        \varphi(e)=X\frac{\partial}{\partial Y},
        &&
        \varphi(f)=Y\frac{\partial}{\partial X},
        &&
        \varphi(h)=X\frac{\partial}{\partial X}-Y\frac{\partial}{\partial Y},
    \end{align}
    is a representation of $\sl(2,\C)$. 
    This means that 
    \begin{gather*}
    \varphi(e)(X^aY^b)=bX^{a+1}Y^{b-1},
    \quad
    \varphi(f)(X^aY^b)=aX^{a-1}Y^{b+1},
    \shortintertext{and that}
    \varphi(h)(X^aY^b)=(a-b)X^aY^b.
    \end{gather*}
\end{exercise}

In the basis $\{X^d,X^{d-1}Y,X^{d-2}Y^2,\dots,XY^{d-1},Y^d\}$, 
\begin{align*}
\varphi(e)=\left(\begin{smallmatrix}
0 & 1 & 0 & \cdots & 0\\
0 & 0 & 2 & \cdots & 0\\
\vdots & \vdots & \vdots & \ddots & \vdots\\
0 & 0 & 0 & \cdots & d\\
0 & 0 & 0 & \cdots & 0
\end{smallmatrix}\right),
&& 
\varphi(f)=\left(\begin{smallmatrix}
0 & 0 & \cdots & 0 & 0\\
d & 0 & \cdots & 0 & 0\\
0 & d-1 & \cdots & 0 & 0\\
\vdots & \vdots & \ddots & \vdots & \vdots\\
0 & 0 & \cdots & 1 & 0
\end{smallmatrix}\right),
&&
\varphi(h)=\left(\begin{smallmatrix}
d & 0 & \cdots & 0 & 0\\
0 & d-2 & \cdots & 0 & 0\\
\vdots & \vdots & \ddots & \vdots & \vdots\\
0 & 0 & \cdots & -d+2 & 0\\
0 & 0 & \cdots & 0 & -d
\end{smallmatrix}\right).
\end{align*}

\begin{exercise}
    Prove that $V_d$ is generated (as an $\sl(2,\C)$-module) by
    $X^aY^b$ for some $a$ and $b$ such that $a+b=d$. 
\end{exercise}

\begin{exercise}
    Prove that each $V_d$ is a simple $\sl(2,\C)$-module.
\end{exercise}

\begin{exercise}
    Prove that any $V$ finite-dimensional simple 
    $\sl(2,\C)$-module is isomorphic to $V_d$ for some $d$. 
\end{exercise}

The following result is a particular
case of Weyl's theorem in the context of $\sl(2,\C)$-modules. 

\begin{theorem}
    Any finite-dimensional $\sl(2,\C)$-module is absolutely reducible. 
\end{theorem}

\begin{proof}
    Let $V$ be a finite-dimensional $\sl(2,\C)$-module. 
    We proceed in several steps.
    
    \begin{claim}
        The elmenet $Z=\frac12h^2+h+2fe$ commutes with every $X\in\sl(2,\C)$. 
    \end{claim}
    
    We first compute
    \begin{equation}
        \begin{aligned}
            \label{eq:Casimir}
            ZX-XZ &= \frac12h^2X-\frac12Xh^2+[h,X]+2feX-2Xfe\\
            &=\frac12h[e,X]-\frac12[X,h]h+[h,X]+2f[e,X]-2[X,f]e.
        \end{aligned}
    \end{equation}
    Now one checks that 
    Equation \eqref{eq:Casimir} is zero if $X\in\{h,e,f\}$ and
    the claim follows. 
    %See \cite[Theorem V.4.6]{MR1321145} or \cite[Theorem 1.67]{MR1920389}. 
    
    \begin{claim}
        If $\dim V=n+1$, then $Z$ acts as the scalar 
        $\frac12n^2+n$, which is not zero unless $V$ is the trivial module. 
    \end{claim}
    
    By Schur's lemma, $Z$ acts by a scalar. Since $V$ is a simple
    $\sl(2,\C)$-module of dimension $n+1$, $V\simeq V_{n+1}$. In particular, 
    $hv_0=nv_0$ and $ev_0=0$. 
    
    \begin{claim}
        Let $U\subseteq V$ be a submodule of codimension one. Then 
        there exists a submodule $W$ of $V$ such that $V=U\oplus W$ 
        and $\dim W=1$. 
    \end{claim}
    
    We split the proof of the claim into several steps. 
    
    First we assume that
    $\dim U=1$. The quotient module $V/U$ is one-dimensional and hence simple. 
    Thus $\sl(2,\C)V\subseteq U$ and $\sl(2,\C)U=\{0\}$. Hence 
    \[
    [X,Y]V\subseteq XYV-YXV\subseteq XU+YU=\{0\}.
    \]
    Since $\sl(2,\C)=[\sl(2,\C),\sl(2,\C)]$, we conclude that
    $\sl(2,\C)V=\{0\}$. Thus any complement of $U$ will serve as $W$. 
    
    \medskip
    We now finish the proof of the theorem. 
\end{proof}

\topic{Enveloping algebras}

Let $L$ be a finite-dimensional Lie algebra with
basis $\{x_1,\dots,x_n\}$. Write
\[
[x_i,x_j]=\sum_{k=1}^n c^k_{ij}x_k
\]
for scalars $c_{ij}^k\in\C$. 
These scalars are called 
the \textbf{structure constants} of $L$. 

\index{Universal enveloping algebra}
The \textbf{universal enveloping algebra} of $L$ 
is the associative algebra $U(L)$ with generators 
$x_1,\dots,x_n$ and relations 
\[
x_ix_j-x_jx_i=\sum_{k=1}^n c_{ij}^kx_k.
\]

Our definition depends on the choice of the basis of the Lie algebra $L$. 
However, it is possible to define $U(L)$ as the quotient 
of the tensor algebra $T(L)$ by the ideal $I$ 
generated by $x\otimes y-y\otimes x-[x,y]$ for all $x,y\in L$.  

\begin{example}
    If $L$ is an abelian Lie algebra, then $U(L)$ is 
    the symmetric algebra $S(L)$. 
\end{example}

\begin{example}
    The universal enveloping algebra $U(\sl(2,\C))$ 
    is the algebra with generators $e,f,h$ and relations 
    \[
    ef-fe=h,\quad
    hf-fh=-2f,\quad
    he-eh=2e.
    \]
\end{example}


The universal enveloping algebra satisfies a \emph{universal property}. 
Let $L$ be a Lie algebra. If $A$ is an associative algebra, then
$A$ has the structure of a Lie algebra with
bracket 
$[a,b]=ab-ba$. If $f\colon L\to A$ is a homomorphism of Lie algebras, 
then there exists a unique algebra homomorphism
$\varphi\colon U(L)\to A$ such that 
\[
\begin{tikzcd}
	L & A \\
	& U(L)
	\arrow["f", from=1-1, to=1-2]
	\arrow["{\iota }"', from=1-1, to=2-2]
	\arrow["\varphi"', dashed, from=2-2, to=1-2]
\end{tikzcd}
\]
commutes, where $\iota\colon L\to U(L)$ 
denotes the canonical map. 

%\begin{exercise}
%    Let $L$ and $L_1$ be Lie algebras. Prove that 
%    $U(L\oplus L_1)\simeq U(L)\otimes U(L_1)$. 
%\end{exercise}

\begin{theorem}[Poincar\'e--Birkhoff--Witt]
    \index{Poincar\'e--Birkhoff--Witt theorem}
    \index{PBW theorem}
    \label{thm:PBW}
    Let $L$ be a finite-dimensional Lie algebra and 
    $\{x_1,\dots,x_n\}$ be an ordered basis of $L$. Then 
    \[
    \{x_1^{a_1}\cdots x_n^{a_n}:a_1,\dots,a_n\geq0\}
    \]
    is a basis of $U(L)$. 
\end{theorem}

See for example \cite[\S17.4]{MR499562}. 

\begin{exercise}
    Let $L$ be a finite-dimensional Lie algebra and $U(L)$ be its
    universal enveloping algebra. Prove that there exists a bijective
    correspondence between (simple) $L$-modules and (simple) $U(L)$-modules. 
\end{exercise}
